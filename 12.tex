\section{Lecture -- Week 12}


%\subsection{The inverse Galois problem}



\subsection{Group cohomology}

Let $G$ be a group and $A$ be a \emph{(left) $G$-module}. This means that $A$ is an abelian
group together with a map
\[
G\times A\to A,\quad
(g,a)\mapsto g\cdot a
\]
such that $1\cdot a=a$ for all $a\in A$, $(gh)\cdot a=g\cdot (h\cdot a)$ for 
all $g,h\in G$ and $a\in A$ and $g\cdot (a+b)=g\cdot a+g\cdot b$ for
all $g\in G$ and $a,b\in A$. 

\begin{example}
    The group $\Gal(\C/\R)$ acts on $\C$ and $\C^\times$. Moreover, 
    it acts trivially on $\R$ and $\R^\times$. 
\end{example}

More generally, if $E/K$ is a finite Galois extension, then 
the Galois group $\Gal(E/K)$ acts on $E$ and $E^\times$. 

\begin{definition}
    Let $G$ be a group and $M$ and $N$ be $G$-modules. A map 
    $f\colon M\to N$ is a \emph{homomorphism} of $G$-modules
    if $f(\sigma\cdot m)=\sigma\cdot f(m)$ for all $m\in M$ and $\sigma\in G$.
\end{definition}

\begin{definition}
    Let $G$ be a group and $M$ be a $G$-module.
    The submodule of \emph{$G$-invariants} is defined as
    \[
    M^G=\{m\in M:\sigma\cdot m=m\text{ for all $\sigma\in G$}\}.
    \]
\end{definition}

Note that $M^G$ is the largest submodule of the $G$-module 
$M$ where $G$ acts trivially. For example, if 
$G=\Gal(E/K)$, then $E^G=K$. 

\begin{proposition}
\label{pro:H0}
    Let $G$ be a group. If the sequence
of $G$-modules and $G$-module homomorphism
\[
\begin{tikzcd}
    0 & P & M & N & 0
    \arrow[from=1-1, to=1-2]
    \arrow["\alpha", from=1-2, to=1-3]
    \arrow["\beta", from=1-3, to=1-4]
    \arrow[from=1-4, to=1-5]
    \end{tikzcd}\]
is exact, then 
\[
\begin{tikzcd}
                        0 & P^G & M^G & N^G 
                        \arrow[from=1-1, to=1-2]
                        \arrow["\alpha^0", from=1-2, to=1-3]
                        \arrow["\beta^0", from=1-3, to=1-4]
        \end{tikzcd}
        \]
is exact, where $\alpha^0$ is the restriction $\alpha|_{P^G}$ of $\alpha$ to $P^G$ and
$\beta^0$ is the restriction $\beta|_{M^G}$ of $\beta$ to $M^G$. 
\end{proposition}

\begin{proof}
    Since $\alpha$ is injective, the restriction $\alpha^0$ is injective. 

    Note that 
    $\ker\beta^0=\ker\beta\cap M^G\subseteq\ker\beta$. 
    
    We claim 
    that $\alpha^0(P^G)=\alpha(P)\cap M^G$. If $m\in\alpha(P)\cap M^G$, then 
    $\alpha(p)=m$ for some $p\in P$ and $\sigma\cdot m=m$. Since
    \[
    \alpha(p)=m=\sigma\cdot m=\sigma\cdot\alpha(p)=\alpha(\sigma\cdot p),
    \]
    $\sigma\cdot p-p\in\ker\alpha=\{0\}$. Hence $\sigma\cdot p=p$ and
    $p\in P^G$. Conversely, if $m\in\alpha^0(P^G)$, then 
    $m=\alpha(p)$ for some $p\in P^G$. If $\sigma\in G$, then
    \[
    \sigma\cdot m=\sigma\cdot\alpha(p)=\alpha(\sigma\cdot p)=\alpha(p)=m.
    \]
    Hence $m\in M^G\cap\alpha(P)$.

    Now
    \[
    \alpha^0(P^G)=\alpha(P)\cap M^G=\ker\beta\cap M^G=\ker\beta^0.\qedhere  
    \]
\end{proof}

Note that in the previous proposition, we did not prove that
the map $\beta|_{M^G}$ is surjective. 

\begin{example}
    Let $G=\Gal(\C/\R)$. Consider the following exact sequence
    of $G$-modules:
    \[
    \begin{tikzcd}
    1 & \{-1,1\} & \C^\times & \C^\times & 1
    \arrow[from=1-1, to=1-2]
    \arrow[from=1-2, to=1-3]
    \arrow["\beta", from=1-3, to=1-4]
    \arrow[from=1-4, to=1-5]
    \end{tikzcd}    
    \]
    where $\beta(z)=z^2$. Note that $\beta$ is surjective. Take invariants 
    to obtain the sequence  
    \[
    \begin{tikzcd}
     0 & \{-1,1\} & \R^\times & \R^\times 
     \arrow[from=1-1, to=1-2]
     \arrow[from=1-2, to=1-3]
     \arrow["\beta^0", from=1-3, to=1-4]
     \end{tikzcd}
     \]
     where $\beta^0(x)=x^2$. Note that $\beta^0$ is not surjective! 
\end{example}

\begin{definition}
    Let $G$ be a group and $N$ be a $G$-module. 
    We define 
    \begin{align*}
        H^0(G,M)&=M^G,\\
        C^1(G,M)&=\{\phi\colon G\to M:\phi\text{ is a map}\},\\
        Z^1(G,M)&=\{\phi\in C^1(G,M):\phi(\sigma\tau)=\phi(\sigma)+\sigma\cdot\phi(\tau)\text{ for all $\sigma,\tau\in G$}\},
        \end{align*}    
\end{definition}

Note that $Z^1(G,M)$ is an abelian group with the operation
\[
(\phi+\phi_1)(\sigma)=\phi(\sigma)+\phi_1(\sigma).
\]
Moreover, if $\phi\in Z^1(G,M)$, then 
$\phi(1_G)=0_M$. To prove this fact, note that  
\[
\phi(1_G)=\phi(1_G1_G)=\phi(1_G)+1_G\cdot\phi(1_G)=\phi(1_G)+\phi(1_G)
\]
implies
that $\phi(1_G)=0_M$. 

\begin{example}
\label{exa:BinZ}
    Let $G$ be a group and $M$ be a $G$-module. Fix $m\in M$. Then
    the map $\phi\colon G\to M$, $\phi(\sigma)=\sigma\cdot m-m$, is an element 
    of $Z^1(G,M)$, because 
    \begin{align*}
    \phi(\sigma\tau)&=(\sigma\tau)\cdot m-m\\
    &=(\sigma\tau)\cdot m-\sigma\cdot m+\sigma\cdot m-m\\
    &=\sigma\cdot (\tau\cdot m-m)+\sigma\cdot m-m\\
    &=\sigma\cdot \phi(\tau)+\phi(\sigma)   
    \end{align*}
    for all $\sigma,\tau\in G$.
\end{example}

\begin{definition}
    Let $G$ be a group and $M$ be a $G$-module. The set
    $B^1(G,M)$ of \emph{coboundaries} is the set 
    of elements $\phi\in C^1(G,M)$ such that there is a fixed 
    $m\in M$ such that
    $\phi(\sigma)=\sigma\cdot m=m$ for all $\sigma\in G$.
\end{definition}

We proved in Example \ref{exa:BinZ} that  
$B^1(G,M)\subseteq Z^1(G,M)$. A direct calculation shows that, in fact, 
$B^1(G,M)$ is a subgroup of $Z^1(G,M)$. 

\begin{definition}
    Let $G$ be a group and $M$ be a $G$-module. The 
    \emph{first cohomology group} of $G$ with coefficients
    in $M$ is defined as the quotient
    \[
    H^1(G,M)=Z^1(G,M)/B^1(G,M).
    \]
\end{definition}

\begin{example}
    If $G$ acts trivially on $M$, then 
    \[
    H^0(G,M)=M^G=M,
    \quad 
    B^1(G,M)=\{0\},
    \quad 
    Z^1(G,M)=\Hom(G,M).
    \]
    Hence 
    $H^1(G,M)\simeq\Hom(G,M)$.
\end{example}

\begin{example}
    Let $G=\Gal(\C/\R)=\{\id,\gamma\}$, where $\gamma\colon\C\to\C$, $z\mapsto\overline{z}$, is the complex conjugation. Then
    \[
    H^0(G,\R^\times)=\left(\R^\times\right)^G=\R^\times.
    \]
    Since $G$ acts trivially on $\R^\times$, 
    \[
    H^1(G,\R^\times)=\Hom(G,\R^\times)\simeq\Hom(G,\{-1,1\})\simeq\Z/2.
    \]
\end{example}

The following lemma will be useful. 

\begin{lemma}
\label{lem:H1_maps}
    Let $G$ be a group and 
    $\alpha\colon M\to N$ be a homomorphism of $G$-modules. 
    Then 
    \[
    \alpha^1\colon H^1(G,M)\to H^1(G,N),\quad 
    \phi+B^1(G,M)\mapsto \alpha\circ\phi+B^1(G,N),
    \]
    is a group homomorphism. 
\end{lemma}

\begin{proof}
    Let us prove that the map $\alpha^1$ is well-defined. If 
    $\phi-\phi'\in B^1(G,M)$, then 
    there exists a fixed 
    $m\in M$ such that 
    $(\phi-\phi')(\sigma)=\sigma\cdot m-m$ for all $\sigma\in G$. 
    Let $n=\alpha(m)\in N$. 
    For $\sigma\in G$, 
    \[
    \alpha((\phi-\phi')(\sigma))=\alpha(\sigma\cdot m-m)
    =\sigma\cdot \alpha(m)-\alpha(m)=\sigma\cdot n-n. 
    \]
    Thus $\alpha\circ\phi-\alpha\circ\phi'\in B^1(G,N)$. 

    We now prove that $\alpha^1$ is a group homomorphism. If 
    $\phi,\phi'\in Z^1(G,M)$, then 
    \begin{align*}
    \alpha^1(\phi+B^1(G,M)+\phi'+B^1(G,M))
    &=\alpha^1(\phi+\phi'+B^1(G,M))\\
    &=\alpha\circ(\phi+\phi')+B^1(G,N)\\
    &=\alpha\circ\phi+\alpha\circ\phi'+B^1(G,N)\\
    &=\alpha\circ\phi+B^1(G,N)+\alpha\circ\phi'+B^1(G,N)\\
    &=\alpha^1(\phi+B^1(G,M))+\alpha^1(\phi'+B^1(G,M)).\qedhere
    \end{align*}
\end{proof}

We will provide a detailed proof of the upcoming result. 
The theorem will be established by applying a \emph{diagram chasing} technique, a widely used tool in homological algebra.
The proof is tedious and may seem intricate, but the method essentially 
involves ``chasing'' elements around a (commutative) diagram until we attain the desired result. 

\begin{theorem}
Let $G$ be a group and 
\[
\begin{tikzcd}
    0 & P & M & N & 0
    \arrow[from=1-1, to=1-2]
    \arrow["\alpha", from=1-2, to=1-3]
    \arrow["\beta", from=1-3, to=1-4]
    \arrow[from=1-4, to=1-5]
    \end{tikzcd}
\] 
be an exact sequence of $G$-modules and $G$-module homomorphism. 
Then there exists a \emph{connection homomorphism} $\delta$ such that the sequence 
\begin{equation}
\label{eq:long}
\begin{tikzcd}
 0\rar & H^0(G,P)\rar["\alpha^0"] & H^0(G,M) \rar["\beta^0"]
             \ar[draw=none]{d}[name=X, anchor=center]{}
    & H^0(G,N) \ar[rounded corners,
            to path={ -- ([xshift=2ex]\tikztostart.east)
                      |- (X.center) \tikztonodes
                      -| ([xshift=-2ex]\tikztotarget.west)
                      -- (\tikztotarget)}]{dll}[at end]{\delta} \\      
  &H^1(G,P) \rar["\alpha^1"] & H^1(G,M) \rar["\beta^1"] & H^1(G,N)
\end{tikzcd}
\end{equation}
of abelian groups
and group homomorphisms is exact. 
\end{theorem}

\begin{proof}
    By Proposition \ref{pro:H0}, the long 
    sequence~\eqref{eq:long} is exact at $H^0(G,P)=P^G$, 
    $H^0(G,M)=M^G$ and 
    $H^0(G,N)=N^G$. Note that, in particular, 
    $\alpha\colon P\to\alpha(P)$ is a bijective group homomorphism. 


    Let us construct the connecting
    homomorphism $\delta\colon H^0(G,N)\to H^1(G,P)$. For 
    $n\in N^G$, let $m\in M$ be such that 
    $\beta(m)=n$. We define $\delta(n)=\phi+B^1(G,P)$, where 
    \[
    \phi(\sigma)=\alpha^{-1}(\sigma\cdot m-m).
    \]
    Note that $\sigma\cdot m-m\in\im\alpha=\ker\beta$, as 
    \[
    \beta(\sigma\cdot m-m)=\sigma\cdot \beta(m)-\beta(m)=\sigma\cdot n-n=0.
    \]

    Let us prove that the map $\delta$ is well-defined: if $m,m'\in M$ are such that
    \[
    \beta(m)=\beta(m')=n,
    \]
    then  $m-m'\in\ker\beta=\alpha(P)$. 
    For $\sigma\in G$, write $\phi'(\sigma)=\sigma\cdot m'-m'$. 
    Since  
    $m-m'=\alpha(p)$ for some $p\in P$ and  
    $\alpha^{-1}$ is a homomorphism of $G$-modules, 
    \begin{align*}
    \phi(\sigma)-\phi'(\sigma)&=\alpha^{-1}(\sigma\cdot m-m)
    -\alpha^{-1}(\sigma\cdot m'-m')\\
    &=\alpha^{-1}(\sigma\cdot (m-m'))-\alpha^{-1}(m-m')\\
    &=\alpha^{-1}(\sigma\cdot\alpha(p)-\alpha(p))\\
    &=\sigma\cdot p-p.
    \end{align*}
    Thus $\phi-\phi'\in B^1(G,P)$. 

    Note that $\phi\in Z^1(G,P)$, because
    \begin{align*}
    \phi(\sigma\tau)&=\alpha^{-1}((\sigma\tau)\cdot m-m)\\
    &=\alpha^{-1}((\sigma\tau)\cdot m-\sigma\cdot m+\sigma\cdot m-m)\\
    &=\alpha^{-1}(\sigma\cdot (\tau\cdot m-m))+\alpha^{-1}(\sigma\cdot m-m)\\
    &=\sigma\cdot\phi(\tau)+\phi(\sigma)
    \end{align*}
    holds for all $\sigma,\tau\in G$. 
    
    We now prove that the sequence \eqref{eq:long}
    is exact at $H^0(G,N)=N^G$. We need to prove that $\ker\delta=\im\beta^0$. To prove $\supseteq$ note that if $m\in M^G$ is such that 
    $\delta(\beta(m))=\phi+B^1(G,P)$, then 
    \[
    \phi(\sigma)=\alpha^{-1}(\sigma\cdot m-m)=0.
    \]
    Conversely, if $n\in\ker\delta$, then there exists $m\in M$ such that $\beta(m)=n$ and
    \[
    \delta(\beta(m))=\phi+B^1(G,P),
    \]
    where $\phi\in B^1(G,P)$ and 
    $\phi(\sigma)=\alpha^{-1}(\sigma\cdot m-m)$
    for all $\sigma\in G$. Since $\phi\in B^1(G,P)$, there
    exists $p\in P$ such that $\phi(\sigma)=\sigma\cdot p-p$ for all $\sigma\in G$. Note that 
    \[
    \beta(m-\alpha(p))=\beta(m)-\beta(\alpha(p))=\beta(m)=n.
    \]
    Moreover, $m-\alpha(p)\in M^G$, as $\sigma\cdot (m-\alpha(p))=m-\alpha(p)$. Hence $n\in\im\beta^0$. 

    We now prove that \eqref{eq:long} is exact at $H^1(G,P)$, that 
    is $\im\delta=\ker\alpha^1$. To prove
    $\subseteq$ note that 
    for $n\in N^G$, $\delta(n)=\phi+B^1(G,P)$, where 
    $\phi(\sigma)=\alpha^{-1}(\sigma\cdot m-m)$ for all $\sigma\in G$
    and some $m\in M$ such that $\beta(m)=n$. In particular, 
    $\alpha\circ\phi\in B^1(G,M)$. 
    Then
    \[
    \alpha^1(\phi+B^1(G,P))=\alpha\circ\phi+B^1(G,M)=B^1(G,M).
    \]
    To prove $\supseteq$, let $\phi+B^1(G,P)\in\ker\alpha^1$. Then
    $\alpha\circ\phi\in B^1(G,M)$, that is $\alpha(\phi(\sigma))=\sigma\cdot m-m$ for 
    all $\sigma\in G$ and some $m\in M$. Note that 
    \[
    \delta(\beta(m))=\psi+B^1(G,P),
    \]
    where $\psi(\sigma)=\alpha^{-1}(\sigma\cdot m-m)$. This implies that 
    $\alpha(\psi(\sigma))=\alpha(\phi(\sigma))$ for all $\sigma\in G$. Since 
    $\alpha$ is injective, $\psi=\phi$. Therefore 
    $\phi+B^1(G,P)$ belongs to the image of $\delta$. 

    Finally, we prove that the sequence \eqref{eq:long} is exact
    at $H^1(G,M)$, that is $\im\alpha^1=\ker\beta^1$. To prove $\subseteq$  
    note that 
    \[
    \beta^1(\alpha^1(\phi+B^1(G,P)))=\beta^1(\alpha\circ\phi+B^1(G,M))
    =(\beta\circ\alpha)\circ\phi+B^1(G,N)=B^1(G,N).
    \]
    Conversely, let $\phi+B^1(G,M)\in\ker\beta_1$. Then $\beta\circ\phi\in B^1(G,N)$. Thus
    there exists $n\in N$ such that $\beta(\phi(\sigma))=\sigma\cdot n-n$ 
    for all $\sigma\in G$. Since $\beta$ is surjective, 
    $n=\beta(m)$ for some $m\in M$. Hence 
    \[
    \beta(\phi(\sigma))=\sigma\cdot n-n=\sigma\cdot \beta(m)-\beta(m)
    =\beta(\sigma\cdot m-m).
    \]
    For each $\sigma\in G$, 
    $\phi(\sigma)-(\sigma\cdot m-m)\in\ker\beta=\im\alpha$. 
    and therefore  $\phi(\sigma)-(\sigma\cdot m-m)=\alpha(\rho_\sigma)$. 
    This defines a map $\rho\colon G\to P$, $\sigma\mapsto\rho_\sigma$.
    We claim that $\rho\in Z^1(G,P)$. If $\sigma,\tau\in G$, then
    \begin{align*}    
    \alpha(\rho_{\sigma\tau})&=\phi(\sigma\tau)-( (\sigma\tau)\cdot m-m)\\
    &=\phi(\sigma)+\sigma\cdot\phi(\tau)-(\sigma\cdot (\tau\cdot m-m)+\sigma\cdot m-m)\\
    &=\alpha(\rho_\sigma)+\sigma\cdot\alpha(\rho_\tau).
    \end{align*}
    Since $\alpha$ is injective, it follows that $\rho\in Z^1(G,P)$.
    Now
    \[
    \alpha_1(\rho+B^1(G,P))=\alpha\circ\rho+B^1(G,M)=\phi+B^1(G,M).\qedhere
    \]
\end{proof}

\begin{theorem}
    Let $G$ be a finite group and $M$ be a $G$-module. 
    Then 
    \[
    |G|H^1(G,M)=\{0\}.
    \]
\end{theorem}

\begin{proof}
    Let $n=|G|$. It is enough to prove that 
    if $\phi\in Z^1(G,M)$, then $n\phi\in B^1(G,M)$. Let $\phi\in Z^1(G,M)$ and 
    $\sigma\in G$. Then 
    \[
    \phi(\sigma\tau)=\phi(\sigma)+\sigma\cdot\phi(\tau)
    \]
    for all $\tau\in G$. Summing over all possible $\tau\in G$, we obtain that 
    \begin{equation}
        \label{eq:|G|H1(G,M)=0}    
        \sum_{\tau\in G}\phi(\tau)=\sum_{\tau\in G}\phi(\sigma\tau)
        =\sum_{\tau\in G}\sigma\cdot\phi(\tau)+\sum_{\tau\in G}\phi(\sigma)=n\phi(\sigma).
    \end{equation}
    Let $m=-\sum_{\tau\in G}\phi(\tau)\in M$. Then 
    \eqref{eq:|G|H1(G,M)=0} can be rewritten as 
    \[
    -m=\sum_{\tau\in G}\phi(\tau)=\sigma\cdot \sum_{\tau\in G}\phi(\tau)+n\phi(\sigma)
    =-\sigma\cdot m+n\phi(\sigma).
    \]
    Thus $n\phi(\sigma)=\sigma\cdot m-m$ and hence $n\phi\in B^1(G,M)$.  
\end{proof}

\begin{exercise}
    Let $G$ be a finite group and 
    $M$ be a finite $G$-module of size coprime to $|G|$. Prove that 
    $H^1(G,M)=\{0\}$. 
\end{exercise}

\begin{exercise}
    Let $G$ be a finite group and 
    $M$ be a finitely generated $G$-module. Prove that
    $H^1(G,M)$ is finite.
\end{exercise}