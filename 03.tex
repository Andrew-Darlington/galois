\chapter{}

%\begin{theorem}[Galois]
%	\index{Galois' theorem}
%	For every prime number $p$ and every $m\geq1$
%	there exists a field of size $p^m$. 
%\end{theorem}
%
%\begin{proof}
%\end{proof}
%
%
Algebraic field extensions form a nice class of extensions. The same happens
with finite field extensions. 

\begin{proposition}
	Let $F/K$ is a subextension of $E/K$. Then $E/K$ is algebraic 
	if and only if $E/F$ and $F/K$ are algebraic. 
\end{proposition}

\begin{proof}
    We know that if $E/K$ is algebraic, then $E/F$ and $F/K$ are both algebraic. 
    Let us assume that $E/F$ and $F/K$ are both algebraic. Let $x\in E$ and 
    let $L$ be the subextension over $K$ generated by the coefficients of $f(x,F)$, 
    the minimal polynomial of $x$ over $F$. Then $L/K$ is finite, since it is generated
    by finitely many algebraic elements. Moreover, $x$ is algebraic over $L$. Since 
    \[
    [L(x):K]=[L(x):L][L:K]<\infty,
    \]
    $L(x)/K$ is algebraic. In particular, $x$ is algebraic over $K$. 
\end{proof}

\begin{exercise}
	Let $F/K$ is a subextension of $E/K$. Prove that $E/K$ is finite 
	if and only if $E/F$ and $F/K$ are finite. 
\end{exercise}

Let $K\subseteq F$ and $L\subseteq E$. The composite 
of $F$ and $L$ is defined as 
\[
FL=K(F\cup L)=F(L)=L(F)
\]
and it is equal to the smallest field that contains $F$ and $L$. 

\begin{exercise}
    If $F=\Q(\sqrt{2})$ and $L=\Q(\sqrt{3})$, then $FL=\Q(\sqrt{2},\sqrt{3})$. 
    Compute $[\Q(\sqrt{2},\sqrt{3}):\Q]$ and 
    $\Q(\sqrt{2})\cap\Q(\sqrt{3})$. 
\end{exercise}

\begin{exercise}
    Let $\xi\in\C$ be a primitive cubic root of one. 
    If $F=\Q(\sqrt[3]{2})$ and $L=\Q(\xi)$, then $FL=\Q(\sqrt[3]{2},\xi)$. 
    Compute $[\Q(\sqrt[3]{2},\xi):\Q]$ and 
    $\Q(\sqrt[3]{2})\cap\Q(\xi)$. 
\end{exercise}

\begin{exercise}
	Let $E/K$ and $F/K$ be extensions, where both $E$ and $F$ are subfields of 
	a field $L$. If $F/K$ is algebraic, then $EF/E$ is algebraic.
\end{exercise}

% \begin{proof}
%     If $F/K$ is algebraic, then $EF/E=E(F)/E$ is algebraic, as it is generated by 
%     algebraic elements over $E$.  
% \end{proof}

\begin{exercise}
	Let $E/K$ and $F/K$ be extensions, where both $E$ and $F$ are subfields of 
	a field $L$. If $F/K$ is finite, then $EF/E$ is finite.
\end{exercise}

The solution to the previous exercise shows, in particular, that $[EF:E]\leq [F:K]$. 



%\begin{theorem}[Galois]
%	\index{Galois' theorem}
%	For every prime number $p$ and every $m\geq1$
%	there exists a field of size $p^m$. 
%\end{theorem}
%
%\begin{proof}
%\end{proof}
%
%
% Algebraic field extensions form a nice class of extensions. The same happens
% with finite field extensions. 

% \begin{proposition}
% 	Let $F/K$ is a subextension of $E/K$. Then $E/K$ is algebraic (resp. finite)
% 	if and only if $E/F$ and $F/K$ are algebraic (resp. finite). 
% \end{proposition}

% \begin{proof}
% 	From the formula 
% 	$[E:K]=[E:F][F:K]$ it follows that 
% 	$E/K$ is finite if and only if $E/F$ and $F/K$ are
% 	both finite. 

% 	If $E/K$ is algebraic, then $E/F$ and $F/K$ are both algebraic. Conversely,
% 	suppose now that both $E/F$ and $F/K$ are algebraic. For $x\in E$ let $L$
% 	be the extension of $K$ generated by the coefficients of $f(x,F)$, the
% 	minimal polynomial of $x$ over $F$. Then $L$ is finite, as it is generated
% 	by finitely many algebraic elements. Moreover, $x$ is algebraic over $L$.
% 	Since $[L(x):K]=[L(x):L][L:K]<\infty$, $L(x)/K$ is algebraic. In
% 	particular, $x$ is algebraic over $K$. 
% \end{proof}

% \begin{proposition}
% 	Let $E/K$ and $F/K$ be extensions, where both $E$ and $F$ are subfields of
% 	a field $L$. If $F/K$ is algebraic (resp. finite), then $EF/E$ is algebraic
% 	(resp. finite).
% \end{proposition}

% \begin{proof}
% 	Now we prove that if $F/K$ is finite, then $EF/E$ is finite. For that purpose,
% 	we show that $[EF:E]<[F:K]$. Recall that $EF=E(F)$. The elements of $F$ are
% 	algebraic over $K$, so they are algebraic over $E$. In particular, $E(F)/E$ is algebraic
% 	and $E(F)=E[F]$. Let $z\in EF$, say $z=\sum_i x_it_i$ for some $x_i\in E$ and $t_i\in F$. 
% 	The extension $F/K$ is finite, so let $\{f_1,\dots,f_m\}$ be a basis of $F$ over $K$. Then
% 	each $t_i$ can be written as $t_i=\sum_ja_{ij}f_j$ for some $a_{ij}\in K$. Then
% 	\[
% 		z=\sum_j\left(\sum_i a_{ij}x_i\right)f_j
% 	\]
% 	and thus $\{f_1,\dots,f_m\}$ generates $EF$ as a vector space over $E$. 
% \end{proof}

% \[\begin{tikzcd}
% 	& EF \\
% 	E && F \\
% 	& K
% 	\arrow[no head, from=1-2, to=2-1]
% 	\arrow[no head, from=2-1, to=3-2]
% 	\arrow[no head, from=3-2, to=2-3]
% 	\arrow[no head, from=1-2, to=2-3]
% \end{tikzcd}\]

\begin{lemma}
	Let $\sigma\colon K\to L$ be a field homomorphism. Then there exists an extension
	$E/K$ and a field isomorphism $\varphi\colon E\to L$
	such that $\varphi|_K=\sigma$. 
\end{lemma}

\begin{proof}
	Let $A$ be a set in bijection with $L\setminus\sigma(K)$ and disjoint with $K$. 
	Let $E=K\cup A$. If $\theta\colon A\to L\setminus\sigma(K)$ is bijective, then 
	let 
	\[
		\varphi\colon E\to L,
		\quad
		\varphi(x)=\begin{cases}
			\sigma(x) & \text{if $x\in K$},\\
			\theta(x) & \text{if $x\in A$}.
		\end{cases}
	\]
	Then $\varphi$ is a bijective map such that $\varphi|_K=\sigma$. 
	Transport the operations of $L$ onto $E$, that is 
	to define binary operations on $E$ as follows: 
	\begin{align*}
		&(x,y)\mapsto x\oplus y=\varphi^{-1}(\varphi(x)+\varphi(y)), && 
		(x,y)\mapsto x\odot y=\varphi^{-1}(\varphi(x)\varphi(y)).
	\end{align*}
	Then, for example, 
	\[
		x\oplus y=\varphi^{-1}(\varphi(x)+\varphi(y))=\varphi^{-1}(\sigma(x)+\sigma(y))
		=\varphi^{-1}(\sigma(x+y))=\varphi^{-1}(\varphi(x+y))=x+y
	\]
	for all $x,y\in K$. 
\end{proof}

If $\sigma\colon A\to B$ is a ring homomorphism, then $\sigma$ induces a ring
homomorphism $\overline{\sigma}\colon A[X]\to B[X]$,
$\sum_ia_iX^i\mapsto\sum\sigma(a_i)X^i$. 

\begin{theorem}
	Let $K$ be a field and $f\in K[X]$ be such that $\deg f>0$. Then 
	there exists an extension $E/K$ such that $f$ admits a root in $E$. 
\end{theorem}

\begin{proof}
	We may assume that $f$ is irreducible over $K$. Let $L=K[X]/(f)$ and 
	$\pi\colon K[X]\to L$ be the canonical map. Then $L$ 
	is a field. The field homomorphism $\sigma\colon K\to L$, $a\mapsto \pi(aX^0)$. 
	Let $g=\overline{\sigma}(f)\in L[X]$. 

	We claim that $\pi(X)$ is a root of $g$ in $L$. Suppose that $f=\sum_i a_iX^i$. 
	Then 
	\begin{align*}
		g(\pi(X))&=\overline{\sigma}(f)(\pi(X))\\
		&=\sum_i \sigma(a_i)\pi(X)^i
		=\sum_i\pi(a_iX^0)\pi(X^i)=\pi(\sum a_iX^i)=\pi(f)=0.
	\end{align*}
	The previous lemma states that 
	there exists an extension $E/K$ and an isomorphism $\varphi\colon E\to L$
	such that $\varphi|_K=\sigma$. If $u=\pi(X)$, then $\varphi^{-1}(u)$ is a root of $f$ in $E$, 
	as 
	\begin{align*}
		\varphi(f(\varphi^{-1}(u)))&=\varphi\left(\sum_ia_i\varphi^{-1}(u)^i\right)
		=\varphi\left(\sum_ia_i\varphi^{-1}(u^i)\right)\\
		&=\sum_i\varphi(a_i)u^i=\sum_i\sigma(a_i)u^i=g(u)=0.\qedhere
	\end{align*}
\end{proof}

As a corollary, if $K$ is a field and $f_1,\dots,f_n\in K[X]$ are polynomials 
of positive degree, then there exists an extension $E/K$  such that 
each $f_i$ admits a root in $E$. This is proved by induction on $n$.  

\begin{definition}
	A field $K$ is \textbf{algebraically closed} if each $f\in K[X]$ 
	of positive degree admits a root in $K$. 
\end{definition}

The \emph{fundamental theorem of algebra} states that $\C$ is algebraically closed. A
typical proof uses complex analysis.  Later we will give a proof of this result
using Galois theory. 

\begin{proposition}
	The following statements are equivalent:
	\begin{enumerate}
		\item $K$ is algebraically closed.
		\item If $f\in K[X]$ is irreducible, then $\deg f=1$.
		\item If $f\in K[X]$ is non-zero, then $f$ decomposes linearly in $K[X]$, that is
			\[
				f=a\prod_{i=1}^n(X-\alpha_i)^{m_i}
			\]
			for some $a\in K$ and $\alpha_1,\dots,\alpha_n\in K$. 
		\item If $E/K$ is algebraic, then $E=K$. 
	\end{enumerate}
\end{proposition}

\begin{proof}
	$1)\implies 2\implies 3)$ are exercises.  
	
	Let us prove that $3)\implies
	4)$. Let $x\in E$. Decompose $f(x,K)$ linearly in $K[X]$ as
	$f(x,K)=a\prod_{i=1}^n(X-\alpha_i)$ and evaluate on $x$ to obtain that
	$x=\alpha_j$ for some $j$. 
	
	To prove that $4)\implies 1)$ let $f\in K[X]$ be
	such that $\deg f>0$. There exists an extension $E/K$ such that $f$ has a
	root $x$ in $E$. The extension $K(x)/K$ is algebraic and hence $K(x)=K$, so
	$x\in K$. 
\end{proof}



\topic{Artin's theorem}

\begin{definition}
	The \textbf{algebraic closure} of a field $K$ is an algebraic extension $C/K$ 
	such that $C$ is algebraically closed. 
\end{definition}

For example, $\C/\R$ is an algebraic closure but $\C/\Q$ it is not. 

\begin{proposition}
\label{pro:Artin}
	Let $C$ be algebraically closed and $\sigma\colon K\to C$ be a field homomorphism. If $E/K$ 
	is algebraic, then there exists a field homomorphism 
	$\varphi\colon E\to C$ such that 
	$\varphi|_K=\sigma$. 
\end{proposition}

\begin{proof}
	Suppose first that $E=K(x)$ and let $f=f(x,K)$. Let $\overline{\sigma}(f)\in C[X]$ 
	and let $y\in C$ be a root of $\overline{\sigma}(f)$. If $z\in E$, then $z=g(x)$ for
	some $g\in K[X]$. Let $\varphi\colon E\to C$, $z\mapsto \overline{\sigma}(g)(y)$. 

	The map $\varphi$ is well-defined. If $z=h(x)$ for some $h\in K[X]$, then
	\[
	0=g(x)-h(x)=(g-h)(x)
	\]
	and thus $f$ divides $g-h$. In particular, $\overline{\sigma}(f)$ divides
    $\overline{\sigma}(g-h)=\overline{\sigma}(g)-\overline{\sigma}(h)$ and hence
    $(\overline{\sigma}(g)-\overline{\sigma}(h))(y)=0$. 

	It is an exercise to show that the map $\varphi$ is a ring homomorphism.
	
	Let $a\in K$. Since $a=(aX^0)(x)$, it follows that $\varphi|_K=\sigma$, as 
	\[
	\varphi(a)=\overline{\sigma}(aX^0)(y)=(\sigma(a)X^0)(y)=\sigma(a)
	\]
	and 
	$\varphi(x)=\overline{\sigma}(X)(y)=y$. 
	
	Let us now prove the proposition in full generality. Let 
	$X$ be the set of pairs $(F,\tau)$, where $F$ is a subfield of $E$ that contains $K$ and
	$\tau\colon F\to C$ is a field homomorphism such that $\tau|_K=\sigma$. Note that
	$(K,\sigma)\in X$, so $X$ is non-empty. Moreover, $X$ is partially ordered by
	\[
	(F,\tau)\leq (F_1,\tau_1)\Longleftrightarrow F\subseteq F_1\text{ and }\tau_1|_F=\tau.
	\]
	If $\{(F_i,\tau_i):i\in I\}$ is a chain in $X$, then $F=\cup_{i\in I}F_i$ is a subfield of $E$
	that contains $K$. Moreover, if $z\in F$, then $z\in F_i$ for some $i\in I$ and 
	then one defines $\tau(z)=\tau_i(z)$. It is an exercise to prove that $\tau$ is well-defined.
	Since $(F,\tau)\in X$ is an upper bound, Zorn's lemma implies that there exists
	a maximal element 
	$(E_1,\theta)\in X$. We claim that $E=E_1$. If not, let $z\in E\setminus E_1$. 
	Since we know the proposition is true for the extension $E_1(z)/K$, 
	let  
	$\rho\colon E_1(z)\to C$ be a field homomorphism such that $\rho|_{E_1}=\sigma$. Then, in particular, 
	$\rho|_K=\sigma$. This implies that $(E_1(z),\rho)\in X$ and hence
	$(E_1,\theta)<(E_1(z),\rho)$, a contradiction to the maximality of $(E_1,\theta)$. 
\end{proof}

The previous proposition will be used to prove 
that the algebraic closure always exists. 

\begin{theorem}[Artin]
	\index{Artin's theorem}
	Let $K$ be a field. Then $K$ admits an algebraic closure $C/K$. If $C_1/K$
	is an algebraic closure, then the extensions $C/K$ and $C_1/K$ are
	isomorphic. 
\end{theorem}

\begin{proof}
    Let us first prove the uniqueness. The previous proposition implies the existence of 
    an extensions homomorphism $\varphi\colon C\to C_1$. Let $y\in C_1$ and $f=f(y,K)$ be 
    the minimal polynomial of $y$ in $K$. Since $f$ admits a factorization
    \[
        f=\lambda\prod (X-\alpha_i)^{m_i}
    \]
    in $C[X]$, it follows that
    \[
    f=\overline{\varphi}(f)=\prod (X-\varphi(\alpha_i))^{m_i}
    \]
    Since $0=f(y)$, we conclude that $y=\varphi(\alpha_j)$ for some $j$. In particular, $\varphi$ is
    surjective and hence $\varphi$ is bijective. 
    
    We now prove the existence. Let us assume that $K$ admits an extension $E/K$ 
    with $E$ algebraically closed. Let 
    \[
    	F=\overline{K}_E=\{x\in E:x\text{ is algebraic over }K\}. 
    \]
    Then $F/K$ is algebraic. Let $g\in F[X]$ be such that
    $\deg g>0$. Since $E$ is algebraically closed, $g$ admits a root $\alpha$ in $E$. In particular, $\alpha$
    is algebraic over $F$ and hence $\alpha$ is algebraic over $K$. This implies that $\alpha\in F$, thus
    $F$ is algebraically closed. This proves that $F/K$ is an algebraic closure. 
    
    Let us prove that there exists an extension $E_1/K$ such that
    every polynomial $f\in K[X]$ with $\deg f>0$ has a root in $E_1$. Let 
    $\{f_i:i\in I\}$ be the family of monic irreducible polynomials with coefficients in $K$. 
    We may think that $f_i=f_i(X_i)$. 
    Let $R=K[\{X_i:i\in I\}]$ and let $J$ be the ideal of $R$ 
    generated by the $f_i(X_i)$. We claim that $J\ne R$. If not, $1\in J$, so
    \[
    1=\sum_{i=1}^m g_jf_{i_j}(X_j)
    \]
    for some $g_1,\dots,g_m\in R$. There exists an extension $F/K$ such that
    $f_{i_j}$ has a root $\alpha_j$ in $F$ for all $j$. Let 
    \[
    \sigma\colon R\to F,\quad
    \sigma(X_k)=\begin{cases}
        \alpha_j & \text{if $k=i_j$},\\
        0 & \text{if $k\not\in\{i_1,\dots,i_m\}$}.
        \end{cases}
    \]
    Then $1=\sigma(1)=\sum_{j=1}^m\sigma(g_j)f_{i_j}(\alpha_j)$, a contradiction. 
    
    Since $J$ is a proper ideal, it is contained in a maximal ideal $M$. Let $L=R/M$ 
    and let $\sigma\colon K\to L$ be given by...
    Then $\pi(X_i)$ is a root of $\overline{\sigma}(f_i)$ for all $i$ 
    and there exists an extension $E_1/K$ such that
    every $f_i$ has a root in $E_1$. Proceeding in this way, we construct
    a sequence
    \[
    E_1\subseteq E_2\subseteq\cdots
    \]
    of fields such that every polynomial of positive degree and coefficients in $E_k$ 
    admits a root in $E_{k+1}$. Let $E=\cup E_k$. We claim that $E$ is algebraically closed. In fact, 
    let $g\in E[X]$ be such that $\deg g>0$. Then, since $g\in E_r[X]$ for some $r$, it follows
    that $g$ has a root in $E_{r+1}\subseteq E$. 
\end{proof}

\topic{Decomposition fields}

\begin{definition}
\index{Decomposition field}
	Let $K$ be a field and $f\in K[X]$ be such that $\deg f>0$. A \textbf{decomposition field}
	of $f$ over $K$ is field $E$ that contains $K$ and that satisfies the following properties:
	\begin{enumerate}
		\item $f$ factorizes linearly in $E[X]$. 
		\item if $F$ is a field such that $K\subseteq F\subseteq E$ and 
			$f$ factorizes
			linearly in $F[X]$, then $F=E$. 
	\end{enumerate}
\end{definition}

Easy examples: 

\begin{example}
	$\C$ is a decomposition field of $X^2+1\in\R[X]$. 
\end{example}

\begin{example}
	$\Q[\sqrt{2}]$ is a decomposition field of $X^2-2\in\Q[X]$. 
\end{example}

\begin{example}
	$\Q(\sqrt[3]{2})$ is not a decomposition field of $X^3-2\in\Q[X]$. However, if
	$\omega$ ia a primitive cubic root of one, then 
	$\Q(\sqrt[3]{2},\omega)$ is is a decomposition field of $X^3-2\in\Q[X]$. 
\end{example}

\begin{proposition}
	$E$ is a decomposition field of $f\in K[X]$ if and only if
	$f$ factorizes linearly in $E[X]$ and $E=K(x_1,\dots,x_n)$ where 
	$x_1,\dots,x_n$ are roots of $f$. 
\end{proposition}

\begin{proof}
    Let $f=a\prod_{i=1}^r(X-x_i)^{n_i}$ and $F=K(x_1,\dots,x_r)$ with $x_1,\dots,x_r\in E$. Since $f$
    factorizes linearly in $F[X]$, it follows that $F=E$. 
    Conversely, let $E=K(x_1,\dots,x_r)$ and assume that $f$ factorizes linearly
    in $F[X]$. Then, in particular, $x_1,\dots,x_r\in F$. Hence $E\subseteq F$ and
    $F=E$. 
\end{proof}

One immediately obtains the following consequence:
If $E$ is a decomposition field of $f\in K[X]$, then $E/K$ is finite. 

\begin{theorem}
    Let $f\in K[X]$. There exists a (unique up to extension isomorphism) 
    decomposition field of $f$ over $K$. 
\end{theorem}

\begin{proof}
    Let $C/K$ be an algebraic closure. Write $f=a\prod_{i=1}^r(X-x_i)^{n_i}$ in $C[X]$. 
    Then $E=K(x_1,\dots,x_r)$ is a decomposition field of $f$ over $K$. Let us prove
    uniqueness: if $E_1/K$ is a decomposition field of $f$ over $K$, 
    then $E_1/K$ is algebraic and thus Proposition
    \ref{pro:Artin} implies that 
    there exists $\varphi\in\Hom(E_1/K,C/K)$ such that $\varphi|_K$ is the identity.
    Factorize $f$ linearly in $E_1[X]$ and apply $\overline{\varphi}$:
    \[
    f=a\prod_{j=1}^s(X-y_j)^{m_j}
    \implies
    \overline{\varphi}(f)=\varphi(a)\prod_{j=1}^s(X-\varphi(y_j))^{m_j}
    \]
    so $f$ factorizes linearly in $\varphi(E_1)$. Moreover, 
    $E_1=K(y_1,\dots,y_s)$ and it follows that 
    $\varphi(E_1)=K(\varphi(y_1),\dots,\varphi(y_s))$. Thus
    $\varphi(E_1)$ is a decomposition field of $f$. Since  
    $\varphi(E_1)\subseteq C$, it follows that $\varphi(E_1)=E$. 
\end{proof}

\begin{exercise}
If $E/K$ is finite and $\varphi\in\Hom(E/K,E/K)$, 
then $\varphi$ is an isomorphism. 
\end{exercise}

Let $C$ be an algebraic closure of $K$ and 
$G=\Gal(C/K)$. The group $G$ acts on $C$
\[
\sigma\cdot x=\sigma(x),\quad
\sigma\in G,\,x\in C.
\]
The orbits 
are of the form 
\[
O_G(x)=\{\sigma(x):\sigma\in G\}
=\{y\in C:y=\sigma(x)\text{ for some $\sigma\in G$}\}
\]
The elements $x,y\in C$ are \textbf{conjugate} 
if $y=\sigma(x)$ for some $\sigma\in G$. 

\begin{proposition}
    Let $C$ be an algebraic closure of $K$ and $x,y\in C$. Then 
    $x$ and $y$ are conjugate if and only if $f(x,K)=f(y,K)$. In particular, 
    $O_G(x)$ is finite. 
\end{proposition}

\begin{proof}
    Let $G=\Gal(C/K)$. 
    If $x$ and $y$ are conjugate, say $y=\sigma(x)$ for some $\sigma\in G$, 
    let us write $g=f(x,K)$ as 
    \[
    g=X^n+\sum_{i=0}^{n-1} a_iX^i. 
    \]
    Then $0=g(x)=x^n+\sum_{i=0}^{n-1}a_ix^i$ and hence $y$ is
    a root of $g$, as 
    \begin{align*}
    0=\sigma\left(x^n+\sum_{i=0}^{n-1}a_ix^i\right)
    &=\sigma(x)^n+\sum_{i=0}^{n-1}\sigma(a_i)\sigma(x)^i\\
    &=\sigma(x)^n+\sum_{i=0}^{n-1}a_i\sigma(x)^i
    =y^n+\sum_{i=0}^{n-1}a_iy^i.
    \end{align*}
    Thus $f(y,K)=g$. 
    
    Conversely, assume that $f(x,K)=f(y,K)$. Let
    $g=f(x,K)=f(y,K)$ and let 
    \[
    \varphi\colon K[x]\to K[y],
    \quad
    h(x)\mapsto h(y).
    \]
    Let us show that the map $\varphi$ is well-defined: we need to show 
    that if 
    $h_1(x)=h_2(x)$, then $h_1(y)=\varphi(h_1(x))=\varphi(h_2(x))=h_2(y)$. 
    If $h_1(x)=h_2(x)$, then 
    \[
    (h_1-h_2)(x)=h_1(x)-h_2(x)=0.
    \]
    Thus implies
    that $g$ divides $h_1-h_2$. In particular, $h_1(y)=h_2(y)$.
    
    A straightforward calculation shows that $\varphi$ is a field 
    homomorphism such that $\varphi|_K=\id$, so $\varphi$ is 
    an extension homomorphism such that $\varphi(x)=y$. There exists
    $\sigma\in\Hom(C/K,C/K)$ such that 
    $\sigma|_{K[x]}=\varphi$. Since $\sigma$ is a bijective, 
    $\sigma(x)=\varphi(x)=y$ and hence 
    $O_G(x)=O_G(y)$. 
\end{proof}

\begin{proposition}
    Let $C$ be an algebraic closure of $K$ and $x$. Then 
    \[
    f(x,K)=\prod_{y\in O_G(x)}(X-y)^m
    \]
    for some $m$. 
\end{proposition}

\begin{proof}
    For each $y\in O_G(x)$ let $m_y$ be the multiplicity
    of $y$ in $f(x,K)$. 
    Then, for example,
    $f(x,K)=(X-x)^{m_x}g$ for some $g$. If $y\in O_G(x)$, 
    then $y=\sigma(x)$ for some $\sigma\in\Gal(C/K)$. Since
    \[
    \overline{\sigma}(f(x,K))=f(x,K)=(X-y)^{m_x}\overline{\sigma}(g), 
    \]
    it follows that $m_y\geq m_x$. By symmetry, 
    we conclude that $m_x=m_y$. 
\end{proof}

The previous proposition shows, in particular, 
that all the roots of 
an irreducible polynomial $f\in K[X]$ 
in an algebraic closure $C$ of $K$
have the same multiplicity. This is clearly 
not true if $f$ is not irreducible. Find an example.

%Take 
%for example $f=(X-1)^2(X^2+1)\in\Q[X]$. 

% pag 20

\begin{definition}
\index{Decomposition field}
    Let $K$ be a field and $\{f_i:i\in I\}$ be a non-empty 
    family of polynomials of positive degree
    with coefficients in $K$. A \textbf{decomposition field} 
    of the $\{f_i:i\in I\}$ is an extension $E/K$
    such that every $f_i$ factorizes linearly in $E[X]$ and 
    if $F/K$ is a subextension of $E/K$ such that every $f_i$ 
    factorizes linearly in $F[X]$, then $F=E$. 
\end{definition}

\begin{exercise}
    Prove that $E/K$ is a decomposition field of
    $\{f_i:i\in I\}$ if and only if every $f_i$ factorizes linearly 
    in $E[X]$ and $E=K(S)$ where $S=\{\text{roots of $f_i$ for all $i$}\}$. 
\end{exercise}

\begin{exercise}
    Prove that if $E/K$ is a decomposition field
    of $\{f_i:i\in I\}$, then $E/K$ is algebraic. If, moreover, 
    $I$ is finite, then $E/K$ is a decomposition field
    of $\prod_{i\in I}f_i$. 
\end{exercise}

\begin{exercise}
    Prove that 
    there exists a decomposition field of $\{f_i:i\in I\}$ 
    and it is unique up to extension isomorphism. 
\end{exercise}

\topic{Normal extensions}

\begin{proposition}
    Let $E/K$ be an algebraic extension and $\sigma\in\Hom(E/K,E/K)$. 
    Then $\sigma$ is bijective. 
\end{proposition}

\begin{proof}
    Let $x\in E$ and 
    $C$ be an algebraic closure of $K$ that contains $E$. There
    exists $\varphi\colon C\to C$ such that $\varphi|_E=\sigma$. 
\end{proof}