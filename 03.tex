\section{26/02/2024}




%\begin{theorem}[Galois]
%	\index{Galois' theorem}
%	For every prime number $p$ and every $m\geq1$
%	there exists a field of size $p^m$. 
%\end{theorem}
%
%\begin{proof}
%\end{proof}
%
%
% Algebraic field extensions form a nice class of extensions. The same happens
% with finite field extensions. 

% \begin{proposition}
% 	Let $F/K$ is a subextension of $E/K$. Then $E/K$ is algebraic (resp. finite)
% 	if and only if $E/F$ and $F/K$ are algebraic (resp. finite). 
% \end{proposition}

% \begin{proof}
% 	From the formula 
% 	$[E:K]=[E:F][F:K]$ it follows that 
% 	$E/K$ is finite if and only if $E/F$ and $F/K$ are
% 	both finite. 

% 	If $E/K$ is algebraic, then $E/F$ and $F/K$ are both algebraic. Conversely,
% 	suppose now that both $E/F$ and $F/K$ are algebraic. For $x\in E$ let $L$
% 	be the extension of $K$ generated by the coefficients of $f(x,F)$, the
% 	minimal polynomial of $x$ over $F$. Then $L$ is finite, as it is generated
% 	by finitely many algebraic elements. Moreover, $x$ is algebraic over $L$.
% 	Since $[L(x):K]=[L(x):L][L:K]<\infty$, $L(x)/K$ is algebraic. In
% 	particular, $x$ is algebraic over $K$. 
% \end{proof}

% \begin{proposition}
% 	Let $E/K$ and $F/K$ be extensions, where both $E$ and $F$ are subfields of
% 	a field $L$. If $F/K$ is algebraic (resp. finite), then $EF/E$ is algebraic
% 	(resp. finite).
% \end{proposition}

% \begin{proof}
% 	Now we prove that if $F/K$ is finite, then $EF/E$ is finite. For that purpose,
% 	we show that $[EF:E]<[F:K]$. Recall that $EF=E(F)$. The elements of $F$ are
% 	algebraic over $K$, so they are algebraic over $E$. In particular, $E(F)/E$ is algebraic
% 	and $E(F)=E[F]$. Let $z\in EF$, say $z=\sum_i x_it_i$ for some $x_i\in E$ and $t_i\in F$. 
% 	The extension $F/K$ is finite, so let $\{f_1,\dots,f_m\}$ be a basis of $F$ over $K$. Then
% 	each $t_i$ can be written as $t_i=\sum_ja_{ij}f_j$ for some $a_{ij}\in K$. Then
% 	\[
% 		z=\sum_j\left(\sum_i a_{ij}x_i\right)f_j
% 	\]
% 	and thus $\{f_1,\dots,f_m\}$ generates $EF$ as a vector space over $E$. 
% \end{proof}

% \[\begin{tikzcd}
% 	& EF \\
% 	E && F \\
% 	& K
% 	\arrow[no head, from=1-2, to=2-1]
% 	\arrow[no head, from=2-1, to=3-2]
% 	\arrow[no head, from=3-2, to=2-3]
% 	\arrow[no head, from=1-2, to=2-3]
% \end{tikzcd}\]

\begin{lemma}
\label{lem:exists_bijective}
	Let $\sigma\colon K\to L$ be a field homomorphism. Then there exists an extension
	$E/K$ and a field isomorphism $\varphi\colon E\to L$
	such that $\varphi|_K=\sigma$. 
\end{lemma}

\begin{proof}
    Note that $\sigma\colon K\to\sigma(K)$ is bijective. 
    Let $A$ be a set in bijection with $L\setminus\sigma(K)$ and disjoint with $K$. 
	Let $E=K\cup A$. If $\theta\colon A\to L\setminus\sigma(K)$ is bijective, then 
	let 
	\[
		\varphi\colon E\to L,
		\quad
		\varphi(x)=\begin{cases}
			\sigma(x) & \text{if $x\in K$},\\
			\theta(x) & \text{if $x\in A$}.
		\end{cases}
	\]
	Then $\varphi$ is a bijective map such that $\varphi|_K=\sigma$. 
	Transport the operations of $L$ onto $E$, that is 
	to define binary operations on $E$ as follows: 
	\begin{align*}
		&(x,y)\mapsto x\oplus y=\varphi^{-1}(\varphi(x)+\varphi(y)), && 
		(x,y)\mapsto x\odot y=\varphi^{-1}(\varphi(x)\varphi(y)).
	\end{align*}
	Then, for example, 
	\[
		x\oplus y=\varphi^{-1}(\varphi(x)+\varphi(y))=\varphi^{-1}(\sigma(x)+\sigma(y))
		=\varphi^{-1}(\sigma(x+y))=\varphi^{-1}(\varphi(x+y))=x+y
	\]
	for all $x,y\in K$. 
\end{proof}

If $\sigma\colon A\to B$ is a ring homomorphism, then $\sigma$ induces a ring
homomorphism $\overline{\sigma}\colon A[X]\to B[X]$,
$\sum_ia_iX^i\mapsto\sum_i\sigma(a_i)X^i$. 

\begin{theorem}
	Let $K$ be a field and $f\in K[X]$ be such that $\deg f>0$. Then 
	there exists an extension $E/K$ such that $f$ admits a root in $E$. 
\end{theorem}

\begin{proof}
	We may assume that $f$ is irreducible over $K$. Let $L=K[X]/(f)$ and 
	$\pi\colon K[X]\to L$ be the canonical map. Then $L$ 
	is a field (the reader should explain why). 
	Let $\sigma\colon K\to L$, $a\mapsto \pi(aX^0)$, and 
	$g=\overline{\sigma}(f)\in L[X]$. 

	We claim that $\pi(X)$ is a root of $g$ in $L$. Suppose that $f=\sum_i a_iX^i$. 
	Then 
	\begin{align*}
		g(\pi(X))&=\overline{\sigma}(f)(\pi(X))\\
		&=\sum_i \sigma(a_i)\pi(X)^i
		=\sum_i\pi(a_iX^0)\pi(X^i)=\pi(\sum a_iX^i)=\pi(f)=0.
	\end{align*}
	The previous lemma states that 
	there exists an extension $E/K$ and an isomorphism $\varphi\colon E\to L$
	such that $\varphi|_K=\sigma$. Note that
	$\varphi(x)=0$ if and only if $x=0$. If $u=\pi(X)$, then $\varphi^{-1}(u)$ is a root of $f$ in $E$, 
	as 
	\begin{align*}
		\varphi(f(\varphi^{-1}(u)))&=\varphi\left(\sum_ia_i\varphi^{-1}(u)^i\right)
		=\varphi\left(\sum_ia_i\varphi^{-1}(u^i)\right)\\
		&=\sum_i\varphi(a_i)u^i=\sum_i\sigma(a_i)u^i=g(u)=0.\qedhere
	\end{align*}
\end{proof}

As a corollary, if $K$ is a field and $f_1,\dots,f_n\in K[X]$ are polynomials 
of positive degree, then there exists an extension $E/K$  such that 
each $f_i$ admits a root in $E$. This is proved by induction on $n$.  

\begin{definition}
	A field $K$ is \textbf{algebraically closed} if each $f\in K[X]$ 
	of positive degree admits a root in $K$. 
\end{definition}

The \emph{fundamental theorem of algebra} states that $\C$ is algebraically closed. A
typical proof uses complex analysis.  Later we will give a proof of this result
using Galois theory. 

\begin{proposition}
	The following statements are equivalent:
	\begin{enumerate}
		\item $K$ is algebraically closed.
		\item If $f\in K[X]$ is irreducible, then $\deg f=1$.
		\item If $f\in K[X]$ is non-zero, then $f$ decomposes linearly in $K[X]$, that is
			\[
				f=a\prod_{i=1}^n(X-\alpha_i)^{m_i}
			\]
			for some $a\in K$ and $\alpha_1,\dots,\alpha_n\in K$. 
		\item If $E/K$ is algebraic, then $E=K$. 
	\end{enumerate}
\end{proposition}

\begin{proof}
	$1)\implies 2\implies 3)$ are exercises.  
	
	Let us prove that $3)\implies
	4)$. Let $x\in E$. Decompose $f(x,K)$ linearly in $K[X]$ as
	\[
        f(x,K)=a\prod_{i=1}^n(X-\alpha_i)^{m_i}
        \]
        and evaluate on $x$ to obtain that
	$x=\alpha_j$ for some $j$. 
	
	To prove that $4)\implies 1)$ let $f\in K[X]$ be
	such that $\deg f>0$. There exists an extension $E/K$ such that $f$ has a
	root $x$ in $E$. The extension $K(x)/K$ is algebraic and hence $K(x)=K$, so
	$x\in K$. 
\end{proof}



\subsection{Artin's theorem}

\begin{definition}
	The \textbf{algebraic closure} of a field $K$ is an algebraic extension $C/K$ 
	such that $C$ is algebraically closed. 
\end{definition}

For example, $\C/\R$ is an algebraic closure but $\C/\Q$ is not. 

\begin{proposition}
\label{pro:Artin}
	Let $C$ be algebraically closed and $\sigma\colon K\to C$ be a field homomorphism. If $E/K$ 
	is algebraic, then there exists a field homomorphism 
	$\varphi\colon E\to C$ such that 
	$\varphi|_K=\sigma$. 
\end{proposition}

\begin{proof}
	Suppose first that $E=K(x)$ and let $f=f(x,K)$. Let $\overline{\sigma}(f)\in C[X]$ 
	and let $y\in C$ be a root of $\overline{\sigma}(f)$. If $z\in E$, then $z=g(x)$ for
	some $g\in K[X]$. Let $\varphi\colon E\to C$, $z\mapsto \overline{\sigma}(g)(y)$. 

	The map $\varphi$ is well-defined. If $z=h(x)$ for some $h\in K[X]$, then
	\[
	0=g(x)-h(x)=(g-h)(x)
	\]
	and thus $f$ divides $g-h$. In particular, $\overline{\sigma}(f)$ divides
    $\overline{\sigma}(g-h)=\overline{\sigma}(g)-\overline{\sigma}(h)$ and hence
    \[
    (\overline{\sigma}(g)-\overline{\sigma}(h))(y)=0.
    \]

	It is an exercise to show that the map $\varphi$ is a ring homomorphism.
	
	Let $a\in K$. It follows that $\varphi|_K=\sigma$, as 
	\[
	\varphi(a)=\overline{\sigma}(aX^0)(y)=\sigma(a)
	\]
	%and 
	%$\varphi(x)=\overline{\sigma}(X)(y)=y$. 
	
	Let us now prove the proposition in full generality. Let 
	$X$ be the set of pairs $(F,\tau)$, where $F$ is a subfield of $E$ that contains $K$ and
	$\tau\colon F\to C$ is a field homomorphism such that $\tau|_K=\sigma$. Note that
	$(K,\sigma)\in X$, so $X$ is non-empty. Moreover, $X$ is partially ordered by
	\[
	(F,\tau)\leq (F_1,\tau_1)\Longleftrightarrow F\subseteq F_1\text{ and }\tau_1|_F=\tau.
	\]
	If $\{(F_i,\tau_i):i\in I\}$ is a chain in $X$, then $F=\cup_{i\in I}F_i$ is a subfield of $E$
	that contains $K$. Moreover, if $z\in F$, then $z\in F_i$ for some $i\in I$ and 
	then one defines $\tau(z)=\tau_i(z)$. It is an exercise to prove that $\tau$ is well-defined.
	Since $(F,\tau)\in X$ is an upper bound, Zorn's lemma implies that there exists
	a maximal element 
	$(E_1,\theta)\in X$. We claim that $E=E_1$. If not, let $z\in E\setminus E_1$. 
	Since we know the proposition is true for the extension $E_1(z)/E_1$, 
	let  
	$\rho\colon E_1(z)\to C$ be a field homomorphism such that $\rho|_{E_1}=\theta$. Then, in particular, 
	$\rho|_K=\sigma$. This implies that $(E_1(z),\rho)\in X$ and hence
	$(E_1,\theta)<(E_1(z),\rho)$, a contradiction to the maximality of $(E_1,\theta)$. 
\end{proof}

