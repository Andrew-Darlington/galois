\chapter{}

\topic{Fields}

Recall that a \textbf{field} is a commutative
ring such that $1\ne 0$ and 
that every non-zero element is invertible. Examples
of (infinite) fields are $\Q$, $\R$ and $\C$. If $p$
is a prime number, then $\Z/p$ is a field. 

\begin{example}
	The abelian group $\Z/2\times\Z/2$ is a field
	with multiplication
	\[
		(a,b)(c,d)=(ac+bd,ad+bc+bd).
	\]
\end{example}

\begin{example}
	$\Q(i)=\{a+bi:a,b\in\Q\}$ and 
	$\Q(\sqrt{2})$ are fields.
\end{example}

\begin{exercise}
	\label{xca:Q(i)}
	Prove that $\Q(i)$ and $\Q(\sqrt{2})$ are not isomorphic as fields.
\end{exercise}

If $R$ is a ring, there exists a unique ring homomorphism
$\Z\to R$, $m\mapsto m1$. The image $\{m1:m\in\Z\}$ 
of this homomorphism is a subring 
of $R$ and it is known as the \textbf{ring of integers} of $R$. The
kernel is a subgroup of $\Z$ and hence  it is generated by
some $t\in\Z$. The integer $t$ is 
the \textbf{characteristic} of the ring $R$. 

\begin{exercise}
	The characteristic of a field is either zero or
	a prime number. 
\end{exercise}

Recall that a commutative ring $R$ is an \textbf{integral 
domain} if $xy=0\implies x=0$ or $y=0$. Fields
are integral domains. 

\begin{exercise}
	Let $K$ be a field. Prove that
	the following statements are equivalent:
	\begin{enumerate}
		\item $K$ is of characteristic zero.
		\item The additive order of $1$ is infinite. 
		\item The additive order of each $x\ne0$ is infinite.
		\item The ring of integers of $K$ is isomorphic to $\Z$.
	\end{enumerate}
\end{exercise}

\begin{exercise}
	Let $K$ be a field. Prove that
	the following statements are equivalent:
	\begin{enumerate}
		\item $K$ is of characteristic $p$.
		\item The additive order of $1$ is $p$. 
		\item The additive order of each $x\ne0$ is $p$.
		\item The ring of integers of $K$ is isomorphic to $\Z/p$.
	\end{enumerate}
\end{exercise}

The following exercise is important. 

\begin{exercise}
	Prove that if $K$ is a finite field, then
	$|K|=p^m$ for some prime number $p$ and some $m\geq1$. 
\end{exercise}

\begin{definition}
	\index{Subfield}
	A \textbf{subfield} of a ring $R$ is a subring of $R$ 
	that is also a field.
\end{definition}

Note that if $K$ is a subfield of $E$, then
the characteristic of $K$ coincides
with the chacteristic 
of $E$. Moreover, if $K\to L$ is a field homomorphis, then
$K$ and $L$ have the same characteristic. 

\begin{exercise}
	Let $K$ be a field of characteristic $p$. Prove
	that $K\to K$, $x\mapsto x^{p^n}$, is a field homomorphism
	for all $n\in\Z_{\geq 0}$. 
\end{exercise}

Note that finite fields are of characteristic $p$. 

Let $K$ be a subfield of a field $E$. Then $E$ 
is a $K$-vector space with the usual scalar multiplication
$K\times E\to E$, 
$(\lambda, x)\mapsto \lambda x$.

\begin{definition}
	A field $K$ is \textbf{prime} if there are no
	proper subfields of $K$. 
\end{definition}

Examples of prime fields are $\Q$ and $\Z/p$ for $p$ a prime number.

\begin{proposition}
	Let $K$ be a field. The following statements hold:
	\begin{enumerate}
		\item $K$ contains a unique prime field, it is known as the 
			\textbf{prime subfield} of $K$.
		\item The prime subfield of $K$ is either isomorphic to $\Q$ if 
			the characteristic of $K$ is zero, or it is isomorphic to $\Z/p$ for
			some prime number $p$ if the characteristic of $K$ is $p$. 
	\end{enumerate}
\end{proposition}

\begin{proof}
	To prove the first claim let $L$ be the intersection
	of all the subfields of $K$. Then $L$ is a subfield of $K$. 
	If $F$ is a subfield of $L$, then $F$ is a subfield
	of $K$. Thus $L\subseteq F$ and hence $F=L$, which proves
	that $L$ is prime. If $L_1$ is a subfield of $K$
	and $L_1$ is prime, then $L\subseteq L_1$ and 
	hence $L=L_1$. 

	Let $K_0$ be the prime field of $K$.  Suppose that $K$ is of characteristic
	$p>0$. Then $K_\Z$ is a field isomorphic to $\Z/p$ and hence $K_0\simeq
	K_\Z$. Suppose now that the characteristic of $K$ is zero. Then $K_\Z$. Let
	$L=\{m1/n1:m,n\in\Z,n\ne 0\}$. We claim that $K_0=L$. Since $K_\Z\subseteq
	K_0$, it follows that $L\subseteq K_0$. Hence $L=K_0$, as $L$ is a subfield
	of $K$.  
\end{proof}

\begin{definition}
	Let $E$ be a field and $K$ be a subfield of $E$. Then 
	$E$ is an \textbf{extension} of $K$. We will use
	the notation $E/K$. 
\end{definition}

If $E$ is an extension of $K$, then $E$ is a
$K$-vector space. 

\begin{definition}
	The degree of an extension $E$ of $K$ 
	is the integer $\dim_KE$. It will be denoted by $[E:K]$. 
\end{definition}

We say that $E$ is a finite extension of $K$ 
if $[E:K]$ is finite. 

\begin{example}
	Let $K$ be a field. Then $[K:K]=1$. Conversely, 
	if $E$ is an extension of $K$ and $[E:K]=1$, then $K=E$. 
	If not, let $x\in E\setminus K$. We claim that
	$\{1,x\}$ is linearly independent over $K$. Indeed, 
	if $a1+bx=0$ for some $a,b\in K$, then $bx=-a$. If 
	$b\ne 0$, then $x=-a/b\in K$, a contradiction. If $b=0$, then 
	$a=0$. 
\end{example}

We know that $[\C:\R]=2$. 

\begin{example}
	A basis of $\Q(\sqrt{2})$ over $\Q$ 
	is given by $\{1,\sqrt{2}\}$. Then 
	$[\Q(\sqrt{2}):\Q]=2$. 
\end{example}

\begin{example}
	Since $\Q$ is numerable and 
	$\R$ is not, $[\R:\Q]>\aleph_0$. If $\{x_i:i\in\Z_{>0}\}$ 
	is a numerable basis of $\R$ over $\Q$, for each
	$n$ consider the $\Q$-vector space
	$V_n$ generated by $\{x_1,\dots,x_n\}$. Then 
	\[
		\R=\bigcup_{n\geq1}V_n,
	\]
	is numerable, as each $V_n$ is numerable, a contradiction.
\end{example}

If $E$ is an extension of $K$ and $E$ is finite,
then $[E:K]$ is finite. 

\begin{proposition}
	Let $K$ be a finite field. Then $|K|=p^m$ 
	for some prime number $p$ and some $m\geq1$. 
\end{proposition}

\begin{proof}
	We know that the prime subfield of $K$ is isomorphic to $\Z/p$. 
	In particular, $|K_0|=p$. Since $K$ is finite, 
	$[K:K_0]=m$ for some $m$. If $\{x_1,\dots,x_m\}$ is a basis
	of $K$ over $K_0$, then each element
	of $K$ can be written uniquely as
	$\sum_{i=1}^ma_ix_i$ for some $a_1,\dots,a_m\in K_0$. Then
	$K\simeq K_0^m$ and hence $|K|=|K_0^m|=p^m$. 
\end{proof}

\begin{definition}
	Let $E$ be an extension of $K$. A \textbf{subextension} $F$ 
	of $K$ is a subfield $F$ of $E$ that contains $K$, that is
	$K\subseteq F\subseteq E$. 
\end{definition}

\begin{definition}
	Let $E$ and $E_1$ be extensions over $K$. An extension
	\textbf{homomorphism} $E\to E_1$ is a 
	field homomorphism $\sigma\colon E\to E_1$ such that 
	$\sigma(x)=x$ for all $x\in K$. 
\end{definition}

To describe the homomorphism $\sigma\colon E\to E_1$ of the extensions over $K$
one typically writes the commutative diagram 
\[
	\begin{tikzcd}
	K & K \\
	E & {E_1} 
	\arrow["\sigma", from=2-1, to=2-2]
	\arrow[equal, no head, from=1-1, to=1-2]
	\arrow[hook, from=1-1, to=2-1]
	\arrow[hook, from=1-2, to=2-2]
\end{tikzcd}
\]
We write $\Hom(E/K,E_1/K)$ to denote
the set of homomorphism $E\to E_1$ of extensions of $K$. Note
that if $\sigma\in\Hom(E/K,E_1/K)$, then
$\sigma$ is a $K$-linear map, as
\[
	\sigma(\lambda x)=\sigma(\lambda)\sigma(x)=\lambda\sigma(x)
\]
for all $\lambda\in K$ and $x\in E$. 

\begin{example}
	The conjugation map $\C\to\C$, $z\mapsto\overline{z}$, 
	is an endomorphism of $\C$ as an extension over $\R$. Let 
	$\varphi\in\Hom(\C/\R,\C/\R)$. Then 
	\[
	\varphi(x+iy)=\varphi(x)+\varphi(i)\varphi(y)=x+\varphi(i)y
	\]
	for all $x,y\in\R$. Since $\varphi(i)^2=\varphi(i^2)=\varphi(-1)=-1$, 
	it follows that $\varphi(i)\in\{-i,i\}$. Thus either 
	$\varphi(x+iy)=x+iy$ or $\varphi(x+iy)=x-iy$. 
\end{example}

\begin{exercise}
	Prove that if $K$ is a field and $\sigma\colon K\to K$ is a field homomorphism, 
then $\sigma\in\Hom(K/K_0,K/K_0)$. 
\end{exercise}

If $E/K$ is an extension, then
\[
	\Aut(E/K)=\{\sigma:\sigma\colon E\to E\text{ is a bijective extension homomorphism}\}
\]
is a group with composition.

\begin{definition}
	Let $E/K$ be an extension. The \textbf{Galois group}
	of $E/K$ is the group
	$\Aut(E/K)$ and it will be denoted by $\Gal(E/K)$. 
\end{definition}

A typicall example: $\Gal(\C/\R)\simeq\Z/2$. 

\begin{example}
	Let $\theta=\sqrt[3]{2}$ and let $E=\{a+b\theta+c\theta^2:a,b,c\in\Q\}$. Note that 
\[
	a+b\theta+c\theta^2=0 \Longleftrightarrow a=b=c=0. 
\]
In fact, if $abc\ne 0$, then $aX^2+bX+c\ne 0$ and 
thus $X^3-2=q(X)(aX^2+bX+c)+r(X)$ for some polynomials
$q(X)\in\Q[X]$ and $r(X)=eX+f\in\Q[X]$. Evaluate in $\theta$ 
to obtain that $r(\theta)=0$ and hence $r(X)=0$ in $\Q[X]$. This implies  
that $aX^2+bX+c$ divides $X^3-2$, a contradiction since
$X^3-2$ is irreducible in $\Q[X]$. 

Then $E$ is an extension of $\Q$ such that $[E:\Q]=3$. We claim
that $\Gal(E/\Q)$ is trivial. If 
$\sigma\in\Gal(E/\Q)$ and $z=a+b\theta+c\theta^2$, then
$\sigma(z)=a+b\sigma(\theta)+c\sigma^2(\theta)$. Since
$\sigma(\theta)^3=\sigma(\theta^3)=\sigma(2)=2$, it follows
that $\sigma(\theta)=\theta$ and therefore
$\sigma=\id$. 
\end{example}

If $E/K$ is an extension and $S$ is a subset of $E$, then
there exists a unique smallest 
subextension $F/K$ of $E/K$ such that
$S\subseteq F$. In fact, 
\[
	F=\bigcap\{T:\text{$T$ is a subfield of $E$ that contains $K\cup S$}\} 
\]
If $L/K$ is a subextension of $E/K$ such that 
$S\subseteq L$, then $F\subseteq L$ by definition. The 
extension $F$ is known as the \textbf{subextension generated by} 
$S$ and
it will be denoted by $K(S)$. 
If $S=\{x_1,\dots,x_n\}$ is finite,
then $K(S)=K(x_1,\dots,x_n)$ is said to be of \textbf{finite type}. 

\begin{example}
	If $\{e_1,\dots,e_n\}$ is a basis of $E$ over $K$, 
	then $E=K(e_1,\dots,e_n)$. 
\end{example}

\begin{example}
	The field $\Q(\sqrt{2})$ is precisely the extension 
	of $\R/\Q$ generated by $\sqrt{2}$. 
\end{example}

Let $E/K$ be an extension and $S$ and $T$ be subsets of $E$.
Then 
\[
	K(S\cup T)=K(S)(T)=K(T)(S).
\]
If, moreover, 
$S\subseteq T$, then $K(S)\subseteq K(T)$. 

\begin{definition}
	Let $E/K$ be an extension. An element $x\in E$
	is \textbf{algebraic} over $K$ if there
	exists a non-zero polynomial 
	$f(X)\in K[X]$ such that $f(x)=0$. If $x$ is
	not algebraic over $K$, 
	then it is called \textbf{trascendent} over $K$.
\end{definition}

If $E/K$ is an extension, then 
\[
	\overline{K}_E=\{x\in E:x\text{ is algebraic over }K\}
\]
is the \textbf{algebraic closure} of $K$ in $E$. 

\begin{definition}	
	An extension $E/K$ is \textbf{algebraic} if 
	every $x\in E$ is algebraic over $K$. 
\end{definition}

If $K$ is a field, every $x\in K$ is algebraic over $K$,
as $x$ is a root of $X-x\in K[X]$. In particular, $K/K$ is
an algebraic extension. 

\begin{example}
	$\C/\R$ is an algebraic extension. If $z\in\C\setminus\R$, then
	$z$ is a root of the polynomial 
	$X^2+(z+\overline{z})X+|z|^2\in\R[X]$. 
\end{example}

If $F/K$ is an algebraic extension and $x\in E$ is algebraic
over $K$, then $x$ is algebraic over $E$. 

\begin{example}
	$\Q(\sqrt{2})/\Q$ is algebraic, as the number
	$a+b\sqrt{2}$ is a root of the polynomial
	$X^2-2aX+(a^2-2b^2)\in\Q[X]$. 
\end{example}

The extension $\C/\Q$ is not algebraic. 

If $E/K$ is an extension and $x\in E$ is algebraic
over $K$, then the evaluation homomorphism 
$K[X]\to E$, $f(X)\mapsto f(x)$, is not injective. In particular,
its kernel is a non-zero ideal and hence 
it is generated by a monic polynomial $f(X)$. This polynomial
is known as the \textbf{minimal polynomial} of $x$ over $X$ and
it will be denoted by $f(x,K)$. The \textbf{degree } 
of $x$ over $K$ is then $\deg f(x,K)$. 

\begin{proposition}
	Let $E/K$ be an extension and $x\in E$. 
	\begin{enumerate}
		\item If $g\in K[X]$ is such that $g(x)=0$, then $f(x,K)$ divides $g$. 
		\item If $g(x)=0$ and $g\ne 0$, then $\deg g\geq\gr f(x,K)$.
		\item $f(x,K)$ is irreducible in $K[X]$.
		\item If $g(x)=0$ and $g(X)$ is monic and irreducible, then
			$g=f(x,K)$. 
		\item If $F/K$ is a subextension of $E/K$, then $f(x,F)$ divides
			$f(x,K)$. 
	\end{enumerate}
\end{proposition}

\begin{proof}
\end{proof}

Some easy examples: $f(i,\R)=X^2+1$ and 
$f(\sqrt[3]{2},\Q)=X^3-2$. 

\begin{example}
	Let us compute 
	$f(\sqrt{2}+\sqrt{3},\Q)$. Let $\alpha=\sqrt{2}+\sqrt{3}$. 
	Then 
	\begin{align*}
		\alpha-\sqrt{2}=\sqrt{3} & \implies 
		(\alpha-\sqrt{2})^2=3 \implies \alpha^2-2\sqrt{2}\alpha+2=3\\
		&\implies \alpha^2-1=2\sqrt{2}\alpha \implies
		(\alpha^2-1)^2=8\alpha^2\implies
		\alpha^4-10\alpha^2+1=0.
	\end{align*}
	Thus $\alpha$ is a root of $g=X^4-10X^2+1$. To prove that $g=f(\alpha,\Q)$ 
	it is enough to prove that 
	$g$ is irreducible in $\Q[X]$. First note that 
	the roots
	of $g$ are $\sqrt{2}+\sqrt{3}$, $\sqrt{2}-\sqrt{3}$, 
	$-\sqrt{2}+\sqrt{3}$ and $-\sqrt{2}-\sqrt{3}$. This means that
	if $g$ is not irreducible, 
	then $g=hh_1$ for some polynomials $h,h_1\in\Q[X]$ such that
	$\deg h=\deg h_1=2$. This is not possible, as 
	$(\sqrt{2}+\sqrt{3})+(\sqrt{2}-\sqrt{3})=2\sqrt{2}\not\in\Q$, 
	$(\sqrt{2}+\sqrt{3})+(-\sqrt{2}+\sqrt{3})=2\sqrt{3}\not\in\Q$ and 
	$(\sqrt{2}+\sqrt{3})(-\sqrt{2}-\sqrt{3})=-5-2\sqrt{6}\not\in\Q$.
\end{example}

\begin{proposition}
	Let $F/K$ be a subextension and $E/K$. Then
	\[
	[E:K]=[E:F][F:K].
	\]
\end{proposition}

\begin{proof}
	Let $\{e_i:i\in I\}$ be a basis of $E$ over $K$
	and $\{f_j:j\in J\}$ be a basis of $F$ over $K$. If $x\in E$,
	then $x=\sum_i \lambda_ie_i$ (finite sum) 
	for some $\lambda_i\in F$. For each $i$, 
	$\lambda_i=\sum_j a_{ij}f_j$ (finite sum)
	for some $a_{ij}\in K$. Then 
	$x=\sum_i\sum_j a_{ij}(f_je_i)$. This means
	that $\{f_je_i:i\in I,j\in J\}$ generates
	$E$ as a $K$-vector space. Let us prove that 
	$\{f_je_i:i\in I,j\in J\}$
	is linearly independent. If $\sum_i\sum_j a_{ij}(f_je_i)=0$ (finite sum)
	for some $a_{ij}\in K$, 
	then
	\begin{align*}
		0=\sum_i\left(\sum_j a_{ij}f_j\right)e_i&\implies
		\sum_j a_{ij}f_j=0\text{ for all $i\in I$}\\
		&\implies 
		a_{ij}=0\text{ for all $i\in I$ and $j\in J$}.\qedhere
	\end{align*}
\end{proof}

We state a lemma:

\begin{lemma}
If $A$ is a finite-dimensional commutative algebra over $K$ 
and $A$ is an integral domain, then $A$ is a field. 
\end{lemma}

\begin{proof}
	Let $a\in A\setminus\{0\}$. We need to prove that there exists $b\in A$
	such that $ab=1$. Let $\theta\colon A\to A$, $x\mapsto ax$. Clearly
	$\theta$ is an algebra homomorphism. It is injective, since $A$ is an
	integral domain.  Since $\dim_KA<\infty$, it follows that $\theta$ is an
	isomorphism. In particular, $\theta(A)=A$, which means that there exists
	$b\in A$ such that $1=ab$. 
\end{proof}

\begin{proposition}
	Let $E/K$ be an extension and $x\in E\setminus K$.
	The following statements are equivalent:
	\begin{enumerate}
		\item $x$ is algebraic over $K$.
		\item $\dim_KK[x]<\infty$.
		\item $K[x]$ is a field.
		\item $K[x]=K(x)$. 
	\end{enumerate}
\end{proposition}

\begin{proof}
	We first prove $1)\implies 2)$. Let $z\in K[x]$, say $z=h(x)$ for some $h\in K[X]$. There exists
	$g\in K[X]$ such that $g\ne 0$ and $g(x)=0$. Divide $h$ by $g$ to obtain 
	polynomials $q,r\in K[X]$ such that $h=gq+r$, where $r=0$ or $\deg r<\deg g$. This implies that
	\[
		z=h(x)=g(x)q(x)+r(x)=r(x).
	\]
	If $\deg g=m$, then $r=\sum_{i=0}^{m-1}a_iX^i$ for some $a_0,\dots,a_{m-1}\in K$. Thus
	$z=\sum_{i=0}^{m-1}a_ix^i$, so $K[x]\subseteq\langle 1,x,\dots,x^{m-1}\rangle$. 

	The previous lemma proves that $2)\implies 3)$. 

	It is trivial that $3)\implies 4)$. 

	It remains to prove that $4)\implies 1)$. Let us prove that $K(x)\subseteq K[x]$. 
	Since $x\ne 0$, $1/x\in K[x]$. There exists $a_0,\dots,a_n\in K$ such that
	$1/x=a_0+a_1x+\cdots+a_nx^n$. Thus
	\[
		a_nx^{n+1}+\cdots+a_1x^2+a_0x-1\ne 0
	\]
	and $x$ is a root of $a_nX^{n+1}+\cdots+a_0X-1\in K[X]$. 
\end{proof}

Note that if $x$ is algebraic over $K$, then
$K[x]\simeq K[X]/(f(x,K))$. 

\begin{corollary}
	If $E/K$ is finite, then $E/K$ is algebraic. 
\end{corollary}

\begin{proof}
\end{proof}

We note that the converse of the previous corollary does not hold. 

\begin{corollary}
	If $E/K$ is an extension and $x_1,\dots,x_n\in E$ 
	are algebraic over $K$, then 
	$K(x_1,\dots,x_n)/K$ is finite and
	$K(x_1,\dots,x_m)=K[x_1,\dots,x_n]$. 
\end{corollary}

\begin{proof}
\end{proof}

\begin{corollary}
	Let $E=K(S)$. Then $E/K$ is algebraic if and only if
	$x$ is algebraic over $K$ for all $x\in S$. 
\end{corollary}

\begin{proof}
\end{proof}

\begin{corollary}
	If $E/K$ is  an extension, then $\overline{K}_E$ 
	is a subfield of $E$ that contains $K$. Moreover, 
	$K(\overline{K}_E)/K$ is algebraic. 
\end{corollary}	

\begin{proof}
\end{proof}

\begin{corollary}
\end{corollary}


%\begin{theorem}[Galois]
%	\index{Galois' theorem}
%	For every prime number $p$ and every $m\geq1$
%	there exists a field of size $p^m$. 
%\end{theorem}
%
%\begin{proof}
%\end{proof}
%
%
