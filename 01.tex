\chapter{}

\topic{Fields}

Recall that a \textbf{field} is a commutative
ring such that $1\ne 0$ and 
that every non-zero element is invertible. Examples
of (infinite) fields are $\Q$, $\R$ and $\C$. If $p$
is a prime number, then $\Z/p$ is a field. 

\begin{example}
	The abelian group $\Z/2\times\Z/2$ is a field
	with multiplication
	\[
		(a,b)(c,d)=(ac+bd,ad+bc+bd).
	\]
\end{example}

\begin{example}
	$\Q(i)=\{a+bi:a,b\in\Q\}$ and 
	$\Q(\sqrt{2})$ are fields.
\end{example}

\[
\begin{tikzcd}
	{\mathbb{C}} \\
	&& {\mathbb{R}} \\
	{\mathbb{Q}(i)} && {\mathbb{Q}(\sqrt{2})} \\
	& {\mathbb{Q}}
	\arrow[no head, from=3-3, to=4-2]
	\arrow[no head, from=2-3, to=3-3]
	\arrow[no head, from=1-1, to=3-1]
	\arrow[no head, from=3-1, to=4-2]
	\arrow[no head, from=1-1, to=2-3]
\end{tikzcd}
\]


\begin{exercise}
	\label{xca:Q(i)}
	Prove that $\Q(i)$ and $\Q(\sqrt{2})$ are not isomorphic as fields.
\end{exercise}

If $R$ is a ring, there exists a unique ring homomorphism
$\Z\to R$, $m\mapsto m1$. The image $\{m1:m\in\Z\}$ 
of this homomorphism is a subring 
of $R$ and it is known as the \textbf{ring of integers} of $R$. The
kernel is a subgroup of $\Z$ and is generated by
some $t\geq0$. The integer $t$ is 
the \textbf{characteristic} of the ring $R$. 

\begin{exercise}
	The characteristic of a field is either zero or
	a prime number. 
\end{exercise}

Recall that a commutative ring $R$ is an \textbf{integral 
domain} if $xy=0\implies x=0$ or $y=0$. Fields
are integral domains. 

\begin{exercise}
	Let $K$ be a field. Prove that
	the following statements are equivalent:
	\begin{enumerate}
		\item $K$ is of characteristic zero.
		\item The additive order of $1$ is infinite. 
		\item The additive order of each $x\ne0$ is infinite.
		\item The ring of integers of $K$ is isomorphic to $\Z$.
	\end{enumerate}
\end{exercise}

\begin{exercise}
	Let $K$ be a field. Prove that
	the following statements are equivalent:
	\begin{enumerate}
		\item $K$ is of characteristic $p$.
		\item The additive order of $1$ is $p$. 
		\item The additive order of each $x\ne0$ is $p$.
		\item The ring of integers of $K$ is isomorphic to $\Z/p$.
	\end{enumerate}
\end{exercise}

% The following exercise is important. 

% \begin{exercise}
% 	Prove that if $K$ is a finite field, then
% 	$|K|=p^m$ for some prime number $p$ and some $m\geq1$. 
% \end{exercise}

\begin{definition}
	\index{Subfield}
	A \textbf{subfield} of a ring $R$ is a subring of $R$ 
	that is also a field.
\end{definition}

Note that if $K$ is a subfield of $E$, then
the characteristic of $K$ coincides
with the characteristic 
of $E$. Moreover, if $K\to L$ is a field homomorphism, then
$K$ and $L$ have the same characteristic. 

\begin{exercise}
	Let $K$ be a field of characteristic $p$. Prove
	that $K\to K$, $x\mapsto x^{p^n}$, is a field homomorphism
	for all $n\in\Z_{\geq 0}$. 
\end{exercise}

Note that finite fields are of characteristic $p$. 

Let $K$ be a subfield of a field $E$. Then $E$ 
is a $K$-vector space with the usual scalar multiplication
$K\times E\to E$, 
$(\lambda, x)\mapsto \lambda x$.

\begin{definition}
	A field $K$ is \textbf{prime} if there are no
	proper subfields of $K$. 
\end{definition}

Examples of prime fields are $\Q$ and $\Z/p$ for a prime number $p$.

\begin{proposition}
	Let $K$ be a field. The following statements hold:
	\begin{enumerate}
		\item $K$ contains a unique prime field, it is known as the 
			\textbf{prime subfield} of $K$.
		\item The prime subfield of $K$ is either isomorphic to $\Q$ if 
			the characteristic of $K$ is zero, or it is isomorphic to $\Z/p$ for
			some prime number $p$ if the characteristic of $K$ is $p$. 
	\end{enumerate}
\end{proposition}

\begin{proof}
	To prove the first claim let $L$ be the intersection
	of all the subfields of $K$. Then $L$ is a subfield of $K$. 
	If $F$ is a subfield of $L$, then $F$ is a subfield
	of $K$. Thus $L\subseteq F$ and hence $F=L$, which proves
	that $L$ is prime. If $L_1$ is a subfield of $K$
	and $L_1$ is prime, then $L\subseteq L_1$ and 
	hence $L=L_1$. 

	Let $K_0$ be the prime field of $K$. Suppose that $K$ is of characteristic
	$p>0$. Then the ring $K_\Z$ of integers of $K$ 
	is a field isomorphic to $\Z/p$ and hence $K_0\simeq
	K_\Z$. Suppose now that the characteristic of $K$ is zero. Let
	$E=\{m1/n1:m,n\in\Z,n\ne 0\}$. We claim that $K_0=E$. Since $K_\Z\subseteq
	K_0$, it follows that $E\subseteq K_0$. Hence $E=K_0$, as $E$ is a subfield
	of $K$.  
\end{proof}

\begin{definition}
	Let $E$ be a field and $K$ be a subfield of $E$. Then 
	$E$ is a \textbf{field extension} of $K$. We will use
	the notation $E/K$. 
\end{definition}

If $E$ is an extension of $K$, then $E$ is a
$K$-vector space. 

\begin{definition}
	The degree of an extension $E$ of $K$ 
	is the integer $\dim_KE$. It will be denoted by $[E:K]$. 
\end{definition}

We say that $E$ is a finite extension of $K$ 
if $[E:K]$ is finite. 

\begin{example}
	Let $K$ be a field. Then $[K:K]=1$. Conversely, 
	if $E$ is an extension of $K$ and $[E:K]=1$, then $K=E$. 
	If not, let $x\in E\setminus K$. We claim that
	$\{1,x\}$ is linearly independent over $K$. Indeed, 
	if $a1+bx=0$ for some $a,b\in K$, then $bx=-a$. If 
	$b\ne 0$, then $x=-a/b\in K$, a contradiction. If $b=0$, then 
	$a=0$. 
\end{example}

We know that $[\C:\R]=2$. 

\begin{example}
	A basis of $\Q(\sqrt{2})$ over $\Q$ 
	is given by $\{1,\sqrt{2}\}$. Then 
	$[\Q(\sqrt{2}):\Q]=2$. The calculations 
	can be easily done by computer: 
\begin{lstlisting}
julia> E, a = quadratic_field(2)
(Real quadratic field defined by x^2 - 2, sqrt(2))

julia> characteristic(E)
0

julia> K = prime_field(E)
Rational Field

julia> degree(E)
2

julia> basis(E)
2-element Vector{nf_elem}:
 1
 sqrt(2)
 
julia> one(K)==one(E)
true

julia> zero(K)==zero(E)
true
\end{lstlisting}
\end{example}

\begin{example}
	Since $\Q$ is numerable and 
	$\R$ is not, $[\R:\Q]>\aleph_0$. If $\{x_i:i\in\Z_{>0}\}$ 
	is a numerable basis of $\R$ over $\Q$, for each
	$n$ consider the $\Q$-vector space
	$V_n$ generated by $\{x_1,\dots,x_n\}$. Then 
	\[
		\R=\bigcup_{n\geq1}V_n,
	\]
	is numerable, as each $V_n$ is numerable, a contradiction.
\end{example}

If $E$ is an extension of $K$ and $E$ is finite,
then $[E:K]$ is finite. 

\begin{proposition}
	Let $K$ be a finite field. Then $|K|=p^m$ 
	for some prime number $p$ and some $m\geq1$. 
\end{proposition}

\begin{proof}
	We know the prime subfield $K_0$ of $K$ is isomorphic to $\Z/p$. 
	In particular, $|K_0|=p$. Since $K$ is finite, 
	$[K:K_0]=m$ for some $m$. If $\{x_1,\dots,x_m\}$ is a basis
	of $K$ over $K_0$, then each element
	of $K$ can be written uniquely as
	$\sum_{i=1}^ma_ix_i$ for some $a_1,\dots,a_m\in K_0$. Then
	$K\simeq K_0^m$ and hence $|K|=|K_0^m|=p^m$. 
\end{proof}

We now perform some basic calculations 
with a finite field of eight elements: 
\begin{lstlisting}
julia> E, x = FiniteField(2, 3, "x")
(Finite field of degree 3 over F_2, x)

julia> characteristic(E)
2

julia> prime_field(E)
Galois field with characteristic 2

julia> degree(E)
3

julia> size(E)
8

julia> [z for z in E]
8-element Vector{fq_nmod}:
 0
 1
 x
 x + 1
 x^2
 x^2 + 1
 x^2 + x
 x^2 + x + 1
\end{lstlisting}

%julia> F, y = FiniteField(5, 2, "y")
%(Finite field of degree 2 over F_5, y)
%julia> degree(F)
%2
%julia> prime_field(F)
%Galois field with characteristic 5
%julia> size(F)
%25


\begin{definition}
	Let $E$ be an extension of $K$. A \textbf{subextension} $F/K$ 
	of $E/K$ is a subfield $F$ of $E$ that contains $K$, that is
	$K\subseteq F\subseteq E$. 
\end{definition}

\begin{definition}
	Let $E$ and $E_1$ be extensions over $K$. An extension
	\textbf{homomorphism} $E\to E_1$ is a 
	field homomorphism $\sigma\colon E\to E_1$ such that 
	$\sigma(x)=x$ for all $x\in K$. 
\end{definition}

To describe the homomorphism $\sigma\colon E\to E_1$ of the extensions over $K$
one typically writes the commutative diagram 
\[
	\begin{tikzcd}
	K & K \\
	E & {E_1} 
	\arrow["\sigma", from=2-1, to=2-2]
	\arrow[equal, no head, from=1-1, to=1-2]
	\arrow[hook, from=1-1, to=2-1]
	\arrow[hook, from=1-2, to=2-2]
\end{tikzcd}
\]
We write $\Hom(E/K,E_1/K)$ to denote
the set of homomorphism $E\to E_1$ of extensions of $K$. Note
that if $\sigma\in\Hom(E/K,E_1/K)$, then
$\sigma$ is a $K$-linear map, as
\[
	\sigma(\lambda x)=\sigma(\lambda)\sigma(x)=\lambda\sigma(x)
\]
for all $\lambda\in K$ and $x\in E$. 

\begin{example}
	The conjugation map $\C\to\C$, $z\mapsto\overline{z}$, 
	is an endomorphism of $\C$ as an extension over $\R$. Let 
	$\varphi\in\Hom(\C/\R,\C/\R)$. Then 
	\[
	\varphi(x+iy)=\varphi(x)+\varphi(i)\varphi(y)=x+\varphi(i)y
	\]
	for all $x,y\in\R$. Since $\varphi(i)^2=\varphi(i^2)=\varphi(-1)=-1$, 
	it follows that $\varphi(i)\in\{-i,i\}$. Thus either 
	$\varphi(x+iy)=x+iy$ or $\varphi(x+iy)=x-iy$. 
\end{example}

\begin{exercise}
	Prove that if $K$ is a field and $\sigma\colon K\to K$ is a field homomorphism, 
then $\sigma\in\Hom(K/K_0,K/K_0)$. 
\end{exercise}

If $E/K$ is an extension, then
\[
	\Aut(E/K)=\{\sigma:\sigma\colon E\to E\text{ is a bijective extension homomorphism}\}
\]
is a group with composition.

\begin{definition}
	Let $E/K$ be an extension. The \textbf{Galois group}
	of $E/K$ is the group
	$\Aut(E/K)$ and it will be denoted by $\Gal(E/K)$. 
\end{definition}

A typical example: $\Gal(\C/\R)\simeq\Z/2$. 

As an example, we show with the computer that $\Gal(\Q(\sqrt{2})/\Q)\simeq\Z/2$:
\begin{lstlisting}
julia> E, x = quadratic_field(2)
(Real quadratic field defined by x^2 - 2, sqrt(2))
julia> characteristic(E)
0
julia> G, C = galois_group(E);
julia> describe(G)
"C2"
julia> order(G)
2
\end{lstlisting}

\begin{example}
	Let $\theta=\sqrt[3]{2}$ and let $E=\{a+b\theta+c\theta^2:a,b,c\in\Q\}$. Note that 
\[
	a+b\theta+c\theta^2=0 \Longleftrightarrow a=b=c=0. 
\]
% In fact, if $abc\ne 0$, then $aX^2+bX+c\ne 0$ and 
% thus $X^3-2=q(X)(aX^2+bX+c)+r(X)$ for some polynomials
% $q(X)\in\Q[X]$ and $r(X)=eX+f\in\Q[X]$. Evaluate in $\theta$ 
% to obtain that $r(\theta)=0$ and hence $r(X)=0$ in $\Q[X]$. This implies  
% that $aX^2+bX+c$ divides $X^3-2$, a contradiction since
% $X^3-2$ is irreducible in $\Q[X]$. 
Then $E$ is an extension of $\Q$ such that $[E:\Q]=3$. We claim
that $\Gal(E/\Q)$ is trivial. If 
$\sigma\in\Gal(E/\Q)$ and $z=a+b\theta+c\theta^2$, then
$\sigma(z)=a+b\sigma(\theta)+c\sigma^2(\theta)$. Since
$\sigma(\theta)^3=\sigma(\theta^3)=\sigma(2)=2$, it follows
that $\sigma(\theta)=\theta$ and therefore
$\sigma=\id$. 
\end{example}

\begin{exercise}
    Prove that the polynomial $X^3-2$ is irreducible in $\Q[X]$.  
\end{exercise}

The previous exercise can easily be solved using
computers: 
\begin{lstlisting}
julia> R, x = PolynomialRing(QQ, "x");
julia> is_irreducible(x^3-2)
true
\end{lstlisting}

The following exercise is known as the 
\emph{Eisenstein's irreducibility criterion}:

\begin{exercise}
    \index{Eisenstein's criterion}    
    Let $A$ be a unique factorization domain and $K$ be its fraction field. 
    Let $f=\sum_{i=0}^n a_iX^i\in K[X]$ be a polynomial of degree $n>0$. 
    Assume that there exists a prime element $p\in A$ such that
    $p\mid a_i$ for all $i\in\{0,1,\dots,n-1\}$, $p\nmid a_n$ and
    $p^2\nmid a_0$. Then $f$ is irreducible in $K[X]$. 
\end{exercise}

\begin{exercise}
    Prove that
    the polynomials 
    $f=X^{10}+60X^7+82X^6-36X^3+2$ and 
    $g=3X^{10}+15X^2-45$ are irreducible in $\Z[X]$. 
\end{exercise}

\begin{exercise}
    Is the polynomial $f=3(X^{10}+5X^2-15)$ irreducible in $\Z[X]$? 
\end{exercise}

If $E/K$ is an extension and $S$ is a subset of $E$, then
there exists a unique smallest 
subextension $F/K$ of $E/K$ such that
$S\subseteq F$. In fact, 
\[
	F=\bigcap\{T:\text{$T$ is a subfield of $E$ that contains $K\cup S$}\} 
\]
If $L/K$ is a subextension of $E/K$ such that 
$S\subseteq L$, then $F\subseteq L$ by definition. The 
extension $F$ is known as the \textbf{subextension generated by} 
$S$ and
it will be denoted by $K(S)$. 
If $S=\{x_1,\dots,x_n\}$ is finite,
then $K(S)=K(x_1,\dots,x_n)$ is said to be of \textbf{finite type}. 

\begin{example}
	If $\{e_1,\dots,e_n\}$ is a basis of $E$ over $K$, 
	then $E=K(e_1,\dots,e_n)$. 
\end{example}

\begin{example}
	The field $\Q(\sqrt{2})$ is precisely the extension 
	of $\R/\Q$ generated by $\sqrt{2}$. 
\end{example}

Let $E/K$ be an extension and $S$ and $T$ be subsets of $E$.
Then 
\[
	K(S\cup T)=K(S)(T)=K(T)(S).
\]
If, moreover, 
$S\subseteq T$, then $K(S)\subseteq K(T)$. 

\topic{Algebraic extensions}

\begin{definition}
\index{Algebraic!element}
\index{Trascendental!element}
	Let $E/K$ be an extension. An element $x\in E$
	is \textbf{algebraic} over $K$ if there
	exists a non-zero polynomial 
	$f(X)\in K[X]$ such that $f(x)=0$. If $x$ is
	not algebraic over $K$, 
	then it is called \textbf{trascendental} over $K$.
\end{definition}

If $E/K$ is an extension, let 
\[
	\overline{K}_E=\{x\in E:x\text{ is algebraic over }K\}. 
\]
%is the \textbf{algebraic closure} of $K$ in $E$. 

\begin{definition}	
\index{Algebraic!extension}
	An extension $E/K$ is \textbf{algebraic} if 
	every $x\in E$ is algebraic over $K$. 
\end{definition}

If $K$ is a field, every $x\in K$ is algebraic over $K$,
as $x$ is a root of $X-x\in K[X]$. In particular, $K/K$ is
an algebraic extension. 

\begin{example}
	$\C/\R$ is an algebraic extension. If $z\in\C\setminus\R$, then
	$z$ is a root of the polynomial 
	$X^2-(z+\overline{z})X+|z|^2\in\R[X]$. 
\end{example}

If $F/K$ is an algebraic extension $x\in E$ is algebraic
over $K$ for some field $E\supseteq F$, 
then $x$ is algebraic over $F$. 

\begin{example}
	$\Q(\sqrt{2})/\Q$ is algebraic, as the number
	$a+b\sqrt{2}$ is a root of the polynomial
	$X^2-2aX+(a^2-2b^2)\in\Q[X]$. 
\end{example}

\index{Lindemann's theorem}
\index{Hermite's theorem}
The extension $\C/\Q$ is not algebraic. For example, Hermite proved 
that $e$ is transcendental 
over $\Q$; see \cite[Therem 24.4]{MR3379917}. Lindemann's theorem 
states that $\pi$ is 
not algebraic $\Q$; see \cite[Theorem 24.5]{MR3379917}. 

\begin{example}
    Let $a=\sqrt{2}$ and $b=\sqrt[3]{3}$. Both $a$ and $b$ are algebraic numbers over $\Q$. 
    Let us show that $a+b$ is also algebraic. Let $f(X)=X^3-3\in\Q[X]$. Then $f(b)=0$. Note
    that the polynomial 
    \[
    g(X)=f(X-a)=X^3-3aX^2+3aX-a^3-3\in\Q(a)[X]
    \]
    is such that $g(a+b)=0$. How can we find a polynomial 
    with coefficients in $\Q$ that vanishes on $a+b$? We do the ``conjugation" trick:
    \[
    h(X)=f(X-a)f(X+a)=X^6-6X^4-6X^3+12X^2-36X+1\in\Q[X].
    \]
    Note that $h(a+b)=0$. How can you prove that $ab$ is also algebraic over $\Q$?     
\end{example}
