\section{04/03/2024}

The previous proposition will be used to prove 
that the algebraic closure always exists. 

\begin{theorem}[Artin]
	\index{Artin's theorem}
	Let $K$ be a field. Then $K$ admits an algebraic closure $C/K$. If $C_1/K$
	is an algebraic closure, then the extensions $C/K$ and $C_1/K$ are
	isomorphic. 
\end{theorem}

\begin{proof}
    Let us first prove the uniqueness. The previous proposition implies the existence of 
    an extensions homomorphism $\varphi\colon C\to C_1$. Let $y\in C_1$ and $f=f(y,K)$ be 
    the minimal polynomial of $y$ in $K$. Since $f$ admits a factorization
    \[
        f=\lambda\prod (X-\alpha_i)^{m_i}
    \]
    in $C[X]$, it follows that
    \[
    f=\overline{\varphi}(f)=\varphi(\lambda)\prod (X-\varphi(\alpha_i))^{m_i}
    \]
    Since $0=f(y)$, we conclude that $y=\varphi(\alpha_j)$ for some $j$. In particular, $\varphi$ is
    surjective and hence $\varphi$ is bijective. 
    
    We now prove the existence. Let us assume that $K$ admits an extension $E/K$ 
    with $E$ algebraically closed. We will prove later that this extension indeed exists; at the moment,
    we only want to get an algebraic extension from this setting. Let 
    \[
    	F=\{x\in E:x\text{ is algebraic over }K\}. 
    \]
    Then $F/K$ is algebraic. Let $g\in F[X]$ be such that
    $\deg g>0$. Since $E$ is algebraically closed, $g$ admits a root $\alpha$ in $E$. In particular, $\alpha$
    is algebraic over $F$ and hence $\alpha$ is algebraic over $K$. This implies that $\alpha\in F$, thus
    $F$ is algebraically closed. This proves that $F/K$ is an algebraic closure. 
    
    Let us prove that there exists an extension $E_1/K$ such that
    every polynomial $f\in K[X]$ with $\deg f>0$ has a root in $E_1$. Let 
    $\{f_i:i\in I\}$ be the family of monic irreducible polynomials with coefficients in $K$. 
    We may think that $f_i=f_i(X_i)$. 
    Let $R=K[\{X_i:i\in I\}]$ and let $J$ be the ideal of $R$ 
    generated by the $f_i(X_i)$. We claim that $J\ne R$. If not, $1\in J$, so
    \[
    1=\sum_{j=1}^m g_jf_{i_j}(X_{i_j})
    \]
    for some $g_1,\dots,g_m\in R$. There exists an extension $F/K$ such that
    $f_{i_j}$ has a root $\alpha_j$ in $F$ for all $j$. Let 
    \[
    \tau\colon R\to F,\quad
    \tau(X_k)=\begin{cases}
        \alpha_j & \text{if $k=i_j$},\\
        0 & \text{if $k\not\in\{i_1,\dots,i_m\}$}.
        \end{cases}
    \]
    Then $\tau$ is a ring homomorphism and 
    \[
    1=\tau(1)=\sum_{j=1}^m\tau(g_j)f_{i_j}(\alpha_{j})=0,
    \]
    a contradiction. 
    
    Since $J$ is a proper ideal, it is contained in a maximal ideal $M$. Let $L=R/M$ 
    and let $\sigma\colon K\to L$ be the composition $K\hookrightarrow R\to R/M=L$, 
    where $\pi\colon R\to R/M$ is the canonical map.  
    As we did before,
    $\pi(X_i)$ is a root of $\overline{\sigma}(f_i)$ for all $i$. And 
    there exists an extension $E_1/K$ such that
    every $f_i$ has a root in $E_1$. Proceeding in this way, we construct
    a sequence
    \[
    E_1\subseteq E_2\subseteq\cdots
    \]
    of fields such that every polynomial of positive degree and coefficients in $E_k$ 
    admits a root in $E_{k+1}$. Let $E=\cup E_k$. We claim that $E$ is algebraically closed. In fact, 
    let $g\in E[X]$ be such that $\deg g>0$. Then, since $g\in E_r[X]$ for some $r$, it follows
    that $g$ has a root in $E_{r+1}\subseteq E$. 
\end{proof}

\subsection{Decomposition fields}

\begin{definition}
\index{Decomposition field}
	Let $K$ be a field and $f\in K[X]$ be such that $\deg f>0$. A \textbf{decomposition field}
	of $f$ over $K$ is field $E$ that contains $K$ and that satisfies the following properties:
	\begin{enumerate}
		\item $f$ factorizes linearly in $E[X]$. 
		\item If $F$ is a field such that $K\subseteq F\subseteq E$ and 
			$f$ factorizes
			linearly in $F[X]$, then $F=E$. 
	\end{enumerate}
\end{definition}

Easy examples: 

\begin{example}
	$\C$ is a decomposition field of $X^2+1\in\R[X]$. 
\end{example}

\begin{example}
	$\Q[\sqrt{2}]$ is a decomposition field of $X^2-2\in\Q[X]$. 
\end{example}

\begin{example}
    The decomposition field of $f=X^2-2$ over $\Z/7$ is 
    precisely $\Z/7$, as 3 and 4 are the roots of $f$ in $\Z/7$. 
\end{example}

\begin{example}
	$\Q(\sqrt[3]{2})$ is not a decomposition field of $X^3-2\in\Q[X]$. However, if
	$\omega$ is a primitive cubic root of one, then 
	$\Q(\sqrt[3]{2},\omega)$ is a decomposition field of the polynomial $X^3-2\in\Q[X]$. 
\end{example}

\begin{proposition}
	$E$ is a decomposition field of $f\in K[X]$ if and only if
	$f$ factorizes linearly in $E[X]$ and $E=K(x_1,\dots,x_n)$, where 
	$x_1,\dots,x_n$ are the roots of $f$. 
\end{proposition}

\begin{proof}
    Let $f=a\prod_{i=1}^r(X-x_i)^{n_i}$ and $F=K(x_1,\dots,x_r)$ with $x_1,\dots,x_r\in E$. Since $f$
    factorizes linearly in $F[X]$, it follows that $F=E$. 
    Conversely, let $E=K(x_1,\dots,x_r)$ and assume that $f$ factorizes linearly
    in $F[X]$. Then, in particular, $x_1,\dots,x_r\in F$. Hence $E\subseteq F$ and
    $F=E$. 
\end{proof}

One immediately obtains the following consequence:
If $E$ is a decomposition field of $f\in K[X]$, then $E/K$ is finite. 

\begin{theorem}
    Let $f\in K[X]$ be such that $\deg f>0$. There exists a (unique up to extension isomorphism) 
    decomposition field of $f$ over $K$. 
\end{theorem}

\begin{proof}
    Let $C/K$ be an algebraic closure. Write 
    \[
    f=a\prod_{i=1}^r(X-x_i)^{n_i}
    \]
    in $C[X]$. 
    Then $E=K(x_1,\dots,x_r)$ is a decomposition field of $f$ over $K$. 
    
    Let us prove the 
    uniqueness: if $E_1/K$ is a decomposition field of $f$ over $K$, 
    then $E_1/K$ is algebraic and thus Proposition
    \ref{pro:Artin} implies that 
    there exists $\varphi\in\Hom(E_1/K,C/K)$, that is $\varphi\colon E_1\to C$ is a field
    homomorphism such that $\varphi|_K$ is the identity.
    Factorize $f$ linearly in $E_1[X]$ and apply $\overline{\varphi}$:
    \[
    f=a\prod_{j=1}^s(X-y_j)^{m_j}
    \implies
    f=\overline{\varphi}(f)=\varphi(a)\prod_{j=1}^s(X-\varphi(y_j))^{m_j}
    \]
    so $f$ factorizes linearly in $\varphi(E_1)[X]$. Moreover, 
    $E_1=K(y_1,\dots,y_s)$ and  
    \[
    \varphi(E_1)=K(\varphi(y_1),\dots,\varphi(y_s)).
    \]
    Thus
    $\varphi(E_1)$ is a decomposition field of $f$. Since  
    $\varphi(E_1)\subseteq C$, it follows that $\varphi(E_1)=E$. 
\end{proof}

\begin{exercise}
\label{xca:C/K_bijective}
If $C$ is an algebraic closure of $K$ and 
$\varphi\in\Hom(C/K,C/K)$, 
then $\varphi$ is an isomorphism. 
%$E/K$ is finite and $\varphi\in\Hom(E/K,E/K)$, 
%then $\varphi$ is an isomorphism. 
\end{exercise}


Let $C$ be an algebraic closure of $K$ and 
$G=\Gal(C/K)$. The group $G$ acts on $C$
\[
\sigma\cdot x=\sigma(x),\quad
\sigma\in G,\,x\in C.
\]
The orbits 
are of the form 
\[
O_G(x)=\{\sigma(x):\sigma\in G\}
=\{y\in C:y=\sigma(x)\text{ for some $\sigma\in G$}\}
\]
The elements $x,y\in C$ are \textbf{conjugate} 
if $y=\sigma(x)$ for some $\sigma\in G$. 

\begin{proposition}
\label{pro:conjugate}
    Let $C$ be an algebraic closure of $K$ and $x,y\in C$. Then 
    $x$ and $y$ are conjugate if and only if $f(x,K)=f(y,K)$. In particular, 
    $O_G(x)$ is finite. 
\end{proposition}

\begin{proof}
    Let $G=\Gal(C/K)$. 
    If $x$ and $y$ are conjugate, say $y=\sigma(x)$ for some $\sigma\in G$, 
    let us write $g=f(x,K)$ as 
    \[
    g=X^n+\sum_{i=0}^{n-1} a_iX^i. 
    \]
    Then $0=g(x)=x^n+\sum_{i=0}^{n-1}a_ix^i$ and hence $y$ is
    a root of $g$, as 
    \begin{align*}
    0=\sigma\left(x^n+\sum_{i=0}^{n-1}a_ix^i\right)
    &=\sigma(x)^n+\sum_{i=0}^{n-1}\sigma(a_i)\sigma(x)^i\\
    &=\sigma(x)^n+\sum_{i=0}^{n-1}a_i\sigma(x)^i
    =y^n+\sum_{i=0}^{n-1}a_iy^i.
    \end{align*}
    Thus $f(y,K)=g$. 
    
    Conversely, assume that $f(x,K)=f(y,K)$. Let
    $g=f(x,K)=f(y,K)$ and let 
    \[
    \varphi\colon K[x]\to K[y],
    \quad
    h(x)\mapsto h(y).
    \]
    Let us show that the map $\varphi$ is well-defined: we need to show 
    that if 
    $h_1(x)=h_2(x)$, then 
    \[
    h_1(y)=\varphi(h_1(x))=\varphi(h_2(x))=h_2(y).
    \]
    If $h_1(x)=h_2(x)$, then 
    \[
    (h_1-h_2)(x)=h_1(x)-h_2(x)=0.
    \]
    This implies
    that $g$ divides $h_1-h_2$. In particular, $h_1(y)=h_2(y)$.
    
    A straightforward calculation shows that $\varphi$ is a field 
    homomorphism such that $\varphi|_K=\id$, this means that
    $\varphi$ is 
    an extension homomorphism such that $\varphi(x)=y$. There exists
    $\sigma\in\Hom(C/K,C/K)$ such that 
    $\sigma|_{K[x]}=\varphi$. Since $\sigma$ is bijective 
    (this is left as an exercise, you did something similar before), 
    $\sigma(x)=\varphi(x)=y$ and hence 
    $O_G(x)=O_G(y)$. 
\end{proof}

\begin{proposition}
    Let $C$ be an algebraic closure of $K$ and $x\in C$. Then 
    \[
    f(x,K)=\prod_{y\in O_G(x)}(X-y)^m
    \]
    for some $m$. 
\end{proposition}

\begin{proof}
    For each $y\in O_G(x)$ let $m_y$ be the multiplicity
    of $y$ in $f(x,K)$. 
    Then, for example,
    $f(x,K)=(X-x)^{m_x}g$ for some $g$. If $y\in O_G(x)$, 
    then $y=\sigma(x)$ for some $\sigma\in\Gal(C/K)$. Since
    \[
    \overline{\sigma}(f(x,K))=f(x,K)=(X-y)^{m_x}\overline{\sigma}(g), 
    \]
    it follows that $m_y\geq m_x$. By symmetry, 
    we conclude that $m_x=m_y$. 
\end{proof}

The previous proposition shows, in particular, 
that all the roots of 
an irreducible polynomial $f\in K[X]$ 
in an algebraic closure $C$ of $K$
have the same multiplicity. This is  
not true if $f$ is not irreducible. Find an example.

%Take 
%for example $f=(X-1)^2(X^2+1)\in\Q[X]$. 

% pag 20

\begin{definition}
\index{Decomposition field}
    Let $K$ be a field and $\{f_i:i\in I\}$ be a non-empty 
    family of polynomials of positive degree
    with coefficients in $K$. A \textbf{decomposition field} 
    of $\{f_i:i\in I\}$ is an extension $E/K$
    such that every $f_i$ factorizes linearly in $E[X]$ and 
    if $F/K$ is a sub extension of $E/K$ such that every $f_i$ 
    factorizes linearly in $F[X]$, then $F=E$. 
\end{definition}

\begin{exercise}
    Prove that $E/K$ is a decomposition field of
    $\{f_i:i\in I\}$ if and only if every $f_i$ factorizes linearly 
    in $E[X]$ and $E=K(S)$ where $S=\{\text{roots of $f_i$ for all $i$}\}$. 
\end{exercise}

\begin{exercise}
    Prove that if $E/K$ is a decomposition field
    of $\{f_i:i\in I\}$, then $E/K$ is algebraic. If, moreover, 
    $I$ is finite, then $E/K$ is a decomposition field
    of $\prod_{i\in I}f_i$. 
\end{exercise}

\begin{exercise}
    Prove that 
    there exists a decomposition field of $\{f_i:i\in I\}$ 
    and it is unique up to extension isomorphism. 
\end{exercise}

\begin{exercise}
    Let $f=X^3-X-1\in (\Z/3)[X]$ and $E$ be a decomposition field of $f$. 
    Compute $[E:\Z/3]$. 
\end{exercise}

% it is 3, as E is (\Z/3)[\alpha], where $\alpha$ is a root of $f$ (in some algebraic closure)
What about the decomposition field of $f=X^3-X-1\in \Q[X]$?

\begin{exercise}
    Let $f=X^4-5x^2+5\in \Q[X]$ and $E$ be a decomposition field of $f$. 
    Compute $[E:\Q]$ and $\Gal(E/K)$. 
\end{exercise}

% degree 4, cyclic of order 4


