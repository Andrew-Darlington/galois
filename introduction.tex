\section*{Introduction}

The notes correspond to the bachelor 
course \emph{Galois theory} of the 
Vrije Universiteit Brussel, 
Faculty of Sciences, 
Department of Mathematics and Data Sciences. The course
is divided into twelve two-hour lectures. 

The material is somewhat standard. Basic texts on fields and Galois theory 
are for example \cite{MR1645586} and 
\cite{MR3379917}. 

As usual, we also mention a set of 
\href{https://kconrad.math.uconn.edu/blurbs/}{great expository papers} by 
Keith Conrad, the notes are extremely well-written and useful  
at every stage of a mathematical career. 

% Several chapters contain optional paragraphs that give examples of 
% how to apply \href{https://oscar.computeralgebra.de/}{OSCAR Computer Algebra System}
% to concrete problems in Galois theory. 

The notes include Magma code, which we use to verify examples and offer alternative solutions to certain exercises. Magma is a powerful software tool designed for working with algebraic structures. There is a free \href{https://magma.maths.usyd.edu.au/calc/}{online} version of Magma available.

 
Thanks go to Wouter Appelmans, Andrew Darlington, Luca Descheemaeker, 
Alejandro de la Cueva Merino, 
Wannes Malfait, Manet Michiels, Silvia Properzi, 
Lukas Simons. 

%Arne van Antwerpen, 
%and Geoffrey Jassens. 

This version 
was compiled on \today~at~\currenttime.
% \bigskip
% \begin{flushright}
% Leandro Vendramin\\Brussels, Belgium\par
% \end{flushright}
