\chapter{}

\topic{The fundamental theorem of algebra}

We now present an easy proof of the fundamental theorem 
of algebra based on the ideas of Galois Theory. 
We need the following well-known facts:
\begin{enumerate}
\item Every real polynomial of odd degree admits a real root. This means that $\R$ 
does not admit extension of odd degree $>1$. 
\item Every complex number admits a square root in $\C$. This means that $\C$ 
does not admit degree-two extensions.
\end{enumerate}

\begin{theorem}
The field $\C$ is algebraically closed.
\end{theorem}

\begin{proof}
    Let $E/\C$ be an algebraic finite extension. Then $E/\R$ 
    is finite separable extension of even degree. There exists a Galois
    extension 
    $L/\R$ such that $E\subseteq L$, so $[L:\R]$ is even. Let $G=\Gal(L/\R)$. 
    Then $|G|=2^ms$ for some odd number $s$. If $T$ is a 2-Sylow subgroup
    of $G$, 
    then there exists a subextension $F/\R$ of degree $s$. Since 
    $\R$ does not admit extensions of odd degree $>1$, $s=1$ and
    hence $G$ is a $2$-group. In particular, $|\Gal(L/\C)|=2^{m-1}$. If $m>1$, 
    let $U$ be a subgroup of $\Gal(L/\C)$ of order $2^{m-2}$. Then $U$ corresponds 
    to a subextension $L_1/\C$ of degree two, a contradiction. Hence $m=1$ 
    and $[L:\C]=1$, so $L=\C$ and $E=\C$. 
\end{proof}

\topic{Purely inseparable extensions}

Let $E/K$ be an algebraic extension. 
In page \ref{separable} we defined the 
\textbf{separable closure} of $K$ with respect to $E$ as 
the field 
\[
    F=\{x\in E:x\text{ is separable over }K\}.
\]
Note that $K\subseteq F\subseteq E$ 
and $F=K(F)$. Moreover, 
$F/K$ is separable and 
$E/F$ is a \textbf{purely inseparable} extension, meaning that
for every $x\in E\setminus F$, the polynomial $f(x,F)$ is not separable. 

The number $[E:F]$ is known as the \textbf{degree of inseparability} of $E/K$. 
We write $[E:K]_{\operatorname{ins}}=[E:F]$.
Clearly, $E/K$ is separable if and only if $[E:K]_{\operatorname{ins}}=1$ and 
$E/K$ is purely inseparable if and only if $[E:K]_{\operatorname{ins}}=[E:K]$. 

\begin{proposition}
Let $K$ be a field of characteristic $p>0$ and
$E/K$ be an algebraic extension. The following statements are equivalent:
\begin{enumerate}
    \item $E/K$ is purely inseparable.
    \item If $x\in E$, then $x^{p^m}\in K$ for some $m\geq0$.
    \item If $x\in E$, then $f(x,K)=X^{p^m}-a$ for some $a\in K$ and $m\geq0$. 
    \item $\gamma(E/K)=1$. 
\end{enumerate}
\end{proposition}

\begin{proof}
    We first prove $1)\implies 2)$. We proceed by induction on the degree of $x$. The result
    is clearly true for elements of degree one. So assume the result holds for element of degree $\leq n$ 
    for some $n\geq1$. 
    If $x\in E$ is such that $\deg(x,K)=n+1$, then, since $f(x,K)=g(X^p)$, the element 
    $x^p$ has degree $\leq n$. By the inductive hypothesis, $x^{p^{m+1}}=(x^p)^{p^m}\in K$.  
    
    We now prove $2)\implies 3)$. Let $x\in E$ and $m$ be the minimal positive integer 
    such that $x^{p^m}\in K$. Then
    $x$ is a root of $X^{p^m}-x^{p^m}\in K[X]$. Since 
    $X^{p^m}-x^{p^m}=(X-x)^{p^m}$, it follows that $f(x,K)=(X-x)^r$ for some
    $r\in\{1,\dots,p^m\}$. Write $r=p^st$ for some integer $t$ coprime with $p$ and $s$ such that
    $0\leq s\leq m$. Let $a,b\in\Z$ be such that $ar+bp^m=p^s$. Then 
    \[
    x^{p^s}=x^{ar+bp^m}=\left(x^r\right)^a\left(x^{p^m}\right)^b\in K.
    \]
    The minimality of $m$ implies that $s\geq m$ and hence $s=m$. Now $p^mt=p^st=r\leq p^m$, so $t=1$. 
    This means $f(x,K)=X^{p^m}-x^{p^m}$. 
    
    We now prove $3)\implies 4)$. Let $C/K$ be an algebraic closure that contains $E$
    and $\sigma\in\Hom(E/K,C/K)$. Let $x\in E$. We claim that $\sigma(x)=x$. Since 
    $f(x,K)=X^{p^m}-a$, 
    \[
    \left(\sigma(x)\right)^{p^m}=\sigma\left(x^{p^m}\right)=\sigma(a)=a=x^{p^m}.
    \]
    It follows that $\sigma(x)=x$. 
    
    Finally, we prove that $4)\implies1)$. If $x\in E$ is separable over $K$, then, 
    as $\sigma(x)=x$ for all $\sigma\in\Gal(E/K)$, 
    \[
    f(x,K)=\prod_{y\in O_{\Gal(E/K)}(x)}(X-y)=X-x\in\ K[X].
    \]
    Thus $x\in K$ and hence $E/K$ is purely inseparable. 
\end{proof}

Some consequences:

\begin{exercise}
    Let $E/K$ be finite and purely inseparable. Then $[E:K]=p^s$ for some prime number $p$ and some $s$.
    Moreover, $x^{[E:K]}\in K$. 
\end{exercise}

For the first part of previous exercise write $E=K(x_1,\dots,x_n)$ and proceed by induction on $n$. 

\begin{exercise}
    Let $K$ be of characteristic $p>0$ and 
    $E/K$ be a finite extension such that $[E:K]$ is not divisible by $p$. Then 
    $E/K$ is separable. 
\end{exercise}

Let $E/K$ be finite and $F$ be the separable closure of $K$ in $E$. 
Since 
\begin{gather*}
\gamma(E/K)=\gamma(E/F)\gamma(F/K)=\gamma(F/K),
\shortintertext{it follows that}
[E:K]=[E:F]\gamma(E/K)=[E:K]_{\operatorname{ins}}\gamma(E/K).
\end{gather*}

\topic{Norm and trace}

\begin{definition}
\index{Trace}
\index{Norm}
    Let $E/K$ be a finite extension and $C/K$ be an algebraic closure 
    that contains $E$. Let $A=\Hom(E/K,C/K)$. For $x\in E$
    we define the \textbf{trace} of $x$ in $E/K$ 
    as 
    \[
    \trace_{E/K}(x)=[E:K]_{\operatorname{ins}}\sum_{\varphi\in A}\varphi(x)
    \]
    and the \textbf{norm} of $x$ in $E/K$ as
    \[
    \norm_{E/K}(x)=\left(\prod_{\varphi\in A}\varphi(x)\right)^{[E:K]_{\operatorname{ins}}}.
    \]
\end{definition}

As an optional exercise, one can show that these definitions do not depend on the algebraic closure. 

We collect some basic properties as an exercise:

\begin{exercise}
    Let $E/K$ be a finite extension. The following statements hold:
    \begin{enumerate}
        \item If $E/K$ is not separable, then $\trace_{E/K}(x)=0$ for all $x\in E$.
        \item If $x\in K$, then $\trace_{E/K}(x)=[E:K]x$.
        \item $\trace_{E/K}(x)\in K$ for all $x\in E$.
        \item $\norm_{E/K}(x)=0$ if and only if $x=0$. 
        \item If $x\in K$, then $\norm_{E/K}(x)=x^{[E:K]}$. 
        \item $\norm_{E/K}(x)\in K$ for all $x\in E$. 
    \end{enumerate}
\end{exercise}

One proves, moreover, that both 
$\trace_{E/K}\colon E\to K$ and $\norm_{E/K}\colon E\to K$ 
are linear forms in $E$. 

\begin{exercise}
        Let $E/K$ be a finite extension and
        $x\in E$. If
        \[
        f(x,K)=X^n+a_{n-1}X^{n-1}+\cdots+a_1X+a_0,
        \]
        then 
        $\norm_{E/K}(x)=\left((-1)^na_0\right)^{[E:K(x)]}$ and 
        $\trace_{E/K}(x)=-[E:K(x)]a_{n-1}$. 
\end{exercise}

\begin{example}
    Let $E=\Q(\sqrt{2},\sqrt{3})$. Then 
    \begin{align*}
    &\trace_{E/\Q}(\sqrt{2})=0,
    &&
    \norm_{E/\Q}(\sqrt{2})=4,\\
    &\trace_{E/\Q(\sqrt{2})}(\sqrt{2})=2\sqrt{2},
    &&\norm_{E/\Q(\sqrt{2})}(\sqrt{2})=2.    
    \end{align*}
\end{example}

\begin{example}
    If $E/K$ is a finite Galois extension, then 
    \[
    \trace_{E/K}(x)=\sum_{\sigma\in\Gal(E/K)}\sigma(x)
    \quad\text{and}\quad
    \trace_{E/K}(x)=\prod_{\sigma\in\Gal(E/K)}\sigma(x)
    \]
    for all $x\in E$. In particular, since $E=K(y)$ for some
    $y$ by Proposition \ref{pro:monogenic}, 
    \[
    \trace_{E/K}(y)=-a_{n-1}
    \quad\text{and}\quad
    \norm_{E/K}(y)=(-1)^na_0,
    \]
    where
    $f(y,K)=X^n+a_{n-1}X^{n-1}+\cdots+a_1X+a_0$.
\end{example}        

\topic{Finite fields}

In this section, $p$ will be a prime number. 

\begin{proposition}
    Let $m$ be a positive integer. 
    Up to isomorphism, there exists a unique 
    field $F_m$ of size $p^m$. 
\end{proposition}

\begin{proof}
    Let $C$ be an algebraic closure of the field $\Z/p$ and 
    let $F_m=\{x\in C:x^{p^m}=x\}$ be the set of roots of $X^{p^m}-X$. Since 
    the polynomial $X^{p^m}-X$ has no multiple roots, $|F_m|=p^m$. Moreover, 
    $F_m$ is the unique subfield of $C$ of size $p^m$. 
    
    To prove the uniqueness it is enough to note that if $K$ is a field of $p^m$ elements, then
    $K$ is the decomposition field of $X^{p^m}-X$ over $\Z/p$.  
\end{proof}

Let $K=\Z/p$ and $C$ be an algebraic closure of $K$. 
We claim that $C=\cup_k F_k$. If $x\in C$, then $x$ is algebraic over $K$. 
Since $K(x)/K$ is finite, $K(x)$ is a finite field, say 
$|K|=p^r$ for some $r$. Then $x^{p^r}=x$ and hence $x\in F_r$. 

\begin{exercise}
    Prove the following statements:
    \begin{enumerate}
        \item If $x\in F_r$, then $x^{p^{rk}}=x$ for all $k\geq0$.
        \item If $m\mid n$, then $F_m\subseteq F_n$. 
        \item $F_m\cap F_n=F_{\gcd(m,n)}$.
        \item $F_m\subseteq F_n$ if and only if $m\mid n$. 
    \end{enumerate}
\end{exercise}

\begin{proposition}
    Every finite extension of a finite field is cyclic. 
\end{proposition}

\begin{proof}
    Let $K=\Z/p$. It is enough to show that $F_n/F_m$ is cyclic if $m$ divides $n$. 
    
    We first prove that $F_n/K$ is cyclic. Let $\sigma\in\Gal(F_n/K)$ be given by $\sigma(x)=x^p$. 
    Note that $|\Gal(F_n/K)|=[F_n:K]=n$. Since 
    $\sigma^i(x)=x^{p^i}$ for all $i\geq 0$, in particular, 
    $\sigma^n(x)=x^{p^n}=x$. Thus $\sigma^n=\id$ and hence $|\sigma|$ divides $n$. Let 
    $s=|\sigma|$. We know that $F_n^{\times}=F_n\setminus\{0\}$ is
    cyclic, say $F_n^{\times}=\langle g\rangle$. Since $|g|=p^n-1$, 
    \[
    g=\sigma^s(g)=g^{p^s}
    \]
    and hence $p^s\equiv 1\bmod (p^n-1)$. Thus $p^n-1$ divides $p^s-1$ and
    hence $n$ divides $s$. Therefore $n=s$ and $\Gal(F_n/K)=\langle\sigma\rangle$. 
    
    For the general case note thast If $m$ divides $n$, 
    then $\Gal(F_n/F_m)$ is a subgroup of $\Gal(F_n/K)$. Since  $\Gal(F_n/K)$ is cyclic, 
    the claim follows.
\end{proof}

\index{Frobenius automorphism}
If $K=\Z/p$ and 
$m$ divides $n$, the subextension $F_m$ corresponds 
to the unique
subgroup of index $m$ of $\Gal(F_n/K)=\langle\sigma\rangle$. This subgroup
is $\langle\sigma^m\rangle$, where
\[
\sigma^m(x)=x^{p^m}=x^{|F_m|}.
\]
Note that $\Gal(F_n/F_m)=\langle\sigma^m\rangle$. 
The map $\sigma^m$ is known as 
the \textbf{Frobenius automorphism}. 

\begin{exercise}
    Let $E/K$ be an extension of finite fields . Then $E/K$ 
    is cyclic and $\Gal(E/K)=\langle\tau\rangle$, where $\tau(x)=x^{|K|}$. 
\end{exercise}

% page 96
% number of irreducible polynomials
% Moebius inversion formula in commutative rings

\topic{Cyclotomic extensions}

For $n\geq1$ let $G_n(K)=\{x\in K:x^n=1\}$ be the 
set of $n$-roots of one in $K$. Note that
$G_n(K)$ is a cyclic subgroup of $K^{\times}$ and that 
$|G_n(K)|$ divides $n$. 

\begin{example}
    $G_n(\R)=\{-1,1\}$ if $n$ is odd and $G_{n}=\{1\}$ is $n$ is even.
\end{example}

\begin{exercise}
    Let $K$ be a field of characteristic $p>0$. Let $n=p^sm$ for some $m$ not divisible by $p$. 
    Then $G_n(K)=G_m(K)$. 
\end{exercise}

\begin{exercise}
    Let $q$ be a prime number. Then $G_n(\Z/q)\simeq\Z/\gcd(n,q-1)$. 
\end{exercise}

Similarly, one can prove that if $K$ is a finite field, then $G_n(K)$ is a cyclic group
of order $\gcd(n,|K^{\times}|)$. 

\begin{example}
    If $C$ is algebraically closed of characteristic coprime with $n$, 
    then $G_n(C)$ is cyclic of order $n$, as $X^n-1$ 
    has all his roots in $C$ and does not contain multiple roots. 
\end{example}

Let $K$ be an algebraically closed field and $n$ be
such that $n$ is coprime with the characteristic of $K$. The set of 
\textbf{primitive $n$-roots} is defined as 
\[
H_n(K)=\{x\in G_n(K):|x|=n\}.
\]

\begin{definition}
\index{Cyclotomic polynomial}
    Let $K$ be an algebraically closed field and $n$ be
    such that $n$ is coprime with the characteristic of $K$. The \textbf{$n$-th cyclotomic
    polynomial} is defined as 
    \[
    \Phi_n=\prod_{x\in H_n(K)}(X-x)\in K[X].
    \]
\end{definition}

\index{Euler's $\phi$ function}
For $n\geq1$ the Euler's function is defined as 
\[
\varphi(n)=|\{k:1\leq k\leq n,\;\gcd(k,n)=1\}|.
\]
For example, $\varphi(4)=2$, $\varphi(8)=\varphi(10)=4$ and $\varphi(p)=p-1$ for every prime $p$. 

\begin{proposition}
    Let $K$ be an algebraically closed field and $n$ be
    such that $n$ is coprime with the characteristic of $K$. Let $A$ be
    the ring of integers of $K$. 
    \begin{enumerate}
        \item $\deg\Phi_n=\varphi(n)$.
        \item $\Phi_n\in A[X]$.
    \end{enumerate}
\end{proposition}

\begin{proof}
    The first statement is clear. Let us prove 2) by induction on $n$. The case $n=1$ is
    trivial, as $\Phi_1=X-1$. Assume that $\Phi_d\in A[X]$ for all $d$ such that $d<n$. 
    In particular,
    \[
    \gamma=\prod_{\substack{d\mid n\\d\ne n}}\Phi_d\in A[X].
    \]
    Since $\gamma$ is monic, it follows that 
    $\frac{X^n-1}{\gamma}\in A[X]$. Now the claim follows from 
    \[
    X^n-1=\prod_{d\mid n}\Phi_d=\Phi_n\prod_{\substack{d\mid n\\d\ne n}}\Phi_d=\Phi_n\gamma.\qedhere
    \]
\end{proof}

By taking degree in the equality 
$X^n-1=\prod_{d\mid n}\Phi_d$ 
one gets 
\[
n=\sum_{d\mid n}\varphi(d).
\]

\begin{definition}
\label{defn:cyclotomic}
\index{Extension!cyclotomic}
    Let $n\geq2$ and $K$ be a field of characteristic coprime with $n$. A 
    \textbf{cyclotomic extension} of $K$ of index $n$ is a 
    decomposition field of $X^n-1$ over $K$. 
\end{definition}

Let $C$ be an algebraic closure of $K$ and $n\geq2$ be coprime with the characteristic of $K$. 
If follows from Definition \ref{defn:cyclotomic} 
that a cyclotomic extension of index $n$ is of the form 
$K(\omega)/K$ for some $\omega\in H_n(K)$. 

\begin{proposition}
    A cyclotomic extension of index $n$ is abelian and of degree a divisor of $\varphi(n)$. 
\end{proposition}

\begin{proof}
    Let $C$ be an algebraic closure of $K$ and $n\geq2$ be coprime with the characteristic of $K$. 
    Let $\omega\in H_n(C)$ and $K(\omega)/K$ be a cyclotomic extension. Then $K(\omega)/K$
    is a Galois extension, as it is a decomposition field of a separable polynomial. 
    Let $U=\mathcal{U}(\Z/n)$ be the group of units of $\Z/n$ and 
    \[
    \lambda\colon \Gal(K(\omega)/K)\to U,
    \quad
    \sigma\mapsto m_{\sigma},
    \]
    where $m_{\sigma}$ is such that $\sigma(\omega)=\omega^{m_{\sigma}}$. The map $\lambda$ is well-defined and
    it is a group homomorphism, as if $\sigma,\tau\in\Gal(K(\omega)/K)$, then, since 
    \[
        (\tau\sigma)(\omega)=\tau(\sigma(\omega))=\tau(\omega^{m_\sigma})=\left(\omega^{m_\sigma}\right)^{m_\tau}=\omega^{m_\sigma m_\tau},
    \]
    it follows that $\lambda(\sigma)\lambda(\tau)=\lambda(\sigma\tau)$. Since 
    $\lambda$ is injective, $\Gal(K(\omega)/K)$ is isomorphic to a subgroup 
    of the abelian group $U$. Hence $\Gal(K(\omega)/K)$ is abelian. Moreover, 
    $[K(\omega):K]=|\Gal(K(\omega)/K)|$ is a divisor of $|U|=\varphi(n)$. 
\end{proof}

\begin{exercise}
    Prove that a cyclotomic extension $K(\omega)/K$ has degree $\varphi(n)$ if and only if 
    $\Phi_n$ is irreducible over $K$. 
\end{exercise}


\begin{example}

\end{example}

\topic{Hilbert's theorem}
