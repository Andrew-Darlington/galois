\chapter{}

\topic{Finite fields}

In this section, $p$ will be a prime number. 

\begin{proposition}
    Let $m$ be a positive integer. 
    Up to isomorphism, there exists a unique 
    field $F_m$ of size $p^m$. 
\end{proposition}

\begin{proof}
    Let $C$ be an algebraic closure of the field $\Z/p$ and 
    let $F_m=\{x\in C:x^{p^m}=x\}$ be the set of roots of $X^{p^m}-X$. Since 
    the polynomial $X^{p^m}-X$ has no multiple roots, $|F_m|=p^m$. Moreover, 
    $F_m$ is the unique subfield of $C$ of size $p^m$. 
    
    To prove the uniqueness, it is enough to note that 
    if $K$ is a field of $p^m$ elements, then
    $K$ is the decomposition field of $X^{p^m}-X$ over $\Z/p$.  
\end{proof}

Let $K=\Z/p$ and $C$ be an algebraic closure of $K$. 
We claim that $C=\cup_k F_k$. If $x\in C$, then $x$ is algebraic over $K$. 
Since $K(x)/K$ is finite, $K(x)$ is a finite field, say 
$|K|=p^r$ for some $r$. Then $x^{p^r}=x$ and hence $x\in F_r$. 

\begin{exercise}
    Prove the following statements:
    \begin{enumerate}
        \item If $x\in F_r$, then $x^{p^{rk}}=x$ for all $k\geq0$.
        \item If $m\mid n$, then $F_m\subseteq F_n$. 
        \item $F_m\cap F_n=F_{\gcd(m,n)}$.
        \item $F_m\subseteq F_n$ if and only if $m\mid n$. 
    \end{enumerate}
\end{exercise}

\begin{proposition}
    Every finite extension of a finite field is cyclic. 
\end{proposition}

\begin{proof}
    Let $K=\Z/p$. It is enough to show that $F_n/F_m$ is cyclic if $m$ divides $n$. 
    
    We first prove that $F_n/K$ is cyclic. 
    Let 
    \[
    \sigma\colon F_n\to F_n,\quad 
    x\mapsto x^p.
    \]
    Then 
    $\sigma\in\Gal(F_n/K)$ (it is bijective because all field homomorphisms 
    are injective and $F_n$ is finite). 

    Note that 
    $F_n/K$ is a Galois extension, as $F_n$ is the splitting
    field over $K$ 
    of the separable polynomial $X^{p^n}-X\in K[X]$. 
    Thus $|\Gal(F_n/K)|=[F_n:K]=n$. 
    
    We claim that $\sigma$ generated $\Gal(F_n/K)$. Since 
    $\sigma^i(x)=x^{p^i}$ for all $i\geq 0$, in particular, 
    $\sigma^n(x)=x^{p^n}=x$. Thus $\sigma^n=\id$ and hence $|\sigma|$ divides $n$. Let 
    $s=|\sigma|$. We know that $F_n^{\times}=F_n\setminus\{0\}$ is
    cyclic, say $F_n^{\times}=\langle g\rangle$. Since $|g|=p^n-1$, 
    \[
    g=\sigma^s(g)=g^{p^s}
    \]
    and hence $p^s\equiv 1\bmod (p^n-1)$. Thus $p^n-1$ divides $p^s-1$ and
    hence $n$ divides $s$. Therefore $n=s$ and $\Gal(F_n/K)=\langle\sigma\rangle$. 
    
    For the general case note that if $m$ divides $n$, 
    then $\Gal(F_n/F_m)$ is a subgroup of $\Gal(F_n/K)$. Since  $\Gal(F_n/K)$ is cyclic, 
    the claim follows.
\end{proof}

\index{Frobenius automorphism}
If $K=\Z/p$ and 
$m$ divides $n$, the subextension $F_m$ corresponds 
to the unique
subgroup of index $m$ of $\Gal(F_n/K)=\langle\sigma\rangle$. This subgroup
is $\langle\sigma^m\rangle$, where
\[
\sigma^m(x)=x^{p^m}=x^{|F_m|}.
\]
Note that $\Gal(F_n/F_m)=\langle\sigma^m\rangle$. 
The map $\sigma^m$ is known as 
the \textbf{Frobenius automorphism}. 

\begin{exercise}
    Let $E/K$ be an extension of finite fields. Then $E/K$ 
    is cyclic and $\Gal(E/K)=\langle\tau\rangle$, where $\tau(x)=x^{|K|}$. 
\end{exercise}

% page 96
% number of irreducible polynomials
% Moebius inversion formula in commutative rings



\topic{Cyclotomic extensions}

For $n\geq1$ let $G_n(K)=\{x\in K:x^n=1\}$ be the 
set of $n$-roots of one in $K$. Note that
$G_n(K)$ is a cyclic subgroup of $K^{\times}$ and that 
$|G_n(K)|$ divides $n$. 

\begin{example}
    $G_n(\R)=\{-1,1\}$ if $n$ is odd and $G_{n}=\{1\}$ if $n$ is even.
\end{example}

\begin{exercise}
    Let $K$ be a field of characteristic $p>0$. Let $n=p^sm$ for some $m$ not divisible by $p$. 
    Then $G_n(K)=G_m(K)$. 
\end{exercise}

\begin{exercise}
    Let $q$ be a prime number. Then $G_n(\Z/q)\simeq\Z/\gcd(n,q-1)$. 
\end{exercise}

Similarly, one can prove that if $K$ is a finite field, then $G_n(K)$ is a cyclic group
of order $\gcd(n,|K^{\times}|)$. 

\begin{example}
    If $C$ is algebraically closed of characteristic coprime with $n$, 
    then $G_n(C)$ is cyclic of order $n$, as $X^n-1$ 
    has all its roots in $C$ and does not contain multiple roots. 
\end{example}

Let $K$ be an algebraically closed field and $n$ be
such that $n$ is coprime with the characteristic of $K$. The set of 
\textbf{primitive $n$-roots} is defined as 
\[
H_n(K)=\{x\in G_n(K):|x|=n\}.
\]

\begin{definition}
\index{Cyclotomic polynomial}
    Let $K$ be an algebraically closed field and $n$ be
    such that $n$ is coprime with the characteristic of $K$. The \textbf{$n$-th cyclotomic
    polynomial} is defined as 
    \[
    \Phi_n=\prod_{x\in H_n(K)}(X-x)\in K[X].
    \]
\end{definition}

\index{Euler's $\phi$ function}
For $n\geq1$ the Euler's function is defined as 
\[
\varphi(n)=|\{k:1\leq k\leq n,\;\gcd(k,n)=1\}|.
\]
For example, $\varphi(4)=2$, $\varphi(8)=\varphi(10)=4$ and $\varphi(p)=p-1$ for every prime $p$. 

\begin{proposition}
    Let $K$ be an algebraically closed field and $n$ be
    such that $n$ is coprime with the characteristic of $K$. Let $A$ be
    the ring of integers of $K$. 
    \begin{enumerate}
        \item $\deg\Phi_n=\varphi(n)$.
        \item $\Phi_n\in A[X]$.
    \end{enumerate}
\end{proposition}

\begin{proof}
    The first statement is clear. Let us prove 2) by induction on $n$. The case $n=1$ is
    trivial, as $\Phi_1=X-1$. Assume that $\Phi_d\in A[X]$ for all $d$ such that $d<n$. 
    In particular,
    \[
    \gamma=\prod_{\substack{d\mid n\\d\ne n}}\Phi_d\in A[X].
    \]
    Since $\gamma$ is monic, it follows that 
    $\frac{X^n-1}{\gamma}\in A[X]$. Now the claim follows from 
    \[
    X^n-1=\prod_{d\mid n}\Phi_d=\Phi_n\prod_{\substack{d\mid n\\d\ne n}}\Phi_d=\Phi_n\gamma.\qedhere
    \]
\end{proof}

By taking degree in the equality 
$X^n-1=\prod_{d\mid n}\Phi_d$ 
one gets 
\[
n=\sum_{d\mid n}\varphi(d).
\]

\begin{definition}
\label{defn:cyclotomic}
\index{Extension!cyclotomic}
    Let $n\geq2$ and $K$ be a field of characteristic coprime with $n$. A 
    \textbf{cyclotomic extension} of $K$ of index $n$ is a 
    decomposition field of $X^n-1$ over $K$. 
\end{definition}

Let $C$ be an algebraic closure of $K$ and $n\geq2$ be coprime with the characteristic of $K$. 
If follows from Definition \ref{defn:cyclotomic} 
that a cyclotomic extension of index $n$ is of the form 
$K(\omega)/K$ for some $\omega\in H_n(K)$. 

\begin{proposition}
    A cyclotomic extension of index $n$ is abelian and of degree a divisor of $\varphi(n)$. 
\end{proposition}

\begin{proof}
    Let $C$ be an algebraic closure of $K$ and $n\geq2$ be coprime with the characteristic of $K$. 
    Let $\omega\in H_n(C)$ and $K(\omega)/K$ be a cyclotomic extension. Then $K(\omega)/K$
    is a Galois extension, as it is a decomposition field of a separable polynomial. 
    Let $U=\mathcal{U}(\Z/n)$ be the group of units of $\Z/n$ and 
    \[
    \lambda\colon \Gal(K(\omega)/K)\to U,
    \quad
    \sigma\mapsto m_{\sigma},
    \]
    where $m_{\sigma}$ is such that $\sigma(\omega)=\omega^{m_{\sigma}}$. The map $\lambda$ is well-defined and
    it is a group homomorphism, as if $\sigma,\tau\in\Gal(K(\omega)/K)$, then, since 
    \[
        (\tau\sigma)(\omega)=\tau(\sigma(\omega))=\tau(\omega^{m_\sigma})=\left(\omega^{m_\sigma}\right)^{m_\tau}=\omega^{m_\sigma m_\tau},
    \]
    it follows that $\lambda(\sigma)\lambda(\tau)=\lambda(\sigma\tau)$. Since 
    $\lambda$ is injective, $\Gal(K(\omega)/K)$ is isomorphic to a subgroup 
    of the abelian group $U$. Hence $\Gal(K(\omega)/K)$ is abelian. Moreover, 
    $[K(\omega):K]=|\Gal(K(\omega)/K)|$ is a divisor of $|U|=\varphi(n)$. 
\end{proof}

\begin{exercise}
    Prove that a cyclotomic extension $K(\omega)/K$ has degree $\varphi(n)$ if and only if 
    $\Phi_n$ is irreducible over $K$. 
\end{exercise}

Note that $\Phi_n$ is irreducible over $\Q$. Some concrete examples:
\[
\Phi_1=X-1,
\quad
\Phi_2=X+1,
\quad
\Phi_3=X^2+X+1,
\quad
\Phi_6=X^2-X+1.
\]
If $p$ is a prime number, then $\Phi_p=X^{p-1}+\cdots+X+1$. 

\begin{example}
    $\Phi_5$ is irreducible over $\Z/2$. First note that
    $\Phi_5=X^{4}+\cdots+X+1$ does not have roots in $\Z/2$. If 
    $\Phi_5$ is reducible, then, since
    $X^2+X+1$ is the unique degree-two 
    monic irreducible polynomial 
    over $\Z/2$, it follows that
    \[
    \Phi_5=(X^2+X+1)(X^2+X+1)=(X^2+X+1)^2=X^4+X^2+1,
    \]
    a contradiction.
\end{example}

\begin{exercise}
Prove that
$\Phi_{12}=X^4-X^2+1$ is not irreducible over $\Z/5$. 
\end{exercise}

\topic{Hilbert's theorem 90}

Let $G$ be a group and $A$ be a (left) $G$-module. This means that $A$ is an abelian
group and there exists a map
\[
G\times A\to A,\quad
(g,a)\mapsto g\cdot a
\]
such that $1\cdot a=a$ for all $a\in A$, $(gh)\cdot a=g\cdot (h\cdot a)$ for 
all $g,h\in G$ and $a\in A$ and $g\cdot (a+b)=g\cdot a+g\cdot b$ for
all $g\in G$ and $a,b\in A$. 

\begin{definition}
    \index{Derivation}
    Let $A$ be a $G$-module. 
    A \textbf{derivation} of $A$ is a map $d\colon G\to A$ such that
    $d(gh)=g\cdot d(h)+d(g)$ for all $g,h\in G$. 
\end{definition}

\index{Inner derivation}
Let $A$ be a $G$-module and $a\in A$. The map 
$d(g)=g\cdot a-a$ is a derivation of $A$. 
Such derivations as known as \textbf{inner derivations}. 

\begin{exercise}
    Let $E/K$ be a finite Galois extension and $G=\Gal(E/K)$. 
    Then the (multiplicative) group $E^\times$ is a $G$-module with
    $\sigma\cdot x=\sigma(x)$. 
\end{exercise}

In the context of the previous exercise, 
let $d\colon G\to E^\times$ be a derivation. For $\tau\in G$ let 
$x_\tau=d(\tau)$. Then
\[
x_{\sigma\tau}=d(\sigma\tau)=(\sigma\cdot d(\tau))d(\sigma)=\sigma(x_\tau)x_\sigma.
\]
Is this true that $x_\sigma=\frac{\sigma(c)}{c}=(\sigma\cdot c)c^{-1}$ 
for some $c\in E^\times$?

\begin{proposition}
    Let $E/K$ be a finite Galois extension and $G=\Gal(E/K)$. 
    \begin{enumerate}
        \item Let $\{x_\tau:\tau\in G\}\subseteq E^\times$ be such that 
            $x_{\sigma\tau}=\sigma(x_\tau)x_\sigma$. Then there exists
            $c\in E^\times$ such that $x_\sigma=\frac{\sigma(c)}{c}$ for all $\sigma\in G$. 
        \item Let $\{x_\tau:\tau\in G\}\subseteq E$ be such that 
            $x_{\sigma\tau}=\sigma(x_\tau)+x_\sigma$. Then there exists
            $c\in E$ such that $x_\sigma=\sigma(c)-c$ for all $\sigma\in G$.  
    \end{enumerate}
\end{proposition}

\begin{proof}
    We prove 1). By Dedekind's theorem, the homomorphism 
    $\sum_{\tau\in G}x_{\tau}^{-1}\tau$ is non-zero. Thus there exists
    $a\in E^{\times}$ such that $\sum_{\tau\in G}x_{\tau}^{-1}\tau(a)=c\in E^{\times}$. 
    If $\sigma\in G$, 
    then 
    \[
    \sigma(c)=\sum_{\tau\in G}\sigma(x_{\tau}^{-1})(\sigma\tau)(a)=
    \sum_{\tau\in G}x_{\sigma\tau}^{-1}x_{\tau}\sigma\tau(a)
    =x_{\sigma}\sum_{\tau\in G}x_{\sigma\tau}^{-1}\sigma\tau(a)=x_{\sigma}c.
    \]
    
    We now prove 2). By Dedekind's theorem, $\sum_{\tau\in G}\tau\ne0$. Since
    it is a non-zero 
    linear form in $E$, it is surjective. In particular, there exists 
    $a\in E$ such that $\sum_{\tau\in G}\tau(a)=1$. Let $c=-\sum_{\tau\in G}x_\tau\tau(a)$. 
    If $\sigma\in G$, then 
    \begin{align*}
        \sigma(c) &= -\sum_{\tau\in G}\sigma(x_{\tau})\sigma\tau(a)\\
        &=-\sum_{\tau\in G}(x_{\sigma\tau}-x_{\sigma})\sigma\tau(a)\\
        &=-\sum_{\tau\in G}x_{\sigma\tau}\sigma\tau(a)+\sum_{\tau\in G}x_{\sigma}\sigma\tau(a)
        =c+x_{\sigma}.
    \end{align*}
    Hence $x_{\sigma}=\sigma(c)-c$. 
\end{proof}

The following result is known as Hilbert's 90 theorem. 

\begin{theorem}[Hilbert's theorem 90]
\index{Hilbert's theorem 90}
    Let $E/K$ be a cyclic extension of degree $n$ with Galois group $G=\Gal(E/K)=\langle\sigma\rangle$. 
    \begin{enumerate}
        \item Let $x\in E^{\times}$ be such that $\norm_{E/K}(x)=1$, 
        There exists $b\in E^{\times}$ such that $x=\sigma(b)/b$. 
        \item Let  $x\in E$ be such that $\trace_{E/K}(x)=0$. 
        There exists $b\in E$ such that $x=\sigma(b)-b$. 
    \end{enumerate}
\end{theorem}

\begin{proof}
    Let us prove 1). Note that $G=\{\sigma^i:0\leq i<n\}$. For $i\in\{0,\dots,n-1\}$ let 
    $x_{\sigma^i}=\prod_{k=0}^{i-1}\sigma^k(x)$. In particular, $x_{\sigma}=x$. We now
    check that $\{x_{\sigma^i}\}$ satisfy the assumptions of the previous proposition:
    \begin{align*}
        \sigma^j(x_{\sigma^i}) &= \prod_{k=0}^{i-1}\sigma^{k+j}(x)
        =\prod_{k=j}^{i+j-1}\sigma^k(x)\\
        &=\prod_{k=0}^{i+j-1}\sigma^k(x)\left(\prod_{k=0}^{j-1}\sigma^k(x)\right)^{-1}
        =\prod_{k=0}^{i+j-1}\sigma^k(x)x_{\sigma^j}^{-1}.
    \end{align*}
    If $i+j<n$, then 
    \[
    \sigma^j\sigma^i=\sigma^{i+j}
    \implies
    \sigma^j(x_{\sigma^i})=x_{\sigma^i\sigma^j}x_{\sigma^j}^{-1}
    \implies
    x_{\sigma^j\sigma^i}=\sigma^j(x_{\sigma^i})x_{\sigma^j}.
    \]
    If $i+j=n+r$ for some $r\in\{0,\dots,n-1\}$, then, since $i+j<2n$ and
    $\sigma^i\sigma^j=\sigma^r$, 
    \[
    \sigma^j(x_{\sigma^i})=\prod_{k=0}^{n-1}\sigma^k(x)\prod_{k=n}^{n+r-1}\sigma^k(x)x_{\sigma^j}^{-1}
    =\prod_{k=0}^{r-1}\sigma^k(x)x_{\sigma^j}^{-1}
    =x_{\sigma^r}x_{\sigma^j}^{-1}=x_{\sigma^j\sigma^i}x_{\sigma^j}^{-1}
    \]
    and hence $x_{\sigma^j\sigma^i}=\sigma^j(x_{\sigma^i})x_{\sigma^j}$. By 
    the previous lemma, there exists $c\in E^{\times}$ such that 
    $x_{\sigma^i}=\frac{\sigma^i(c)}{c}$. In particular, $x=x_{\sigma}=\sigma(c)/c$. 
    
    The second statement is similar and it is left as an exercise. 
\end{proof}

If $A$, $B$ and $C$ are groups (written multiplicatively) 
and $f\colon A\to B$ and $g\colon B\to C$ are group homomorphism, 
the sequence 
\[\begin{tikzcd}
	A & B & C 
	\arrow["f", from=1-1, to=1-2]
	\arrow["g", from=1-2, to=1-3]
\end{tikzcd}\]
of groups and homomorphisms 
is said to be \textbf{exact} if $f(A)=\ker g$. 
For example, the sequence 
\[\begin{tikzcd}
	1 & B & C 
	\arrow[from=1-1, to=1-2]
	\arrow["g", from=1-2, to=1-3]
\end{tikzcd}\]
is exact if and only if $g$ is injective, as the first map represents the trivial homomorphism. 
Similarly, the sequence 
\[\begin{tikzcd}
	A & B & 1
	\arrow["f", from=1-1, to=1-2]
	\arrow[from=1-2, to=1-3]
\end{tikzcd}\]
is exact if and only if $f$ is surjective. 

\begin{corollary}
Let $E/K$ be finite and cyclic with $\Gal(E/K)=\langle\sigma\rangle$. 
\begin{enumerate}
    \item The sequence 
    \[\begin{tikzcd}
	1 & {K^{\times}} & {E^{\times}} & {E^{\times}} & {K^{\times}}
	\arrow[from=1-1, to=1-2]
	\arrow[hook, from=1-2, to=1-3]
	\arrow["\rho", from=1-3, to=1-4]
	\arrow["{\norm_{E/K}}", from=1-4, to=1-5]
    \end{tikzcd}\]
    is exact, where $\rho(z)=\sigma(z)/z$. 
     \item The sequence 
    \[\begin{tikzcd}
	0 & {K} & {E} & {E} & {K}
	\arrow[from=1-1, to=1-2]
	\arrow[hook, from=1-2, to=1-3]
	\arrow["\lambda", from=1-3, to=1-4]
	\arrow["{\trace{E/K}}", from=1-4, to=1-5]
    \end{tikzcd}\]
    is exact, where $\lambda(z)=\sigma(z)-z$. 
\end{enumerate}
\end{corollary}

\begin{proof}
    We only prove 1). Note that 
    $K^{\times}\hookrightarrow E^{\times}$ is the inclusion map. If 
    $z\in\ker\rho$, then $\sigma(z)=z$ and hence $z\in K$. The 
    sequence 
    \[\begin{tikzcd}
	{E^{\times}} & {E^{\times}} & {K^{\times}}
	\arrow["\rho", from=1-1, to=1-2]
	\arrow["{\norm_{E/K}}", from=1-2, to=1-3]
    \end{tikzcd}\]
    is exact by Hilbert's theorem 90. 
\end{proof}

% 107/136