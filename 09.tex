\section{}

\subsection{Finite fields}

In this section, $p$ will be a prime number. 

\begin{proposition}
    Let $m$ be a positive integer. 
    Up to isomorphism, there exists a unique 
    field $F_m$ of size $p^m$. 
\end{proposition}

\begin{proof}
    Let $C$ be an algebraic closure of the field $\Z/p$ and 
    let $F_m=\{x\in C:x^{p^m}=x\}$ be the set of roots of $X^{p^m}-X$. Since 
    the polynomial $X^{p^m}-X$ has no multiple roots, $|F_m|=p^m$. Moreover, 
    $F_m$ is the unique subfield of $C$ of size $p^m$. 
    
    To prove the uniqueness, it is enough to note that 
    if $K$ is a field of $p^m$ elements, then
    $K$ is the decomposition field of $X^{p^m}-X$ over $\Z/p$.  
\end{proof}

Let $K=\Z/p$ and $C$ be an algebraic closure of $K$. 
We claim that $C=\cup_k F_k$. If $x\in C$, then $x$ is algebraic over $K$. 
Since $K(x)/K$ is finite, $K(x)$ is a finite field, say 
$|K|=p^r$ for some $r$. Then $x^{p^r}=x$ and hence $x\in F_r$. 

\begin{exercise}
    Prove the following statements:
    \begin{enumerate}
        \item If $x\in F_r$, then $x^{p^{rk}}=x$ for all $k\geq0$.
        \item If $m\mid n$, then $F_m\subseteq F_n$. 
        \item $F_m\cap F_n=F_{\gcd(m,n)}$.
        \item $F_m\subseteq F_n$ if and only if $m\mid n$. 
    \end{enumerate}
\end{exercise}

\begin{proposition}
    Every finite extension of a finite field is cyclic. 
\end{proposition}

\begin{proof}
    Let $K=\Z/p$. It is enough to show that $F_n/F_m$ is cyclic if $m$ divides $n$. 
    
    We first prove that $F_n/K$ is cyclic. 
    Let 
    \[
    \sigma\colon F_n\to F_n,\quad 
    x\mapsto x^p.
    \]
    Then 
    $\sigma\in\Gal(F_n/K)$ (it is bijective because all field homomorphisms 
    are injective and $F_n$ is finite). 

    Note that 
    $F_n/K$ is a Galois extension, as $F_n$ is the splitting
    field over $K$ 
    of the separable polynomial $X^{p^n}-X\in K[X]$. 
    Thus $|\Gal(F_n/K)|=[F_n:K]=n$. 
    
    We claim that $\sigma$ generated $\Gal(F_n/K)$. Since 
    $\sigma^i(x)=x^{p^i}$ for all $i\geq 0$, in particular, 
    $\sigma^n(x)=x^{p^n}=x$. Thus $\sigma^n=\id$ and hence $|\sigma|$ divides $n$. Let 
    $s=|\sigma|$. We know that $F_n^{\times}=F_n\setminus\{0\}$ is
    cyclic, say $F_n^{\times}=\langle g\rangle$. Since $|g|=p^n-1$, 
    \[
    g=\sigma^s(g)=g^{p^s}
    \]
    and hence $p^s\equiv 1\bmod (p^n-1)$. Thus $p^n-1$ divides $p^s-1$ and
    hence $n$ divides $s$. Therefore $n=s$ and $\Gal(F_n/K)=\langle\sigma\rangle$. 
    
    For the general case note that if $m$ divides $n$, 
    then $\Gal(F_n/F_m)$ is a subgroup of $\Gal(F_n/K)$. Since  $\Gal(F_n/K)$ is cyclic, 
    the claim follows.
\end{proof}

\index{Frobenius automorphism}
If $K=\Z/p$ and 
$m$ divides $n$, the subextension $F_m$ corresponds 
to the unique
subgroup of index $m$ of $\Gal(F_n/K)=\langle\sigma\rangle$. This subgroup
is $\langle\sigma^m\rangle$, where
\[
\sigma^m(x)=x^{p^m}=x^{|F_m|}.
\]
Note that $\Gal(F_n/F_m)=\langle\sigma^m\rangle$. 
The map $\sigma^m$ is known as 
the \textbf{Frobenius automorphism}. 

\begin{exercise}
    Let $E/K$ be an extension of finite fields. Then $E/K$ 
    is cyclic and $\Gal(E/K)=\langle\tau\rangle$, where $\tau(x)=x^{|K|}$. 
\end{exercise}

% page 96
% number of irreducible polynomials
% Moebius inversion formula in commutative rings



\subsection{Cyclotomic extensions}

For $n\geq1$ let $G_n(K)=\{x\in K:x^n=1\}$ be the 
set of $n$-roots of one in $K$. Note that
$G_n(K)$ is a cyclic subgroup of $K^{\times}$ and that 
$|G_n(K)|$ divides $n$. 

\begin{example}
    $G_n(\R)=\{-1,1\}$ if $n$ is odd and $G_{n}=\{1\}$ if $n$ is even.
\end{example}

\begin{exercise}
    Let $K$ be a field of characteristic $p>0$. Let $n=p^sm$ for some $m$ not divisible by $p$. 
    Then $G_n(K)=G_m(K)$. 
\end{exercise}

\begin{exercise}
    Let $q$ be a prime number. Then $G_n(\Z/q)\simeq\Z/\gcd(n,q-1)$. 
\end{exercise}

Similarly, one can prove that if $K$ is a finite field, then $G_n(K)$ is a cyclic group
of order $\gcd(n,|K^{\times}|)$. 

\begin{example}
    If $C$ is algebraically closed of characteristic coprime with $n$, 
    then $G_n(C)$ is cyclic of order $n$, as $X^n-1$ 
    has all its roots in $C$ and does not contain multiple roots. 
\end{example}

Let $K$ be an algebraically closed field and $n$ be
such that $n$ is coprime with the characteristic of $K$. The set of 
\textbf{primitive $n$-roots} is defined as 
\[
H_n(K)=\{x\in G_n(K):|x|=n\}.
\]

\begin{definition}
\index{Cyclotomic polynomial}
    Let $K$ be an algebraically closed field and $n$ be
    such that $n$ is coprime with the characteristic of $K$. The \textbf{$n$-th cyclotomic
    polynomial} is defined as 
    \[
    \Phi_n=\prod_{x\in H_n(K)}(X-x)\in K[X].
    \]
\end{definition}

\index{Euler's $\phi$ function}
For $n\geq1$ the Euler's function is defined as 
\[
\varphi(n)=|\{k:1\leq k\leq n,\;\gcd(k,n)=1\}|.
\]
For example, $\varphi(4)=2$, $\varphi(8)=\varphi(10)=4$ and $\varphi(p)=p-1$ for every prime $p$. 

\begin{proposition}
    Let $K$ be an algebraically closed field and $n$ be
    such that $n$ is coprime with the characteristic of $K$. Let $A$ be
    the ring of integers of $K$. 
    \begin{enumerate}
        \item $\deg\Phi_n=\varphi(n)$.
        \item $\Phi_n\in A[X]$.
    \end{enumerate}
\end{proposition}

\begin{proof}
    The first statement is clear. Let us prove 2) by induction on $n$. The case $n=1$ is
    trivial, as $\Phi_1=X-1$. Assume that $\Phi_d\in A[X]$ for all $d$ such that $d<n$. 
    In particular,
    \[
    \gamma=\prod_{\substack{d\mid n\\d\ne n}}\Phi_d\in A[X].
    \]
    Since $\gamma$ is monic, it follows that 
    $\frac{X^n-1}{\gamma}\in A[X]$. Now the claim follows from 
    \[
    X^n-1=\prod_{d\mid n}\Phi_d=\Phi_n\prod_{\substack{d\mid n\\d\ne n}}\Phi_d=\Phi_n\gamma.\qedhere
    \]
\end{proof}

By taking degree in the equality 
$X^n-1=\prod_{d\mid n}\Phi_d$ 
one gets 
\[
n=\sum_{d\mid n}\varphi(d).
\]

\begin{definition}
\label{defn:cyclotomic}
\index{Extension!cyclotomic}
    Let $n\geq2$ and $K$ be a field of characteristic coprime with $n$. A 
    \textbf{cyclotomic extension} of $K$ of index $n$ is a 
    decomposition field of $X^n-1$ over $K$. 
\end{definition}

Let $C$ be an algebraic closure of $K$ and $n\geq2$ be coprime with the characteristic of $K$. 
If follows from Definition \ref{defn:cyclotomic} 
that a cyclotomic extension of index $n$ is of the form 
$K(\omega)/K$ for some $\omega\in H_n(K)$. 

\begin{proposition}
    A cyclotomic extension of index $n$ is abelian and of degree a divisor of $\varphi(n)$. 
\end{proposition}

\begin{proof}
    Let $C$ be an algebraic closure of $K$ and $n\geq2$ be coprime with the characteristic of $K$. 
    Let $\omega\in H_n(C)$ and $K(\omega)/K$ be a cyclotomic extension. Then $K(\omega)/K$
    is a Galois extension, as it is a decomposition field of a separable polynomial. 
    Let $U=\mathcal{U}(\Z/n)$ be the group of units of $\Z/n$ and 
    \[
    \lambda\colon \Gal(K(\omega)/K)\to U,
    \quad
    \sigma\mapsto m_{\sigma},
    \]
    where $m_{\sigma}$ is such that $\sigma(\omega)=\omega^{m_{\sigma}}$. The map $\lambda$ is well-defined and
    it is a group homomorphism, as if $\sigma,\tau\in\Gal(K(\omega)/K)$, then, since 
    \[
        (\tau\sigma)(\omega)=\tau(\sigma(\omega))=\tau(\omega^{m_\sigma})=\left(\omega^{m_\sigma}\right)^{m_\tau}=\omega^{m_\sigma m_\tau},
    \]
    it follows that $\lambda(\sigma)\lambda(\tau)=\lambda(\sigma\tau)$. Since 
    $\lambda$ is injective, $\Gal(K(\omega)/K)$ is isomorphic to a subgroup 
    of the abelian group $U$. Hence $\Gal(K(\omega)/K)$ is abelian. Moreover, 
    $[K(\omega):K]=|\Gal(K(\omega)/K)|$ is a divisor of $|U|=\varphi(n)$. 
\end{proof}

\begin{exercise}
    Prove that a cyclotomic extension $K(\omega)/K$ has degree $\varphi(n)$ if and only if 
    $\Phi_n$ is irreducible over $K$. 
\end{exercise}

Note that $\Phi_n$ is irreducible over $\Q$. Some concrete examples:
\[
\Phi_1=X-1,
\quad
\Phi_2=X+1,
\quad
\Phi_3=X^2+X+1,
\quad
\Phi_6=X^2-X+1.
\]
If $p$ is a prime number, then $\Phi_p=X^{p-1}+\cdots+X+1$. 

\begin{example}
    $\Phi_5$ is irreducible over $\Z/2$. First note that
    $\Phi_5=X^{4}+\cdots+X+1$ does not have roots in $\Z/2$. If 
    $\Phi_5$ is reducible, then, since
    $X^2+X+1$ is the unique degree-two 
    monic irreducible polynomial 
    over $\Z/2$, it follows that
    \[
    \Phi_5=(X^2+X+1)(X^2+X+1)=(X^2+X+1)^2=X^4+X^2+1,
    \]
    a contradiction.
\end{example}

\begin{exercise}
Prove that
$\Phi_{12}=X^4-X^2+1$ is not irreducible over $\Z/5$. 
\end{exercise}

\subsection{Hilbert's theorem 90}

\begin{theorem}[Hilbert]
    Let $E/K$ be a cyclic extension. Assume that 
    $\Gal(E/K)$ is generated by $\tau$. For 
    $a\in E$, $\norm_{E/K}(a)=1$ if an only 
    if $a=b/\tau(b)$ for some $b\in L\setminus\{0\}$. 
\end{theorem}

\begin{proof}
    Let $n=|G|$. We first prove $\impliedby$. If $a=b/\tau(b)$ and $b\ne 0$, then 
    \[
    \norm_{E/K}(a)=a\tau(a)\tau^2(a)\cdots\tau^{n-1}(a)
    =\frac{b}{\tau(b)}\frac{\tau(b)}{\tau^2(b)}\cdots\frac{\tau^{n-1}(b)}{\tau^n(b)}=1.
    \]

    Now we prove $\implies$. Let $a\in E$ be such that $\norm_{E/K}(a)=1$. For 
    $c\in E$ let 
    \begin{align*}
        d_0 &= ac,\\
        d_1 &= a\tau(a)\tau(c),\\
        d_2 &= a\tau(a)\tau^2(a)\tau^2(c),\\
        &\vdots\\
        d_{n-1} &= \underbrace{a\tau(a)\cdots\tau^{n-1}(a)}_{=\norm_{E/K}(a)}\tau^{n-1}(c)=\tau^{n-1}(c).
    \end{align*}
    Then 
    \[
    a\tau(d_j)=a\tau(a)\cdots\tau^{j+1}(a)=d_{j+1}
    \]
    for all $j\in\{0,\dots,n-2\}$. Let $b=d_0+\cdots+d_{n-1}$. Then 
    $b\ne 0$, otherwise, if $b=0$, then, for every $c\in E$, 
    \begin{align*}
    0&=ac+(a\tau(a))\tau(c)+\cdots+(a\tau(a)\cdots\tau^{n-1}(a))\tau^{n-1}(c)
    \end{align*}
    implies that $a=0$ by Dedekind's theorem, a contradiction. So let $c\in E$ be
    such that $b\ne 0$. Then 
    \begin{align*}
    \tau(b)&=\tau(d_0)+\cdots+\tau(d_{n-1})\\
    &=\tau(ac)+\tau(a\tau(c))+\cdots+\tau(\tau^{n-1}(c))\\
    &=\frac{1}{a}(d_1+\cdots+d_{n-1})+\tau^n(c)\\
    &=\frac{1}{a}(d_0+\cdots+d_{n-1})\\
    &=b/a.\qedhere
    \end{align*}
\end{proof}

\begin{exercise}
    Let $E/K$ be a cyclic extension. Assume that 
    $\Gal(E/K)$ is generated by $\tau$. Prove that for 
    $a\in E$, $\trace_{E/K}(a)=0$ if an only 
    if $a=b-\tau(b)$ for some $b\in L\setminus\{0\}$.  
\end{exercise}

\begin{corollary}
    Let $a,b,c\in \Z$ be such that $a^2+b^2=c^2$. Then 
    \[
    (a,b,c)=\lambda(r^2-s^2,-2rs, r^2+s^2)
    \]
    for some $r,s\in\Z$ and some $\lambda\in\Z$.
\end{corollary}

\begin{proof}
    We work with the extension $\Q(i)/\Q$. Note that 
    $\Gal(\Q(i),\Q)=\{\id,\gamma\}$ is cyclic, where 
    $\gamma\colon\Q(i)\to\Q(i)$, $z\mapsto\overline{z}$, is the complex conjugation. 
    We may assume that $c\ne 0$, otherwise $a=b=0$ and the result is trivial.  
    Write $(a/c)^2+(b/c)^2=1$ and let $\alpha=(a/c)+(b/c)i\in\Q(i)$. Then
    $\norm_{\Q(i)/\Q}(\alpha)=1$. 
    By Hilbert's theorem, 
    there exists $\beta\in\Q(i)\setminus\{0\}$ such that 
    \[
    \alpha=a+bi=\frac{\gamma(\beta)}{\beta}.
    \]
    Note that if $m\in\Z\setminus\{0\}$, then 
    $\frac{m\gamma(\beta)}{m\beta}=\frac{\gamma(\beta)}{\beta}$. 
    There exists $m\in\Z\setminus\{0\}$ such that 
    $m\beta\in\Z[i]$, say $m\beta=r+is$ with $r,s\in\Z$. Then
    \[
    \alpha=\frac{\gamma(\beta)}{\beta}=\frac{\gamma(m\beta)}{m\beta}=
    \frac{r-is}{r+is}=\frac{r^2-s^2-2rsi}{r^2+s^2}.
    \]
    From this the claim follows. 
\end{proof}

\begin{exercise}
% https://abel.math.harvard.edu/~elkies/Misc/hilbert.pdf
Let $A,B\in\Z$ be such that $A^2-4B$ is not a square. Prove that 
a solution $(x,y,z)$ of $x^2 + Axy + By^2 = z^2$
is proportional to 
\[
(r^2-Bs^2,2rs+As^2,r^2+Ars+Bs^2).
\]
\end{exercise}

\subsection{Group cohomology}

Let $G$ be a group and $A$ be a (left) $G$-module. This means that $A$ is an abelian
group and there exists a map
\[
G\times A\to A,\quad
(g,a)\mapsto g\cdot a
\]
such that $1\cdot a=a$ for all $a\in A$, $(gh)\cdot a=g\cdot (h\cdot a)$ for 
all $g,h\in G$ and $a\in A$ and $g\cdot (a+b)=g\cdot a+g\cdot b$ for
all $g\in G$ and $a,b\in A$. 

\begin{example}
    The group $\Gal(\C/\R)$ acts on $\C$ and $\C^\times$. Moreover, 
    it acts trivially on $\R$ and $\R^\times$. 
\end{example}

More generally, if $E/K$ is a finite Galois extension, then 
the Galois group $\Gal(E/K)$ acts on $E$ and $E^\times$. 

\begin{definition}
    Let $G$ be a group and $M$ and $N$ be $G$-modules. A map 
    $f\colon M\to N$ is a \textbf{homomorphism} of $G$-modules
    if $f(\sigma\cdot m)=\sigma\cdot f(m)$ for all $m\in M$ and $\sigma\in G$.
\end{definition}

\begin{definition}
    Let $G$ be a group and $M$ be a $G$-module.
    The submodule of \textbf{$G$-invariants} is defined as
    \[
    M^G=\{m\in M:\sigma\cdot m=m\text{ for all $\sigma\in G$}\}.
    \]
\end{definition}

Note that $M^G$ is the largest submodule of the $G$-module 
$M$ where $G$ acts trivially. For example, if 
$G=\Gal(E/K)$, then $E^G=K$. 

\begin{proposition}
\label{pro:H0}
    Let $G$ be a group. If the sequence
of $G$-modules and $G$-module homomorphism
\[
\begin{tikzcd}
    0 & P & M & N & 0
    \arrow[from=1-1, to=1-2]
    \arrow["\alpha", from=1-2, to=1-3]
    \arrow["\beta", from=1-3, to=1-4]
    \arrow[from=1-4, to=1-5]
    \end{tikzcd}\]
is exact, then 
\[
\begin{tikzcd}
                        0 & P^G & M^G & N^G 
                        \arrow[from=1-1, to=1-2]
                        \arrow["\alpha^0", from=1-2, to=1-3]
                        \arrow["\beta^0", from=1-3, to=1-4]
        \end{tikzcd}
        \]
is exact, where $\alpha^0$ is the restriction $\alpha|_{P^G}$ of $\alpha$ to $P^G$ and
$\beta^0$ is the restriction $\beta|_{M^G}$ of $\beta$ to $M^G$. 
\end{proposition}

\begin{proof}
    Since $\alpha$ is injective, the restriction $\alpha^0$ is injective. 

    Note that 
    $\ker\beta^0=\ker\beta\cap M^G\subseteq\ker\beta$. 
    
    We claim 
    that $\alpha^0(P^G)=\alpha(P)\cap M^G$. If $m\in\alpha(P)\cap M^G$, then 
    $\alpha(p)=m$ for some $p\in P$ and $\sigma\cdot m=m$. Since
    \[
    m=\sigma\cdot m=\sigma\cdot\alpha(p)=\alpha(\sigma\cdot p),
    \]
    $\sigma\cdot p-p\in\ker\alpha=\{0\}$. Hence $\sigma\cdot p=p$ and
    $p\in P^G$. Conversely, if $m\in\alpha^0(P^G)$, then 
    $m=\alpha(p)$ for some $p\in P^G$. If $\sigma\in G$, then
    \[
    \sigma\cdot m=\sigma\cdot\alpha(p)=\alpha(\sigma\cdot p)=\alpha(p)=m.
    \]
    Hence $m\in M^G\cap\alpha(P)$.

    Now
    \[
    \alpha^0(P^G)=\alpha(P)\cap M^G=\ker\beta\cap M^G=\ker\beta^0.\qedhere  
    \]
\end{proof}

Note that in the previous proposition, we did not prove that
the map $\beta|_{M^G}$ is subjective. 

\begin{example}
    Let $G=\Gal(\C/\R)$. Consider the following exact sequence
    of $G$-modules:
    \[
    \begin{tikzcd}
    1 & \{-1,1\} & \C^\times & \C^\times & 1
    \arrow[from=1-1, to=1-2]
    \arrow[from=1-2, to=1-3]
    \arrow["\beta", from=1-3, to=1-4]
    \arrow[from=1-4, to=1-5]
    \end{tikzcd}    
    \]
    where $\beta(z)=z^2$. Note that $\beta$ is subjective. Take invariants 
    to obtain the sequence  
    \[
    \begin{tikzcd}
     0 & \{-1,1\} & \R^\times & \R^\times 
     \arrow[from=1-1, to=1-2]
     \arrow[from=1-2, to=1-3]
     \arrow["\beta^0", from=1-3, to=1-4]
     \end{tikzcd}
     \]
     where $\beta^0(x)=x^2$. Note that $\beta^0$ is not surjective! 
\end{example}

\begin{definition}
    Let $G$ be a group and $N$ be a $G$-module. 
    We define 
    \begin{align*}
        H^0(G,M)&=M^G,\\
        C^1(G,M)&=\{\phi\colon G\to M:\phi\text{ is a map}\},\\
        Z^1(G,M)&=\{\phi\in C^1(G,M):\phi(\sigma\tau)=\phi(\sigma)+\sigma\cdot\phi(\tau)\text{ for all $\sigma,\tau\in G$}\},
        \end{align*}    
\end{definition}

Note that $Z^1(G,M)$ is an abelian group with the operation
\[
(\phi+\phi_1)(\sigma)=\phi(\sigma)+\phi_1(\sigma).
\]
Moreover, if $\phi\in Z^1(G,M)$, then 
$\phi(1_G)=0_M$. To prove this fact, note that  
\[
\phi(1_G)=\phi(1_G1_G)=\phi(1_G)+1_G\cdot\phi(1_G)=\phi(1_G)+\phi(1_G)
\]
implies
that $\phi(1_G)=0_M$. 

\begin{example}
\label{exa:BinZ}
    Let $G$ be a group and $M$ be a $G$-module. Fix $m\in M$. Then
    the map $\phi\colon G\to M$, $\phi(\sigma)=\sigma\cdot m-m$, is an element 
    of $Z^1(G,M)$, because 
    \begin{align*}
    \phi(\sigma\tau)&=(\sigma\tau)\cdot m-m\\
    &=(\sigma\tau)\cdot m-\sigma\cdot m+\sigma\cdot m-m\\
    &=\sigma\cdot (\tau\cdot m-m)+\sigma\cdot m-m\\
    &=\sigma\cdot \phi(\tau)+\phi(\sigma)   
    \end{align*}
    for all $\sigma,\tau\in G$.
\end{example}

\begin{definition}
    Let $G$ be a group and $M$ be a $G$-module. The set
    $B^1(G,M)$ of \textbf{coboundaries} is the set 
    of elements $\phi\in C^1(G,M)$ such that there is a fixed 
    $m\in M$ such that
    $\phi(\sigma)=\sigma\cdot m=m$ for all $\sigma\in G$.
\end{definition}

We proved in Example \ref{exa:BinZ} that  
$B^1(G,M)\subseteq Z^1(G,M)$. A direct calculation shows that, in fact, 
$B^1(G,M)$ is a subgroup of $Z^1(G,M)$. 

\begin{definition}
    Let $G$ be a group and $M$ be a $G$-module. The 
    \textbf{first cohomology group} of $G$ with coefficients
    in $M$ is defined as the quotient
    \[
    H^1(G,M)=Z^1(G,M)/B^1(G,M).
    \]
\end{definition}

\begin{example}
    If $G$ acts trivially on $M$, then 
    \[
    H^0(G,M)=M^G=M,
    \quad 
    B^1(G,M)=\{0\},
    \quad 
    Z^1(G,M)=\Hom(G,M).
    \]
    Hence 
    $H^1(G,M)\simeq\Hom(G,M)$.
\end{example}

\begin{example}
    Let $G=\Gal(\C/\R)=\{\id,\gamma\}$, where $\gamma\colon\C\to\C$, $z\mapsto\overline{z}$, is the complex conjugation. Then
    \[
    H^0(G,\R^\times)=\left(\R^\times\right)^G=\R^\times.
    \]
    Since $G$ acts trivially on $\R^\times$, 
    \[
    H^1(G,\R^\times)=\Hom(G,\R^\times)\simeq\Hom(G,\{-1,1\})\simeq\Z/2.
    \]
\end{example}

The following lemma will be useful. 

\begin{lemma}
\label{lem:H1_maps}
    Let $G$ be a group and 
    $\alpha\colon M\to N$ be a homomorphism of $G$-modules. 
    Then 
    \[
    \alpha^1\colon H^1(G,M)\to H^1(G,N),\quad 
    \phi+B^1(G,M)\mapsto \alpha\circ\phi+B^2(G,N),
    \]
    is a group homomorphism. 
\end{lemma}

\begin{proof}
    Let us prove that the map $\alpha^1$ is well-defined. If 
    $\phi-\phi'\in B^1(G,M)$, then 
    there exists a fixed 
    $m\in M$ such that 
    $(\phi-\phi')(\sigma)=\sigma\cdot m-m$ for all $\sigma\in G$. 
    Let $n=\alpha(m)\in N$. 
    For $\sigma\in G$, 
    \[
    \alpha((\phi-\phi')(\sigma))=\alpha(\sigma\cdot m-m)
    =\sigma\cdot \alpha(m)-\alpha(m)=\sigma\cdot n-n\in B^1(G,N). 
    \]
    Thus $\alpha\circ\phi-\alpha\circ\phi'\in B^1(G,N)$. 

    We now prove that $\alpha^1$ is a group homomorphism. If 
    $\phi,\phi'\in Z^1(G,M)$, then 
    \begin{align*}
    \alpha^1(\phi+B^1(G,M)+\phi'+B^1(G,M))
    &=\alpha^1(\phi+\phi'+B^1(G,M))\\
    &=\alpha\circ(\phi+\phi')+B^1(G,N)\\
    &=\alpha\circ\phi+\alpha\circ\phi'+B^1(G,N)\\
    &=\alpha\circ\phi+B^1(G,N)+\alpha\circ\phi'+B^1(G,N)\\
    &=\alpha^1(\phi+B^1(G,M))+\alpha^1(\phi'+B^1(G,M)).\qedhere
    \end{align*}
\end{proof}

We will provide a detailed proof of the upcoming result. 
The theorem will be established by applying a \textbf{diagram chasing} technique, a widely used tool in homological algebra.
The proof is tedious and may seem intricate, but the method essentially 
involves ``chasing'' elements around a (commutative) diagram until we attain the desired result. 

\begin{theorem}
Let $G$ be a group and 
\[
\begin{tikzcd}
    0 & P & M & N & 0
    \arrow[from=1-1, to=1-2]
    \arrow["\alpha", from=1-2, to=1-3]
    \arrow["\beta", from=1-3, to=1-4]
    \arrow[from=1-4, to=1-5]
    \end{tikzcd}
\] 
be an exact sequence of $G$-modules and $G$-module homomorphism. 
Then there exists a \textbf{connection homomorphism} $\delta$ such that the sequence 
\begin{equation}
\label{eq:long}
\begin{tikzcd}
 0\rar & H^0(G,P)\rar["\alpha^0"] & H^0(G,M) \rar["\beta^0"]
             \ar[draw=none]{d}[name=X, anchor=center]{}
    & H^0(G,N) \ar[rounded corners,
            to path={ -- ([xshift=2ex]\tikztostart.east)
                      |- (X.center) \tikztonodes
                      -| ([xshift=-2ex]\tikztotarget.west)
                      -- (\tikztotarget)}]{dll}[at end]{\delta} \\      
  &H^1(G,P) \rar["\alpha^1"] & H^1(G,M) \rar["\beta^1"] & H^1(G,N)
\end{tikzcd}
\end{equation}
of abelian groups
and group homomorphisms is exact. 
\end{theorem}

\begin{proof}
    By Proposition \ref{pro:H0}, the 
    sequence is exact at $H^0(G,P)=P^G$, 
    $H^0(G,M)=M^G$ and 
    $H^0(G,N)=N^G$. Note that, in particular, 
    $\alpha\colon P^G\to\alpha(P^G)$ is a bijective group homomorphism. 


    Let us construct the connecting
    homomorphism $\delta\colon H^0(G,N)\to H^1(G,P)$. For 
    $n\in N^G$, let $m\in M^G$ be such that 
    $\beta(m)=n$. We define $\delta(n)=\phi+B^1(G,P)$, where 
    \[
    \phi(\sigma)=\alpha^{-1}(\sigma\cdot m-m).
    \]


    Let us prove that the map $\delta$ is well-defined: if $m,m'\in M^G$ are such that
    $\beta(m)=\beta(m')=n$, then  $m-m'\in\ker\beta=\alpha(P^G)$. Thus 
    $m-m'=\alpha(p)$ for some $p\in P^G$ and hence, 
    since $\alpha^{-1}$ is a homomorphism of $G$-modules, 
    \begin{align*}
    \phi(\sigma)-\phi'(\sigma)&=\alpha^{-1}(\sigma\cdot m-m)
    -\alpha^{-1}(\sigma\cdot m'-m')\\
    &=\alpha^{-1}(\sigma\cdot (m-m'))-\alpha^{-1}(m-m')\\
    &=\alpha^{-1}(\sigma\cdot\alpha(p)-\alpha(p))\\
    &=\sigma\cdot p-p.
    \end{align*}
    Thus $\phi-\phi'\in B^1(G,P)$. 

    Note that $\phi\in Z^1(G,P)$, because
    \begin{align*}
    \phi(\sigma\tau)&=\alpha^{-1}((\sigma\tau)\cdot m-m)\\
    &=\alpha^{-1}((\sigma\tau)\cdot m-\sigma\cdot m+\sigma\cdot m-m)\\
    &=\alpha^{-1}(\sigma\cdot (\tau\cdot m-m))+\alpha^{-1}(\sigma\cdot m-m)\\
    &=\sigma\cdot\phi(\sigma)+\phi(\tau)
    \end{align*}
    holds for all $\sigma,\tau\in G$. 
    
    We now prove that the sequence \eqref{eq:long}
    is exact at $H^0(G,N)=N^G$. We need to prove that $\ker\delta=\im\beta$. If $m\in M^G$ is such that 
    $\delta(\beta(m))=\phi+B^1(G,P)$, then $\phi(\sigma)=\alpha^{-1}(\sigma\cdot m-m)=0$. Conversely, 
    if $n\in\ker\delta$, then there exists $m\in M^G$ such that $n=\beta(m)$ and 
    $\delta(n)=\phi+B^1(G,P)$, where $\phi(\sigma)=\sigma\cdot m-m$
    for all $\sigma\in G$. 

    We now prove that \eqref{eq:long} is exact at $H^1(G,P)$, that 
    is $\im\delta=\ker\alpha^1$. To prove
    $\subseteq$ note that 
    for $n\in N^G$, $\delta(n)=\phi+B^1(G,P)$, where 
    $\phi(\sigma)=\alpha^{-1}(\sigma\cdot m-m)$ for all $\sigma\in G$
    and some $m\in M$ such that $\beta(m)=n$. In particular, 
    $\alpha\circ\phi\in B^1(G,M)$. 
    Then
    \[
    \alpha^1(\phi+B^1(G,P))=\alpha\circ\phi+B^1(G,M)=B^1(G,M).
    \]
    To prove $\supseteq$, let $\phi+B^1(G,P)\in\ker\alpha^1$. Then
    $\alpha\circ\phi\in B^1(G,M)$, that is $\alpha(\phi(\sigma))=\sigma\cdot m-m$ for 
    all $\sigma\in G$ and some $m\in M$. Note that 
    \[
    \delta(\beta(m))=\psi+B^1(G,P),
    \]
    where $\psi(\sigma)=\alpha^{-1}(\sigma\cdot m-m)$. This implies that 
    $\alpha(\psi(\sigma))=\alpha(\phi(\sigma))$ for all $\sigma\in G$. Since 
    $\alpha$ is injective, $\psi=\phi$. Therefore 
    $\phi+B^1(G,P)$ belongs to the image of $\delta$. 

    Finally, we prove that the sequence \eqref{eq:long} is exact
    at $H^1(G,M)$, that is $\im\alpha^1=\ker\beta^1$. To prove $\subseteq$  
    note that 
    \[
    \beta^1(\alpha^1(\phi+B^1(G,P)))=\beta^1(\alpha\circ\phi+B^1(G,M))
    =(\beta\circ\alpha)\circ\phi+B^1(G,N)=B^1(G,N).
    \]
    Conversely, let $\phi+B^1(G,M)\in\ker\beta_1$. Then $\beta\circ\phi\in B^1(G,N)$. Thus
    there exists $n\in N$ such that $\beta(\phi(\sigma))=\sigma\cdot n-n$ 
    for all $\sigma\in G$. Since $\beta$ is surjective, 
    $n=\beta(m)$ for some $m\in M$. Hence 
    \[
    \beta(\phi(\sigma))=\sigma\cdot n-n=\sigma\cdot \beta(m)-\beta(m)
    =\beta(\sigma\cdot m-m).
    \]
    For each $\sigma\in G$, 
    $\phi(\sigma)-(\sigma\cdot m-m)\in\ker\beta=\im\alpha$. 
    and therefore  $\phi(\sigma)-(\sigma\cdot m-m)=\alpha(\rho_\sigma)$. 
    This defines a map $\rho\colon G\to P$, $\sigma\mapsto\rho_\sigma$.
    We claim that $\rho\in Z^1(G,P)$. If $\sigma,\tau\in G$, then
    \begin{align*}    
    \alpha(\rho_{\sigma\tau})&=\phi(\sigma\tau)-( (\sigma\tau)\cdot m-m)\\
    &=\phi(\sigma)+\sigma\cdot\phi(\tau)-(\sigma\cdot (\tau\cdot m-m)+\sigma\cdot m-m)\\
    &=\alpha(\rho_\sigma)+\sigma\cdot\alpha(\rho_\tau).
    \end{align*}
    Now
    \[
    \alpha_1(\rho+B^1(G,P))=\alpha\circ\rho+B^1(G,M)=\phi+B^1(G,M).\qed
    \]
\end{proof}

\begin{theorem}
    Let $G$ be a finite group and $M$ be a $G$-module. 
    Then 
    \[
    |G|H^1(G,M)=\{0\}.
    \]
\end{theorem}

\begin{proof}
    Let $n=|G|$. It is enough to prove that 
    if $\phi\in Z^1(G,M)$, then $n\phi\in B^1(G,M)$. Let $\phi\in Z^1(G,M)$ and 
    $\sigma\in G$. Then 
    \[
    \phi(\sigma\tau)=\phi(\sigma)+\sigma\cdot\phi(\tau)
    \]
    for all $\tau\in G$. Summing over all possible $\sigma\in G$, we obtain that 
    \begin{equation}
        \label{eq:|G|H1(G,M)=0}    
        \sum_{\tau\in G}\phi(\sigma\tau)=\sum_{\tau\in G}\sigma\cdot\phi(\tau)+\sum_{\tau\in G}\phi(\sigma).
    \end{equation}
    Let $m=-\sum_{\tau\in G}\phi(\tau)\in M$. Then 
    \eqref{eq:|G|H1(G,M)=0} can be rewritten as 
    \[
    -m=\sum_{\tau\in G}\phi(\tau)=\sigma\cdot \sum_{\tau\in G}\phi(\tau)+n\phi(\sigma)
    =-\sigma\cdot m+n\phi(\sigma).
    \]
    Thus $n\phi(\sigma)=\sigma\cdot m-m$ and hence $n\phi\in B^1(G,M)$.  
\end{proof}

\begin{exercise}
    Let $G$ be a finite group and 
    $M$ be a finite $G$-module of size coprime to $|G|$. Prove that 
    $H^1(G,M)=\{0\}$. 
\end{exercise}

\begin{exercise}
    Let $G$ be a finite group and 
    $M$ be a finitely generated $G$-module. Prove that
    $H^1(G,M)$ is finite.
\end{exercise}

%\begin{proof}
%    
%    By the previous theorem, $H^1(G,M)$ is a
%\end{proof}
%
% \begin{definition}
%     \index{Derivation}
%     Let $A$ be a $G$-module. 
%     A \textbf{derivation} of $A$ is a map $d\colon G\to A$ such that
%     $d(gh)=g\cdot d(h)+d(g)$ for all $g,h\in G$. 
% \end{definition}

% \index{Inner derivation}
% Let $A$ be a $G$-module and $a\in A$. The map 
% $d(g)=g\cdot a-a$ is a derivation of $A$. 
% Such derivations as known as \textbf{inner derivations}. 

% \begin{exercise}
%     Let $E/K$ be a finite Galois extension and $G=\Gal(E/K)$. 
%     Then the (multiplicative) group $E^\times$ is a $G$-module with
%     $\sigma\cdot x=\sigma(x)$. 
% \end{exercise}

% In the context of the previous exercise, 
% let $d\colon G\to E^\times$ be a derivation. For $\tau\in G$ let 
% $x_\tau=d(\tau)$. Then
% \[
% x_{\sigma\tau}=d(\sigma\tau)=(\sigma\cdot d(\tau))d(\sigma)=\sigma(x_\tau)x_\sigma.
% \]
% Is this true that $x_\sigma=\frac{\sigma(c)}{c}=(\sigma\cdot c)c^{-1}$ 
% for some $c\in E^\times$?

% \begin{proposition}
%     Let $E/K$ be a finite Galois extension and $G=\Gal(E/K)$. 
%     \begin{enumerate}
%         \item Let $\{x_\tau:\tau\in G\}\subseteq E^\times$ be such that 
%             $x_{\sigma\tau}=\sigma(x_\tau)x_\sigma$. Then there exists
%             $c\in E^\times$ such that $x_\sigma=\frac{\sigma(c)}{c}$ for all $\sigma\in G$. 
%         \item Let $\{x_\tau:\tau\in G\}\subseteq E$ be such that 
%             $x_{\sigma\tau}=\sigma(x_\tau)+x_\sigma$. Then there exists
%             $c\in E$ such that $x_\sigma=\sigma(c)-c$ for all $\sigma\in G$.  
%     \end{enumerate}
% \end{proposition}

% \begin{proof}
%     We prove 1). By Dedekind's theorem, the homomorphism 
%     $\sum_{\tau\in G}x_{\tau}^{-1}\tau$ is non-zero. Thus there exists
%     $a\in E^{\times}$ such that $\sum_{\tau\in G}x_{\tau}^{-1}\tau(a)=c\in E^{\times}$. 
%     If $\sigma\in G$, 
%     then 
%     \[
%     \sigma(c)=\sum_{\tau\in G}\sigma(x_{\tau}^{-1})(\sigma\tau)(a)=
%     \sum_{\tau\in G}x_{\sigma\tau}^{-1}x_{\sigma}\sigma\tau(a)
%     =x_{\sigma}\sum_{\tau\in G}x_{\sigma\tau}^{-1}\sigma\tau(a)=x_{\sigma}c.
%     \]
    
%     We now prove 2). By Dedekind's theorem, $\sum_{\tau\in G}\tau\ne0$. Since
%     it is a non-zero 
%     linear form in $E$, it is surjective. In particular, there exists 
%     $a\in E$ such that $\sum_{\tau\in G}\tau(a)=1$. Let $c=-\sum_{\tau\in G}x_\tau\tau(a)$. 
%     If $\sigma\in G$, then 
%     \begin{align*}
%         \sigma(c) &= -\sum_{\tau\in G}\sigma(x_{\tau})\sigma\tau(a)\\
%         &=-\sum_{\tau\in G}(x_{\sigma\tau}-x_{\sigma})\sigma\tau(a)\\
%         &=-\sum_{\tau\in G}x_{\sigma\tau}\sigma\tau(a)+\sum_{\tau\in G}x_{\sigma}\sigma\tau(a)
%         =c+x_{\sigma}.
%     \end{align*}
%     Hence $x_{\sigma}=\sigma(c)-c$. 
% \end{proof}

% The following result is known as Hilbert's 90 theorem. 

% \begin{theorem}[Hilbert's theorem 90]
% \index{Hilbert's theorem 90}
%     Let $E/K$ be a cyclic extension of degree $n$ with Galois group $G=\Gal(E/K)=\langle\sigma\rangle$. 
%     \begin{enumerate}
%         \item Let $x\in E^{\times}$ be such that $\norm_{E/K}(x)=1$, 
%         There exists $b\in E^{\times}$ such that $x=\sigma(b)/b$. 
%         \item Let  $x\in E$ be such that $\trace_{E/K}(x)=0$. 
%         There exists $b\in E$ such that $x=\sigma(b)-b$. 
%     \end{enumerate}
% \end{theorem}

% \begin{proof}
%     Let us prove 1). Note that $G=\{\sigma^i:0\leq i<n\}$. For $i\in\{0,\dots,n-1\}$ let 
%     $x_{\sigma^i}=\prod_{k=0}^{i-1}\sigma^k(x)$. In particular, $x_{\sigma}=x$. We now
%     check that $\{x_{\sigma^i}\}$ satisfy the assumptions of the previous proposition:
%     \begin{align*}
%         \sigma^j(x_{\sigma^i}) &= \prod_{k=0}^{i-1}\sigma^{k+j}(x)
%         =\prod_{k=j}^{i+j-1}\sigma^k(x)\\
%         &=\prod_{k=0}^{i+j-1}\sigma^k(x)\left(\prod_{k=0}^{j-1}\sigma^k(x)\right)^{-1}
%         =\prod_{k=0}^{i+j-1}\sigma^k(x)x_{\sigma^j}^{-1}.
%     \end{align*}
%     If $i+j<n$, then 
%     \[
%     \sigma^j\sigma^i=\sigma^{i+j}
%     \implies
%     \sigma^j(x_{\sigma^i})=x_{\sigma^i\sigma^j}x_{\sigma^j}^{-1}
%     \implies
%     x_{\sigma^j\sigma^i}=\sigma^j(x_{\sigma^i})x_{\sigma^j}.
%     \]
%     If $i+j=n+r$ for some $r\in\{0,\dots,n-1\}$, then, since $i+j<2n$ and
%     $\sigma^i\sigma^j=\sigma^r$, 
%     \[
%     \sigma^j(x_{\sigma^i})=\prod_{k=0}^{n-1}\sigma^k(x)\prod_{k=n}^{n+r-1}\sigma^k(x)x_{\sigma^j}^{-1}
%     =\prod_{k=0}^{r-1}\sigma^k(x)x_{\sigma^j}^{-1}
%     =x_{\sigma^r}x_{\sigma^j}^{-1}=x_{\sigma^j\sigma^i}x_{\sigma^j}^{-1}
%     \]
%     and hence $x_{\sigma^j\sigma^i}=\sigma^j(x_{\sigma^i})x_{\sigma^j}$. By 
%     the previous lemma, there exists $c\in E^{\times}$ such that 
%     $x_{\sigma^i}=\frac{\sigma^i(c)}{c}$. In particular, $x=x_{\sigma}=\sigma(c)/c$. 
    
%     The second statement is similar and it is left as an exercise. 
% \end{proof}

% If $A$, $B$ and $C$ are groups (written multiplicatively) 
% and $f\colon A\to B$ and $g\colon B\to C$ are group homomorphism, 
% the sequence 
% \[\begin{tikzcd}
% 	A & B & C 
% 	\arrow["f", from=1-1, to=1-2]
% 	\arrow["g", from=1-2, to=1-3]
% \end{tikzcd}\]
% of groups and homomorphisms 
% is said to be \textbf{exact} if $f(A)=\ker g$. 
% For example, the sequence 
% \[\begin{tikzcd}
% 	1 & B & C 
% 	\arrow[from=1-1, to=1-2]
% 	\arrow["g", from=1-2, to=1-3]
% \end{tikzcd}\]
% is exact if and only if $g$ is injective, as the first map represents the trivial homomorphism. 
% Similarly, the sequence 
% \[\begin{tikzcd}
% 	A & B & 1
% 	\arrow["f", from=1-1, to=1-2]
% 	\arrow[from=1-2, to=1-3]
% \end{tikzcd}\]
% is exact if and only if $f$ is surjective. 

% \begin{corollary}
% Let $E/K$ be finite and cyclic with $\Gal(E/K)=\langle\sigma\rangle$. 
% \begin{enumerate}
%     \item The sequence 
%     \[\begin{tikzcd}
% 	1 & {K^{\times}} & {E^{\times}} & {E^{\times}} & {K^{\times}}
% 	\arrow[from=1-1, to=1-2]
% 	\arrow[hook, from=1-2, to=1-3]
% 	\arrow["\rho", from=1-3, to=1-4]
% 	\arrow["{\norm_{E/K}}", from=1-4, to=1-5]
%     \end{tikzcd}\]
%     is exact, where $\rho(z)=\sigma(z)/z$. 
%      \item The sequence 
%     \[\begin{tikzcd}
% 	0 & {K} & {E} & {E} & {K}
% 	\arrow[from=1-1, to=1-2]
% 	\arrow[hook, from=1-2, to=1-3]
% 	\arrow["\lambda", from=1-3, to=1-4]
% 	\arrow["{\trace{E/K}}", from=1-4, to=1-5]
%     \end{tikzcd}\]
%     is exact, where $\lambda(z)=\sigma(z)-z$. 
% \end{enumerate}
% \end{corollary}

% \begin{proof}
%     We only prove 1). Note that 
%     $K^{\times}\hookrightarrow E^{\times}$ is the inclusion map. If 
%     $z\in\ker\rho$, then $\sigma(z)=z$ and hence $z\in K$. The 
%     sequence 
%     \[\begin{tikzcd}
% 	{E^{\times}} & {E^{\times}} & {K^{\times}}
% 	\arrow["\rho", from=1-1, to=1-2]
% 	\arrow["{\norm_{E/K}}", from=1-2, to=1-3]
%     \end{tikzcd}\]
%     is exact by Hilbert's theorem 90. 
% \end{proof}

% % 107/136