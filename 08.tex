\chapter{}



\topic{Galois' correspondence}

\begin{theorem}[Galois]
\index{Galois' theorem}
    Let $E/K$ be a finite Galois extension and $G=\Gal(E/K)$. 
    There exists a bijective correspondence
    \[
    \{F:K\subseteq F\subseteq E\text{ subfields}\}\leftrightarrow
    \{S:S\text{ is a subgroup of $G$}\}
    \]
    The correspondence is given by $F\mapsto \Gal(E/F)$ and 
    $\prescript{S}{}{E}\mapsfrom S$. Moreover, 
    normal subextensions of $E/K$ correspond 
    to normal subgroups of $G$. 
\end{theorem}

\[
\begin{tikzcd}
	&& E \\
	& F && {\{1\}} \\
	K && S \\
	& G
	\arrow[no head, from=1-3, to=2-2]
	\arrow[no head, from=2-2, to=3-1]
	\arrow[no head, from=3-1, to=4-2]
	\arrow[no head, from=4-2, to=3-3]
	\arrow[no head, from=2-2, to=3-3]
	\arrow[no head, from=1-3, to=2-4]
	\arrow[no head, from=2-4, to=3-3]
\end{tikzcd}
\]

\begin{proof}
Let $\alpha$ and $\beta$ be the maps $\alpha(F)=\Gal(E/F)$ and $\beta(S)=\prescript{S}{}{E}$. A routine
exercise shows that $\alpha$ and $\beta$ are well-defined. 
We first note that
\begin{align*}
   &\beta(\alpha(F))=\beta(\Gal(E/F))=\prescript{\Gal(E/F)}{}{E}=F
\end{align*}
since $E/F$ is a Galois extension. Moreover,
\begin{align*}
   &\alpha(\beta(S))=\alpha(\prescript{S}{}{E})=\Gal(E/\prescript{S}{}{E})=S
\end{align*}
by Artin's theorem, as $S$ is finite. 

Let $F$ be a subfield of $E$ containing $K$ and 
$S=\alpha(F)$. Then
\[
[F:K]=\frac{[E:K]}{[E:F]}=\frac{|G|}{|S|}=(G:S).
\]

Let $C$ be an algebraic closure of $K$ that contains $E$. 
If $S=\Gal(E/F)$, then $F=\prescript{S}{}{E}$. 

We need to prove that $F/K$ is normal if and only if $S$ is normal in $G$. 
Let us first prove $\implies$. Let $\tau\in S$ and $\sigma\in G$. Since
$F/K$ is normal, $\sigma|_F\in\Aut(F)$. Thus $\sigma^{-1}(F)=F$. In particular, 
if $x\in F$, then $\sigma^{-1}(x)\in F$ and 
\[
\sigma\tau\sigma^{-1}(x)=\sigma\sigma^{-1}(x)=x.
\]
Conversely, let $\varphi\in\Hom(F/K,C/K)$. There exists 
$\Phi\colon E\to C$ such that $\Phi|_F=\varphi$. Since $E/K$ is normal, 
$\Phi(E)=E$ and hence $\Phi\in G$. We claim that $\varphi(x)\in F$ for all $x\in F$. 
Note that $F=\prescript{S}{}{E}$, so
 \[
 \tau\varphi(x)=\tau\Phi(x)=\Phi\Phi^{-1}\tau\Phi(x)=\Phi(x)=\varphi(x)
 \]
 for all $\tau\in S$, as $\Phi^{-1}\tau\Phi\in S$. This means that $\varphi(x)\in\prescript{S}{}{E}=F$. 
 
 Let us compute $\Gal(F/K)$. Since $F/K$ is normal, 
 the map 
 $\lambda\colon G\to\Gal(F/K)$, $\sigma\mapsto\sigma|_F$, 
 is a surjective group homomorphism such that $\ker\lambda=S$. The first isomorphism 
 theorem implies that $\Gal(F/K)\simeq G/S$. 
\end{proof}

Some easy consequences.

\begin{exercise}
    If $E/K$ is a Galois extension of degree $n$ and
    $p$ is a prime number dividing $n$, then $E/K$ admits
    a subextension of degree $n/p$. 
\end{exercise}
    
\begin{exercise}
    If $E/K$ is a Galois extension of degree $p^\alpha m$ with
    $p$ a prime number coprime with $m$, then $E/K$ admits 
    a subextension of degree $m$. 
    %This follows from Sylow's theorem
    %and Galois's theorem.
\end{exercise}

\begin{definition}
\index{Extension!abelian}
    An extension $E/K$ is \textbf{abelian} if $E/K$ is a Galois extension
    with $\Gal(E/K)$ abelian.
\end{definition}

\begin{exercise}
    If $E/K$ is an abelian extension of degree $n$ and $d$ divides
    $n$, then $E/K$ admits a subextension of degree $d$. 
\end{exercise}

\begin{definition}
    \index{Extension!cyclic}
    An extension $E/K$ is \textbf{cyclic} if $E/K$ is 
    a Galois extension with $\Gal(E/K)$ cyclic. 
\end{definition}

\begin{example}
    The extension $\Q(\sqrt{2},\sqrt{3})/\Q$ admits
    exactly three non-trivial subextensions: 
    \[
    \Q(\sqrt{2})/\Q,
    \quad
    \Q(\sqrt{3})/\Q,
    \quad 
    \Q(\sqrt{6})/\Q,
    \]
    as $\Gal(\Q(\sqrt{2},\sqrt{3})/Q)\simeq C_2\times C_2$. 
    % Note that if $\sigma\in\Gal(\Q(\sqrt{2},\sqrt{3})/Q)$, then
    % $\sigma(\sqrt{2})\in\{\sqrt{2},-\sqrt{2}\}$ and
    % $\sigma(\sqrt{3})\in\{\sqrt{3},-\sqrt{3}\}$.
\end{example}

\begin{example}
    Let $\omega\in\C\setminus\{1\}$ be such that $\omega^5=1$.
    Then 
    \[
    f(\omega,\Q)=1+X+X^2+X^3+X^4
    \]
    and $\Q(\omega)/\Q$ has
    degree four. 
    Moreover, $\Q(\omega)/\Q$ is a Galois extension
    and 
    $\Gal(\Q(\omega)/\Q)\simeq C_4$. If $\sigma\in \Gal(\Q(\omega)/\Q)$,
    then $\sigma(\omega)=\omega^i$ for some $i\in\{1,\dots,4\}$. 
    Moreover, for every $i\in\{1,\dots,4\}$ 
    the map $\omega\mapsto\omega^i$ induces an automorphism
    of $\Q(\omega)/\Q$. Thus $|\Gal(\Q(\omega)/\Q)|=4$. Now 
    \[
    \sigma_i^k=\id\Longleftrightarrow
    \omega^{i^k}=\sigma_i^k(\omega)=\omega\Longleftrightarrow
    i^k\equiv1\bmod 5.
    \]
    Thus the map $\sigma_2$ given 
    by $\omega\mapsto\omega^2$ has order four. 
    
    Since $\Gal(\Q(\omega)/\Q)=\langle\sigma\rangle$,
    where $\sigma(\omega)=\omega^2$, 
    is cyclic of order four, 
    the extension $\Q(\omega)/\Q$ has a unique degree-two 
    subtextension $F/\Q$. Note that $|\langle\sigma^2\rangle|=2$ 
    and $\sigma^2(\omega)=\omega^4=\omega^{-1}$. Thus 
    $F=\prescript{\langle\sigma^2\rangle}{}{\Q(\omega)}$. Let 
    $\theta=\omega+\omega^{-1}$. Then 
    \[
    \theta^2=\omega^2+\omega^3+2=-(1+\omega+\omega^{-1})+2=1-\theta
    \]
    and hence $\theta$ is a root of $X^2+X-1$. It follows that 
    \[
    \theta\in\{(-1+\sqrt{5})/2,(-1-\sqrt{5})/2\}.
    \]
    Therefore
    $F=\Q(\sqrt{5})$. 
\end{example}

Let us mention some other consequences.

\begin{exercise}
\label{xca:composite}
    Let $E/K$ be a finite Galois extension 
    and $F_1,\dots,F_n$ fields 
    such that $K\subseteq F_i\subseteq E$ for 
    all $i\in\{1,\dots,n\}$. For every 
    $i$ let $S_i=\Gal(E/F_i)$. Then
    \[
    \Gal\left(E/\bigcap_{i=1}^nF_i\right)=\left\langle\bigcup_{i=1}^nS_i\right\rangle,
    \quad
    \Gal\left(E/\prod_{i=1}^nF_i\right)=\bigcap_{i=1}^nS_i.
    \]
\end{exercise}

% \begin{sol}{xca:composite}
%     We solve the exercise for $n=2$. The general case 
%     follows by induction. 
    
%     Let us prove 
%     first that
%     $\Gal(E/F)\cap\Gal(E/L)=\Gal(E/FL)$:
%     \begin{align*}
%         \sigma\in\Gal(E/F)\cap\Gal(E/L) & \Longleftrightarrow
%         \sigma|_F=\id_F\text{ and }\sigma|_L=\id_L\\
%         &\Longleftrightarrow \sigma|_{FL}=\id_{FL}\\
%         &\Longleftrightarrow \sigma\in\Gal(E/FL).
%     \end{align*}
    
%     Now we prove that $\Gal(E/F\cap L)\simeq\langle\Gal(E/F)\cup\Gal(E/L)\rangle$. Let 
%     $S=\Gal(E/F)$ and $T=\Gal(E/L)$. Then
%     \begin{align*}
%         x\in \prescript{S}{}{E}\cup \prescript{T}{}{E} &\Longleftrightarrow \sigma(x)=x\text{ for all $\sigma\in S\cup T$}\\
%         &\Longleftrightarrow \sigma(x)=x\text{ for all $\sigma\in\langle S\cup T\rangle$}\\
%         &\Longleftrightarrow x\in \prescript{\langle S\cup T\rangle}{}{E}.
%     \end{align*}
% \end{sol}

The following statement is a concrete application of the 
previous exercise.

\begin{exercise}
    Let $E/K$ be a finite Galois extension and $G=\Gal(E/K)$.
    Assume that $G$ is the direct product
    $G=S\times T$
    of the groups $S$ and $T$. Let 
    $F=\prescript{S}{}{E}$ and
    $L=\prescript{T}{}{E}$. Then $F\cap L=K$ and $FL=E$.
\end{exercise}

\begin{proposition}
Let $E_1/K,\dots,E_r/K$ be Galois extensions. 
If $E=\prod_{i=1}^rE_i$, then $E/K$ is a Galois extension. If, moreover, each $E_i/K$ is finite,
then 
\[
\theta\colon \Gal(E/K)\to \Gal(E_1/K)\times\cdots\times\Gal(E_r/K),
\quad
\sigma\mapsto(\sigma|_{E_1},\dots,\sigma|_{E_r}),
\]
is an injective group homomorphism.
\end{proposition}

\begin{proof}
    We only do the first part in the case $r=2$, the general case is left as an exercise. Since $E_1/K$ is algebraic, 
    then $E_1E_2/E_2$ is algebraic. Since $E_2/K$ is algebraic, $E_1E_2/K$ is algebraic. Similarly, 
    $E_1E_2/K$ is separable. 
    
    Let $C/K$ be an algebraic closure such that $E_1E_2\subseteq C$. If $\sigma\in\Hom(E_1E_2/K,C/K)$, then 
    $\sigma(E_1E_2)\subseteq\sigma(E_1)\sigma(E_2)=E_1E_2$ (do this calculation as an exercise using the fact that
    $E_1/K$ and $E_2/K$ are normal extensions). 
    Thus $E_1E_2/K$ is normal. 
    
    If both $E_1/K$ and $E_2/K$ are finite, then $E_1E_2/K$ is finite. 
    
    Then $\theta$ is a group homomorphism. We claim that the map $\theta$ is injective. Let $\sigma\in\ker\theta$. Then
    $\sigma|_{E_i}=\id_{E_i}$ for all $i\in\{1,\dots,r\}$. Let $S=\langle\sigma\rangle\subseteq\Gal(E/K)$ and
    $F=\prescript{S}{}{E}$. Then $E_i\subseteq F$ for all $i\in\{1,\dots,r\}$ and
    hence $E\subseteq F$. It follows that $F=E=\prescript{\{\id\}}{}{E}$ and therefore $S=\{\id\}$, so 
    $\sigma=\id$. 
\end{proof}

\begin{exercise}
    Let $E_1/K,\dots,E_r/K$ be finite Galois extensions such that for each $j$ 
    one has $E_j\cap (E_1\cdots E_{j-1}E_{j+1}\cdots E_r)=K$. Then 
    \[
    \Gal(E/K)\simeq\Gal(E_1/K)\times\cdots\times\Gal(E_r/K).
    \]
    In this case, $[E:K]=\prod_{i=1}^r[E_i:K]$. 
\end{exercise}

\topic{The fundamental theorem of algebra}

We now present an easy proof of the fundamental theorem 
of algebra based on the ideas of Galois Theory. 
We need the following well-known facts:
\begin{enumerate}
\item Every real polynomial of odd degree admits a real root. This means that $\R$ 
does not admit extension of odd degree $>1$. 
\item Every complex number admits a square root in $\C$. This means that $\C$ 
does not admit degree-two extensions.
\end{enumerate}

\begin{theorem}
The field $\C$ is algebraically closed.
\end{theorem}

\begin{proof}
    Let $E/\C$ be an algebraic finite extension. Then $E/\R$ 
    is finite separable extension of even degree. There exists a Galois
    extension 
    $L/\R$ such that $E\subseteq L$, so $[L:\R]$ is even. Let $G=\Gal(L/\R)$. 
    Then $|G|=2^ms$ for some odd number $s$. If $T$ is a 2-Sylow subgroup
    of $G$, 
    then there exists a subextension $F/\R$ of degree $s$. Since 
    $\R$ does not admit extensions of odd degree $>1$, $s=1$ and
    hence $G$ is a $2$-group. Since 
    $L/\R$ is a Galois extension, $L/\C$ is a Galois extension. 
    In particular, $|\Gal(L/\C)|=2^{m-1}$. If $m>1$, 
    let $U$ be a subgroup of $\Gal(L/\C)$ of order $2^{m-2}$. Then $U$ corresponds 
    to a subextension $L_1/\C$ of degree two, a contradiction. Hence $m=1$ 
    and $[L:\C]=1$, so $L=\C$ and $E=\C$. 
\end{proof}

\topic{Purely inseparable extensions}

Let $E/K$ be an algebraic extension. 
In page \ref{separable} we defined the 
\textbf{separable closure} of $K$ with respect to $E$ as 
the field 
\[
    F=\{x\in E:x\text{ is separable over }K\}.
\]
Note that $K\subseteq F\subseteq E$ 
and $F=K(F)$. Moreover, 
$F/K$ is separable and 
$E/F$ is a \textbf{purely inseparable} extension, meaning that
for every $x\in E\setminus F$, the polynomial $f(x,F)$ is not separable. 

The number $[E:F]$ is known as the \textbf{degree of inseparability} of $E/K$. 
We write $[E:K]_{\operatorname{ins}}=[E:F]$.
Clearly, $E/K$ is separable if and only if $[E:K]_{\operatorname{ins}}=1$ and 
$E/K$ is purely inseparable if and only if $[E:K]_{\operatorname{ins}}=[E:K]$. 

\begin{proposition}
Let $K$ be a field of characteristic $p>0$ and
$E/K$ be an algebraic extension. The following statements are equivalent:
\begin{enumerate}
    \item $E/K$ is purely inseparable.
    \item If $x\in E$, then $x^{p^m}\in K$ for some $m\geq0$.
    \item If $x\in E$, then $f(x,K)=X^{p^m}-a$ for some $a\in K$ and $m\geq0$. 
    \item $\gamma(E/K)=1$. 
\end{enumerate}
\end{proposition}

\begin{proof}
    We first prove $1)\implies 2)$. 
    Let $x\in E$ and $f=f(x,K)$. Assume $x$ is not separable. Then
    $f(x)=0$ and $f'(x)=0$, as $x$ is not a simple root. Since $\deg f'<\deg f$
    and $f$ is the minimal polynomial of $x$, it follows that 
    $f'=0$. The coefficients of $f'$ are of the form $ka_k$. 
    Since $E$ is a field, $a_k=0$ if $k$ is not divisible by $p$. If $a_k\ne 0$, 
    then $k=pm$ for some $m\geq0$. It follows that $f=g(X^p)$
    for some $g\in K[X]$ with $\deg g<\deg f$. 
%Since $f\in K[X]$, 
%the set
%$\left\{k\geq 0:f\in K\left[X^{p^k}\right]\right\}$ is non-empty
%and hence 
%\[
%m=\max\left\{k\geq 0:f\in K\left[X^{p^k}\right]\right\}
%\]
%is well-defined. Thus $f\in K[X^{p^m}]$. This means that 
%there exists a polynomial $h$ such that $\deg f=p^m\deg h$, so
%$p^m$ divides $\deg f$. In conclusion, $f=g(X^p)$ 
%for some $g\not\in K[X^p]$. 
    We now proceed by induction on the degree of $x$. The result
    is true for elements of degree one. So assume the result holds for the element of degree $\leq n$ 
    for some $n\geq1$. 
    If $x\in E$ is such that $\deg f(x,K)=n+1$, then, since $f(x,K)=g(X^p)$, the element 
    $x^p$ has degree $\leq n$. By the inductive hypothesis, $x^{p^{m+1}}=(x^p)^{p^m}\in K$.  

    We now prove $2)\implies 3)$. Let $x\in E$ and $m$ be the minimal positive integer 
    such that $x^{p^m}\in K$. Then
    $x$ is a root of $X^{p^m}-x^{p^m}\in K[X]$. Since 
    $X^{p^m}-x^{p^m}=(X-x)^{p^m}$, it follows that 
    \[
    f(x,K)=(X-x)^r=X^r+\cdots+(-1)^rx^r
    \]
    for some
    $r\in\{1,\dots,p^m\}$. Write $r=p^st$ for some integer $t$ coprime with $p$ and $s$ such that
    $0\leq s\leq m$. Let $a,b\in\Z$ be such that $ar+bp^m=p^s$. Then 
    \[
    x^{p^s}=x^{ar+bp^m}=\left(x^r\right)^a\left(x^{p^m}\right)^b\in K.
    \]
    The minimality of $m$ implies that $s\geq m$ and hence $s=m$. Now $p^mt=p^st=r\leq p^m$, so $t=1$. 
    This means $f(x,K)=X^{p^m}-x^{p^m}$. 
    
    We now prove $3)\implies 4)$. Let $C/K$ be an algebraic closure that contains $E$
    and $\sigma\in\Hom(E/K,C/K)$. Let $x\in E$. We claim that $\sigma(x)=x$. Since 
    $f(x,K)=X^{p^m}-a$, 
    \[
    \left(\sigma(x)\right)^{p^m}=\sigma\left(x^{p^m}\right)=\sigma(a)=a=x^{p^m}.
    \]
    It follows that $\sigma(x)$ is a root of $X^{p^m}-x^{p^m}=(X-x)^{p^m}$. 
    Thus $\sigma(x)=x$. 
    
    Finally, we prove that $4)\implies1)$. Let 
    $C$ be an algebraic closure of $K$ containing $E$. 
    Then $\Gal(E/K)=\Hom(E/K,C/K)=\{\id\}$, as $\gamma(E/K)=1$. 
    If $x\in E$ is separable over $K$, then
    \[
    f(x,K)=\prod_{y\in O_{\Gal(E/K)}(x)}(X-y)=X-x\in\ K[X].
    \]
    Thus $x\in K$ and hence $E/K$ is purely inseparable. 
\end{proof}

Some consequences:

\begin{exercise}
    Let $K$ be a field of characteristic $p>0$ and 
    $E/K$ be finite and purely inseparable. Then $[E:K]=p^s$ for some prime number $p$ and some $s$.
    Moreover, $x^{[E:K]}\in K$. 
\end{exercise}

For the first part of the previous exercise, write $E=K(x_1,\dots,x_n)$ and proceed by induction on $n$. 

\begin{exercise}
    Let $K$ be of characteristic $p>0$ and 
    $E/K$ be a finite extension such that $[E:K]$ is not divisible by $p$. Then 
    $E/K$ is separable. 
\end{exercise}

Let $K$ be of characteristic $p>0$, $E/K$ be finite and $F$ be the separable closure of $K$ in $E$. 
Since 
\begin{gather*}
\gamma(E/K)=\gamma(E/F)\gamma(F/K)=\gamma(F/K),
\shortintertext{it follows that}
[E:K]=[E:F]\gamma(E/K)=[E:K]_{\operatorname{ins}}\gamma(E/K).
\end{gather*}

\topic{Norm and trace}

\begin{definition}
\index{Trace}
\index{Norm}
    Let $E/K$ be a finite extension and $C/K$ be an algebraic closure 
    that contains $E$. Let $A=\Hom(E/K,C/K)$. For $x\in E$
    we define the \textbf{trace} of $x$ in $E/K$ 
    as 
    \[
    \trace_{E/K}(x)=[E:K]_{\operatorname{ins}}\sum_{\varphi\in A}\varphi(x)
    \]
    and the \textbf{norm} of $x$ in $E/K$ as
    \[
    \norm_{E/K}(x)=\left(\prod_{\varphi\in A}\varphi(x)\right)^{[E:K]_{\operatorname{ins}}}.
    \]
\end{definition}

As an optional exercise, one can show that these definitions do not depend on the algebraic closure. 

We collect some basic properties as an exercise:

\begin{exercise}
\label{xca:norm_and_trace}
    Let $E/K$ be a finite extension. The following statements hold:
    \begin{enumerate}
        \item If $E/K$ is not separable, then $\trace_{E/K}(x)=0$ for all $x\in E$.
        \item If $x\in K$, then $\trace_{E/K}(x)=[E:K]x$.
        \item $\trace_{E/K}(x)\in K$ for all $x\in E$.
        \item $\norm_{E/K}(x)=0$ if and only if $x=0$. 
        \item If $x\in K$, then $\norm_{E/K}(x)=x^{[E:K]}$. 
        \item $\norm_{E/K}(x)\in K$ for all $x\in E$. 
    \end{enumerate}
\end{exercise}

One proves, moreover, that  
$\trace_{E/K}\colon E\to K$ 
satisfies
\[
\trace_{E/K}(x+\lambda y)=
\trace_{E/K}(x)+\lambda\trace_{E/K}(y)
\]
for all $x,y\in E$ and $\lambda\in K$, that is to say that 
$\trace_{E/K}\colon E\to K$ 
is a 
linear form in $E$ The norm  
$\norm_{E/K}\colon E^{\times}\to K^{\times}$ 
is a group homomorphism. 

\begin{exercise}
        Let $E/K$ be a finite extension and
        $x\in E$. If
        \[
        f(x,K)=X^n+a_{n-1}X^{n-1}+\cdots+a_1X+a_0,
        \]
        then 
        $\norm_{E/K}(x)=\left((-1)^na_0\right)^{[E:K(x)]}$ and 
        $\trace_{E/K}(x)=-[E:K(x)]a_{n-1}$. 
\end{exercise}

\begin{example}
    Let $E=\Q(\sqrt{2},\sqrt{3})$. Then 
    \begin{align*}
    &\trace_{E/\Q}(\sqrt{2})=0,
    &&
    \norm_{E/\Q}(\sqrt{2})=4,\\
    &\trace_{E/\Q(\sqrt{2})}(\sqrt{2})=2\sqrt{2},
    &&\norm_{E/\Q(\sqrt{2})}(\sqrt{2})=2.    
    \end{align*}
\end{example}

\begin{example}
    If $E/K$ is a finite Galois extension, then 
    \[
    \trace_{E/K}(x)=\sum_{\sigma\in\Gal(E/K)}\sigma(x)
    \quad\text{and}\quad
    \trace_{E/K}(x)=\prod_{\sigma\in\Gal(E/K)}\sigma(x)
    \]
    for all $x\in E$. In particular, since $E=K(y)$ for some
    $y$ by Proposition \ref{pro:monogenic}, 
    \[
    \trace_{E/K}(y)=-a_{n-1}
    \quad\text{and}\quad
    \norm_{E/K}(y)=(-1)^na_0,
    \]
    where
    $f(y,K)=X^n+a_{n-1}X^{n-1}+\cdots+a_1X+a_0$.
\end{example}        
