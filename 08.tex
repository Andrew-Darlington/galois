\chapter{}



\begin{theorem}[Artin]
\index{Artin's theorem}
\label{thm:ArtinGalois}
    Let $E$ be a field and $G$ be a finite group of automorphisms of $E$. 
    If $K=\prescript{G}{}{E}$, then $E/K$ is a Galois extension,
    $[E:K]=|G|$ and $\Gal(E/K)=G$. 
\end{theorem}

Before proving the theorem, we need a lemma.

\begin{lemma}
    Let $E/K$ be a separable extension such that $\deg(x,K)\leq m$
    for all $x\in E$. Then $E/K$ is finite and $[E:K]\leq m$. 
\end{lemma}

\begin{proof}
   Let $z\in E$ be of maximal degree. If $x\in E$, 
   then $K(x,z)/K$ is separable. Then $K(x,z)=K(y)$ for some $y$. 
   It follows that 
   \[
   K(z)\subseteq K(x,z)=K(y).
   \]
   Since 
   $\deg(z,K)\leq\deg(y,K)$, 
   $\deg(z,K)=\deg(y,K)$. Hence 
   $K(y)=K(z)$. In particular, $x\in K(z)$ and
   therefore $E=K(z)$. 
\end{proof}

Now we are ready to prove Artin's theorem: 

\begin{proof}[Proof of Theorem \ref{thm:ArtinGalois}]
    Note that $G\subseteq\Gal(E/K)$. Let $x\in E$ and 
    \[
    f_x=\prod_{y\in O_G(x)}(X-y).
    \]
    Since $f_x\in K[X]$, it follows
    that the extension $E/K$ is normal and separable (as it is a decomposition
    field of a family of separable polynomials), so $E/K$ is a Galois extension. Moreover, 
    \[
    \deg(x,K)\leq \deg f_x=|O_G(x)|\leq |G|.
    \]
    By the previous lemma, $E/K$ is finite and $[E:K]\leq |G|$. This
    implies that
    $|G(E/K)|=[E:K]\leq |G|$ and hence $|G(E/K)|=|G|$. 
\end{proof}

\begin{example}
    Let $E=K(X,Y)$ and $\sigma\colon K[X,Y]\to E$ be the ring homomorphism given by $\sigma(X)=Y$ and $\sigma(Y)=X$. Note that $\sigma$ is bijective, as $\sigma^2=\id$. The map $\sigma$ induces
    a field homomorphism $\overline{\sigma}\colon E\to E$ such that 
    $\overline{\sigma}^2=\id$. Recall that such a homomorphism is given by 
    $f/g\mapsto \sigma(f)/\sigma(g)$. Let $G=\langle\overline{\sigma}\rangle$. Then $|G|=2$. 
    We claim that $\prescript{G}{}{E}=K(X+Y,XY)$. Let $F=K(X+Y,XY)$. We only prove
    that $\prescript{G}{}{E}\subseteq F$, as the other inclusion is trivial. Artin's theorem
    implies that $[E:\prescript{G}{}{E}]=2$ and $E=F(X)$, as $X$ is a root
    of the polynomial $Z^2-(X+Y)Z+XY$. Then $[E:F]\leq 2$ and $[\prescript{G}{}{E}:F]=1$.
\end{example}

\topic{Galois' correspondence}

\begin{theorem}[Galois]
\index{Galois' theorem}
    Let $E/K$ be a finite Galois extension and $G=\Gal(E/K)$. 
    There exists a bijective correspondence
    \[
    \{F:K\subseteq F\subseteq E\text{ subfields}\}\leftrightarrow
    \{\text{subgroups of $G$}\}
    \]
    The correspondence is given by $F\mapsto G(E/F)$ and 
    $\prescript{S}{}{E}\mapsfrom S$. Moreover, 
    normal subextensions of $E/K$ correspond 
    to normal subgroups of $G$. 
\end{theorem}

\[
\begin{tikzcd}
	&& E \\
	& F && {\{1\}} \\
	K && S \\
	& G
	\arrow[no head, from=1-3, to=2-2]
	\arrow[no head, from=2-2, to=3-1]
	\arrow[no head, from=3-1, to=4-2]
	\arrow[no head, from=4-2, to=3-3]
	\arrow[no head, from=2-2, to=3-3]
	\arrow[no head, from=1-3, to=2-4]
	\arrow[no head, from=2-4, to=3-3]
\end{tikzcd}
\]

\begin{proof}
Let $\alpha$ and $\beta$ be the maps $\alpha(F)=\Gal(E/F)$ and $\beta(S)=\prescript{S}{}{E}$. A routine
exercise shows that $\alpha$ and $\beta$ are well-defined. 
We first note that
\begin{align*}
   &\beta(\alpha(F))=\beta(\Gal(E/F))=\prescript{\Gal(E/F)}{}{E}=F
\end{align*}
since $E/F$ is a Galois Extension. Moreover,
\begin{align*}
   &\alpha(\beta(S))=\alpha(\prescript{S}{}{E})=\Gal(E/\prescript{S}{}{E})=S
\end{align*}
by Artin's theorem, as $S$ is finite. 

Let $F$ be a subfield of $E$ containing $K$ and 
$S=\alpha(F)$. Then
\[
[F:K]=\frac{[E:K]}{[E:F]}=\frac{|G|}{|S|}=(G:S).
\]

Let $C$ be an algebraic closure of $K$ that contains $E$. 
If $S=\Gal(E/F)$, then $F=\prescript{S}{}{E}$. 

We need to prove that $F/K$ is normal if and only if $S$ is normal in $G$. 
Let us first prove $\implies$. Let $\tau\in S$ and $\sigma\in G$. Since
$F/K$ is normal, $\sigma|_F\in\Aut(F)$. Thus $\sigma^{-1}(F)=F$. In particular, 
if $x\in F$, then $\sigma^{-1}(x)\in F$ and 
\[
\sigma\tau\sigma^{-1}(x)=\sigma\sigma^{-1}(x)=x.
\]
Conversely, let $\varphi\in\Hom(F/K,C/K)$. There exists 
$\Phi\colon E\to C$ such that $\Phi|_F=\varphi$. Since $E/K$ is normal, 
$\Phi(E)=E$ and hence $\Phi\in G$. We claim that $\varphi(x)\in F$ for all $x\in F$. 
Note that $F=\prescript{S}{}{E}$, so
 \[
 \tau\varphi(x)=\tau\Phi(x)=\Phi\Phi^{-1}\tau\Phi(x)=\Phi(x)=\varphi(x)
 \]
 for all $\tau\in S$, as $\Phi^{-1}\tau\Phi\in S$. This means that $\varphi(x)\in\prescript{S}{}{E}=F$. 
 
 Let us compute $\Gal(F/K)$. Since $F/K$ is normal, 
 the map 
 $\lambda\colon G\to\Gal(F/K)$, $\sigma\mapsto\sigma|_F$, 
 is a surjective group homomorphism such that $\ker\lambda=S$. The first isomorphism 
 theorem implies that $\Gal(F/K)\simeq G/S$. 
\end{proof}

Some easy consequences.

\begin{exercise}
    If $E/K$ is a Galois extension of degree $n$ and
    $p$ is a prime number dividing $n$, then $E/K$ admits
    a subextension of degree $n/p$. 
\end{exercise}
    
\begin{exercise}
    If $E/K$ is a Galois extension of degree $p^\alpha m$ with
    $p$ a prime number coprime with $m$, then $E/K$ admits 
    a subextension of degree $m$. 
    %This follows from Sylow's theorem
    %and Galois's theorem.
\end{exercise}

\begin{definition}
\index{Extension!abelian}
    An extension $E/K$ is \textbf{abelian} if $E/K$ is a Galois extension
    with $\Gal(E/K)$ abelian.
\end{definition}

\begin{exercise}
    If $E/K$ is an abelian extension of degree $n$ and $d$ divides
    $n$, then $E/K$ admits a subextension of degree $d$. 
\end{exercise}

\begin{definition}
    \index{Extension!cyclic}
    An extension $E/K$ is \textbf{cyclic} if $E/K$ is 
    a Galois extension with $\Gal(E/K)$ cyclic. 
\end{definition}

\begin{example}
    The extension $\Q(\sqrt{2},\sqrt{3})/\Q$ admits
    exactly three non-trivial subextensions: 
    \[
    \Q(\sqrt{2})/\Q,
    \quad
    \Q(\sqrt{3})/\Q,
    \quad 
    \Q(\sqrt{6})/\Q,
    \]
    as $\Gal(\Q(\sqrt{2},\sqrt{3})/Q)\simeq C_2\times C_2$. 
    % Note that if $\sigma\in\Gal(\Q(\sqrt{2},\sqrt{3})/Q)$, then
    % $\sigma(\sqrt{2})\in\{\sqrt{2},-\sqrt{2}\}$ and
    % $\sigma(\sqrt{3})\in\{\sqrt{3},-\sqrt{3}\}$.
\end{example}

\begin{example}
    Let $\omega\in\C\setminus\{1\}$ be such that $\omega^5=1$.
    Then 
    \[
    f(\omega,\Q)=1+X+X^2+X^3+X^4
    \]
    and $\Q(\omega)/\Q$ has
    degree four. 
    Moreover, $\Q(\omega)/\Q$ is a Galois extension
    and 
    $\Gal(\Q(\omega)/\Q)\simeq C_4$. If $\sigma\in \Gal(\Q(\omega)/\Q)$,
    then $\sigma(\omega)=\omega^i$ for some $i\in\{1,\dots,4\}$. 
    Moreover, for every $i\in\{1,\dots,4\}$ 
    the map $\omega\mapsto\omega^i$ induces an automorphism
    of $\Q(\omega)/\Q$. Thus $|\Gal(\Q(\omega)/\Q)|=4$. Now 
    \[
    \sigma_i^k=\id\Longleftrightarrow
    \omega^{i^k}=\sigma_i^k(\omega)=\omega\Longleftrightarrow
    i^k\equiv1\bmod 5.
    \]
    Thus the map $\sigma_2$ given 
    by $\omega\mapsto\omega^2$ has order four. 
    
    Since $\Gal(\Q(\omega)/\Q)=\langle\sigma\rangle$,
    where $\sigma(\omega)=\omega^2$, 
    is cyclic of order four, 
    the extension $\Q(\omega)/\Q$ has a unique degree-two 
    subtextension $F/\Q$. Note that $|\langle\sigma^2\rangle|=2$ 
    and $\sigma^2(\omega)=\omega^4=\omega^{-1}$. Thus 
    $F=\prescript{\langle\sigma^2\rangle}{}{\Q(\omega)}$. Let 
    $\theta=\omega+\omega^{-1}$. Then 
    \[
    \theta^2=\omega^2+\omega^3+2=-(1+\omega+\omega^{-1})+2=1-\theta
    \]
    and hence $\theta$ is a root of $X^2+X-1$. Since $\theta\not\in\Q$, 
    it follows that 
    \[
    \theta\in\{(-1+\sqrt{5})/2,(-1-\sqrt{5})/2\}.
    \]
    Therefore
    $F=\Q(\sqrt{5})$. 
\end{example}

Let us mention some other consequences.

\begin{exercise}
    Let $E/K$ be a finite Galois extension 
    and $F_1,\dots,F_n$ fields 
    such that $K\subseteq F_i\subseteq E$ for 
    all $i\in\{1,\dots,n\}$. For every 
    $i$ let $S_i=\Gal(E/F_i)$. Then
    \[
    \Gal\left(E/\bigcap_{i=1}^nF_i\right)=\left\langle\bigcup_{i=1}^nS_i\right\rangle,
    \quad
    \Gal\left(E/\prod_{i=1}^nF_i\right)=\bigcap_{i=1}^nS_i.
    \]
\end{exercise}

The following statement is a concrete application of the 
previous exercise.

\begin{exercise}
    Let $E/K$ be a finite Galois extension and $G=\Gal(E/K)$.
    Assume that $G$ is the direct product
    $G=S\times T$
    of the groups $S$ and $T$. Let 
    $F=\prescript{S}{}{E}$ and
    $L=\prescript{T}{}{E}$. Then $F\cap L=K$ and $FL=E$.
\end{exercise}

\begin{proposition}
Let $E_1/K,\dots,E_r/K$ be Galois extensions. 
If $E=\prod_{i=1}^rE_i$, then $E/K$ is a Galois extension. If, moreover, each $E_i/K$ is finite,
then 
\[
\theta\colon \Gal(E/K)\to \Gal(E_1/K)\times\cdots\times\Gal(E_r/K),
\quad
\sigma\mapsto(\sigma|_{E_1},\dots,\sigma|_{E_r}),
\]
is an injective group homomorphism.
\end{proposition}

\begin{proof}
    We only do the first part in the case $r=2$, the general case is left as an exercise. Since $E_1/K$ is algebraic, 
    then $E_1E_2/E_2$ is algebraic. Since $E_2/K$ is algebraic, $E_1E_2/K$ is algebraic. Similarly, 
    $E_1E_2/K$ is separable. 
    
    Let $C/K$ be an algebraic closure such that $E_1E_2\subseteq C$. If $\sigma\in\Hom(E_1E_2/K,C/K)$, then 
    $\sigma(E_1E_2)\subseteq\sigma(E_1)\sigma(E_2)=E_1E_2$ (do this calculation as an exercise). 
    Thus $E_1E_2/K$ is normal. 
    
    If both $E_1/K$ and $E_2/K$ are finite, then $E_1E_2/K$ is finite. 
    
    Clearly, $\theta$ is a group homomorphism. We claim that the map $\theta$ is injective. Let $\sigma\in\ker\theta$. Then
    $\sigma|_{E_i}=\id_{E_i}$ for all $i\in\{1,\dots,r\}$. Let $S=\langle\sigma\rangle\subseteq\Gal(E/K)$ and
    $F=\prescript{S}{}{E}$. Then $E_i\subseteq F$ for all $i\in\{1,\dots,r\}$ and
    hence $E\subseteq F$. It follows that $F=E=\prescript{\{\id\}}{}{E}$ and therefore $S=\{\id\}$, so 
    $\sigma=\id$. 
\end{proof}

\begin{exercise}
    Let $E_1/K,\dots,E_r/K$ be finite Galois extensions such that for each $j$ 
    one has $E_j\cap (E_1\cdots E_{j-1}E_{j+1}\cdots E_r)=K$. Then 
    \[
    \Gal(E/K)\simeq\Gal(E_1/K)\times\cdots\times\Gal(E_r/K).
    \]
    In this case, $[E:K]=\prod_{i=1}^r[E_i:K]$. 
\end{exercise}

