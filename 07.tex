\chapter{}



\begin{theorem}[Artin]
\index{Artin's theorem}
\label{thm:ArtinGalois}
    Let $E$ be a field and $G$ be a finite group of automorphisms of $E$. 
    If $K=\prescript{G}{}{E}$, then $E/K$ is a Galois extension,
    $[E:K]=|G|$ and $\Gal(E/K)=G$. 
\end{theorem}

Before proving the theorem, we need a lemma.

\begin{lemma}
    Let $E/K$ be a separable extension such that $\deg(x,K)\leq m$
    for all $x\in E$. Then $E/K$ is finite and $[E:K]\leq m$. 
\end{lemma}

\begin{proof}
   Let $z\in E$ be of maximal degree. If $x\in E$, 
   then $K(x,z)/K$ is separable. Then $K(x,z)=K(y)$ for some $y$. 
   It follows that 
   \[
   K(z)\subseteq K(x,z)=K(y).
   \]
   Since 
   $\deg(z,K)\leq\deg(y,K)$, it follows that
   $\deg(z,K)=\deg(y,K)$ and hence 
   $K(y)=K(z)$. In particular, $x\in K(z)$ and
   therefore $E=K(z)$. 
\end{proof}

Now we are ready to prove Artin's theorem: 

\begin{proof}[Proof of Theorem \ref{thm:ArtinGalois}]
    Note that $G\subseteq\Gal(E/K)$. Let $x\in E$ and 
    \[
    f_x=\prod_{y\in O_G(x)}(X-y).
    \]
    Since $f_x\in K[X]$, it follows
    that $E/K$ is normal and separable, so $E/K$ is a Galois extension. Moreover, 
    \[
    \deg(x,K)\leq \deg f_x=|O_G(x)|\leq |G|.
    \]
    By the previous lemma, $E/K$ is finite and $[E:K]\leq |G|$. This
    implies that
    $|G(E/K)|=[E:K]\leq |G|$ and hence $|G(E/K)|=|G|$. 
\end{proof}

\begin{example}
    Let $E=K(X,Y)$ and $\sigma\colon K[X,Y]\to E$ be the ring homomorphism given by $\sigma(X)=Y$ and $\sigma(Y)=X$. Note that $\sigma$ is bijective, as $\sigma^2=\id$. The map $\sigma$ induces
    a field homomorphism $\overline{\sigma}\colon E\to E$ such that 
    $\overline{\sigma}^2=\id$. Recall that such a homomorphism is given by 
    $f/g\mapsto \sigma(f)/\sigma(g)$. Let $G=\langle\overline{\sigma}\rangle$. Then $|G|=2$. 
    We claim that $\prescript{G}{}{E}=K(X+Y,XY)$. Let $F==K(X+Y,XY)$. We only prove
    that $\prescript{G}{}{E}\subseteq F$, as the other inclusion is trivial. Artin's theorem
    implies that $[E:\prescript{G}{}{E}]=2$ and $E=F(X)$, as $X$ is a root
    of the polynomial $Z^2-(X+Y)Z+XY$. Then $[E:F]\leq 2$ and $[\prescript{G}{}{E}:F]=1$.
\end{example}

\topic{Galois' correspondence}

\begin{theorem}[Galois]
\index{Galois' theorem}
    Let $E/K$ be a finite Galois extension and $G=\Gal(E/K)$. 
    There exists a bijective correspondence
    \[
    \{F:K\subseteq F\subseteq E\text{ subfields}\}\to 
    \{\text{subgroups of $G$}\}
    \]
    The correspondence is given by $F\mapsto G(E/F)$ and 
    $\prescript{S}{}{E}\mapsfrom S$. Moreover, 
    normal subextensions of $E/K$ correspond 
    to normal subgroups of $G$. 
\end{theorem}

\[
\begin{tikzcd}
	&& E \\
	& F && {\{1\}} \\
	K && S \\
	& G
	\arrow[no head, from=1-3, to=2-2]
	\arrow[no head, from=2-2, to=3-1]
	\arrow[no head, from=3-1, to=4-2]
	\arrow[no head, from=4-2, to=3-3]
	\arrow[no head, from=2-2, to=3-3]
	\arrow[no head, from=1-3, to=2-4]
	\arrow[no head, from=2-4, to=3-3]
\end{tikzcd}
\]

\begin{proof}
We first note that
\begin{align*}
   &\beta(\alpha(F))=\beta(\Gal(E/F))=\prescript{\Gal(E/F)}{}{E}=F
\end{align*}
since $E/F$ is a Galois Extension. Moreover,
\begin{align*}
   &\alpha(\beta(S))=\alpha(\prescript{S}{}{E})=\Gal(E/\prescript{S}{}{E})=S
\end{align*}
by Artin's theorem, as $S$ is finite. 

Let $F$ be a subfield of $E$ containing $K$ and 
$S=\alpha(F)$. Then
\[
[F:K]=\frac{[E:K]}{[E:F]}=\frac{|G|}{|S|}=(G:S).
\]

Let $C$ be an algebraic closure of $K$ that contains $E$. 
If $S=\Gal(E/F)$, then $F=\prescript{S}{}{E}$. 

We need to prove that $F/K$ is normal if and only if $S$ is normal in $G$. 
Let us first prove $\implies$. Let $\tau\in S$ and $\sigma\in G$. Since
$F/K$ is normal, $\sigma|_F\in\Aut(F)$. Thus $\sigma^{-1}(F)=F$. In particular, 
if $x\in F$, then $\sigma^{-1}(x)\in F$ and 
\[
\sigma\tau\sigma^{-1}(x)=\sigma\sigma^{-1}(x)=x.
\]
Conversely, let $\varphi\in\Hom(F/K,C/K)$. There exists 
$\Phi\in\colon E\to C$ such that $\Phi|_F=\varphi$. Since $E/K$ is normal, 
$\Phi(E)=E$ and hence $\Phi\in G$. We claim that $\varphi(x)\in F$ for all $x\in F$
 for all $x\in F$. Note that $F=\prescript{S}{}{E}$, so
 \[
 \tau\varphi(x)=\tau\Phi(x)=\Phi\Phi^{-1}\tau\Phi(x)=\Phi(x)=\varphi(x)
 \]
 for all $\tau\in S$, as $\Phi^{-1}\tau\Phi\in S$. 
 
 Let us compute $\Gal(F/K)$. Since $F/K$ is normal, 
 the map 
 $\lambda\colon G\to\Gal(F/K)$, $\sigma\mapsto\sigma|_F$, 
 is a surjective group homomorphism such that $\ker\lambda=S$. The first isomortphism 
 theorem implies that $\Gal(F/K)\simeq G/S$. 
\end{proof}
