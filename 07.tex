\chapter{}

\index{Extension!separable}
\label{separable}
If $E/K$ is algebraic, then 
\[
    F=\{x\in E:x\text{ is separable over }K\}
\]
is a subfield of $E$ that contains $K$. It is known as 
the \textbf{separable closure} of $K$ with respect to $E$. 
Note that $F=K(F)$, as
$K(F)$ is separable because it is generated by separable elements. Moreover, 
$F/K$ is separable and 
$E/F$ is a \textbf{purely inseparable} extension, meaning that
for every $x\in E\setminus F$, the polynomial $f(x,F)$ is not separable. 

\begin{proposition}
\label{pro:monogenic}
    If $E/K$ is separable and finite, then $E=K(x)$ for some $x\in E$. 
\end{proposition}

\begin{proof}
    Let us assume that $K$ is finite. Then $E$ is finite and hence 
    the multiplicative group $E^{\times}=E\setminus\{0\}$ 
    is cyclic, say $E^{\times}=\langle x\rangle$. It follows
    that $E=K(x)$. 
    
    Let us now assume that $K$ is infinite. We first consider the case 
    $E=K(x,y)$. The general case $E=K(x_1,\dots,x_n)$ is left as an exercise, one needs to proceed by induction. 
    Let $n=[E:K]$ and 
    $C$ be an algebraic closure of $K$ containing $E$. 
    Write $\Hom(E/K,C/K)=\{\sigma_1,\dots,\sigma_n\}$. Let 
    \[
    f=\prod_{1\leq i<j\leq n}\left((\sigma_i(y)-\sigma_j(y))
    +X(\sigma_i(x)-\sigma_j(x))\right)\in C[X].
    \]
    Then $f\ne 0$, as $f$ is a product of non-zero polynomials. Since $K$ is infinite, 
    there exists a non-zero 
    $c\in K$ such that $f(c)\ne 0$. For any $r,s\in\{1,\dots,n\}$ with 
    $r\ne s$,
    \[
        \sigma_r(y)-\sigma_s(y)+c(\sigma_r(x)-\sigma_s(x))\ne 0,
    \]
    as $f(c)\ne0$. It follows that $\sigma_r(y+cx)\ne\sigma_s(y+cx)$. Thus $\gamma(K(y+cx)/K)\geq n$. 
    Now 
    \[
    n\geq [K(y+cx):K]=\gamma(K(y+cx)/K)\geq n,
    \]
    so $[K(y+cx):K]=n$ and
    hence $K(y+cx)=E$. 
\end{proof}

For example, $\Q(\sqrt{2},i)=\Q(\sqrt{2}+i)$. 

\begin{proposition}
    Let $E/K$ be a finite extension. Then $E=K(x)$ for some $x\in E$ 
    if and only if $E/K$ admits finitely many subextensions. 
\end{proposition}

\begin{proof}
     We may assume that $K$ is infinite; otherwise, the result is trivial. 
    We first prove $\implies$. 
    Let us 
    assume that $E=K(x)$ for some $x$. We claim that the map
    \begin{align*}
    \Psi\colon \{F:K\subseteq F\subseteq E\}&\to\{g\in K[X]:g\text{ is a monic divisor of $f(x,K)$}\},\\
    F&\mapsto f(x,F),
    \end{align*}
    is injective. 
    Take $F_0$ such that $K\subseteq F_0\subseteq F\subseteq E$ and  
    $f(x,F)=f(x,F_0)$. Then  
%    Let $\Psi(F)=g\in F[X]$ and write $g=\sum_{i=0}^ma_iX^i$, where $m=\deg g$. 
%    Thus $a_m=1$. Let $F_0=K(a_0,\dots,a_m)$. Then $F_0\subseteq F$. Since $g=f(x,F)$, the polynomial $g$ is irreducible 
%    in $F[X]$ and hence it is irreducible in $F_0[X]$. Now
    \[
    [E:F_0]=[F_0(x):F_0]=\deg f(x,F_0)=m=[F(x):F]=[E:F]
    \]
    and hence $F=F_0$. It follows that $\Psi$ is injective 
    and therefore there are finitely many fields between $K$ and $E$. 
    
    Let us prove $\impliedby$.  
    As before, let us assume that $E=K(x,y)$. For each $a\in K$ we consider
    the extension $K(ay+x)/K$. By assumption, there exist $a,b\in K$ such that
    $a\ne b$ and $K(x+ay)=K(x+by)=L$. We claim that $L=E$. Note that 
    $x+ay\in L$ and $x+by\in L$, so $(a-b)y\in L$ and hence, since $K\subseteq L$, it follows that
    $y\in L$. Thus $x\in L$ and therefore $L=E$. 
\end{proof}

As a consequence, if $E/K$ is finite and separable, then $E/K$ admits
finitely many subextensions. 

\topic{Galois extensions}

Let $E/K$ be an algebraic extension. Assume that $E=K(S)$ and
let $C$ be an algebraic closure of $K$ containing $E$. Let 
\[
T=\{y\in C:y\text{ is a root of $f(x,K)$ for $x\in S$}\}
\]
and let $L=K(T)$. Then $E\subseteq L$, as $S\subseteq T$. The extension
$L/K$ is normal, as $L/K$ is a decomposition field of the family $\{f(x,K):x\in S\}$. 
Moreover, $L$ is the smallest normal extension of $K$ containing $E$. The field
$L$ is the \textbf{normal closure} of $E$ (with respect to $C$). 

\begin{exercise}
If $E/K$ is finite, then $L/K$ is finite
\end{exercise}

\begin{exercise}
If $E/K$ is separable, then $L/K$ is separable.
\end{exercise}

Let $E/K$ be an extension and $S\subseteq\Gal(E/K)$ be a subset. 
the set 
\[
    \prescript{S}{}{E}=\{x\in E:\sigma(x)=x\text{ for all $\sigma\in S$}\}
\]
is a subfield of $E$ that contains $K$. The subfield $\prescript{S}{}{E}$
is known as the \textbf{fixed field} of $S$. 

\begin{definition}
    \index{Extension!Galois}
    Let $E/K$ be an algebraic extension and $G=\Gal(E/K$). 
    Then $E/K$ is a \textbf{Galois extension} if $\prescript{G}{}{E}=K$. 
\end{definition}

Clearly, $K/K$ is a Galois extension. 
Note that $\Q(\sqrt[3]{2})/\Q$ is not a Galois extension. Why?

\begin{exercise}
    Prove that $\Q(\sqrt{2},\sqrt{3})/\Q$ is a Galois extension. 
\end{exercise}

\begin{exercise}
If the characteristic of $K$ is different from two, 
then every quadratic extension of $K$ is a Galois extension. 
\end{exercise}

\begin{exercise}
    Let $E/K$ be an algebraic extension and $G=\Gal(E/K)$. Let
    $F=\prescript{G}{}{E}$. Prove that $\Gal(E/F)=G$ and hence $E/F$ is a Galois extension. 
\end{exercise}

\begin{proposition}
\label{pro:normal+separable}
    Let $E/K$ be an algebraic extension. Then $E/K$ is a Galois extension
    if and only if $E/K$ is normal and separable. 
\end{proposition}

\begin{proof}
    Let $G=\Gal(E/K)$. Let us first assume that $E/K$ is Galois. For $x\in E$ let 
    $f_x=\prod_{y\in O_G(x)}(X-y)=\sum a_iX^i\in E[X]$. If $\varphi\in G$, then 
    \[
    \overline{\varphi}(f_x)=\prod_{y\in O_G(x)}(X-\varphi(y))=f_x,
    \]
    as if $O_G(x)=\{\sigma_1(x),\dots,\sigma_r(x)\}$, then 
    $\varphi(\sigma_i(x))=(\varphi\sigma_i)(x)=\sigma_j(x)$ for some $j$. 
    Since 
    \[
    \sum a_iX^i=f_x=\overline{\varphi}(f_x)=\sum\varphi(a_i)X^i,
    \]
    it follows that $a_i\in\prescript{G}{}{E}=K$ for all $i$. 
    Thus $f_x\in K[X]$
    and $E/K$ is a decomposition field of the family $\{f_x:x\in E\}$. In particular, 
    $E/K$ is normal. Moreover, $x$ is a simple root of $f_x\in K[X]$ and hence
    $x$ is separable over $K$. 

    Conversely, let $x\in \prescript{G}{}{E}$. Since $E/K$ is normal, then 
    $f(x,K)=\prod_{y\in O_G(x)}(X-y)^m$ for some $m$. Since $E/K$ is separable, 
    $m=1$. Thus $f(x,K)=\prod_{y\in O_G(x)}(X-y)=X-x$ and $x\in K$. 
\end{proof}

\begin{definition}
Let $K$ be a field and $f\in K[X]$. Then $f$ is \textbf{separable}
if all roots of $f$ are simple (in some algebraic closure of $K$). 
\end{definition}

\begin{proposition}
    Let $E/K$ be a finite extension. Then $E/K$ is a Galois extension 
    if and only if $E$ is a decomposition field over $K$ 
    of a separable polynomial $f\in K[X]$. 
\end{proposition}

\begin{proof}
    Let us assume first that $E/K$ is a Galois extension. Since
    $E/K$ is finite and separable, $E=K(x)$ by Proposition \ref{pro:monogenic}. 
    Then $E/K$ is a decomposition field of $f(x,K)$ since
    $E/K$ is normal. Since $E/K$ is separable, $x$ is separable over $K$. Thus $x$ is 
    a simple root of $f(x,K)$ and hence $f(x,K)$ is separable. 
    
    Conversely, let $x_1,\dots,x_r$ be the roots of a separable polynomial $f\in K[X]$.
    Then $E=K(x_1,\dots,x_r)$ is separable and normal.  
\end{proof}

In the previous case, $\Gal(E/K)$ is known as the \textbf{Galois group}
of the polynomial $f$. The notation is $\Gal(f,K)$. If $n=\deg f$ and
$x_1,\dots,x_n$ are the roots of $f$, then any 
$\varphi\in\Gal(f,K)$ permutes the roots of $f$, that is
$\varphi$ permutes the 
set $\{x_1,\dots,x_n\}$. In particular, $\Gal(f,K)$ is isomorphic to a subgroup of
$\Sym_n$ and hence $|\Gal(f,K)|$ divides $n!$. 

\begin{proposition}
    Let $E/K$ be a normal extension and $F$ be the separable
    closure of $K$ with respect to $E$. 
    Then $F/K$ is a Galois extension.
\end{proposition}

\begin{proof} 
    Let $C/K$ be an algebraic closure such that $E\subseteq C$. Let $\sigma\in\Hom(F/K,C/K)$. 
    and let $\varphi\in\Hom(E/K,C/K)$ be such that $\varphi|_F=\sigma$. Since $E/K$ is normal, 
    $\varphi(E)=E$. Let $x\in F$. Then $\sigma(x)=\varphi(x)\in E$. Thus
    $f(\sigma(x),K)=f(x,K)$ and $\sigma(x)$ is separable over $K$, which 
    implies that $\sigma(x)\in F$. Thus $F/K$ is normal. Since $F/K$ is separable, it follows
    that $F/K$ is a Galois extension by Proposition \ref{pro:normal+separable}.
\end{proof}

Some easy facts.

\begin{exercise}
    Let $E/K$ be a separable extension and $L/K$ be the normal 
    closure of $E$ in some algebraic closure $C$
    that contains $E$. Prove that $L/K$ is a Galois extension.
\end{exercise}

\begin{exercise}
    Let $E/K$ be a finite extension. Prove that $E/K$ is Galois
    if and only if $[E:K]=|\Gal(E/K)|$.
\end{exercise}

For the previous exercise, 
note that if $E/K$ is a finite extension, then  
\[
|\Gal(E/K)|\leq\gamma(E/K)\leq [E:K].
\]
The first inequality
is equality if and only if $E/K$ is normal. The second
inequality is equality if and only if $E/K$ is separable.

\begin{exercise}
    Let $E/K$ be a Galois extension and $F/K$ be a subextension of $E/K$. 
    Prove that $E/F$ is a Galois extension. 
\end{exercise}


\begin{theorem}[Artin]
\index{Artin's theorem}
\label{thm:ArtinGalois}
    Let $E$ be a field and $G$ be a finite group of automorphisms of $E$. 
    If $K=\prescript{G}{}{E}$, then $E/K$ is a Galois extension,
    $[E:K]=|G|$ and $\Gal(E/K)=G$. 
\end{theorem}

Before proving the theorem, we need a lemma.

\begin{lemma}
    Let $E/K$ be a separable extension such that $\deg f(x,K)\leq m$
    for all $x\in E$. Then $E/K$ is finite and $[E:K]\leq m$. 
\end{lemma}

\begin{proof}
   Let $z\in E$ be of maximal degree. If $x\in E$, 
   then $K(x,z)/K$ is separable. Then $K(x,z)=K(y)$ for some $y$. 
   It follows that 
   \[
   K(z)\subseteq K(x,z)=K(y).
   \]
   Since 
   $\deg f(z,K)\leq\deg f(y,K)$, 
   $\deg f(z,K)=\deg f(y,K)$. Hence 
   $K(y)=K(z)$. In particular, $x\in K(z)$ and
   therefore $E=K(z)$. 
\end{proof}

Now we are ready to prove Artin's theorem: 

\begin{proof}[Proof of Theorem \ref{thm:ArtinGalois}]
    Note that $G\subseteq\Gal(E/K)$. Let $x\in E$ and 
    \[
    f_x=\prod_{y\in O_G(x)}(X-y).
    \]
    Since $f_x\in K[X]$, the extension $E/K$ is normal and separable (as it is a decomposition
    field of a family of separable polynomials), so $E/K$ is a Galois extension. Moreover, 
    \[
    \deg f(x,K)\leq \deg f_x=|O_G(x)|\leq |G|.
    \]
    By the previous lemma, $E/K$ is finite and $[E:K]\leq |G|$. This
    implies that
    $|\Gal(E/K)|=[E:K]\leq |G|$ and hence $|\Gal(E/K)|=|G|$. 
\end{proof}

\begin{example}
    Let $E=K(X,Y)$ and $\sigma\colon K[X,Y]\to E$ be the ring homomorphism given by $\sigma(X)=Y$ and $\sigma(Y)=X$. Note that $\sigma$ is bijective, as $\sigma^2=\id$. The map $\sigma$ induces
    a field homomorphism $\overline{\sigma}\colon E\to E$ such that 
    $\overline{\sigma}^2=\id$. Recall that such a homomorphism is given by 
    $f/g\mapsto \sigma(f)/\sigma(g)$. Let $G=\langle\overline{\sigma}\rangle$. Then $|G|=2$. 
    We claim that $\prescript{G}{}{E}=K(X+Y,XY)$. Let $F=K(X+Y,XY)$. We only prove
    that $\prescript{G}{}{E}\subseteq F$, as the other inclusion is trivial. Artin's theorem
    implies that $[E:\prescript{G}{}{E}]=2$ and $E=F(X)$, as $X$ is a root
    of the polynomial $Z^2-(X+Y)Z+XY$. Then $[E:F]\leq 2$ and $[\prescript{G}{}{E}:F]=1$.
\end{example}
