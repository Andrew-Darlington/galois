\section{Lecture -- Week 6}

If $t\colon A\to B$ is a surjective map, then 
$a\sim a_1\Longleftrightarrow t(a)=t(a_1)$ 
defines an equivalence relation on $A$. The set $\overline{A}$ 
of equivalence classes is in bijective correspondence with $B$,
$\overline{A}\to B$, $\overline{a}\mapsto t(a)$. 
Moreover, if $|t^{-1}(\{b\})|=m$ for all $b\in B$, then 
$|A|=m|\overline{A}|=m|B|$. 

\begin{proposition}
    Let $E/K$ be algebraic and $F/K$ be a subextension such that 
    $E/F$ is finite. Then $\gamma(E/K)=\gamma(E/F)\gamma(F/K)$. 
\end{proposition}

 \begin{proof}
    Assume first that $E=F(x)$. 
    Let $C$ be an algebraic closure of $K$ containing $E$ 
    and $G=\Gal(C/F)$. Let $f=f(x,F)=\sum b_iX^i$.
    
    The map
    \[
    \lambda\colon \Hom(E/K,C/K)\to\Hom(F/K,C/K),\quad
    \sigma\mapsto\sigma|_F,
    \]
    is well-defined. 
    %It is an exercise to check that the map
    It is surjective: 
    if $\varphi\in\Hom(F/K,C/K)$, then $\varphi\colon F\to C$ is, 
    in particular, a field homomorphism. Since $E/F$ is algebraic, by Proposition \ref{pro:Artin} 
    there exists a field homomorphism 
    $\sigma\colon E\to C$ such that $\sigma|_F=\varphi$. Since $\sigma|_K=\varphi|_K=\id$, in particular 
    $\sigma\in\Hom(E/K,C/K)$. 
    
    For $\varphi\in\Hom(F/K,C/K)$,  
    \[
    \lambda^{-1}(\{\varphi\})=\{\sigma\in\Hom(E/K,C/K):\sigma|_F=\varphi\}
    \]
    and let $R_\varphi$ be the set of roots (in $C$) of the polynomial $\overline{\varphi}(f)=\sum\varphi(b_i)X^i$. 
    
    \begin{claim}
        The map $\alpha\colon \lambda^{-1}(\{\varphi\})\to R_{\varphi}$, $\sigma\mapsto\sigma(x)$, is well-defined. 
    \end{claim}
    
    We need to show that $\sigma(x)$ is a root of $\overline{\varphi}(f)$:
    \begin{align*}
    \overline{\varphi}(f)(\sigma(x))&=\sum \varphi(b_i)\sigma(x)^i
    =\sum\sigma(b_i)\sigma(x^i)\\
    &=\sum\sigma(b_ix^i)=\sigma\left(\sum b_ix^i\right)=\sigma(f(x))=\sigma(0)=0.
    \end{align*}
    
    \begin{claim}
    The map $\beta\colon R_{\varphi}\to \lambda^{-1}(\{\varphi\})$, $y\mapsto\sigma_y$, 
    where $\sigma_y(z)=\overline{\varphi}(h)(y)$
    if $z=h(x)$, is well-defined. 
    \end{claim}
    
    We need to show that if $z=h(x)$ and 
    $z=h_1(x)$ for some $h,h_1\in F[X]$, then 
    \[
    \overline{\varphi}(h)(y)=\overline{\varphi}(h_1)(y).
    \]
    The assumptions imply that 
    $(h-h_1)(x)=0$ and hence $f$ divides $h-h_1$. Since
    $\overline{\varphi}$ is a ring homomorphism, 
    $\overline{\varphi}(f)$ divides $\overline{\varphi}(h)-\overline{\varphi}(h_1)$. 
    This implies $(\overline{\varphi}(h)-\overline{\varphi}(h_1))(y)=0$. We also need to show that 
    $\sigma_y|_F=\varphi$: if $a\in F$, then 
    write $a=aX^0\in F[X]$. Thus 
    $\sigma_y(a)=\overline{\varphi}(aX^0)(y)=\varphi(a)\in C$. 
    It is now an exercise to prove that $\sigma_y\in\Hom(E/K,C/K)$. 
    
    \begin{claim}
        $|\lambda^{-1}(\{\varphi\})|=|R_\varphi|$. 
    \end{claim}
    
    For this we need to show that $\beta$ is
    the inverse of $\alpha$, that is 
    $\alpha\circ\beta=\id$ and $\beta\circ\alpha=\id$. 
    To prove that $\beta\circ\alpha=\id$ 
    let $\sigma$ be such that $\sigma|_F=\varphi$. 
    Then $y=\sigma(x)\in R_\varphi$. Let
    \[
    z=h(x)=\sum a_ix^i\in F[x]=E.
    \]
    Then  
    \[
    \overline{\varphi}(h)(y)=\sum\varphi(a_i)y^i=\sum\sigma(a_i)y^i
    =\sigma\left(\sum a_ix^i\right)=\sigma(z).
    \]
    Conversely, if $y\in R_\varphi$, then
    \[
    \alpha(\sigma_y)=\sigma_y(x)=y,
    \]
    as $\sigma_y(x)=\overline{\varphi}(X)(y)=y$.
    
    \begin{claim}
        If $\phi\in\Gal(C/K)$ is such that $\phi|_F=\varphi$, then 
        $|\phi^{-1}(R_\varphi)|=|R_{\varphi}|$ and 
        \[
        O_{G}(x)=\phi^{-1}(R_\varphi).
        \]
    \end{claim}

    Let us first prove $O_{G}(x)\supseteq \phi^{-1}(R_\varphi)$.
    If $y\in R_{\varphi}$, 
    then 
    \begin{align*}
    f(\phi^{-1}(y))&=\sum b_i\phi^{-1}(y^i)=\phi^{-1}\left(\sum\phi(b_i)y^i\right)\\
&=\phi^{-1}(\overline{\varphi}(f)(y))=\phi^{-1}(0)=0.
    \end{align*}
    Then $f(x,F)=f(\phi^{-1}(y),F)$. By Proposition \ref{pro:conjugate}, $\phi^{-1}(y)\in O_G(x)$. 
    
    Now we prove $O_{G}(x)\subseteq\phi^{-1}(R_\varphi)$.
    Let $z\in O_{G}(x)$. Then $\overline{\varphi}(f)(\phi(z))=0$, as
    \begin{align*}
    \overline{\varphi}(f)(\phi(z))&=\sum\varphi(b_i)\phi(z^i)\\
    &=\sum\phi(b_i)\phi(z^i)
    =\phi\left(\sum b_iz^i\right)
    =\phi(f(z))=\phi(0)=0.
    \end{align*}
    
    \medskip
    Thus $\phi(z)\in R_{\varphi}$ and hence $z\in\phi^{-1}(R_{\varphi})$. 
    It follows that $|\lambda^{-1}(\{\varphi\})|=|O_{G}(x)|$ for
    all $\varphi$. By using the argument
    before the proposition, 
    \begin{align*}
    \gamma(E/K)&=|\Hom(E/K,C/K)|\\
        &=|O_{G}(x)||\Hom(F/K,C/K)|\\
        &=|O_{G}(x)|\gamma(F/K).
    \end{align*}
    Since $\gamma(E/F)=\gamma(F(x)/F)=|O_{G}(x)|$ by Proposition \ref{pro:gamma_orbit}, the claim follows. 
    
    \medskip 
    For the general case, we assume that $E=F(x_1,\dots,x_n)$. We proceed
    by induction on $n$. If $n=0$, then $E=F$ and the result is trivial. 
    If $n>0$, let $L=F[x_1,\dots,x_{n-1}]$ and $E=L(x_n)$. The 
    case proved 
    implies that $\gamma(E/F)=\gamma(E/L)\gamma(L/F)$. By the inductive 
    hypothesis, $\gamma(L/K)=\gamma(L/F)\gamma(F/K)$. Thus 
    \[
    \gamma(E/F)\gamma(F/K)=\gamma(E/L)\gamma(L/F)\gamma(F/K)
    =\gamma(E/L)\gamma(L/K)=\gamma(E/K),
    \]
    again using the previous case. 
\end{proof}

\subsection{Separable extensions}

\begin{definition}
\index{Extension!separable}
\index{Element!separable}
    Let $E/K$ be an extension and $x\in E$ an algebraic element over $K$. Then
    $x$ is \emph{separable} over $K$ if $x$ is a simple root
    of $f(x,K)$. 
\end{definition}

An algebraic extension $E/K$ is \emph{separable} 
if every $x\in E$ is separable over $K$. Then $K/K$ is separable. 

\begin{exercise}
    Prove that 
    an element $x$ is separable over $K$ if and only if $x$ is a simple root
    of a polynomial with coefficients in $K$. 
\end{exercise}

If $F/K$ is a subextension of $E/K$ and $x\in E$ is separable over $K$, then
$x$ is separable over $F$. 

\begin{exercise}
    If $C$ is an algebraic closure of $K$, $x\in C$ and $G=\Gal(C/K)$. 
    Prove that the following statements are equivalent:
    \begin{enumerate}
        \item $x$ is separable over $K$.
        \item Every $y\in O_G(x)$ is separable over $K$.
        \item $\gamma(K(x)/K)=[K(x):K]=\deg f(x,K)$. 
    \end{enumerate}
\end{exercise}

Let $K$ be any field and $g\in K[X]$. Let $z$ be a root of $g$. 
Then $z$ is a multiple root of $g$ if and only if $z$ is a root of $g'$. 

\begin{exercise}
Prove that if $K$ has characteristic zero or $K$ is finite, then 
every algebraic extension of $K$ is separable. 
\end{exercise}

%Let $K$ be a field and $g\in K[X]$. Let $z$ be a root of $g$. Then %$z$ is a multiple root of $g$ if and only if $z$ is a root of $g’$. 

%Assume first that $K$ has characteristic zero. If $x$ is a multiple root of $f(x,K)$, then 
%$f(x,K)$ divides $f(x,K)’$, a contradiction (take degree). 
%
%Assume now that $K$ has characteristic $p>0$. Let $f=f(x,K)$. If $x$ is a multiple root of $f$, we repeat the previous argument. 
%Assume that $f=\sum_{i=0}^n a_iX^i$ and $0=f’=\sum_{i=1}^n a_iiX^{i-1}$. Then $a_i=0$ if $p$ does not divide $i$. Thus
%$f=\sum_{k=0}^r b_kX^{pk}$. If $K$ is finite, the map $\alpha\mapsto\alpha^p$ is an automorphism of $K$. In particular, 
%every element of $K$ is a $p$-power. Thus, for every $k$, there exists $c_k\in K$ such that $b_k=c_k^p$. Therefore 
%$f=\sum_{k=0}^r c_k^pX^{pk}=\left(\sum_{k=0}^r c_kX^k\right)^p$, a contradiction, as $f$ is irreducible.   



%A consequence: 
%Let $E/K$ be a finite extension. Then $E/K$ is separable
%if and only if $\gamma(E/K)=[E:K]$. 
%Let $x\in E$, Then 
%\[
%[E:K]=[E:K(x)][K(x):K]\geq \gamma(E/K(x))\gamma(K(x)/K)=\gamma(E/K).
%\]
%This implies that
%\begin{align*}
%	[E:K(x)][K(x):K]=\gamma(E/K(x))\gamma(K(x)/K).
%\end{align*}
%Then $[E:K(x)]=\gamma(E/K(x))$ and $[K(x):K]=\gamma(K(x)/K)$ because
%$[E:K(x)]\geq \gamma(E/K(x))$ and $[K(x):K]\geq \gamma(K(x)/K)$. Therefore
%$x$ is separable over $K$.
%
%Conversely, let $E=K(x_1,\dots,x_n)$. We proceed by induction on $n$. The case $n=0$ is trivial, as $E=K$. If $n>0$, 
%let $F=K(x_1,\dots,x_{n-1})$ and $E=F(x_n)$. Note that $x_n$ is separable over $K$, so $x_n$ is separable over $F$. Thus
%\[
%[E:K]=[E:F][F:K]=\gamma(E/F)\gamma(F/K)=\gamma(E/K).
%\] 
%
\begin{example}
    Let $E=\Q(\sqrt{2},\sqrt{3})$. Then 
    $[E:\Q]=4$ and 
    $\Gal(E/\Q)\simeq C_2\times C_2$. The extension $E/\Q$ is normal, 
    as it is the splitting field of $(X^2-2)(X^2-3)$ and 
    it is separable as $\Q$ has characteristic zero. 
    % If $\sigma\in\Gal(E/\Q)$, then
    % $\sigma(\sqrt{2})\in\{-\sqrt{2},\sqrt{2}\}$ and 
    % $\sigma(\sqrt{3})\in\{-\sqrt{3},\sqrt{3}\}$. 
\end{example}
% Q<x> := PolynomialRing(Rationals());
% E := SplittingField((x^2-2)*(x^2-3));
% Degree(E);
% GroupName(GaloisGroup(E));


\begin{example}
    Let $E$ be a splitting field of $X^4-2$ over $\Q$. 
    Then $E/\Q$ is normal and separable. Note that
    $E=\Q(\sqrt[4]{2},i)$, so 
    \[
    [E:\Q]=8=|\Gal(E/\Q)|.
    \]
    
    Let us compute
    $\Gal(E/\Q)$. If $\sigma\in\Gal(E/\Q)$, then 
    $\sigma(\sqrt[4]{2})\in\{\sqrt[4]{2},-\sqrt[4]{2},\sqrt[4]{2}i,-\sqrt[4]{2}i\}$ and 
    $\sigma(i)\in\{-i,i\}$. Two examples are 
    \[
    \alpha\colon\begin{cases}
    \sqrt[4]{2}\mapsto\sqrt[4]{2}i,\\
    i\mapsto i,
    \end{cases}
    \quad
    \beta\colon\begin{cases}
    \sqrt[4]{2}\mapsto\sqrt[4]{2},\\
    i\mapsto -i.
    \end{cases}
    \]
    It follows that 
    $\Gal(E/\Q)$ is isomorphic to the group $\langle\alpha,\beta\rangle$, which turns out to be
    isomorphic to the dihedral group
    of eight elements. 
\end{example}
% Q<x> := PolynomialRing(Rationals());
% E := SplittingField(x^4-2);
% Degree(E);
% GroupName(GaloisGroup(E));


Another consequence: If $E=K(S)$, then $E/K$ is separable if and only if
every $x\in S$ is separable over $K$. One first does the case $E=K(x)$ 
and then proceeds by induction. 

\begin{exercise}
\label{xca:separable1}
    Let $K\subseteq F\subseteq E$ be a tower of fields. Prove that 
    $E/K$ is separable if and only if $F/K$ and $E/F$ are separable. 
\end{exercise}

%If $E/K$ is separable, then $F/K$ and $E/F$ are separable (easy). Conversely, assume that $F/K$ and $E/F$ are separable. Let $x\in E$ and
%$f=f(x,K)$. Then $x$ is a simple root of $f$. Let $L=K(S)$, where $S$ is the set of coefficients of $f$. Then $L\subseteq F$, $L/K$ is finite and separable and $x$ is separable over $L$. Thus
%\[
%[L(x):K]=[L(x):L][L:K]=\gamma(L(x)/L)\gamma(L/K)=\gamma(L(x)/K).
%\]
%Hence $L(x)/K$ is separable, and $x$ is separable over $K$. 


\begin{exercise}
\label{xca:separable2}
    Let $E/K$ and $F/K$ be extensions. Prove that if $F/K$ is separable, 
    then $EF/E$ is separable. 
\end{exercise}

% This follows because $EF/E$ is algebraic and 
% $EF/E=E(F)/E$ is generated by separable elements.

\begin{example}
    Let $p$ be a prime and let $t$ be transcendental over $\mathbb{F}_p=\Z/p$. Let $K=\mathbb{F}_p(t^p)$ and let $E=\mathbb{F}_p(t)$. Then $E/K$ has degree $p$ and is not separable.

    Indeed, $E/K$ is algebraic because $f=f(t,K)=X^p-t^p$ (it is an exercise to check that $f$ is irreducible in $K[X]$). Moreover, $f=(X-t)^p \in E[X]$, and so $t$ has multiplicity $p$ in $f$. It follows that $t$ is not separable over $K$. Otherwise there is a polynomial $g \in K[X]$ such that $t$ is a simple root of $g$, and hence, in particular, $g(t)=0$. However, we then must have $f \mid g$, and thus $t$ cannot be a simple root of $g$.

    Note that, in this case, $E/K$ is also not normal, as $\Gal(E/K)=\{\mathrm{id_E}\}$.
\end{example}

\index{Extension!separable}
\label{separable}
If $E/K$ is algebraic, then 
\[
    F=\{x\in E:x\text{ is separable over }K\}
\]
is a subfield of $E$ that contains $K$. It is known as 
the \emph{separable closure} of $K$ with respect to $E$. 
Note that $F=K(F)$, as
$K(F)$ is separable because it is generated by separable elements. Moreover, 
$F/K$ is separable and 
$E/F$ is a \emph{purely inseparable} extension, meaning that
for every $x\in E\setminus F$, $x$ is not separable over $K$. 

\begin{proposition}
\label{pro:monogenic}
    If $E/K$ is separable and finite, then $E=K(x)$ for some $x\in E$. 
\end{proposition}

\begin{proof}
    Let us assume that $K$ is finite. Then $E$ is finite and hence 
    the multiplicative group $E^{\times}=E\setminus\{0\}$ 
    is cyclic, say $E^{\times}=\langle x\rangle$. It follows
    that $E=K(x)$. 
    
    Let us now assume that $K$ is infinite. We first consider the case 
    $E=K(x,y)$. The general case $E=K(x_1,\dots,x_n)$ is left as an exercise, one needs to proceed by induction. 
    Let $n=[E:K]$ and 
    $C$ be an algebraic closure of $K$ containing $E$. 
    Write $\Hom(E/K,C/K)=\{\sigma_1,\dots,\sigma_n\}$. Let 
    \[
    f=\prod_{1\leq i<j\leq n}\left((\sigma_i(y)-\sigma_j(y))
    +X(\sigma_i(x)-\sigma_j(x))\right)\in C[X].
    \]
    Then $f\ne 0$, as $f$ is a product of non-zero polynomials. Since $K$ is infinite, 
    there exists a non-zero 
    $c\in K$ such that $f(c)\ne 0$. For any $r,s\in\{1,\dots,n\}$ with 
    $r\ne s$,
    \[
        \sigma_r(y)-\sigma_s(y)+c(\sigma_r(x)-\sigma_s(x))\ne 0,
    \]
    as $f(c)\ne0$. It follows that $\sigma_r(y+cx)\ne\sigma_s(y+cx)$. Thus $\gamma(K(y+cx)/K)\geq n$. 
    Now 
    \[
    n\geq [K(y+cx):K]=\gamma(K(y+cx)/K)\geq n,
    \]
    so $[K(y+cx):K]=n$ and
    hence $K(y+cx)=E$. 
\end{proof}

For example, $\Q(\sqrt{2},i)=\Q(\sqrt{2}+i)$. 
