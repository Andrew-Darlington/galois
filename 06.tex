\chapter{}

If $E/K$ is algebraic, then $F=\{x\in E:x\text{ is separable over }K\}$ 
is a subfield of $E$ that contains $K$. Note that $F=K(F)$, as
$K(F)$ is separable because it is generated by separable elements. Note that
$F/K$ is separable and 
$E/F$ is a \textbf{purely inseparable} extension, meaning that
for every $x\in E\setminus F$, the polynomial $f(x,F)$ is not separable. 

\begin{proposition}
    If $E/K$ is separable and finite, then $E=K(x)$ for some $x\in E$. 
\end{proposition}

\begin{proof}
    Let us assume that $K$ is finite. Then $E$ is finite and hence 
    the multiplicative group $E^{\times}=E\setminus\{0\}$ 
    is cyclic, say $E^{\times}=\langle x\rangle$. It follows
    that $E=K(x)$. 
    
    Let us now assume that $K$ is infinite. We may assume that 
    $K=K(x,y)$ and let $n=[E:K]$. Let $C$ be an algebraic closure of $K$ containing $E$. 
    Assume that $\Hom(E/K,C/K)=\{\sigma_1,\dots,\sigma_n\}$. Let 
    \[
    f=\prod_{1\leq i<j\leq n}\left(\sigma_i(y)-\sigma_j(y)\right)
    +X\left(\sigma_i(x)-\sigma_j(x)\right)\in C[X].
    \]
    Then $f\ne 0$, as $f$ is a product of non-zero polynomials. Since $K$ is infinite, 
    there exists $c\in K$ such that $f(c)\ne 0$. Let $r,s\in\{1,\dots,n\}$ be such that
    $r\ne s$. Since $c\in K$ and 
    \[
        \sigma_r(y)-\sigma_s(y)+c(\sigma_r(x)-\sigma_s(x))\ne 0,
    \]
    it follows that $\sigma_r(y+cx)\ne\sigma_s(y+cx)$. Thus $\gamma(K(y+cx)/K)\geq n$. 
    Now 
    \[
    n\geq [K(y+cx):K]=\gamma(K(y+cx)/K)\geq n,
    \]
    so $[K(y+cx):K]=n$ and
    hence $K(y+cx)=E$. 
\end{proof}

For example, $\Q(\sqrt{2},i)=\Q(\sqrt{2}+i)$. 

\begin{proposition}
    Let $E/K$ be a finite extension. Then $E=K(x)$ for some $x\in E$ 
    if and only if $E/K$ admits finitely many subextensions. 
\end{proposition}

\begin{proof}
    We may assume that $K$ is infinite, otherwise the result is trivial. Let us 
    assume that $E=K(x)$. If $F/K$ is a subsextension of $E/K$, 
    let $g=f(x,F)$. Then $g$ divides $f(x,K)$. So we constructed a map that assigns 
    to every subextension $F/K$ a (monic) divisor of $f(x,K)$ in $E[X]$. Let us show
    that $g$ completely determines $F/K$. Let $F_0/K$ be the subextension
    generated by the coefficients of $g$. Then $F_0\subseteq F$ and $g$ 
    is irreducible in $F_0[X]$. Thus 
    \[
    [E:F_0]=[F_0(x):F_0]=\deg f(x,F_0)=\deg g=[F(x):F]=[E:F]
    \]
    and hence $F=F_0$. This means that the assignment is injective
    and therefore there are finitely many fields between $K$ and $E$. 
    
    Let us prove the converse. 
    As before let us assume that $E=K(x,y)$. For each $a\in K$ we consider
    the extension $K(ay+x)/K$. By assumption, there exist $a,b\in K$ such that
    $a\ne b$ and $K(x+ay)=K(x+by)=L$. We claim that $L=E$. Note that 
    $x+ay\in L$ and $x+by\in L$, so $(a-b)y\in L$ and hence $y\in L$. Since 
    $K\subseteq L$, it follows that $x\in L$ and therefore $L=E$. 
\end{proof}

As a consequence, if $E/K$ is finite and separable, then $E/K$ admits
finitely many subextensions. 

