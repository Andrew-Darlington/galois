\section*{Some solutions}

% \pagestyle{plain}
% \fancyhf{}
% \fancyhead[LE,RO]{Rings and modules}
% \fancyhead[RE,LO]{Some solutions}
% \fancyfoot[CE,CO]{\leftmark}
% \fancyfoot[LE,RO]{\thepage}

%\addcontentsline{toc}{chapter}{Some solutions}

% \begin{sol}{xca:algebraic_bijective}
%     Let $\sigma\colon C\to C$ be a homomorphism. As $\sigma$ is
%     injective, we need to prove that $\sigma$ is surjective. Let $y\in C$. Note that $y$ is algebraic over $K$. Let 
%     $R$ be the set of roots of the minimal polynomial 
%     $f(y,K)$ of $y$ over $K$. 
%     The map 
%     $\sigma|_R\colon R\to R$ is injective. Since 
%     $R$ is finite, $\sigma|_R$ is then bijective. In particular, 
%     there exists $x\in R$ such that $y=\sigma(x)$.
% \end{sol}


\begin{sol}{xca:Q(i)}
Assume that $\Q[i]$ and $\Q[\sqrt{2}]$ were isomorphic 
and let $\varphi:\Q[i]\to \Q[\sqrt{2}]$
be a field isomorphism.
Then 
    \[
    \varphi(i)^2=
    \varphi(i^2)=
    \varphi(-1)=
    -\varphi(1)=-1.
    \]
But $\varphi(i)\in\Q[\sqrt{2}]$ and
$\Q[\sqrt{2}]\subseteq \R$ where every square is 
positive, a contradiction.
\end{sol}

\begin{sol}{xca: field characteristic}
    Let $t>0$ be the characteristic of a field $K$ and 
    let $\varphi:\Z\to K,$ $x\mapsto x1$.
    Then, by definition, $\ker\varphi$ is 
    the ideal generated by $t$.
    On the other hand, the image $\varphi(\Z)$ is 
    a domain, being a subring of a field
    and is isomorphic to $\Z/\ker\varphi$.
    Therefore, $\ker\varphi$ is a prime ideal of $\Z$,
    i.e. $t$ is a prime number.
\end{sol}

\begin{sol}{xca: char 0}
Let $\varphi:\Z\to K,$ $x\mapsto x1$.

We first prove that \textbf{1)} implies all the other 
properties. So suppose that the characteristic of $K$ is zero, i.e. $\ker\varphi=\{0\}$. 

Then $m1=0$ if and only if $m=0$, i.e.
the order of $1$ is infinite.

Let $0\neq x\in K$. If $mx=0$, then $0=mx=(m1)x$.
But $K$ is a field and $x\neq 0$, hence $m1=0$,
so $m\in\ker\varphi=\{0\}$.
Hence the order of $x$ is infinite.

By definition, the ring of integers of $K$ is
the image of $\varphi$, which 
so it is isomorphic to $\Z/\ker\varphi=\Z$.

Finally, we prove that \textbf{4)} implies \textbf{1)}.
Take $m\in\ker\varphi$. Then $m1=0$,
but 1 has infinite order, hence $m=0$.
Therefore $\ker\varphi=\{0\}$, 
i.e. $K$ has characteristic 0.
\end{sol}

\begin{sol}{xca: char p}
Let $\varphi:\Z\to K,$ $x\mapsto x1$.

We first prove that \textbf{1)} implies all the other 
properties. So suppose that the characteristic of $K$ is $p>0$ i.e. $\ker\varphi$ is the ideal generated by $p$. 

Then $m1=0$ if and only if $p$ divides $m$, i.e.
the order of $1$ is $p$.

Let $0\neq x\in K$. If $mx=0$, then $0=mx=(m1)x$.
But $K$ is a field and $x\neq 0$, hence $m1=0$,
so $p$ divides $m$.
Hence $x$ has order $p$.

By definition, the ring of integers of $K$ is
the image of $\varphi$, which 
so it is isomorphic to $\Z/\ker\varphi\cong \Z/p$.

Finally, we prove that \textbf{4)} implies \textbf{1)}.
Take $m\in\ker\varphi$. Then $m1=0$,
but 1 has order $p$, hence $p$ divides $m$.
Therefore $\ker\varphi$ is generated by $p$, 
i.e. $K$ has characteristic 0.
\end{sol}

\begin{sol}{xca: Frobenius hom}
Let $\Phi: K\to K$ be the map 
$x\mapsto x^{p}$.
Since the map $x\mapsto x^{p^n}$
is exactly $\Phi^n$,
it is enough to prove that $\Phi$ is a field homomorphism.

As $K$ is commutative under multiplication, 
for all $x,y \in K$ 
        \[
        \Phi(xy) = (xy)^p = x^py^p = \Phi(x) \Phi(y).
        \]
Moreover, for all $x,y \in K$ 
\[
\Phi(x+y) =
(x+y)^p \sum_{k=0}^p \binom{p}{k}x^py^{p-k}=
x^p + y^p + \sum_{k=1}^{p-1} \binom{p}{k}x^ky^{p-k} ,
\]
where $\binom{p}{k} = \frac{p!}{k!(p-k)!}$,
which can also be written as
        \[
        p! = \binom{p}{k} \cdot k! \cdot (p-k)!
        \]
But $p$ divides $p!$, so $p$
has to divide at least one 
factor on the right side. 
But $p$ doesn't divide $i$
 for $1 \leq i \leq p-1$, therefore if $k \leq p-1$, $p$ doesn't divide
$k!$ and if $1 \leq  k$, $p$ doesn't divide $(p-k)!$. 
Hence, if $1 \leq k \leq p-1$, $p$ has to divide $\binom{p}{k}$ and 
\[
\sum_{k=1}^{p-1} \binom{p}{k}x^ky^{p-k}=0.
\]
Therefore, $\Phi$ is a field homomorphism.
\end{sol}

\begin{sol}{xca: prime field is fixed}
    By definition $K_0=\{m1\colon m\in \Z\}$ and 
    $\sigma:K\to K$ is a field homomorphism,
    so $\sigma(1)=1$.
    Hence, for every $m\in\Z$
    \[
    \sigma(m1)=m\sigma(1)=m1,
    \]
    i.e. $\sigma|_{K_0}$ is the identity.
\end{sol}

\begin{sol}{xca: X^3-2}
    If $X^3-2$ were reducible, since it has degree 3,
    it would have a linear
    factor in the decomposition in irreducibles.
    Therefore it would have a rational root.
    But the roots of $X^3-2$ are $\sqrt[3]{2},\sqrt[3]{2}\xi,\sqrt[3]{2}\xi^2$,
    where $\xi$ is a primitive third root of unity,
    So all the roots are not in $\Q$, a contradiction.
\end{sol}

\begin{sol}{xca:Eisenstein's criterion}
Recall first the following:
\begin{lemma}[Gauss' Lemma]
    Let $A$ be a unique factorization domain and $K$ be its fraction field.
    A non-constant polynomial $f\in A[X]$ is irreducible if and only if it is primitive and irreducible in $K[X]$.
\end{lemma}

    Suppose that $f$ is reducible in $K[X]$.
    Then $g=c^{-1}f$, where $c$ is the content of $f$,
    would be reducible and primitive.
    Hence, by Gauss' Lemma, $g$ is also 
    reducible in $A[X]$.
    So $c^{-1}f=g=hl$, for some non-constant polynomials $h,l\in A[X]$.
    Now consider $\pi:A\to A/(p)$, $a\mapsto \overline{a}$ the natural projection.
    We know that $\overline{a_i}=0$ for all 
    $i\in\{0,1,\dots, n-1\}$ and $\overline{a_n}\neq 0$.
    Therefore 
    \[
    \overline{\pi}(ch)\overline{\pi}(l)=\overline{c}\:\overline{\pi}(h)\overline{\pi}(l)=\overline{\pi}(f)=\overline{a_n}X^n\in  A/(p)[X].
    \]
    But $A/(p)[X]$ is a UFD so the only possibility
    is that $\overline{\pi}(ch)=\overline{d}X^t$ and $\overline{\pi}(l)=\overline{f}X^s$, for
    some $f,d\in A/(p)\setminus\{\overline{0}\}$
    and $t,s\in\{1,\dots, n-1\}$.
    In particular, $\overline{\pi}(ch)$ and $\overline{\pi}(l)$ have both constant term 
    equal to 0.
    Hence $p$ divides $ch(0)$ and $l(0)$ in $A$.
    Therefore $p^2$ divides $ch(0)l(0)=f(0)$,
    a contradiction.
\end{sol}

\begin{sol}{xca: using Eisenstein}
It is easy to see that $f$ satisfies the Eisenstein criterion for $p=2$
and $g$ satisfies it for $p=5$. 
\end{sol}

\begin{sol}{xca: counterexample os Eisenstein in Z}
    $f=3(X^10+5X^2-15)$ is a product of $3$ and $(X^10+5X^2-15)$,
    which are both non-invertible elements of $\Z[X]$.
    Hence $f$ is reducible.
\end{sol}

\begin{sol}{xca:degree_of_x}
    Let $f=f(x,K)$ be the minimal polynomial of $x$ over $K$ of degree $\deg(f)=n$.
    We claim that $\{1,x,\dots, x^{n-1}\}$ is a basis of $K(x)$ as a $K$-vector space. 

    To prove that $\{1,x,\dots, x^{n-1}\}$ is a generating set, recall that $K(x)=K[x]$, since $x$ is algebraic over $K$. 
    Let $z\in K(x)=K[x]$, say $z=h(x)$ for some $h\in K[X]$. 
    Divide $h$ by $f$ to obtain polynomials $q,r\in K[X]$ 
    such that $h=fq+r$, where either $r=0$ or $\deg r<\deg f=n$. Then 
    \[
		z=h(x)=f(x)q(x)+r(x)=r(x).
	\]
	Write $r=\sum_{i=0}^{n-1}c_iX^i$ for some $c_0,\dots,c_{n-1}\in K$. 
    Thus $z=\sum_{i=0}^{n-1}a_ix^i\in \langle 1,x,\dots,x^{n-1}\rangle$.
        
    We now prove that $\{1,x,\dots, x^{n-1}\}$ is linearly independent. If not, 
    there exists a linear combination
    $0=\sum_{i=0}^{n-1}a_ix^i$ with $a_0,\dots,a_{n-1}\in K$ not all zero. 
    Then $h(X)=\sum_{i=0}^{n-1}a_iX^i\in K[X]\setminus\{0\}$
    has $x$ as a root and 
        \[
        n=\deg(f)\leq \deg(h)\leq n-1,
        \]
       a contradiction. 
\end{sol}

\begin{sol}{xca:dec field X^4-5X^2+5}
Note that, since $f=X^4-5X^2+5$ is an even polynomial
if $\alpha\in \C$ is a root of $f$,
then also $-\alpha$ is a root of $f$.
 Hence, given two roots $\alpha,\beta\in \C$
 such that $\beta\neq-\alpha$,
we have that the decomposition field of $f$ over $\Q$ is
$E=\Q(\alpha,-\alpha,\beta,-\beta)$.
 But $-\alpha,-\beta\in \Q(\alpha,\beta)\subseteq E$
 and so 
\[
 E=\Q(\alpha,-\alpha,\beta,-\beta)\subseteq \Q(\alpha,\beta)\subseteq\Q(\alpha,-\alpha,\beta,-\beta)=E,
 \]
 which means that $E=\Q(\alpha,\beta)$.
 Moreover we can decompose $f$ in $\C[X]$ as 
 \[
 (X-\alpha)(X+\alpha)(X-\beta)(X+\beta)=(X^2-\alpha^2)(X^2-\beta^2)=X^4-(\alpha^2+\beta^2)X^2+\alpha^2\beta^2.
 \]
This implies in particular that $\alpha^2\beta^2=5$, hence $\beta=\pm \frac{\sqrt{5}}{\alpha}\in \Q(\alpha,\sqrt{5})$.

Therefore $E=\Q(\alpha,\beta)\subseteq \Q(\alpha,\sqrt{5})$.
On the other hand $\sqrt{5}=\pm\alpha\beta\in\Q(\alpha,\beta)$,
 hence $\Q(\alpha,\sqrt{5})\subseteq\Q(\alpha,\beta)=E$.
 So we can conclude that $E=\Q(\alpha,\sqrt{5})$.
Using the multiplicative of the degree of finite extension we get that
 \[
 [E:\Q]=[E:\Q(\alpha)][\Q(\alpha):\Q].
\]
But $[\Q(\alpha):\Q]=\deg(f(\alpha,\Q))$.
 Using Eisenstein criterion (Exercise \ref{xca:Eisenstein's criterion}) with $p=5$,
 we have that $f$ is irreducible (and monic), so $f=f(\alpha,\Q)$.
Thus $[\Q(\alpha):\Q]=\deg f=4$.
It remains to compute $[E:\Q(\alpha)]=[\Q(\alpha,\sqrt{5}):\Q(\alpha)]$.
We have the following situation:
\[
\begin{tikzcd}
	& {E=\Q(\alpha,\sqrt{5})}\arrow[rd,no head]\\
	{\Q(\alpha)}\arrow[ru,no head,"\leq 2"]&& {\Q(\sqrt{5})}\arrow[ld,no head,"2"]  \\
	& {\Q}\arrow[lu,no head,"4"] 
 \end{tikzcd}
\]
Observe that $\Q(\alpha,\sqrt{5})$ is equal to the composite of $\Q(\alpha)$ and $\Q(\sqrt{5})$.
 We can use the property of composite extension,
$[LF:L]\leq[F:K]$, to deduce that 
 \[
 [\Q(\alpha,\sqrt{5}):\Q(\alpha)]\leq [\Q(\sqrt{5}):\Q]=2.
 \]
 The last equality is because $f(\sqrt{5},\Q)=X^2-5$,
 as it is monic has $\sqrt{5}$ as a root 
 and it's irreducible 
 (due to Eisenstein's criterion or
 because it is of degree 2 with 2 non-rational roots).
 Finally, we want to understand whether $[\Q(\alpha,\sqrt{5}):\Q(\alpha)]$ is 1 or 2.
 Note that $\alpha^4-5\alpha^2+5=0$, so we can solve the equation for
$\alpha^2$ as it is a root of $X^2-5X+5$, i.e.
 \[
\alpha^2=\frac{5\pm \sqrt{25-20}}{2}=\frac{5\pm \sqrt{5}}{2},
 \]
 hence $\sqrt{5}=\pm (2\alpha^2-5)\in\Q(\alpha)$.
 So $\Q(\alpha,\sqrt{5})\subseteq\Q(\alpha)\subseteq \Q(\alpha,\sqrt{5})$, 
which means that $E=\Q(\alpha)$ and
$[E:\Q]=[\Q(\alpha):\Q]=4$.
\end{sol}


\begin{sol}{xca:Q(sqrt[3]{2},xi) normal}
First of all, note that $\sqrt[3]{2}$ is a root of
the polynomial $f(X)=X^3-2$.
To prove that $\Q(\sqrt[3]{2},\xi)$ is a normal extension 
we use Proposition \ref{pro:normal<=>dec},
so it is enough to prove
that $\Q(\sqrt[3]{2},\xi)$ is the decomposition field of $f$.
We know that the decomposition field $E$ of $f$ over $\Q$ is
$\Q$ extended with the roots of $f$, i.e.
$E=\Q(\sqrt[3]{2},\sqrt[3]{2}\xi,\sqrt[3]{2}\xi^2)$.
But it's easy to see that actually 
\[
\Q(\sqrt[3]{2},\xi)=\Q(\sqrt[3]{2},\sqrt[3]{2}\xi,\sqrt[3]{2}\xi^2)=E.
\]
The inclusion $\subseteq$ is because $\sqrt[3]{2},  \xi=\frac{\sqrt[3]{2}\xi}{\sqrt[3]{2}}\in E$. 
Vice versa $\supseteq$ is due to the fact that 
the roots of $f$ are products of $\sqrt[3]{2}$ and $\xi$, elements in $\Q(\sqrt[3]{2},\xi)$.
\end{sol}


\begin{sol}{xca:Q[sqrt[4]{7}+sqrt{2}]}
Let $\alpha=\sqrt[4]{7}+\sqrt{2}$. Then $(\alpha - \sqrt{2})^4 -7 = 0$.
By expanding the left side, we get
\[
0  =\alpha^4 - 4\sqrt{2}\alpha^3 + 12 \alpha^2 - 8\sqrt{2}\alpha - 3 
 = (\alpha^4 + 12\alpha^2 - 3) -  (4 \alpha^3 + 8 \alpha )\sqrt{2}.
\]
But $4 \alpha^3 + 8 \alpha = 4\alpha (\alpha^2+2) \neq 0$, otherwise $\alpha\in\{0,\pm i\sqrt{2})$.
Therefore $\sqrt{2}=\frac{\alpha^4 + 12\alpha^2 - 3}{4 \alpha^3 + 8 \alpha}\in \Q(\alpha) $.

This allows us to prove that $\Q(\sqrt{2} , \sqrt[4]{7}) = \Q(\alpha)$.
From the definition of $\alpha$ it's clear that 
$\Q(\alpha) \subseteq \Q(\sqrt{2} , \sqrt[4]{7})$.
On the other hand, we just proved that $\sqrt{2} \in \Q(\alpha)$.
As $\sqrt[4]{7} = \alpha - \sqrt{2} \in \Q(\alpha)$,
we also see that $\sqrt[4]{7} \in \Q(\alpha)$.
It follows that $\Q(\sqrt{2} , \sqrt[4]{7}) \subseteq \Q(\alpha)$.

Moreover, $\sqrt{2}\not\in \Q(\sqrt[4]{7})$. 
Otherwise, as $[\Q(\sqrt[4]{7}):\Q] = 4$ and $[\Q(\sqrt{2}):\Q] = 2$
we would get that $[\Q(\sqrt[4]{7}):\Q(\sqrt{2})] = 2$.
Let $f(\sqrt[4]{7},\Q(\sqrt{2})) = X^2 + \beta X + \gamma$,
with $\beta, \gamma \in \Q(\sqrt{2})$. 
So 
$$0=f(\sqrt[4]{7})=\sqrt{7} + \beta \sqrt[4]{7} + \gamma.$$
Therefore 
$$\beta^2 \sqrt{7}  = (- \sqrt{7} - \gamma)^2 =7 + 2\gamma \sqrt{7}  + \gamma^2.$$
So $(\beta^2 - 2 \gamma)\sqrt{7}= \gamma^2 + 7$.
But $\beta^2 - 2 \gamma\neq 0$ because 
$\gamma^2 + \beta \gamma + \frac{\beta^2}{2} = 0$ 
holds only for $\gamma =\frac{\beta}{2} (-1 \pm i)\in \C\setminus\R$,
which is clearly not in $\Q(\sqrt{2})$.
Thus we would have 
$\sqrt{7} = \frac{\gamma^2 + 7}{\beta^2 - 2 \gamma} \in \Q(\sqrt{2})$, which 
is a contradiction.

To sum up we have that $\sqrt{2}\not\in \Q(\sqrt[4]{7})$ and
\[
\begin{tikzcd}
	& {E=\Q(\alpha)=\Q(\sqrt{2},\sqrt[4]{7})}\arrow[rd,no head]\\
	{\Q(\sqrt{2})}\arrow[ru,no head,]&& {\Q(\sqrt[4]{7})}\arrow[ld,no head,"4"]  \\
	& {\Q}\arrow[lu,no head,"2"] 
 \end{tikzcd}
\]
\begin{enumerate}
    \item We know that $\sqrt[4]{7} \in \Q(\alpha)$ 
    which has minimal polynomial $f(\sqrt[4]{7},\Q) = x^4-7$.
    One root of this polynomial is $i\sqrt[4]{7}$.
    This root is not in $\Q(\alpha)\subseteq \R$ as it is in $\C\setminus\R$. 
    Therefore $\Q(\alpha)/\Q$ is not normal by Proposition \ref{pro:linear_factorization}.
    \item  As $\sqrt{2} \not\in \Q(\sqrt[4]{7})$,
    we see that $[\Q(\alpha):\Q(\sqrt[4]{7})] > 1$.
    On the other hand, 
    $$[\Q(\alpha):\Q(\sqrt[4]{7})] \leq [\Q(\sqrt{2}:\Q]=2,$$
    which proves that $[\Q(\alpha):\Q(\sqrt[4]{7})] = 2$.
    Therefore,
    \[
     [E:\Q]=[\Q(\alpha):\Q] = [\Q(\alpha):\Q(\sqrt[4]{7})] \cdot [\Q(\sqrt[4]{7}): \Q] = 2 \cdot 4 = 8.
    \]
    \item Let $\sigma\in G=\Gal(E/\Q)$.
    Since $E=\Q(\sqrt{2},\sqrt[4]{7})$ and 
    $\sqrt{2}$ and $\sqrt[4]{7}$ are independent 
    because $\sqrt{2}\not\in \Q(\sqrt[4]{7})$,
    we know that $\sigma$ is completely determined
    by $\sigma(\sqrt{2})$ and $\sigma(\sqrt[4]{7})$.
    By Proposition \ref{pro:conjugate},
    $\sigma(\sqrt{2})\in E$ has to be a root of $f(\sqrt{2},\Q)=X^2-2$ 
    and $\sigma(\sqrt[4]{7})\in E$ has to be a root of $f(\sqrt[4]{7},\Q)=X^4-7$.
    So $\sigma(\sqrt{2})=\pm\sqrt{2}$ and, since $E\subseteq \R$,
    $$\sigma(\sqrt[4]{7})\in E\cap\{\sqrt[4]{7}i^j\mid j\in\{0,1,2,3\}\}=\{\pm\sqrt[4]{7}\}.$$
    Therefore $G$ contains 4 elements $\sigma_{k,l}$ for $k,l\in \Z/2$
    such that $\sigma_{k,l}(\sqrt{2})=(-1)^k\sqrt{2}$ 
    and $\sigma_{k,l}(\sqrt[4]{7})=(-1)^l\sqrt[4]{7}$.
    This gives directly the isomorphism between $G$ and $\Z/2\times\Z/2$. 
\end{enumerate}
\end{sol}



\begin{sol}{xca:dim}
 Let $\{v_i:i\in I\}$ be a basis of $V$ over $K$. For each $i\in I$
 let $f_i\colon V\to F$, $f_i(v_j)=\delta_{ij}$. Then $\{f_i:i\in I\}$ is linearly 
 independent over $F$. In fact, let 
 $\sum a_if_i=0$, where each $a_i\in F$. Then 
 $a_i=0$ for almost all $i$. If $j\in I$, then 
 \[
 0=\left(\sum a_if_i\right)(v_j)=\sum a_if_i(v_j)=a_j.
 \]
 Now assume that $\dim_KV=n$. Let $\{v_1,\dots,v_n\}$ be a basis of $V$ over $K$.
 We claim that $\{f_1,\dots,f_n\}$ is a basis of $\Hom_K(V,F)$ over $F$. If 
 $g\in\Hom_K(V,F)$, then $g=\sum g(v_i)f_i$. If $1\leq k\leq n$, then
 \[
 \left(\sum g(v_i)f_i\right)(v_k)=\sum g(v_i)f_i(v_k)=g(v_k).
 \]
\end{sol}



\begin{sol}{xca:gamma_C}
We need to find a bijective map 
\[
\Hom(E/K,C/K)\to\Hom(E/K,C_1/K).
\]
If $\sigma\in\Hom(E/K,C/K)$, then $\theta^{-1}\sigma\in\Hom(E/K,C_1/K)$. 
If $\varphi\in\Hom(E/K,C_1/K)$, then $\theta\varphi\in\Hom(E/K,C/K)$. The 
maps $\sigma\mapsto\theta^{-1}\sigma$ and 
$\varphi\mapsto\theta\varphi$ are inverse to each other. 
\end{sol}

\begin{sol}{xca:order_reversing}
    We first prove that for every order
    reversing function $\varphi$
    and every element $s,t$ in its domain,
    \[
    \varphi(s\vee t)\leq \varphi(s)\wedge \varphi(t) \text{ and }
    \varphi(s)\vee \varphi(t)\leq \varphi(s\wedge t).
    \]
    Since that $s\leq s\vee t$
    and $t\leq s\vee t$ and $\varphi$
    is order reversing,
    we have that $\varphi(s\vee t)\leq \varphi(s)$ 
    and $\varphi(s\vee t)\leq \varphi(t)$.
    Hence $\varphi(s\vee t)\leq \varphi(s)\wedge \varphi(t).$
    Moreover $s\wedge t\leq s$
    and $s\wedge t\leq t$.
    So $\varphi(s)\leq \varphi(s\wedge t)$ and $\varphi(t)\leq \varphi(s\wedge t)$.
    Hence 
    $\varphi(s)\vee \varphi(t)\leq \varphi(s\wedge t).$

    We can now apply this result
    for $\varphi=f$, $s=x$ and $t=y$
    obtaining that
    \[
    f(x\vee y)\leq f(x)\wedge f(y), \quad
    f(x)\vee f(y)\leq f(x\wedge y).
    \]
    On the other hand, for $\varphi=f^{-1}$, 
    $s=f(x)$ and $t=f(y)$,
    we obtain that 
    \[
    f^{-1}(f(x)\vee f(y))\leq
    f^{-1}(f(x))\wedge f^{-1}(f(y))=
    x\wedge y,
    \]
    and
    \[
    x\vee y=f^{-1}(f(x))\vee f^{-1}(f(y))\leq f^{-1}\big(f(x)\wedge f(y)\big).
    \]
    Thus, applying $f$, which is order reversing,
    it implies that 
    \[
    f(x\wedge y)\leq f\big(f^{-1}(f(x)\vee f(y))\big)=
    f(x)\vee f(y)
    \]
    and
    \[
    f(x)\wedge f(y)=f\big(f^{-1}\big(f(x)\wedge f(y)\big)\big)\leq f(x\vee y).
    \]
 \end{sol}

 \begin{sol}{xca:Cauchy+Galois}
     Since $E/K$ is a Galois extension,
     the order of $\Gal(E/K)$ is precisely $[E:K]=n$.
     So, by Cauchy's Theorem, there exists a subgroup $S$ 
     of $\Gal(E/K)$
     of order $p$.
     Then, by Galois' Theorem, the subextension
     ${}^SE/K$
     has degree equal to the index of $S$,
     which is $n/p$.
 \end{sol}

 \begin{sol}{xca:Sylow+Galois}
     Since $E/K$ is a Galois extension,
     $|\Gal(E/K)|=[E:K]=p^\alpha m$.
     So, by Sylow's Theorem, there exists a subgroup $P$ 
     of $\Gal(E/K)$
     of order $p^{\alpha}$.
     Then, by Galois' Theorem, the subextension
     ${}^PE/K$
     has degree $(\Gal(E/K):P)=m$.
 \end{sol}

\begin{sol}{xca:separable_charp}
    Write $f=X^n+a_{n-1}X^{n-1}+\cdots+a_1X+a_0$. Then 
    \[ 
    f'=nX^{n-1}+(n-1)a_{n-1}X^{n-2}+\cdots+2a_2X+a_1.
    \]
    Since $f$ is not separable, $f'=0$. Thus $n=ka_k=0$ in $K$ for all $k\in\{0,\dots,n-1\}$. This implies
    that $p$ divides $k$ whenever $a_k\ne 0$. This means that the only terms in $f$ occur in degree 
    that are multiples of $p$. In particular, $n=pm$ for some $m$. Hence 
    \[
    f=X^{pm}+a_{p(n-1)}X^{p(m-1)}+\cdots+a_pX^p+a_0=g(X^p)
    \]
    for some $g\in K[X]$.  
\end{sol}

% \begin{sol}\
%     \begin{enumerate}
%         \item If $E/K$ is not separable, 
%         then $[E:K]_{\operatorname{ins}}=p^s$ for some $s$. 
%         Then the trace is zero because $K$ is of characteristic $p$. 
%         \item 
%     \end{enumerate}
% \end{sol}

\begin{sol}{xca:solvable+simple}
 If $G$ is solvable, then $[G,G]$ is a proper normal subgroup of $G$. 
 Since $G$ is simple, $[G,G]=\{1\}$ and $G$ is abelian. Thus $G$ is cyclic of prime order.
\end{sol}

\begin{sol}{xca:diagonal}
 Assume that $G$ is simple. Let $A=G\times\{1\}$, $B=\{1\}\times G$ and
 $D=\{(x,x):x\in G\}$ the diagonal subgroup of $G\times G$. 
 Since 
 \[
 (g,h)=(g,1)(1,h)=(gh^{-1},1)(h,h)
 \]
 it follows that $G=AB=AD$. Let $M$ be a subgroup of $G\times G$ that contains $D$. 
 Note that
 \[
 M=M\cap (G\times G)=M\cap AD=(M\cap A)D. 
 \]
 Similarly, $M=(M\cap B)D$. Since $A$ is normal in $G\times G$, $M\cap A$ is normal in $G\times G$ 
 and $(M\cap A)B$ is normal in $MB=G\times G$. Using the second isomorphism theorem, we see that
 \[
 M\cap A\simeq \frac{(M\cap A)B}{B}
 \]
 is a normal subgroup of $(G\times G)/B\simeq A$. Since $A\simeq G$ is simple, either 
 $M\cap A=\{1\}$ or $M\cap A=A$. Thus either $M=D$ or $BD=G\times G$. Therefore $D$ is maximal.
 \end{sol}
