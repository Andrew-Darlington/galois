\section*{Some solutions}

% \pagestyle{plain}
% \fancyhf{}
% \fancyhead[LE,RO]{Rings and modules}
% \fancyhead[RE,LO]{Some solutions}
% \fancyfoot[CE,CO]{\leftmark}
% \fancyfoot[LE,RO]{\thepage}

%\addcontentsline{toc}{chapter}{Some solutions}

% \begin{sol}{xca:algebraic_bijective}
%     Let $\sigma\colon C\to C$ be a homomorphism. As $\sigma$ is
%     injective, we need to prove that $\sigma$ is surjective. Let $y\in C$. Note that $y$ is algebraic over $K$. Let 
%     $R$ be the set of roots of the minimal polynomial 
%     $f(y,K)$ of $y$ over $K$. 
%     The map 
%     $\sigma|_R\colon R\to R$ is injective. Since 
%     $R$ is finite, $\sigma|_R$ is then bijective. In particular, 
%     there exists $x\in R$ such that $y=\sigma(x)$.
% \end{sol}
\begin{sol}{xca:degree_of_x}
    Let $f=f(x,K)$ be the minimal polynomial of $x$ over $K$ of degree $\deg(f)=n$.
    We claim that $\{1,x,\dots, x^{n-1}\}$ is a basis of $K(x)$ as a $K$-vector space. 

    To prove that $\{1,x,\dots, x^{n-1}\}$ is a generating set, recall that $K(x)=K[x]$, since $x$ is algebraic over $K$. 
    Let $z\in K(x)=K[x]$, say $z=h(x)$ for some $h\in K[X]$. 
    Divide $h$ by $f$ to obtain polynomials $q,r\in K[X]$ 
    such that $h=fq+r$, where either $r=0$ or $\deg r<\deg f=n$. Then 
    \[
		z=h(x)=f(x)q(x)+r(x)=r(x).
	\]
	Write $r=\sum_{i=0}^{n-1}c_iX^i$ for some $c_0,\dots,c_{n-1}\in K$. 
    Thus $z=\sum_{i=0}^{n-1}a_ix^i\in \langle 1,x,\dots,x^{n-1}\rangle$.
        
    We now prove that $\{1,x,\dots, x^{n-1}\}$ is linearly independent. If not, 
    there exists a linear combination
    $0=\sum_{i=0}^{n-1}a_ix^i$ with $a_0,\dots,a_{n-1}\in K$ not all zero. 
    Then $h(X)=\sum_{i=0}^{n-1}a_iX^i\in K[X]\setminus\{0\}$
    has $x$ as a root and 
        \[
        n=\deg(f)\leq \deg(h)\leq n-1,
        \]
       a contradiction. 
\end{sol}


\begin{sol}{xca:dim}
 Let $\{v_i:i\in I\}$ be a basis of $V$ over $K$. For each $i\in I$
 let $f_i\colon V\to F$, $f_i(v_j)=\delta_{ij}$. Then $\{f_i:i\in I\}$ is linearly 
 independent over $F$. In fact, let 
 $\sum a_if_i=0$, where each $a_i\in F$. Then 
 $a_i=0$ for almost all $i$. If $j\in I$, then 
 \[
 0=\left(\sum a_if_i\right)(v_j)=\sum a_if_i(v_j)=a_j.
 \]
 Now assume that $\dim_KV=n$. Let $\{v_1,\dots,v_n\}$ be a basis of $V$ over $K$.
 We claim that $\{f_1,\dots,f_n\}$ is a basis of $\Hom_K(V,F)$ over $F$. If 
 $g\in\Hom_K(V,F)$, then $g=\sum g(v_i)f_i$. If $1\leq k\leq n$, then
 \[
 \left(\sum g(v_i)f_i\right)(v_k)=\sum g(v_i)f_i(v_k)=g(v_k).
 \]
\end{sol}


\begin{sol}{xca:gamma_C}
We need to find a bijective map 
\[
\Hom(E/K,C/K)\to\Hom(E/K,C_1/K).
\]
If $\sigma\in\Hom(E/K,C/K)$, then $\theta^{-1}\sigma\in\Hom(E/K,C_1/K)$. 
If $\varphi\in\Hom(E/K,C_1/K)$, then $\theta\varphi\in\Hom(E/K,C/K)$. The 
maps $\sigma\mapsto\theta^{-1}\sigma$ and 
$\varphi\mapsto\theta\varphi$ are inverse to each other. 
\end{sol}

% \begin{sol}\
%     \begin{enumerate}
%         \item If $E/K$ is not separable, 
%         then $[E:K]_{\operatorname{ins}}=p^s$ for some $s$. 
%         Then the trace is zero because $K$ is of characteristic $p$. 
%         \item 
%     \end{enumerate}
% \end{sol}

\begin{sol}{xca:solvable+simple}
 If $G$ is solvable, then $[G,G]$ is a proper normal subgroup of $G$. 
 Since $G$ is simple, $[G,G]=\{1\}$ and $G$ is abelian. Thus $G$ is cyclic of prime order.
\end{sol}

\begin{sol}{xca:diagonal}
 Assume that $G$ is simple. Let $A=G\times\{1\}$, $B=\{1\}\times G$ and
 $D=\{(x,x):x\in G\}$ the diagonal subgroup of $G\times G$. 
 Since 
 \[
 (g,h)=(g,1)(1,h)=(gh^{-1},1)(h,h)
 \]
 it follows that $G=AB=AD$. Let $M$ be a subgroup of $G\times G$ that contains $D$. 
 Note that
 \[
 M=M\cap (G\times G)=M\cap AD=(M\cap A)D. 
 \]
 Similarly, $M=(M\cap B)D$. Since $A$ is normal in $G\times G$, $M\cap A$ is normal in $G\times G$ 
 and $(M\cap A)B$ is normal in $MB=G\times G$. Using the second isomorphism theorem, we see that
 \[
 M\cap A\simeq \frac{(M\cap A)B}{B}
 \]
 is a normal subgroup of $(G\times G)/B\simeq A$. Since $A\simeq G$ is simple, either 
 $M\cap A=\{1\}$ or $M\cap A=A$. Thus either $M=D$ or $BD=G\times G$. Therefore $D$ is maximal.
 \end{sol}
