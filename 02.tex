\chapter{}

If $E/K$ is an extension and $S$ is a subset of $E$, then
there exists a unique smallest 
subextension $F/K$ of $E/K$ such that
$S\subseteq F$. In fact, 
\[
	F=\bigcap\{T:\text{$T$ is a subfield of $E$ that contains $K\cup S$}\} 
\]
If $L/K$ is a subextension of $E/K$ such that 
$S\subseteq L$, then $F\subseteq L$ by definition. The 
extension $F$ is known as the \textbf{subextension generated by} 
$S$ and
it will be denoted by $K(S)$. 
If $S=\{x_1,\dots,x_n\}$ is finite,
then $K(S)=K(x_1,\dots,x_n)$ is said to be of \textbf{finite type}. 

\begin{example}
	If $\{e_1,\dots,e_n\}$ is a basis of $E$ over $K$, 
	then $E=K(e_1,\dots,e_n)$. 
\end{example}

\begin{example}
	The field $\Q(\sqrt{2})$ is precisely the extension 
	of $\R/\Q$ generated by $\sqrt{2}$. 
\end{example}

Let $E/K$ be an extension and $S$ and $T$ be subsets of $E$.
Then 
\[
	K(S\cup T)=K(S)(T)=K(T)(S).
\]
If, moreover, 
$S\subseteq T$, then $K(S)\subseteq K(T)$. 

\topic{Algebraic extensions}

\begin{definition}
	Let $E/K$ be an extension. An element $x\in E$
	is \textbf{algebraic} over $K$ if there
	exists a non-zero polynomial 
	$f(X)\in K[X]$ such that $f(x)=0$. If $x$ is
	not algebraic over $K$, 
	then it is called \textbf{trascendent} over $K$.
\end{definition}

If $E/K$ is an extension, let 
\[
	\overline{K}_E=\{x\in E:x\text{ is algebraic over }K\}. 
\]
%is the \textbf{algebraic closure} of $K$ in $E$. 

\begin{definition}	
	An extension $E/K$ is \textbf{algebraic} if 
	every $x\in E$ is algebraic over $K$. 
\end{definition}

If $K$ is a field, every $x\in K$ is algebraic over $K$,
as $x$ is a root of $X-x\in K[X]$. In particular, $K/K$ is
an algebraic extension. 

\begin{example}
	$\C/\R$ is an algebraic extension. If $z\in\C\setminus\R$, then
	$z$ is a root of the polynomial 
	$X^2+(z+\overline{z})X+|z|^2\in\R[X]$. 
\end{example}

If $F/K$ is an algebraic extension and $x\in E$ is algebraic
over $K$, then $x$ is algebraic over $E$. 

\begin{example}
	$\Q(\sqrt{2})/\Q$ is algebraic, as the number
	$a+b\sqrt{2}$ is a root of the polynomial
	$X^2-2aX+(a^2-2b^2)\in\Q[X]$. 
\end{example}

The extension $\C/\Q$ is not algebraic. 

If $E/K$ is an extension and $x\in E$ is algebraic
over $K$, then the evaluation homomorphism 
$K[X]\to E$, $f\mapsto f(x)$, is not injective. In particular,
its kernel is a non-zero ideal and hence 
it is generated by a monic polynomial $f$. 

\begin{definition}
	Let $E/K$ be an extension and $x\in E$ be an algebraic element.  The monic
	polynomial that generates the kernel of $K[X]\to E$, $f\mapsto f(x)$, is
	known as the \textbf{minimal polynomial} of $x$ over $K$ and it will be
	denoted by $f(x,K)$. The \textbf{degree} of $x$ over $K$ is then $\deg
	f(x,K)$. 
\end{definition}

Some basic properties of the minimal polynomial of an algebraic element:

\begin{proposition}
	Let $E/K$ be an extension and $x\in E$. 
	\begin{enumerate}
		\item If $g\in K[X]\setminus\{0\}$ is such that $g(x)=0$, then $f(x,K)$ divides $g$. In particular, 
		$\deg f(x,K)\leq\deg g$. 
%		\item If $g(x)=0$ and $g\ne 0$, then $\deg g\geq\gr f(x,K)$.
		\item $f(x,K)$ is irreducible in $K[X]$.
		%\item If $g(x)=0$ and $g(X)$ is monic and irreducible, then
		%	$g=f(x,K)$. 
		\item If $F/K$ is a subextension of $E/K$, then $f(x,F)$ divides
			$f(x,K)$. 
	\end{enumerate}
\end{proposition}

\begin{proof}
	Write $f=f(x,K)$ to denote the minimal polynomial of $x$. 
	To prove 1) note that $g(x)=0$ implies that	$g$ belongs to the kernel of
	the evaluation map, so $g$ is a multiple of $f$. To prove 2) 
	note that if $f=pq$ for some $p,q\in K[X]$ such that
	$0<\deg p,\deg q<\deg f$, then $f(x)=0$ implies that 
	either $p(x)=0$ or $q(x)=0$, a
	contradiction. Finally we prove 3). Since $f\in K[X]\subseteq F[X]$ 
	and $f(x)=0$, it follows from 1) that $f(x,F)$ divides $f$. 
\end{proof}

Some easy examples: $f(i,\R)=X^2+1$ and 
$f(\sqrt[3]{2},\Q)=X^3-2$. 

\begin{example}
	Let us compute 
	$f(\sqrt{2}+\sqrt{3},\Q)$. Let $\alpha=\sqrt{2}+\sqrt{3}$. 
	Then 
	\begin{align*}
		\alpha-\sqrt{2}=\sqrt{3} & \implies 
		(\alpha-\sqrt{2})^2=3 \implies \alpha^2-2\sqrt{2}\alpha+2=3\\
		&\implies \alpha^2-1=2\sqrt{2}\alpha \implies
		(\alpha^2-1)^2=8\alpha^2\implies
		\alpha^4-10\alpha^2+1=0.
	\end{align*}
	Thus $\alpha$ is a root of $g=X^4-10X^2+1$. To prove that $g=f(\alpha,\Q)$ 
	it is enough to prove that 
	$g$ is irreducible in $\Q[X]$. First note that 
	the roots
	of $g$ are $\sqrt{2}+\sqrt{3}$, $\sqrt{2}-\sqrt{3}$, 
	$-\sqrt{2}+\sqrt{3}$ and $-\sqrt{2}-\sqrt{3}$. This means that
	if $g$ is not irreducible, 
	then $g=hh_1$ for some polynomials $h,h_1\in\Q[X]$ such that
	$\deg h=\deg h_1=2$. This is not possible, as 
	$(\sqrt{2}+\sqrt{3})+(\sqrt{2}-\sqrt{3})=2\sqrt{2}\not\in\Q$, 
	$(\sqrt{2}+\sqrt{3})+(-\sqrt{2}+\sqrt{3})=2\sqrt{3}\not\in\Q$ and 
	$(\sqrt{2}+\sqrt{3})(-\sqrt{2}-\sqrt{3})=-5-2\sqrt{6}\not\in\Q$.
\end{example}

\begin{proposition}
	Let $F/K$ be a subextension and $E/K$. Then
	\[
	[E:K]=[E:F][F:K].
	\]
\end{proposition}

\begin{proof}
	Let $\{e_i:i\in I\}$ be a basis of $E$ over $F$
	and $\{f_j:j\in J\}$ be a basis of $F$ over $K$. If $x\in E$,
	then $x=\sum_i \lambda_ie_i$ (finite sum) 
	for some $\lambda_i\in F$. For each $i$, 
	$\lambda_i=\sum_j a_{ij}f_j$ (finite sum)
	for some $a_{ij}\in K$. Then 
	$x=\sum_i\sum_j a_{ij}(f_je_i)$. This means
	that $\{f_je_i:i\in I,j\in J\}$ generates
	$E$ as a $K$-vector space. Let us prove that 
	$\{f_je_i:i\in I,j\in J\}$
	is linearly independent. If $\sum_i\sum_j a_{ij}(f_je_i)=0$ (finite sum)
	for some $a_{ij}\in K$, 
	then
	\begin{align*}
		0=\sum_i\left(\sum_j a_{ij}f_j\right)e_i&\implies
		\sum_j a_{ij}f_j=0\text{ for all $i\in I$}\\
		&\implies 
		a_{ij}=0\text{ for all $i\in I$ and $j\in J$}.\qedhere
	\end{align*}
\end{proof}

We state a lemma:

\begin{lemma}
If $A$ is a finite-dimensional commutative algebra over $K$ 
and $A$ is an integral domain, then $A$ is a field. 
\end{lemma}

\begin{proof}
	Let $a\in A\setminus\{0\}$. We need to prove that there exists $b\in A$
	such that $ab=1$. Let $\theta\colon A\to A$, $x\mapsto ax$. Clearly
	$\theta$ is an algebra homomorphism. It is injective, since $A$ is an
	integral domain.  Since $\dim_KA<\infty$, it follows that $\theta$ is an
	isomorphism. In particular, $\theta(A)=A$, which means that there exists
	$b\in A$ such that $1=ab$. 
\end{proof}

Let $E/K$ be an extension and $x\in E\setminus K$. 
Then 
\[
K[x]=\{y=f(x):\text{ for some $f\in K[X]$}\}
\]
is a subring of $E$ that contains $K$. More generally,
if $x_1,\dots,x_n\in E$, then
\[
K[x_1,\dots,x_n]=\{f(x_1,\dots,x_n):f\in K[X_1,\dots,X_n]\}
\]
is a subring of $E$. Clearly, $K[x_1,\dots,x_n]$ is a domain
and 
\[
K(x_1,\dots,x_n)=\left\{\frac{f(x_1,\dots,x_n)}{g(x_1,\dots,x_n)}:f,g\in K[X_1,\dots,X_m]\text{ with $g(x_1,\dots,x_n)\ne 0$}\right\}
\]
is the extension of $K$ generated by $x_1,\dots,x_n$. 
Note that 
\[
K(x_1,\dots,x_n)=(K(x_1,\dots,x_{n-1}))(x_n).
\]
The previous construction
can be generalized. Let $I$ be a non-empty set. 
For each $i\in I$ let $X_i$ be an indeterminate. Consider
the polynomial ring $K[\{X_i:i\in I\}]$ and let 
$S=\{x_i:i\in I\}$ be a subset of $E$. There exists a unique 
algebras homomorphism $K[\{X_i:i\in I\}]\to E$ 
such that $X_i\mapsto x_i$ for all $i\in I$. The image 
is denoted by $K[S]$. 

\begin{exercise}
    Prove that $\Q[\sqrt{2}]=\Q(\sqrt{2})$. 
\end{exercise}

\begin{theorem}
	Let $E/K$ be an extension and $x\in E\setminus K$.
	The following statements are equivalent:
	\begin{enumerate}
		\item $x$ is algebraic over $K$.
		\item $\dim_KK[x]<\infty$.
		\item $K[x]$ is a field.
		\item $K[x]=K(x)$. 
	\end{enumerate}
\end{theorem}

\begin{proof}
	We first prove $1)\implies 2)$. Let $z\in K[x]$, say $z=h(x)$ for some $h\in K[X]$. There exists
	$g\in K[X]$ such that $g\ne 0$ and $g(x)=0$. Divide $h$ by $g$ to obtain 
	polynomials $q,r\in K[X]$ such that $h=gq+r$, where $r=0$ or $\deg r<\deg g$. This implies that
	\[
		z=h(x)=g(x)q(x)+r(x)=r(x).
	\]
	If $\deg g=m$, then $r=\sum_{i=0}^{m-1}a_iX^i$ for some $a_0,\dots,a_{m-1}\in K$. Thus
	$z=\sum_{i=0}^{m-1}a_ix^i$, so $K[x]\subseteq\langle 1,x,\dots,x^{m-1}\rangle$. 

	The previous lemma proves that $2)\implies 3)$. 

	It is trivial that $3)\implies 4)$. 

	It remains to prove that $4)\implies 1)$. 
	Since $x\ne 0$, $1/x\in K[x]$. There exists $a_0,\dots,a_n\in K$ such that
	$1/x=a_0+a_1x+\cdots+a_nx^n$. Thus
	\[
		a_nx^{n+1}+\cdots+a_1x^2+a_0x-1- 0
	\]
	so $x$ is a root of $a_nX^{n+1}+\cdots+a_0X-1\in K[X]\setminus\{0\}$. 
\end{proof}

Note that if $x$ is algebraic over $K$, then
$K[x]\simeq K[X]/(f(x,K))$. 

\begin{corollary}
	If $E/K$ is finite, then $E/K$ is algebraic. 
\end{corollary}

\begin{proof}
	Let $n=[E:K]$ and $x\in E$. The set $\{1,x,\dots,x^n\}$ is linearly dependent, 
	so there exist $a_0,\dots,a_n\in K$ not all zero such that
	$a_0+a_1x+\cdots+a_nx^n=0$. Thus $x$ is a root of the non-zero
	polynomial $a_0+a_1X+\cdots+a_nX^n\in K[X]$. 
\end{proof}

We note that the converse of the previous corollary does not hold. 

\begin{corollary}
	If $E/K$ is an extension and $x_1,\dots,x_n\in E$ 
	are algebraic over $K$, then 
	$K(x_1,\dots,x_n)/K$ is finite and
	$K(x_1,\dots,x_m)=K[x_1,\dots,x_n]$. 
\end{corollary}

\begin{proof}
	We proceed by induction on $n$. The case $n=1$ follows immediately from 
	the theorem. So assume the result holds for some $n\geq1$. Since the extensions 
	$K(x_1,\dots,x_n)/K(x_1,\dots,x_{n-1})$ and $K(x_1,\dots,x_{n-1})/K$ are
	both finite, it follows that $K(x_1,\dots,x_n)/K$ is finite. Moreover, 
	\begin{align*}
	K(x_1,\dots,x_n)&=K(x_1,\dots,x_{n-1})(x_n)\\
	&=K(x_1,\dots,x_{n-1})[x_n]=K[x_1,\dots,x_{n-1}][x_n]=K[x_1,\dots,x_n].\qedhere
    \end{align*}
\end{proof}

\begin{corollary}
	Let $E=K(S)$. Then $E/K$ is algebraic if and only if
	$x$ is algebraic over $K$ for all $x\in S$. 
\end{corollary}

\begin{proof}
	Let us prove the non-trivial implication. Let $z\in K(S)$. In particular, 
	there exists a finite subset $T\subseteq S$ such that 
	$z\in K(T)$. The previous corollary implies that $K(T)/K$ is algebraic and
	hence $z$ is algebraic. 
\end{proof}

\begin{corollary}
	If $E/K$ is  an extension, then $\overline{K}_E$ 
	is a subfield of $E$ that contains $K$. Moreover, 
	$K(\overline{K}_E)/K$ is algebraic. 
\end{corollary}	

\begin{proof}
    By definition, $K(\overline{K}_E)/K$ is algebraic. 
    Thus $K(\overline{K}_E)\subseteq\overline{K}_E$. From this it follows that
    $K(\overline{K}_E)=\overline{K}_E$. 
\end{proof}

The following exercise is now almost trivial:

\begin{exercise}
    Let $E/K$ be an extension of finite type. 
    Prove that $E/K$ is algebraic if and only if $E/K$ 
    is finite. 
\end{exercise}

Let $\overline{\Q}=\{\alpha\in\C:\alpha\text{ is algebraic over }\Q\}$. 
Then $\overline{\Q}$ is the field of algebraic numbers. 
Can you compute $[\overline{\Q}:\Q]$?

