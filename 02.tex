\chapter{}

If $E/K$ is an extension and $S$ is a subset of $E$, then
there exists a unique smallest 
subextension $F/K$ of $E/K$ such that
$S\subseteq F$. In fact, 
\[
	F=\bigcap\{T:\text{$T$ is a subfield of $E$ that contains $K\cup S$}\} 
\]
If $L/K$ is a subextension of $E/K$ such that 
$S\subseteq L$, then $F\subseteq L$ by definition. The 
extension $F$ is known as the \textbf{subextension generated by} 
$S$ and
it will be denoted by $K(S)$. 
If $S=\{x_1,\dots,x_n\}$ is finite,
then $K(S)=K(x_1,\dots,x_n)$ is said to be of \textbf{finite type}. 

\begin{example}
	If $\{e_1,\dots,e_n\}$ is a basis of $E$ over $K$, 
	then $E=K(e_1,\dots,e_n)$. 
\end{example}

\begin{example}
	The field $\Q(\sqrt{2})$ is precisely the extension 
	of $\R/\Q$ generated by $\sqrt{2}$. 
\end{example}

Let $E/K$ be an extension and $S$ and $T$ be subsets of $E$.
Then 
\[
	K(S\cup T)=K(S)(T)=K(T)(S).
\]
If, moreover, 
$S\subseteq T$, then $K(S)\subseteq K(T)$. 

\topic{Algebraic extensions}

\begin{definition}
	Let $E/K$ be an extension. An element $x\in E$
	is \textbf{algebraic} over $K$ if there
	exists a non-zero polynomial 
	$f(X)\in K[X]$ such that $f(x)=0$. If $x$ is
	not algebraic over $K$, 
	then it is called \textbf{trascendent} over $K$.
\end{definition}

If $E/K$ is an extension, let 
\[
	\overline{K}_E=\{x\in E:x\text{ is algebraic over }K\}. 
\]
%is the \textbf{algebraic closure} of $K$ in $E$. 

\begin{definition}	
	An extension $E/K$ is \textbf{algebraic} if 
	every $x\in E$ is algebraic over $K$. 
\end{definition}

If $K$ is a field, every $x\in K$ is algebraic over $K$,
as $x$ is a root of $X-x\in K[X]$. In particular, $K/K$ is
an algebraic extension. 

\begin{example}
	$\C/\R$ is an algebraic extension. If $z\in\C\setminus\R$, then
	$z$ is a root of the polynomial 
	$X^2+(z+\overline{z})X+|z|^2\in\R[X]$. 
\end{example}

If $F/K$ is an algebraic extension and $x\in E$ is algebraic
over $K$, then $x$ is algebraic over $E$. 

\begin{example}
	$\Q(\sqrt{2})/\Q$ is algebraic, as the number
	$a+b\sqrt{2}$ is a root of the polynomial
	$X^2-2aX+(a^2-2b^2)\in\Q[X]$. 
\end{example}

The extension $\C/\Q$ is not algebraic. 

If $E/K$ is an extension and $x\in E$ is algebraic
over $K$, then the evaluation homomorphism 
$K[X]\to E$, $f\mapsto f(x)$, is not injective. In particular,
its kernel is a non-zero ideal and hence 
it is generated by a monic polynomial $f$. 

\begin{definition}
	Let $E/K$ be an extension and $x\in E$ be an algebraic element.  The monic
	polynomial that generates the kernel of $K[X]\to E$, $f\mapsto f(x)$, is
	known as the \textbf{minimal polynomial} of $x$ over $K$ and it will be
	denoted by $f(x,K)$. The \textbf{degree} of $x$ over $K$ is then $\deg
	f(x,K)$. 
\end{definition}

Some basic properties of the minimal polynomial of an algebraic element:

\begin{proposition}
	Let $E/K$ be an extension and $x\in E$. 
	\begin{enumerate}
		\item If $g\in K[X]\setminus\{0\}$ is such that $g(x)=0$, then $f(x,K)$ divides $g$. In particular, 
		$\deg f(x,K)\leq\deg g$. 
%		\item If $g(x)=0$ and $g\ne 0$, then $\deg g\geq\gr f(x,K)$.
		\item $f(x,K)$ is irreducible in $K[X]$.
		%\item If $g(x)=0$ and $g(X)$ is monic and irreducible, then
		%	$g=f(x,K)$. 
		\item If $F/K$ is a subextension of $E/K$, then $f(x,F)$ divides
			$f(x,K)$. 
	\end{enumerate}
\end{proposition}

\begin{proof}
	Write $f=f(x,K)$ to denote the minimal polynomial of $x$. 
	To prove 1) note that $g(x)=0$ implies that	$g$ belongs to the kernel of
	the evaluation map, so $g$ is a multiple of $f$. To prove 2) 
	note that if $f=pq$ for some $p,q\in K[X]$ such that
	$0<\deg p,\deg q<\deg f$, then $f(x)=0$ implies that 
	either $p(x)=0$ or $q(x)=0$, a
	contradiction. Finally we prove 3). Since $f\in K[X]\subseteq F[X]$ 
	and $f(x)=0$, it follows from 1) that $f(x,F)$ divides $f$. 
\end{proof}

Some easy examples: $f(i,\R)=X^2+1$ and 
$f(\sqrt[3]{2},\Q)=X^3-2$. 

\begin{example}
	Let us compute 
	$f(\sqrt{2}+\sqrt{3},\Q)$. Let $\alpha=\sqrt{2}+\sqrt{3}$. 
	Then 
	\begin{align*}
		\alpha-\sqrt{2}=\sqrt{3} & \implies 
		(\alpha-\sqrt{2})^2=3 \implies \alpha^2-2\sqrt{2}\alpha+2=3\\
		&\implies \alpha^2-1=2\sqrt{2}\alpha \implies
		(\alpha^2-1)^2=8\alpha^2\implies
		\alpha^4-10\alpha^2+1=0.
	\end{align*}
	Thus $\alpha$ is a root of $g=X^4-10X^2+1$. To prove that $g=f(\alpha,\Q)$ 
	it is enough to prove that 
	$g$ is irreducible in $\Q[X]$. First note that 
	the roots
	of $g$ are $\sqrt{2}+\sqrt{3}$, $\sqrt{2}-\sqrt{3}$, 
	$-\sqrt{2}+\sqrt{3}$ and $-\sqrt{2}-\sqrt{3}$. This means that
	if $g$ is not irreducible, 
	then $g=hh_1$ for some polynomials $h,h_1\in\Q[X]$ such that
	$\deg h=\deg h_1=2$. This is not possible, as 
	$(\sqrt{2}+\sqrt{3})+(\sqrt{2}-\sqrt{3})=2\sqrt{2}\not\in\Q$, 
	$(\sqrt{2}+\sqrt{3})+(-\sqrt{2}+\sqrt{3})=2\sqrt{3}\not\in\Q$ and 
	$(\sqrt{2}+\sqrt{3})(-\sqrt{2}-\sqrt{3})=-5-2\sqrt{6}\not\in\Q$.
\end{example}

\begin{proposition}
	Let $F/K$ be a subextension and $E/K$. Then
	\[
	[E:K]=[E:F][F:K].
	\]
\end{proposition}

\begin{proof}
	Let $\{e_i:i\in I\}$ be a basis of $E$ over $F$
	and $\{f_j:j\in J\}$ be a basis of $F$ over $K$. If $x\in E$,
	then $x=\sum_i \lambda_ie_i$ (finite sum) 
	for some $\lambda_i\in F$. For each $i$, 
	$\lambda_i=\sum_j a_{ij}f_j$ (finite sum)
	for some $a_{ij}\in K$. Then 
	$x=\sum_i\sum_j a_{ij}(f_je_i)$. This means
	that $\{f_je_i:i\in I,j\in J\}$ generates
	$E$ as a $K$-vector space. Let us prove that 
	$\{f_je_i:i\in I,j\in J\}$
	is linearly independent. If $\sum_i\sum_j a_{ij}(f_je_i)=0$ (finite sum)
	for some $a_{ij}\in K$, 
	then
	\begin{align*}
		0=\sum_i\left(\sum_j a_{ij}f_j\right)e_i&\implies
		\sum_j a_{ij}f_j=0\text{ for all $i\in I$}\\
		&\implies 
		a_{ij}=0\text{ for all $i\in I$ and $j\in J$}.\qedhere
	\end{align*}
\end{proof}

We state a lemma:

\begin{lemma}
If $A$ is a finite-dimensional commutative algebra over $K$ 
and $A$ is an integral domain, then $A$ is a field. 
\end{lemma}

\begin{proof}
	Let $a\in A\setminus\{0\}$. We need to prove that there exists $b\in A$
	such that $ab=1$. Let $\theta\colon A\to A$, $x\mapsto ax$. Clearly
	$\theta$ is an algebra homomorphism. It is injective, since $A$ is an
	integral domain.  Since $\dim_KA<\infty$, it follows that $\theta$ is an
	isomorphism. In particular, $\theta(A)=A$, which means that there exists
	$b\in A$ such that $1=ab$. 
\end{proof}

Let $E/K$ be an extension and $x\in E\setminus K$. 
Then 
\[
K[x]=\{y=f(x):\text{ for some $f\in K[X]$}\}
\]
is a subring of $E$ that contains $K$. 

The previous construction
can be generalized. Let $I$ be a non-empty set. 
For each $i\in I$ let $X_i$ be an indeterminate. Consider
the polynomial ring $K[\{X_i:i\in I\}]$ and let 
$S=\{x_i:i\in I\}$ be a subset of $E$. There exists a unique 
algebras homomorphism $K[\{X_i:i\in I\}]\to E$ 
such that $X_i\mapsto x_i$ for all $i\in I$. The image 
is denoted by $K[S]$. 

\begin{theorem}
	Let $E/K$ be an extension and $x\in E\setminus K$.
	The following statements are equivalent:
	\begin{enumerate}
		\item $x$ is algebraic over $K$.
		\item $\dim_KK[x]<\infty$.
		\item $K[x]$ is a field.
		\item $K[x]=K(x)$. 
	\end{enumerate}
\end{theorem}

\begin{proof}
	We first prove $1)\implies 2)$. Let $z\in K[x]$, say $z=h(x)$ for some $h\in K[X]$. There exists
	$g\in K[X]$ such that $g\ne 0$ and $g(x)=0$. Divide $h$ by $g$ to obtain 
	polynomials $q,r\in K[X]$ such that $h=gq+r$, where $r=0$ or $\deg r<\deg g$. This implies that
	\[
		z=h(x)=g(x)q(x)+r(x)=r(x).
	\]
	If $\deg g=m$, then $r=\sum_{i=0}^{m-1}a_iX^i$ for some $a_0,\dots,a_{m-1}\in K$. Thus
	$z=\sum_{i=0}^{m-1}a_ix^i$, so $K[x]\subseteq\langle 1,x,\dots,x^{m-1}\rangle$. 

	The previous lemma proves that $2)\implies 3)$. 

	It is trivial that $3)\implies 4)$. 

	It remains to prove that $4)\implies 1)$. 
	Since $x\ne 0$, $1/x\in K[x]$. There exists $a_0,\dots,a_n\in K$ such that
	$1/x=a_0+a_1x+\cdots+a_nx^n$. Thus
	\[
		a_nx^{n+1}+\cdots+a_1x^2+a_0x-1- 0
	\]
	so $x$ is a root of $a_nX^{n+1}+\cdots+a_0X-1\in K[X]\setminus\{0\}$. 
\end{proof}

Note that if $x$ is algebraic over $K$, then
$K[x]\simeq K[X]/(f(x,K))$. 

\begin{corollary}
	If $E/K$ is finite, then $E/K$ is algebraic. 
\end{corollary}

\begin{proof}
	Let $n=[E:K]$ and $x\in E$. The set $\{1,x,\dots,x^n\}$ is linearly dependent, 
	so there exist $a_0,\dots,a_n\in K$ not all zero such that
	$a_0+a_1x+\cdots+a_nx^n=0$. Thus $x$ is a root of the non-zero
	polynomial $a_0+a_1X+\cdots+a_nX^n\in K[X]$. 
\end{proof}

We note that the converse of the previous corollary does not hold. 

\begin{corollary}
	If $E/K$ is an extension and $x_1,\dots,x_n\in E$ 
	are algebraic over $K$, then 
	$K(x_1,\dots,x_n)/K$ is finite and
	$K(x_1,\dots,x_m)=K[x_1,\dots,x_n]$. 
\end{corollary}

\begin{proof}
	We proceed by induction on $n$. The case $n=1$ follows immediately from 
	the theorem. So assume the result holds for some $n\geq1$. Since the extensions 
	$K(x_1,\dots,x_n)/K(x_1,\dots,x_{n-1})$ and $K(x_1,\dots,x_{n-1})/K$ are
	both finite, it follows that $K(x_1,\dots,x_n)/K$ is finite. Moreover, 
	\begin{align*}
	K(x_1,\dots,x_n)&=K(x_1,\dots,x_{n-1})(x_n)\\
	&=K(x_1,\dots,x_{n-1})[x_n]=K[x_1,\dots,x_{n-1}][x_n]=K[x_1,\dots,x_n].\qedhere
    \end{align*}
\end{proof}

\begin{corollary}
	Let $E=K(S)$. Then $E/K$ is algebraic if and only if
	$x$ is algebraic over $K$ for all $x\in S$. 
\end{corollary}

\begin{proof}
	Let us prove the non-trivial implication. Let $z\in K(S)$. In particular, 
	there exists a finite subset $T\subseteq S$ such that 
	$z\in K(T)$. The previous corollary implies that $K(T)/K$ is algebraic and
	hence $z$ is algebraic. 
\end{proof}

\begin{corollary}
	If $E/K$ is  an extension, then $\overline{K}_E$ 
	is a subfield of $E$ that contains $K$. Moreover, 
	$K(\overline{K}_E)/K$ is algebraic. 
\end{corollary}	

\begin{proof}
    By definition, $K(\overline{K}_E)/K$ is algebraic. 
    Thus $K(\overline{K}_E)\subseteq\overline{K}_E$. From this it follows that
    $K(\overline{K}_E)=\overline{K}_E$. 
\end{proof}

The following exercise is now almost trivial:

\begin{exercise}
    Let $E/K$ be an extension. Prove that $E/K$ is algebraic if and only if $E/K$ 
    is finite of finite type. 
\end{exercise}


%\begin{theorem}[Galois]
%	\index{Galois' theorem}
%	For every prime number $p$ and every $m\geq1$
%	there exists a field of size $p^m$. 
%\end{theorem}
%
%\begin{proof}
%\end{proof}
%
%
Algebraic field extensions form a nice class of extensions. The same happens
with finite field extensions. 

\begin{proposition}
	Let $F/K$ is a subextension of $E/K$. Then $E/K$ is algebraic 
	if and only if $E/F$ and $F/K$ are algebraic. 
\end{proposition}

\begin{proof}
    We know that if $E/K$ is algebraic, then $E/F$ and $F/K$ are both algebraic. 
    Let us assume that $E/F$ and $F/K$ are both algebraic. Let $x\in E$ and 
    let $L$ be the subextension over $K$ generated by the coefficients of $f(x,F)$, 
    the minimal polynomial of $x$ over $F$. Then $L/K$ is finite, since it is generated
    by finitely many algebraic elements. Moreover, $x$ is algebraic over $L$. Since 
    \[
    [L(x):K]=[L(x):L][L:K]<\infty,
    \]
    $L(x)/K$ is algebraic. In particular, $x$ is algebraic over $K$. 
\end{proof}

\begin{exercise}
	Let $F/K$ is a subextension of $E/K$. Prove that $E/K$ is finite 
	if and only if $E/F$ and $F/K$ are finite. 
\end{exercise}

\begin{exercise}
	Let $E/K$ and $F/K$ be extensions, where both $E$ and $F$ are subfields of 
	a field $L$. If $F/K$ is algebraic, then $EF/E$ is algebraic.
\end{exercise}

% \begin{proof}
%     If $F/K$ is algebraic, then $EF/E=E(F)/E$ is algebraic, as it is generated by 
%     algebraic elements over $E$.  
% \end{proof}

\begin{exercise}
	Let $E/K$ and $F/K$ be extensions, where both $E$ and $F$ are subfields of 
	a field $L$. If $F/K$ is finite, then $EF/E$ is finite.
\end{exercise}

The solution to the previous exercise shows, in particular, that $[EF:E]\leq [F:K]$. 



%\begin{theorem}[Galois]
%	\index{Galois' theorem}
%	For every prime number $p$ and every $m\geq1$
%	there exists a field of size $p^m$. 
%\end{theorem}
%
%\begin{proof}
%\end{proof}
%
%
% Algebraic field extensions form a nice class of extensions. The same happens
% with finite field extensions. 

% \begin{proposition}
% 	Let $F/K$ is a subextension of $E/K$. Then $E/K$ is algebraic (resp. finite)
% 	if and only if $E/F$ and $F/K$ are algebraic (resp. finite). 
% \end{proposition}

% \begin{proof}
% 	From the formula 
% 	$[E:K]=[E:F][F:K]$ it follows that 
% 	$E/K$ is finite if and only if $E/F$ and $F/K$ are
% 	both finite. 

% 	If $E/K$ is algebraic, then $E/F$ and $F/K$ are both algebraic. Conversely,
% 	suppose now that both $E/F$ and $F/K$ are algebraic. For $x\in E$ let $L$
% 	be the extension of $K$ generated by the coefficients of $f(x,F)$, the
% 	minimal polynomial of $x$ over $F$. Then $L$ is finite, as it is generated
% 	by finitely many algebraic elements. Moreover, $x$ is algebraic over $L$.
% 	Since $[L(x):K]=[L(x):L][L:K]<\infty$, $L(x)/K$ is algebraic. In
% 	particular, $x$ is algebraic over $K$. 
% \end{proof}

% \begin{proposition}
% 	Let $E/K$ and $F/K$ be extensions, where both $E$ and $F$ are subfields of
% 	a field $L$. If $F/K$ is algebraic (resp. finite), then $EF/E$ is algebraic
% 	(resp. finite).
% \end{proposition}

% \begin{proof}
% 	Now we prove that if $F/K$ is finite, then $EF/E$ is finite. For that purpose,
% 	we show that $[EF:E]<[F:K]$. Recall that $EF=E(F)$. The elements of $F$ are
% 	algebraic over $K$, so they are algebraic over $E$. In particular, $E(F)/E$ is algebraic
% 	and $E(F)=E[F]$. Let $z\in EF$, say $z=\sum_i x_it_i$ for some $x_i\in E$ and $t_i\in F$. 
% 	The extension $F/K$ is finite, so let $\{f_1,\dots,f_m\}$ be a basis of $F$ over $K$. Then
% 	each $t_i$ can be written as $t_i=\sum_ja_{ij}f_j$ for some $a_{ij}\in K$. Then
% 	\[
% 		z=\sum_j\left(\sum_i a_{ij}x_i\right)f_j
% 	\]
% 	and thus $\{f_1,\dots,f_m\}$ generates $EF$ as a vector space over $E$. 
% \end{proof}

% \[\begin{tikzcd}
% 	& EF \\
% 	E && F \\
% 	& K
% 	\arrow[no head, from=1-2, to=2-1]
% 	\arrow[no head, from=2-1, to=3-2]
% 	\arrow[no head, from=3-2, to=2-3]
% 	\arrow[no head, from=1-2, to=2-3]
% \end{tikzcd}\]

\begin{lemma}
	Let $\sigma\colon K\to L$ be a field homomorphism. Then there exists an extension
	$E/K$ and a field isomorphism $\varphi\colon E\to L$
	such that $\varphi|_K=\sigma$. 
\end{lemma}

\begin{proof}
	Let $A$ be a set in bijection with $L\setminus\sigma(K)$ and disjoint with $K$. 
	Let $E=K\cup A$. If $\theta\colon A\to L\setminus\sigma(K)$ is bijective, then 
	let 
	\[
		\varphi\colon E\to L,
		\quad
		\varphi(x)=\begin{cases}
			\sigma(x) & \text{if $x\in K$},\\
			\theta(x) & \text{if $x\in A$}.
		\end{cases}
	\]
	Then $\varphi$ is a bijective map such that $\varphi|_K=\sigma$. 
	Transport the operations of $L$ onto $E$, that is 
	to define binary operations on $E$ as follows: 
	\begin{align*}
		&(x,y)\mapsto x\oplus y=\varphi^{-1}(\varphi(x)+\varphi(y)), && 
		(x,y)\mapsto x\odot y=\varphi^{-1}(\varphi(x)\varphi(y)).
	\end{align*}
	Then, for example, 
	\[
		x\oplus y=\varphi^{-1}(\varphi(x)+\varphi(y))=\varphi^{-1}(\sigma(x)+\sigma(y))
		=\varphi^{-1}(\sigma(x+y))=\varphi^{-1}(\varphi(x+y))=x+y
	\]
	for all $x,y\in K$. 
\end{proof}

If $\sigma\colon A\to B$ is a ring homomorphism, then $\sigma$ induces a ring
homomorphism $\overline{\sigma}\colon A[X]\to B[X]$,
$\sum_ia_iX^i\mapsto\sum\sigma(a_i)X^i$. 

\begin{theorem}
	Let $K$ be a field and $f\in K[X]$ be such that $\deg f>0$. Then 
	there exists an extension $E/K$ such that $f$ admits a root in $E$. 
\end{theorem}

\begin{proof}
	We may assume that $f$ is irreducible over $K$. Let $L=K[X]/(f)$ and 
	$\pi\colon K[X]\to L$ be the canonical map. Then $L$ 
	is a field. The field homomorphism $\sigma\colon K\to L$, $a\mapsto \pi(aX^0)$. 
	Let $g=\overline{\sigma}(f)\in L[X]$. 

	We claim that $\pi(X)$ is a root of $g$ in $L$. Suppose that $f=\sum_i a_iX^i$. 
	Then 
	\begin{align*}
		g(\pi(X))&=\overline{\sigma}(f)(\pi(X))\\
		&=\sum_i \sigma(a_i)\pi(X)^i
		=\sum_i\pi(a_iX^0)\pi(X^i)=\pi(\sum a_iX^i)=\pi(f)=0.
	\end{align*}
	The previous lemma states that 
	there exists an extension $E/K$ and an isomorphism $\varphi\colon E\to L$
	such that $\varphi|_K=\sigma$. If $u=\pi(X)$, then $\varphi^{-1}(u)$ is a root of $f$ in $E$, 
	as 
	\begin{align*}
		\varphi(f(\varphi^{-1}(u)))&=\varphi\left(\sum_ia_i\varphi^{-1}(u)^i\right)
		=\varphi\left(\sum_ia_i\varphi^{-1}(u^i)\right)\\
		&=\sum_i\varphi(a_i)u^i=\sum_i\sigma(a_i)u^i=g(u)=0.\qedhere
	\end{align*}
\end{proof}

As a corollary, if $K$ is a field and $f_1,\dots,f_n\in K[X]$ are polynomials 
of positive degree, then there exists an extension $E/K$  such that 
each $f_i$ admits a root in $E$. This is proved by induction on $n$.  

\begin{definition}
	A field $K$ is \textbf{algebraically closed} if each $f\in K[X]$ 
	of positive degree admits a root in $K$. 
\end{definition}

The \emph{fundamental theorem of algebra} states that $\C$ is algebraically closed. A
typical proof uses complex analysis.  Later we will give a proof of this result
using Galois theory. 

\begin{proposition}
	The following statements are equivalent:
	\begin{enumerate}
		\item $K$ is algebraically closed.
		\item If $f\in K[X]$ is irreducible, then $\deg f=1$.
		\item If $f\in K[X]$ is non-zero, then $f$ decomposes linearly in $K[X]$, that is
			\[
				f=a\prod_{i=1}^n(X-\alpha_i)^{m_i}
			\]
			for some $a\in K$ and $\alpha_1,\dots,\alpha_n\in K$. 
		\item If $E/K$ is algebraic, then $E=K$. 
	\end{enumerate}
\end{proposition}

\begin{proof}
	$1)\implies 2\implies 3)$ are exercises.  
	
	Let us prove that $3)\implies
	4)$. Let $x\in E$. Decompose $f(x,K)$ linearly in $K[X]$ as
	$f(x,K)=a\prod_{i=1}^n(X-\alpha_i)$ and evaluate on $x$ to obtain that
	$x=\alpha_j$ for some $j$. 
	
	To prove that $4)\implies 1)$ let $f\in K[X]$ be
	such that $\deg f>0$. There exists an extension $E/K$ such that $f$ has a
	root $x$ in $E$. The extension $K(x)/K$ is algebraic and hence $K(x)=K$, so
	$x\in K$. 
\end{proof}



\topic{Artin's theorem}

\begin{definition}
	The \textbf{algebraic closure} of a field $K$ is an algebraic extension $C/K$ 
	such that $C$ is algebraically closed. 
\end{definition}

For example, $\C/\R$ is an algebraic closure but $\C/\Q$ it is not. 

\begin{proposition}
	Let $C$ be algebraically closed and $\sigma\colon K\to C$ be a field homomorphism. If $E/K$ 
	is algebraic, then there exists a field homomorphism 
	$\varphi\colon E\to C$ such that 
	$\varphi|_K=\sigma$. 
\end{proposition}

\begin{proof}
	Suppose first that $E=K(x)$ and let $f=f(x,K)$. Let $\overline{\sigma}(f)\in C[X]$ 
	and let $y\in C$ be a root of $\overline{\sigma}(f)$. If $z\in E$, then $z=g(x)$ for
	some $g\in K[X]$. Let $\varphi\colon E\to C$, $z\mapsto \overline{\sigma}(g)(y)$. 

	The map $\varphi$ is well-defined. If $z=h(x)$ for some $h\in K[X]$, then
	\[
	0=g(x)-h(x)=(g-h)(x)
	\]
	and thus $f$ divides $g-h$. In particular, $\overline{\sigma}(f)$ divides
    $\overline{\sigma}(g-h)=\overline{\sigma}(g)-\overline{\sigma}(h)$ and hence
    $(\overline{\sigma}(g)-\overline{\sigma}(h))(y)=0$. 

	It is an exercise to show that the map $\varphi$ is a ring homomorphism.
	
	Let $a\in K$. Since $a=(aX^0)(x)$, it follows that $\varphi|_K=\sigma$, as 
	\[
	\varphi(a)=\overline{\sigma}(aX^0)(y)=(\sigma(a)X^0)(y)=\sigma(a)
	\]
	and 
	$\varphi(x)=\overline{\sigma}(X)(y)=y$. 
	
	Let us now prove the proposition in full generality. Let 
	$X$ be the set of pairs $(F,\tau)$, where $F$ is a subfield of $E$ that contains $K$ and
	$\tau\colon F\to C$ is a field homomorphism such that $\tau|_K=\sigma$. Note that
	$(K,\sigma)\in X$, so $X$ is non-empty. Moreover, $X$ is partially ordered by
	\[
	(F,\tau)\leq (F_1,\tau_1)\Longleftrightarrow F\subseteq F_1\text{ and }\tau_1|_F=\tau.
	\]
	If $\{(F_i,\tau_i):i\in I\}$ is a chain in $X$, then $F=\cup_{i\in I}F_i$ is a subfield of $E$
	that contains $K$. Moreover, if $z\in F$, then $z\in F_i$ for some $i\in I$ and 
	then one defines $\tau(z)=\tau_i(z)$. It is an exercise to prove that $\tau$ is well-defined.
	Since $(F,\tau)\in X$ is an upper bound, Zorn's lemma implies that there exists
	a maximal element 
	$(E_1,\theta)\in X$. We claim that $E=E_1$. If not, let $z\in E\setminus E_1$. 
	Since we know the proposition is true for the extension $E_1(z)/K$, 
	let  
	$\rho\colon E_1(z)\to C$ be a field homomorphism such that $\rho|_{E_1}=\sigma$. Then, in particular, 
	$\rho|_K=\sigma$. This implies that $(E_1(z),\rho)\in X$ and hence
	$(E_1,\theta)<(E_1(z),\rho)$, a contradiction to the maximality of $(E_1,\theta)$. 
\end{proof}

The previous proposition will be used to prove 
that the algebraic closure always exists. 

\begin{theorem}[Artin]
	\index{Artin's theorem}
	Let $K$ be a field. Then $K$ admits an algebraic closure $C/K$. If $C_1/K$
	is an algebraic closure, then the extensions $C/K$ and $C_1/K$ are
	isomorphic. 
\end{theorem}

\begin{proof}
    Let us first prove the uniqueness. The previous proposition implies the existence of 
    an extensions homomorphism $\varphi\colon C\to C_1$. Let $y\in C_1$ and $f=f(y,K)$ be 
    the minimal polynomial of $y$ in $K$. Since $f$ admits a factorization
    \[
        f=\lambda\prod (X-\alpha_i)^{m_i}
    \]
    in $C[X]$, it follows that
    \[
    f=\overline{\varphi}(f)=\prod (X-\varphi(\alpha_i))^{m_i}
    \]
    Since $0=f(y)$, we conclude that $y=\varphi(\alpha_j)$ for some $j$. In particular, $\varphi$ is
    surjective and hence $\varphi$ is bijective. 
    
    We now prove the existence. Let us assume that $K$ admits an extension $E/K$ 
    with $E$ algebraically closed. Let $F=...$. Then $F/K$ is algebraic. Let $g\in F[X]$ be such that
    $\deg g>0$. Since $E$ is algebraically closed, $g$ admits a root $\alpha$ in $E$. In particular, $\alpha$
    is algebraic over $F$ and hence $\alpha$ is algebraic over $K$. This implies that $\alpha\in F$, thus
    $F$ is algebraically closed. This proves that $F/K$ is an algebraic closure. 
    
    Let us prove that there exists an extension $E_1/K$ such that
    every polynomial $f\in K[X]$ with $\deg f>0$ has a root in $E_1$. Let 
    $\{f_i:i\in I\}$ be the family of monic irreducible polynomials with coefficients in $K$. 
    We may think that $f_i=f_i(X_i)$. 
    Let $R=K[\{X_i:i\in I\}]$ and let $J$ be the ideal of $R$ 
    generated by the $f_i(X_i)$. We claim that $J\ne R$. If not, $1\in J$, so
    \[
    1=\sum_{i=1}^m g_jf_{i_j}(X_j)
    \]
    for some $g_1,\dots,g_m\in R$. There exists an extension $F/K$ such that
    $f_{i_j}$ has a root $\alpha_j$ in $F$ for all $j$. Let 
    \[
    \sigma\colon R\to F,\quad
    \sigma(X_k)=\begin{cases}
        \alpha_j & \text{if $k=i_j$},\\
        0 & \text{if $k\not\in\{i_1,\dots,i_m\}$}.
        \end{cases}
    \]
    Then $1=\sigma(1)=\sum_{j=1}^m\sigma(g_j)f_{i_j}(\alpha_j)$, a contradiction. 
    
    Since $J$ is a proper ideal, it is contained in a maximal ideal $M$. Let $L=R/M$ 
    and let $\sigma\colon K\to L$ be given by...
    Then $\pi(X_i)$ is a root of $\overline{\sigma}(f_i)$ for all $i$ 
    and there exists an extension $E_1/K$ such that
    every $f_i$ has a root in $E_1$. Proceeding in this way, we construct
    a sequence
    \[
    E_1\subseteq E_2\subseteq\cdots
    \]
    of fields such that every polynomial of positive degree and coefficients in $E_k$ 
    admits a root in $E_{k+1}$. Let $E=\cup E_k$. We claim that $E$ is algebraically closed. In fact, 
    let $g\in E[X]$ be such that $\deg g>0$. Then, since $g\in E_r[X]$ for some $r$, it follows
    that $g$ has a root in $E_{r+1}\subseteq E$. 
\end{proof}

\topic{Decomposition fields}

\begin{definition}
	Let $K$ be a field and $f\in K[X]$ be such that $\deg f>0$. A \textbf{decomposition field}
	of $f$ over $K$ is field $E$ that contains $K$ and that satisfies the following properties:
	\begin{enumerate}
		\item $f$ factorizes linearly in $E[X]$. 
		\item if $F$ is a field such that $K\subseteq F\subseteq E$ and 
			$f$ factorizes
			linearly in $F[X]$, then $F=E$. 
	\end{enumerate}
\end{definition}

Easy examples: 

\begin{example}
	$\C$ is a decomposition field of $X^2+1\in\R[X]$. 
\end{example}

\begin{example}
	$\Q[\sqrt{2}]$ is a decomposition field of $X^2-2\in\Q[X]$. 
\end{example}

\begin{example}
	$\Q(\sqrt[3]{2})$ is not a decomposition field of $X^3-2\in\Q[X]$. However, if
	$\omega$ ia a primitive cubic root of one, then 
	$\Q(\sqrt[3]{2},\omega)$ is is a decomposition field of $X^3-2\in\Q[X]$. 
\end{example}

\begin{proposition}
	$E$ is a decomposition field of $f\in K[X]$ if and only if
	$f$ factorizes linearly in $E[X]$ and $E=K(x_1,\dots,x_n)$ where 
	$x_1,\dots,x_n$ are roots of $f$. 
\end{proposition}

\begin{proof}
\end{proof}
