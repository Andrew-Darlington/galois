\section{Lecture -- Week 11}

\subsection{Solvable groups}

\index{Commutator subgroup}
Let $G$ be a group. If $x,y\in G$ we define
the \emph{commutator} of $x$ and $y$ as
\[
[x,y]=xyx^{-1}y^{-1}.
\]
Note that $[x,y]=1$ if and only if $xy=yx$. 
Moreover, $[x,y]^{-1}=[y,x]$. 
The
\emph{commutator (or derived) subgroup} $[G,G]$ of $G$ is defined as
the subgroup of $G$ generated by all commutators, i.e. 
\[
[G,G]=\langle [x,y]:x,y\in G\rangle.
\]
This means that every element of $[G,G]$ is a finite product of commutators, 
so every element of $[G,G]$ is of the form $\prod_{i=1}^m [x_i,y_i]$. 
In general, the commutator subgroup is not equal to the set of commutators!

\begin{example}
This example is taken from the book \cite{MR0075938} of Carmichael. 
Let $G$ be the subgroup of $\Sym_{16}$ generated by the permutations
\begin{align*}
&a = (1 3)(2 4),&&
b = (5 7)(6 8),\\
&c = (9 11)(10\,12),&&
d = (13\,15)(14\,16),\\
&e = (1 3)(5 7)(9\,11),&&
f = (1 2)(3 4)(13\,15),\\
&g = (5 6)(7 8)(13\,14)(15\,16),&&
h =(9\,10)(11\,12).
\end{align*}
Then $[G,G]$ has order $16$. However, 
the set $\{[x,y]:x,y\in G\}$ of commutators has 15 elements:
\begin{lstlisting}
> S16 := Sym(16);
> a := S16 ! (1,3)(2,4);
> b := S16 ! (5,7)(6,8);
> c := S16 ! (9,11)(10,12);
> d := S16 ! (13,15)(14,16);
> e := S16 ! (1,3)(5,7)(9,11);
> f := S16 ! (1,2)(3,4)(13,15);
> g := S16 ! (5,6)(7,8)(13,14)(15,16);
> h := S16 ! (9,10)(11,12);
> G := PermutationGroup< 16 | a,b,c,d,e,f,g,h >;
> D := DerivedSubgroup(G);
> #D;
16
> #{ x*y*Inverse(x)*Inverse(y) : x in G, y in G };
15
> c*d in { x : x in D } \
> diff { u*v*Inverse(u)*Inverse(v) : u in G, v in G };
true
\end{lstlisting}
% \begin{lstlisting}
% julia> a = @perm (1,3)(2,4);

% julia> b = @perm (5,7)(6,8);

% julia> c = @perm (9,11)(10,12);

% julia> d = @perm (13,15)(14,16);

% julia> e = @perm (1,3)(5,7)(9,11);

% julia> f = @perm (1,2)(3,4)(13,15);

% julia> g = @perm (5,6)(7,8)(13,14)(15,16);

% julia> h = @perm (9,10)(11,12);
   
% julia> S16 = symmetric_group(16);

% julia> G = sub(S16, [a,b,c,d,e,f,g,h])[1];

% julia> commutators = G -> Set(comm(x,y) for x in G, y in G);

% julia> length(commutators(G))
% 15

% julia> order(derived_subgroup(G)[1])
% 16
% \end{lstlisting}
\end{example}

%julia> c*d in derived_subgroup(G)[1]
%true
%julia> c*d in commutators(G)
%false

\begin{exercise}
    Let $G$ be a group. 
    Prove the following facts:
    \begin{enumerate}
        \item $G$ is abelian if and only if $[G,G]=\{1\}$.
        \item $[G,G]$ is a normal subgroup of $G$. 
        \item $G/[G,G]$ is abelian. 
        \item If $H$ is a subgroup of $G$ and $[G,G]\subseteq H$, then 
        $H$ is normal in $G$.
        \item If $H$ is a normal subgroup of $G$, then 
        $G/H$ is abelian if and only if $[G,G]\subseteq H$. 
    \end{enumerate}
\end{exercise}

\begin{definition}
\index{Derived series}
    Let $G$ be a group. The \emph{derived series} of $G$ 
    is defined as $G^{(0)}=G$ and
    $G^{(k+1)}=[G^{(k)},G^{(k)}]$ for $k\geq 0$. 
\end{definition}

\begin{exercise}
    Prove that $G^{(k)}$ is normal in $G$ for all $k$. 
\end{exercise}

Why derived series? We cannot explain this here, but let us use the following
notation. 
We write $G'=[G,G]$, $G''=[G',G']$... 
Note that 
\[
G\supseteq G'\supseteq G''\supseteq\cdots
\]

%\begin{exercise}
%    $[\Sym_3,\Sym_3]=\Alt_3$. 
%\end{exercise}

\begin{exercise}
    Let $n\geq3$. Prove that 
    $[\Sym_n,\Sym_n]=\Alt_n$. 
\end{exercise}

\begin{example}
    Let $K=\{\id,(12)(34),(13)(24),(14)(23)\}$. Then 
    $K$ is a normal subgroup of $\Alt_4$. 
    One proves that $[\Alt_4,\Alt_4]=K$. 
\end{example}

\index{Group!simple}
A group $G$ is said to be \emph{simple} 
if there are no proper non-trivial normal subgroups of $G$. If $p$
is a prime number, then the group 
$\Z/p$ of integers modulo $p$ 
is a simple group. We will prove
later that $\Alt_n$ is simple if 
$n\geq5$. 

\begin{example}
    Let $n\geq5$. Since $\Alt_n$ is a non-abelian simple group, 
    $[\Alt_n,\Alt_n]=\Alt_n$. 
\end{example}

Let us show that $\Alt_5$ is a non-abelian simple group. Hence
it is not solvable: 
\begin{lstlisting}
> G := Alt(5);
> IsAbelian(G);
false
> IsSimple(G);
true
> IsSolvable(G);
false
\end{lstlisting}


\begin{definition}
\index{Group!solvable}
    A group $G$ is \emph{solvable} if and only if
    $G^{(m)}=\{1\}$ for some $m$. 
\end{definition}

Every abelian group is solvable. 

\begin{exercise}
    Prove that $\Sym_n$ is solvable if and only if $n\leq4$. 
\end{exercise}

Let us compute (with the computer software Oscar) 
the derived series of the symmetric group
$\Sym_4$. The calculation shows that $\Sym_4$ is solvable: 
\begin{lstlisting}
> G := Sym(4);
> DerivedSeries(G);
[
    Symmetric group G acting on a set of cardinality 4
    Order = 24 = 2^3 * 3
        (1, 2, 3, 4)
        (1, 2),
    Permutation group acting on a set of cardinality 4
    Order = 12 = 2^2 * 3
        (1, 2, 3)
        (2, 3, 4),
    Permutation group acting on a set of cardinality 4
    Order = 4 = 2^2
        (1, 4)(2, 3)
        (1, 3)(2, 4),
    Permutation group acting on a set of cardinality 4
    Order = 1
]
> IsSolvable(G);
true
\end{lstlisting}
% S4 := SymmetricGroup(4);
% DerivedSeries(S4);
% for x in DerivedSeries(S4) do
%   GroupName(x);
% end for;
% IsSolvable(S4);


\begin{proposition}
\label{pro:solvable}
    Let $G$ be a group and $H$ be a subgroup of $G$. 
    The following statements hold:
    \begin{enumerate}
        \item If $G$ is solvable, then $H$ is solvable.
        \item If $H$ is normal in $G$ and $G$ is solvable, then $G/H$ is solvable. 
        \item If $H$ is normal in $G$ and $H$ and $G/H$ are solvable, then $G$ is solvable. 
    \end{enumerate}
\end{proposition}

\begin{proof}
    The first statement follows from the fact that
    $H^{(i)}\subseteq G^{(i)}$ holds for all $i$. 
 
    Assume now that $H$ is normal in $G$. 
    Let $Q=G/H$ and $\pi\colon G\to Q$ be the canonical map.
    By induction one proves that $\pi(G^{(i)})=Q^{(i)}$ for all $i\geq0$.
    The case where $i=0$ is trivial, as $\pi$ is surjective. 
    If the result holds for some $i\geq0$, then
    \[
            \pi(G^{(i+1)})=\pi([G^{(i)},G^{(i)}])=[\pi(G^{(i)}),\pi(G^{(i)})]=[Q^{(i)},Q^{(i)}]=Q^{(i+1)}.
    \]

    We now prove 2). Since $G$ is solvable, $G^{(n)}=\{1\}$ for some $n$. 
    Thus $Q$ is solvable, as $Q^{n}=\pi(G^{(n)})=\pi(\{1\})=\{1\}$. 

    We finally prove 3). 
    Since $Q$ is solvable, $Q^{(n)}=\{1\}$ for some $n$.
    Moreover, since $\pi(G^{(n)})=Q^{(n)}=\{1\}$, it follows that
    $G^{(n)}\subseteq H$. Since $H$
    is solvable, 
    \[
        G^{(n+m)}\subseteq (G^{(n)})^{(m)}\subseteq H^{(m)}=\{1\}
    \]
    for some $m$. Thus $G$ is solvable.
\end{proof}

An application:

\begin{proposition}
    Let $G$ be a finite $p$-group. Then $G$ is solvable. 
\end{proposition}

\begin{proof}
    Assume the result is not true. Let $G$ be a finite
    $p$-group of minimal order that is not solvable. Since 
    $G$ is a $p$-group, $Z(G)\ne\{1\}$. Since $|G|$ is minimal, 
    $G/Z(G)$ is a solvable $p$-group. Since $Z(G)$ is abelian, 
    $Z(G)$ is solvable. Now $G$ is solvable 
    by Proposition \ref{pro:solvable}.
\end{proof}

\index{Subgroup!maximal normal}
Let $G$ be a group. A subgroup $N$ of $G$ is said to be \emph{maximal normal} 
if $N$ is a normal subgroup of $G$ and there is no other normal subgroup of $G$ 
containing $N$. 

\begin{exercise}
    If a subgroup $N$ of $G$ is maximal (for the inclusion) and
    normal, then it is maximal normal. Show that the converse does not hold.
\end{exercise}

% $G=\Alt_5$. Since $G$ is simple, the trivial subgroup is the only maximal normal subgroup. However, 
% the trivial subgroup is not a maximal subgroup. 

The following result is a direct consequence of the correspondence theorem:

\begin{exercise}
\label{xca:G/N_simple}
    Let $G$ be a group and $N$ be a normal subgroup of $G$. Prove that 
    $N$ is maximal normal if and only if $G/N$ is simple. 
\end{exercise}

Maximal normal subgroups always exist in finite groups (they could be trivial). 
We can compute maximal normal subgroups as follows:
\begin{lstlisting}
> MaximalNormalSubgroup(Sym(3));
Permutation group acting on a set of cardinality 3
Order = 3
    (1, 2, 3)
julia> maximal_normal_subgroups(quaternion_group(8))
3-element Vector{PcGroup}:
 Group([ y2, x ])
 Group([ y2, y ])
 Group([ y2, x*y ])

> MaximalNormalSubgroup(Alt(4));
Permutation group acting on a set of cardinality 4
Order = 4 = 2^2
    (1, 2)(3, 4)
    (1, 3)(2, 4)
\end{lstlisting}

% This construct Q8
% > S8 := Sym(8);
% > a := S8 ! (1,5,3,7)(2,8,4,6);
% > b := S8 ! (1,2,3,4)(5,6,7,8);
% > G := PermutationGroup< 8 | a,b >;
% Better
% G := Group("Q8");
% It follows from the correspondence theorem. 

\begin{exercise}
\label{xca:solvable+simple}
    Let $G$ be a 
    finite solvable group. Prove that if $G$ is simple, then 
    $G$ is cyclic of prime order. 
\end{exercise}

The following result will be important later:

\begin{proposition}
\label{pro:prime_index}
    Every finite solvable group contains a normal subgroup of prime index. 
\end{proposition}

\begin{proof}
    Let $G$ be a finite solvable group. 
    Let $M$ be a maximal normal subgroup of $G$ (there is at least one, as $G$ is finite). 
    Since $G/M$ is simple and solvable (see Proposition \ref{pro:solvable}), 
    $G/M$ is cyclic of prime order by Exercise \ref{xca:solvable+simple}.  
\end{proof}

We finish this discussion with two important theorems (without proof) 
about finite solvable groups. 

\begin{theorem}[Burnside]
\index{Burnside's theorem}
    Let $p$ and $q$ be prime numbers. If $G$ is a group
    of order $p^aq^b$, then $G$ is solvable. 
\end{theorem}

The proof appears in courses on the 
representation theory of finite groups. 

\begin{theorem}[Feit--Thompson]
\index{Feit--Thompson theorem}
    Every finite group of odd order is solvable.
\end{theorem}

The proof of the theorem is extremely hard. It occupies a full volume of
\emph{Pacific Journal of Mathematics}, see~\cite{MR166261}.



\subsection{Simplicity of the alternating simple group}

We will present a family of non-abelian simple groups. 
We start with some exercises. 

\begin{exercise}
\label{xca:diagonal}
    Let $G$ be a group. 
    Prove that $G$ is simple if and only if 
    $\{(g,g):g\in G\}$ 
    is a maximal subgroup of $G\times G$. 
\end{exercise}

\begin{exercise}
    Prove that $\Alt_n$ is generated by 3-cycles. 
\end{exercise}

\begin{exercise}
    Compute the commutator subgroup of $\Alt_n$ for
    $n\geq2$. 
\end{exercise}

Note that $\Alt_2$ and $\Alt_3$ are abelian. 
For $\Alt_4$, one proves that 
\[
[\Alt_4,\Alt_4]=\{\id,(12)(34),(13)(24),(14)(23)\}.
\]
Finally, $[\Alt_n,\Alt_n]=\Alt_n$ for $n\geq5$. 

Let us compute some commutator subgroups (and the inclusion group homomorphism)  
with the computer:
\begin{lstlisting}
> S3 := Sym(3);
> DerivedSubgroup(S3);
Permutation group acting on a set of cardinality 3
Order = 3
    (1, 2, 3)   
\end{lstlisting}

\begin{exercise}
    Let $n\geq3$. 
    Prove that $[\Sym_n,\Sym_n]=\Alt_n$. 
\end{exercise}

Recall that every normal subgroup is a union of conjugacy classes. The group 
$\Alt_5$ has conjugacy classes of sizes 1, 15, 20, 12 and 12. It follows that 
the only possible normal subgroups of $\Alt_5$ are $\{\id\}$ and $\Alt_5$. 
\begin{lstlisting}
> A5 := Alt(5);
> ConjugacyClasses(A5);
Conjugacy Classes of group A5
-----------------------------
[1]     Order 1       Length 1
        Rep Id(A5)

[2]     Order 2       Length 15
        Rep (1, 2)(3, 4)

[3]     Order 3       Length 20
        Rep (1, 2, 3)

[4]     Order 5       Length 12
        Rep (1, 2, 3, 4, 5)

[5]     Order 5       Length 12
        Rep (1, 3, 4, 5, 2)
\end{lstlisting}

\begin{theorem}[Jordan]
\index{Jordan's theorem}
\label{thm:Jordan}
    Let $n\geq5$. Then $\Alt_n$ is simple. 
\end{theorem}

Before proving the theorem, we need some preliminary results.

Every permutation $\rho\in\Sym_n$ decomposes as a product of disjoint cycles, say
\[
\rho=(a_1\cdots a_r)(b_1\cdots b_s)\cdots (c_1\cdots c_t)
\]
where, by convention, we do not write cycles of length one. 
The cyclic structure of $\rho$ is, by definition, the ordered 
sequence of integers $r,s,\dots t$, where, again by convention,  
we omit fixed points. For example, the cyclic structure of 
the transposition $(ab)$ is 2, 
of $(abc)(d)$ is 3 and of $(123)(45)(789a)(bcd)(d)$ is 2,3,3,4. 

\begin{lemma}
If both permutations $\rho_1\in\Sym_n$ and $\rho_2\in\Sym_n$ have the same
cyclic structure, then 
$\rho_2=\sigma\rho_1\sigma^{-1}$ for some
$\sigma\in\Sym_n$. 
\end{lemma}

\begin{proof}
Assume that 
\begin{align*}
\rho_1&=(a_1\cdots a_r)(b_1\cdots b_s)\cdots (c_1\cdots c_t),\\
\rho_2&=(x_1\cdots x_r)(y_1\cdots y_s)\cdots (z_1\cdots z_t).
\end{align*}
Let  
\begin{align*}
\Fix(\rho_1) &= \{x\in\{1,\dots,n\}:\rho_1(x)=x\}=\{k_1,\dots,k_m\},&&
\Fix(\rho_2)=\{l_1,\dots,l_m\}	
\end{align*}
be the fixed points of the permutations $\rho_1$ and $\rho_2$,
respectively. Then 
\[
\sigma(x)=\begin{cases}
x_j & \text{if $x=a_j$ for some $j$},\\
y_j & \text{if $x=b_j$ for some $j$},\\
\;\vdots\\
z_j & \text{if $x=c_j$ for some $j$},\\
l_j & \text{if $x=k_j$ for some $j$},	
\end{cases}
\]
is such that $\sigma\rho_1\sigma^{-1}=\rho_2$. 
\end{proof}

What happens with the alternating group? 

\begin{lemma}
If $\rho_1,\rho_2\in\Sym_n$ are conjugate in $\Sym_n$ and  $|\Fix(\rho_1)|\geq2$, then 
$\mu\rho_1\mu^{-1}=\rho_2$ for some $\mu\in\Alt_n$.  
\end{lemma}

\begin{proof}
Assume that $\rho_2=\sigma\rho_1\sigma^{-1}$ for some $\sigma\in\Sym_n$. 
There are $a,b\in\{1,\dots,n\}$ such that
$\rho_1(a)=a$, $\rho_1(b)=b$ and $a\ne b$. Let 
\[
\mu=\begin{cases}
\sigma & \text{if $\sigma\in\Alt_n$,}\\
\sigma(ab) & \text{otherwise.}
\end{cases}
\]
Then $\mu\in\Alt_n$ and $\mu\rho_1\mu^{-1}=\rho_2$, as 
$(ab)$ commutes with $\rho_1$. 
\end{proof}

Let us discuss some examples.

\begin{example}
If $\rho_1=(23)(156)$ and $\rho_2=(45)(123)$, then 
$\rho_2=\sigma\rho_1\sigma^{-1}$ for  
\[
\sigma=\binom{123456}{145623}.
\]
\end{example}

\begin{example}
The permutations $\rho_1=(123)\in\Sym_3$ and $\rho_2=(132)\in\Sym_3$ are conjugate 
in $\Sym_3$, as  
$(123)=\sigma(132)\sigma^{-1}$ if $\sigma=(23)$. However, $\rho_1$ and $\rho_2$ are not conjugate in $\Alt_3$. 
\end{example}

Now we are ready to prove the theorem. 

\begin{proof}[Proof of Theorem~\ref{thm:Jordan}]
Let $N\ne\{\id\}$ be a normal subgroup of $\Alt_n$. If $(abc)\in N$, then every 3-cycle belongs to $N$, because all 3-cycles 
are conjugate in $\Sym_n$, and the previous lemma 
states that 
$(ijk)=\mu(abc)\mu^{-1}\in N$ for some
$\mu\in\Alt_n$. Thus $N=\Alt_n$. 

We claim that $N$ contains a 3-cycle. 
Since $N\ne\{\id\}$, there exists $\sigma\in N\setminus\{\id\}$. Let $m=|\sigma|$ and let $p$ be a prime number dividing $m$. 
Then $\tau=\sigma^{m/p}$ has order $p$ and hence
$\tau=\rho_1\cdots\rho_s$, where the $\rho_j$'s are disjoint 
$p$-cycles. 

If $p=2$, then $1=\sgn(\tau)=(-1)^s$. Thus $s$ is even. Write
\[
\tau=(ab)(cd)\rho_3\cdots\rho_s. 
\]
Since $\rho_3\cdots\rho_s$ commutes with $(abc)$ and $(acb)$, 
\[
\underbrace{(abc)\tau(abc)^{-1}\tau^{-1}}_{\in N}
=(abc)(ab)(cd)(acb)(ab)(cd)=(ac)(bd).
%(ac)(bd)\tau=(abc)\tau(abc)^{-1}\in N. 
\]
Hence $(ac)(bd)\in N$. Let 
$e\in\{1,\dots,n\}\setminus\{a,b,c,d\}$. Then 
\[
(ae)(bd)=(aec)\underbrace{(ac)(bd)}_{\in N}(aec)^{-1}\in N
\]
and therefore  
\[
(aec)=(ac)(ae)=(ac)(bd)(ae)(bd)\in N.
\]

If $p=3$, without loss of generality, we may assume that $s\geq2$ (otherwise, $\tau$ would be a 3-cycle). Then
$\tau=(abc)(def)\rho_3\cdots\rho_s$. Since $(bcd)$ commutes with
$\rho_3\cdots\rho_s$ and $N$ is normal in $\Alt_n$, 
\begin{align*}
\underbrace{(bcd)\tau(bcd)^{-1}\tau^{-1}}_{\in N}=(bcd)(abc)(def)(bdc)(acb)(dfe)=(adbce)
%&(adbce)=(bcd)\tau(bcd)^{-1}\tau^{-1}\in N
%\shortintertext{and therefore}
\end{align*}
and therefore 
\begin{align*}
&(adc)=(adb)(adbce)(adb)^{-1}(adbce)^{-1}\in N.
\end{align*}

If $p>3$, then $\tau=(abcd\cdots z)\rho_2\cdots\rho_s$. In particular, $(abc)$ commutes with $\rho_2\cdots\rho_s$. Then 
\[
(abd)=(abc)\tau(abc)^{-1}\tau^{-1}\in N.\qedhere
\]
\end{proof}

As an application, we compute the normal subgroups 
of the symmetric group $\Sym_n$. 

\begin{exercise}
    Compute the list of normal subgroups of $\Sym_n$ for $n\geq2$. 
\end{exercise}

%The first cases
%are easy: $\Sym_2$ is simple, 
%the normal subgroups of $\Sym_3$ 
%are $\{\id\}$, $\Alt_3$ and $\Sym_3$, and
%the normal subgroups of $\Sym_4$ are
%\[
%\{\id\},\{\id,(12)(34),(13)(24),(14)(23)\},
%\Alt_4, \Sym_4.
%\]
%
%\begin{exercise}
%Let $n\geq5$ and $N$ be a normal subgroup of $\Sym_n$. Then either 
%$N=\{\id\}$, $N=\Alt_n$ or $N=\Sym_n$.     
%\end{exercise}
%
%\begin{proposition}
%\index{Subgrupos normales!de $\Sym_n$}
%Sea $n\geq5$ y sea $N$ un subgrupo normal de $\Sym_n$. Entonces $N=\{\id\}$, $N=\Alt_n$ o bien $N=\Sym_n$. 
%\end{proposition}
%
%\begin{proof}
%Sea $N$ un subgrupo normal de $\Sym_n$. Como la restricción $\sgn|_N$ de la función signo a $N$ es un morfismo de grupos, 
%\[
%(N:\ker(\sgn|_N))=|\sgn(N)|\in\{1,2\}.
%\] 
%Si $(N:\ker(\sgn|_N)=1$, entonces $N\subseteq\Alt_n$ y entonces $N$ es normal en $\Alt_n$. 
%Luego $N=\{\id\}$ o bien $N=\Alt_n$ pues $\Alt_n$ es un grupo simple si $n\geq5$. 
%
%Si $(N:\ker(\sgn|_N)=2$, entonces $N\cap\Alt_n=\ker(\sgn|_N)$ es normal en $\Alt_n$ y luego, por la simplicidad de $\Alt_n$ para $n\geq5$, $N\cap\Alt_n=\{\id\}$ o bien $N\cap\Alt_n=\Alt_n$. 
%
%En el primer caso, $|N|=2$ y entonces $N$ contiene una permutación impar que además tiene orden dos, digamos
%$\tau=(ij)\tau_2\cdots\tau_s$, escrita como producto de trasposiciones disjuntas. Entonces
%$\pi=(ik)\tau(ik)\in N$ si $k\not\in\{i,j\}$ y $\pi\ne \tau$ pues $\tau(j)=i$ y $\pi(j)=k$. Luego $|N|\geq3$, una contradicción. 
%
%En el segundo caso, si $N\cap\Alt_n=\Alt_n$, entonces $N=\Alt_n$.
%\
%
%\begin{proof}
%    
%\end{proof}


\subsection{Radical extensions}

\begin{definition}
    \index{Extension!pure}
    \index{Pure extension}
    An extension $E/K$ is said to be \emph{pure} of type $m$ if 
    $E=K(x)$ for some $x$ such that $x^m\in K$. 
\end{definition}

Note that if $E=K(x)$ is a pure extension of type $m$ and $K$ contains 
$m$-th roots of one, then $E/K$ is a splitting field of $X^m-x^m$. 

\begin{definition}
\index{Radical extension}
\index{Radical tower}
\index{Extension!radical}
    The sequence $K=R_0\subseteq R_1\subseteq\cdots\subseteq R_m$ 
    of fields is said to be a \emph{radical tower} if 
    each $R_{i+1}/R_i$ is pure. In this case, $R_m/K$ is a \emph{radical extension}. 
\end{definition}

Note that radical extensions are finite. 

\begin{example}
    Let $E$ be a splitting field of $X^4-2$ over $\Q$. Then $E/\Q$ is radical, 
    as $E=\Q(\sqrt[4]{2},i)$. 
\end{example}

\begin{example}
    Let $\alpha,\beta\in\C$ be such that $\alpha^2=2$ and 
    $\beta^5=1+\alpha$. 
    The number $\sqrt[5]{1+\sqrt{2}}$ belongs to the radical extension $\Q(\alpha,\beta)/\Q$. 
\end{example}

\begin{theorem}
\label{thm:by_radicals}
    Let $K$ be of characteristic zero and 
    $R/K$ be a radical extension. If $E/K$ is a subextension of $R/K$, 
    then $\Gal(E/K)$ is solvable. 
\end{theorem}

\begin{proof}
    Without loss of generality, 
    we may assume that $E/K$ is a Galois extension. To prove this fact, 
    let $G=\Gal(E/K)$ and $F=\prescript{G}{}{E}$. Then
    $E/F$ is a Galois extension and $\Gal(E/F)=G$ by Artin's theorem. 
    Thus, replacing $K$ by $F$ if needed, we may assume that
    $E/K$ is Galois. 
    
    Let $L$ be the normal closure of $R$ in some
    algebraic closure $C$ that contains $R$. Note that 
    if $R=K(x_1,\dots,x_m)$, then 
    \[
    L=K(\{\sigma_i(x_j):1\leq i\leq s,\,1\leq j\leq m\}),
    \]
    where $\Hom(R/K,C/K)=\{\sigma_1,\dots,\sigma_s\}$. 
    
    \begin{claim}
        $L/K$ is radical. 
    \end{claim}
    
    Since $x_j^{a_j}\in K(x_1,\dots,x_{j-1})$ for some integer $a_j$, 
    \[
    \sigma_i(x_j)^{a_j}=\sigma_i\left(x_j^{a_j}\right)\in\sigma_i(K(x_1,\dots,x_{j-1})=K(\sigma_i(x_1),\dots,\sigma_i(x_{j-1}))
    \]
    Thus $L/K$ is radical and Galois. 
    
    We may assume then that $E/K$ and $R/K$ are both Galois. 
    
    Since $\Gal(E/K)\simeq\Gal(R/K)/\Gal(R/E)$, we only need
    to prove that $\Gal(R/K)$ is solvable. 
    
    For a positive integer $n$, 
    let $\xi$ be a primitive $n$-th root of one (in some algebraic closure
    of $K$ that contains $R$). Consider the diagram
    \[
    \begin{tikzcd}
	& {R(\xi)} \\
	R && {K(\xi)} \\
	& K
	\arrow[no head, from=1-2, to=2-3]
	\arrow[no head, from=1-2, to=2-1]
	\arrow[no head, from=2-1, to=3-2]
	\arrow[no head, from=3-2, to=2-3]
    \end{tikzcd}
    \]
    Then
    \begin{enumerate}
        \item $K(\xi)/K$ and $R(\xi)/R$ are abelian.
        \item $R(\xi)/K$ is Galois.
        \item $\Gal(R/K)\simeq\Gal(R(\xi)/K)/\Gal(R(\xi)/R)$. 
        \item $\Gal(K(\xi)/K)\simeq\Gal(R(\xi)/K)/\Gal(R(\xi)/K(\xi))$. 
    \end{enumerate}
    The third item implies that we need to 
    show that $\Gal(R(\xi)/K)$ is solvable. By the fourth item,
    it suffices to show that $\Gal(R(\xi)/K(\xi))$ is solvable (because $\Gal(K(\xi)/K)$ is abelian and hence solvable). 
    
    Since
    $R=K(x_1,\dots,x_m)$,  
    \[
    R(\xi)=K(x_1,\dots,x_m,\xi)=K(\xi)(x_1,\dots,x_m)
    \]
    and hence $R(\xi)/K(\xi)$ is radical. 
    This means that
    without loss of generality, we may assume that
    $K$ contains primitive $n$-roots of one. For example, 
    if $R=K(x_1,\dots,x_m)$ and $x_i^{a_i}\in K(x_1,\dots,x_{i-1})$, 
    then we may assume that $K$ contains a primitive $a_i$-root of one. We proceed by induction on $m$. 
    The case $m=0$ is trivial. Assume that the claim holds for some $m\geq0$. Let 
    $L=K(x_1)$. Then $L/K$ is a splitting field of $X^{a_1}-x_1^{a_1}$, and hence
    $L/K$ is a cyclic extension. Thus $\Gal(L/K)$ is cyclic (and hence, in particular, solvable). 
    Let $H$ be the subgroup that corresponds to $L$, that is
    $H=\Gal(R/L)$ (here, we use Galois' correspondence). Then $H$ is normal in $\Gal(R/K)$. 
    Since $R=K(x_1,\dots,x_m)=L(x_2,\dots,x_m)$, $R/L$ is radical and Galois. By the inductive hypothesis, 
    $\Gal(R/L)$ is solvable. Since 
    \[
    \Gal(L/K)\simeq \Gal(R/K)/\Gal(R/L),
    \]
    it follows that $\Gal(R/K)$ is solvable. 
\end{proof}

\begin{definition}
    Let $f\in K[X]$ and $E$ be a splitting field of $f$ over $K$. 
    We say that $f$ is \emph{solvable by radicals} if
    there is a radical extension $R/K$ such that $E\subseteq R$. 
\end{definition}

The general polynomial of degree two 
is solvable by radicals, as its Galois group 
is solvable (in fact, isomorphic to $\Sym_2$).  

\begin{exercise}
    Prove that $f=X^2-s_1X+s_2\in\Q[X]$ is solvable by radicals. 
%    In this case, if 
%    $F=K(s_1,s_2)$, then $E=F(t_1,t_2)=F(t_1)$, as 
%    $t_1+t_2=s_1$. Moreover,  
%    $F(t_1)=F(u)$, as $u^2=s_1^2-4s_2$. 
\end{exercise}

Theorem \ref{thm:by_radicals} translates into the following result:

\begin{exercise}
    Let $K$ be a field of characteristic zero. 
    If $f\in K[X]$ is solvable by radicals, then $\Gal(f,K)$ is solvable. 
\end{exercise}

As a consequence, the general polynomial of degree $n\geq5$ 
is not solvable by radicals, as its Galois group is isomorphic to 
$\Sym_5$. 

\begin{example}
    Let $p$ be a prime number and $f=X^5-2pX+p\in\Q[X]$. 
    We claim that 
    $f$ is not solvable by radicals. 
    
    By Gauss' theorem, one proves that $f$ has no rational roots. 

    Note that $f'=5X^4-2p$. Then $\alpha=\sqrt[4]{2p/5}$ and $\beta=-\sqrt[4]{2p/5}$ are
    are critical points. Since $f(\alpha)<0$ and $f(\beta)>0$, it follows that $f$ has
    exactly three real roots. Let 
    $x_1,x_2\in\C\setminus\R$ and $x_3,x_4,x_5\in\R$ be the roots
    of $f$. 
    
    By Eisenstein's theorem, $f$ is irreducible. 
    
    Let $E/\Q$ be a splitting field of $f$. 
    Then $\Gal(f,\Q)=\Gal(E/\Q)$ is isomorphic 
    to a subgroup $G$ of $\Sym_5$.
    Since 
    $f$ is irreducible, $5$ divides $[E:\Q]=|G|$. In particular, 
    by Cauchy's theorem, $G$ contains an element $\sigma$ of order five. This element
    is a 5-cycle, so without loss of generality, we may assume that 
    $\sigma=(x_1x_2x_3x_4x_5)$. Note that 
    $(x_1x_2)\in G$. Thus $G\simeq\Sym_5$ and hence
    $G$ is not solvable. 
\end{example}

\begin{exercise}
    Let $f=X^6+2X^5-5X^4+9X^3-5X^2+2X+1\in\Q[X]$. 
    Prove that $f$ is solvable by radicals. 
\end{exercise}

%Let us solve the previous exercise
%with a quick computer calculation:
%\begin{lstisting}
%\end{lstisting}

%\subsection{Constructions (optional)}

It is now time to prove Galois' great theorem on solvability 
of polynomials. 

\begin{theorem}[Galois]
\index{Galois' theorem}
\label{thm:Galois_great}
    Let $K$ be a field of characteristic zero and $f\in K[X]$. 
    Then $f$ is solvable by radicals if and only if $\Gal(f,K)$ is solvable. 
\end{theorem}

We proved in Theorem~\ref{thm:by_radicals} that solvable 
polynomials have solvable Galois groups. 
For the converse, we need two auxiliary results. 

\begin{lemma}
\label{lem:E=K(beta)}
    Let $E/K$ be a Galois extension of prime degree $p$. Assume that $K$ admits a primitive $p$-root of one. 
    Then $E=K(\beta)$ where $\beta^p\in K$. 
\end{lemma}

\begin{proof}
    Assume that $\Gal(E/K)=\langle\sigma\rangle$. 
    Let $\omega\in K$ be a primitive $p$-root of one. Then $\norm_{E/K}(\omega)=\omega^p=1$. By Hilbert's theorem, 
    $\omega=\beta/\sigma(\beta)$ for some $\beta\in E$. Note that $\beta\not\in K$, as $\omega\ne1$. Moreover, 
    \[
    \sigma(\beta^p)=(\beta\omega^{-1})^p=\beta^p\in\prescript{\Gal(E/K)}{}{E}=K.
    \]
    Since $K\subseteq K(\beta)\subseteq E$ and $[E:K]=p$, we conclude that $E=K(\beta)$ with $\beta^p\in K$. 
\end{proof}

\begin{exercise}
\label{xca:embedding}
    Let $E/K$ be a splitting field of $f\in K[X]$ and $K^*/K$ be an extension.
    If $E^*/K^*$ is a splitting field of $f$ containing $E$, then 
    \[
    \Gal(E^*/K^*)\to\Gal(E/K),\quad \sigma\mapsto\sigma|_{E},
    \]
    is an injective group homomorphism. 
\end{exercise}

Now Theorem \ref{thm:Galois_great} will follow from the following theorem. 

\begin{theorem}
    Let $K$ be a field of characteristic zero and $E/K$ be a Galois extension.
    If $\Gal(E/K)$ is solvable, then $E$ can be embedded in a radical extension. 
\end{theorem}

\begin{proof}
    Let $G=\Gal(E/K)$. Since $G$ is solvable, by Proposition \ref{pro:prime_index}, 
    there exists a normal subgroup $H$ of $G$ of prime index $p$. 
    Let $\omega$ be a primitive $p$-th root of one. (It exists because $K$ is a field of characteristic zero.)
    
    We proceed by induction on $[E:K]$. 
    
    If $[E:K]=1$, there is nothing to prove. 
    So assume that $[E:K]>1$. 

    We first assume that $\omega\in K$. 
    The group $\Gal(E/\prescript{H}{}{E})$ is solvable, as it is a subgroup of $G$. Moreover, since 
    \[
    [E:\prescript{H}{}{E}]<[E:K],
    \]
    the inductive hypothesis implies that $E/\prescript{H}{}{E}$ can be embedded in a radical extension, so 
    there exists a radical tower
    \begin{equation}
    \label{eq:radical_tower1}
    \prescript{H}{}{E}\subseteq R_1\subseteq R_2\subseteq\cdots\subseteq R_m,
    \end{equation}
    where $E\subseteq R_m$. Now $E/\prescript{H}{}{E}$ is a Galois extension, as $H$ is normal in $G$. Moreover, 
    \[ 
    [\prescript{H}{}{E}:K]=(G:H)=p.
    \]
    Since $\omega\in K$, Lemma~\ref{lem:E=K(beta)} implies that 
    $\prescript{H}{}{E}=K(\beta)$ for some $\beta$ such that $\beta^p\in K$. The radical tower \ref{eq:radical_tower1} can be 
    extended by adding $K\subseteq \prescript{H}{}{E}$. 

    For the general case, let $K^*=K(\omega)$ and $E^*=E(\omega)$. Then $E^*/K^*$ is a Galois extension with Galois group 
    $\Gal(E^*/K^*)$. By Exercise~\ref{xca:embedding}, $\Gal(E^*/K^*)$ is solvable. By the previous part, 
    $E^*$ and $E$ can be embedded in a radical extension $R^*/K^*$, so there exists a radical tower
    \begin{equation}
    \label{eq:radical_tower2}
    K^*\subseteq R_1^*\subseteq R_2^*\subseteq\cdots\subseteq R_n^*.
    \end{equation}
    Since $K^*=K(\omega)$ is a pure extension, the radical tower \eqref{eq:radical_tower2} 
    can be extended by adding $K\subseteq K^*$. 
\end{proof}