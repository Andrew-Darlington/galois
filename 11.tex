\section{06/05/2024}

\subsection{Radical extensions}

\begin{definition}
    \index{Extension!pure}
    \index{Pure extension}
    An extension $E/K$ is said to be \textbf{pure} of type $m$ if 
    $E=K(x)$ for some $x$ such that $x^m\in K$. 
\end{definition}

Note that if $E=K(x)$ is a pure extension of type $m$ and $K$ contains 
$m$-th roots of one, then $E/K$ is a splitting field of $X^m-x^m$. 

\begin{definition}
\index{Radical extension}
\index{Radical tower}
\index{Extension!radical}
    The sequence $K=R_0\subseteq R_1\subseteq\cdots\subseteq R_m$ 
    of fields is said to be a \textbf{radical tower} if 
    each $R_{i+1}/R_i$ is pure. In this case, $R_m/K$ is a \textbf{radical extension}. 
\end{definition}

Note that radical extensions are finite. 

\begin{example}
    Let $E$ be a decomposition field of $X^4-2$ over $\Q$. Then $E/\Q$ is radical, 
    as $E=\Q(\sqrt[4]{2},i)$. 
\end{example}

\begin{example}
    Let $\alpha,\beta\in\C$ be such that $\alpha^2=2$ and 
    $\beta^5=1+\alpha$. 
    The number $\sqrt[5]{1+\sqrt{2}}$ belongs to the radical extension $\Q(\alpha,\beta)/\Q$. 
\end{example}

\begin{theorem}
\label{thm:by_radicals}
    Let $K$ be of characteristic zero and 
    $R/K$ be a radical extension. If $E/K$ is a subextension of $R/K$, 
    then $\Gal(E/K)$ is solvable. 
\end{theorem}

\begin{proof}
    Without loss of generality, 
    we may assume that $E/K$ is a Galois extension. To prove this fact, 
    let $G=\Gal(E/K)$ and $F=\prescript{G}{}{E}$. Then
    $E/F$ is a Galois extension and $\Gal(E/F)=G$ by Artin's theorem. 
    Thus, replacing $K$ by $F$ if needed, we may assume that
    $E/K$ is Galois. 
    
    Let $L$ be the normal closure of $R$ in some
    algebraic closure $C$ that contains $R$. Note that 
    if $R=K(x_1,\dots,x_m)$, then 
    \[
    L=K(\{\sigma_i(x_j):1\leq i\leq s,\,1\leq j\leq m\}),
    \]
    where $\Hom(R/K,C/K)=\{\sigma_1,\dots,\sigma_s\}$. 
    
    \begin{claim}
        $L/K$ is radical. 
    \end{claim}
    
    Since $x_j^{a_j}\in K(x_1,\dots,x_{j-1})$ for some integer $a_j$, 
    \[
    \sigma_i(x_j)^{a_j}=\sigma_i\left(x_j^{a_j}\right)\in\sigma_i(K(x_1,\dots,x_{j-1})=K(\sigma_i(x_1),\dots,\sigma_i(x_{j-1}))
    \]
    Thus $L/K$ is radical and Galois. 
    
    We may assume then that $E/K$ and $R/K$ are both Galois. 
    
    Since $\Gal(E/K)\simeq\Gal(R/K)/\Gal(R/E)$, we only need
    to prove that $\Gal(R/K)$ is solvable. 
    
    For a positive integer $n$, 
    let $\xi$ be a primitive $n$-th root of one (in some algebraic closure
    of $K$ that contains $R$). Consider the diagram
    \[
    \begin{tikzcd}
	& {R(\xi)} \\
	R && {K(\xi)} \\
	& K
	\arrow[no head, from=1-2, to=2-3]
	\arrow[no head, from=1-2, to=2-1]
	\arrow[no head, from=2-1, to=3-2]
	\arrow[no head, from=3-2, to=2-3]
    \end{tikzcd}
    \]
    Then
    \begin{enumerate}
        \item $K(\xi)/K$ and $R(\xi)/R$ are abelian.
        \item $R(\xi)/K$ is Galois.
        \item $\Gal(R/K)\simeq\Gal(R(\xi)/K)/\Gal(R(\xi)/R)$. 
        \item $\Gal(K(\xi)/K)\simeq\Gal(R(\xi)/K)/\Gal(R(\xi)/K(\xi))$. 
    \end{enumerate}
    The third item implies that we need to 
    show that $\Gal(R(\xi)/K)$ is solvable. By the fourth item,
    it suffices to show that $\Gal(R(\xi)/K(\xi))$ is solvable (because $\Gal(K(\xi)/K)$ is abelian and hence solvable). 
    
    Since
    $R=K(x_1,\dots,x_m)$,  
    \[
    R(\xi)=K(x_1,\dots,x_m,\xi)=K(\xi)(x_1,\dots,x_m)
    \]
    and hence $R(\xi)/K(\xi)$ is radical. 
    This means that
    without loss of generality, we may assume that
    $K$ contains primitive $n$-roots of one. For example, 
    if $R=K(x_1,\dots,x_m)$ and $x_i^{a_i}\in K(x_1,\dots,x_{i-1})$, 
    then we may assume that $K$ contains a primitive $a_i$-root of one. We proceed by induction on $m$. 
    The case $m=0$ is trivial. Assume that the claim holds for some $m\geq0$. Let 
    $L=K(x_1)$. Then $L/K$ is a decomposition field of $X^{a_1}-x_1^{a_1}$, and hence
    $L/K$ is a cyclic extension. Thus $\Gal(L/K)$ is cyclic (and hence, in particular, solvable). 
    Let $H$ be the subgroup that corresponds to $L$, that is
    $H=\Gal(R/L)$ (here, we use Galois' correspondence). Then $H$ is normal in $\Gal(R/K)$. 
    Since $R=K(x_1,\dots,x_m)=L(x_2,\dots,x_m)$, $R/L$ is radical and Galois. By the inductive hypothesis, 
    $\Gal(R/L)$ is solvable. Since 
    \[
    \Gal(L/K)\simeq \Gal(R/K)/\Gal(R/L),
    \]
    it follows that $\Gal(R/K)$ is solvable. 
\end{proof}

\begin{definition}
    Let $f\in K[X]$ and $E$ be a decomposition field of $f$ over $K$. 
    We say that $f$ is \textbf{solvable by radicals} if
    there is a radical extension $R/K$ such that $E\subseteq R$. 
\end{definition}

The general polynomial of degree two 
is solvable by radicals, as its Galois group 
is solvable (in fact, isomorphic to $\Sym_2$).  

\begin{exercise}
    Prove that $f=X^2-s_1X+s_2\in\Q[X]$ is solvable by radicals. 
%    In this case, if 
%    $F=K(s_1,s_2)$, then $E=F(t_1,t_2)=F(t_1)$, as 
%    $t_1+t_2=s_1$. Moreover,  
%    $F(t_1)=F(u)$, as $u^2=s_1^2-4s_2$. 
\end{exercise}

Theorem \ref{thm:by_radicals} translates into the following result:

\begin{exercise}
    Let $K$ be a field of characteristic zero. 
    If $f\in K[X]$ is solvable by radicals, then $\Gal(f,K)$ is solvable. 
\end{exercise}

As a consequence, the general polynomial of degree $n\geq5$ 
is not solvable by radicals, as its Galois group is isomorphic to 
$\Sym_5$. 

\begin{example}
    Let $p$ be a prime number and $f=X^5-2pX+p\in\Q[X]$. 
    We claim that 
    $f$ is not solvable by radicals. 
    
    By Gauss' theorem, one proves that $f$ has no rational roots. 

    Note that $f'=5X^4-2p$. Then $\alpha=\sqrt[4]{2p/5}$ and $\beta=-\sqrt[4]{2p/5}$ are
    are critical points. Since $f(\alpha)<0$ and $f(\beta)>0$, it follows that $f$ has
    exactly three real roots. Let 
    $x_1,x_2\in\C\setminus\R$ and $x_3,x_4,x_5\in\R$ be the roots
    of $f$. 
    
    By Eisenstein's theorem, $f$ is irreducible. 
    
    Let $E/\Q$ be a decomposition field of $f$. 
    Then $\Gal(f,\Q)=\Gal(E/\Q)$ is isomorphic 
    to a subgroup $G$ of $\Sym_5$.
    Since 
    $f$ is irreducible, $5$ divides $[E:\Q]=|G|$. In particular, 
    by Cauchy's theorem, $G$ contains an element $\sigma$ of order five. This element
    is a 5-cycle, so without loss of generality, we may assume that 
    $\sigma=(x_1x_2x_3x_4x_5)$. Note that 
    $(x_1x_2)\in G$. Thus $G\simeq\Sym_5$ and hence
    $G$ is not solvable. 
\end{example}

\begin{exercise}
    Let $f=X^6+2X^5-5X^4+9X^3-5X^2+2X+1\in\Q[X]$. 
    Prove that $f$ is solvable by radicals. 
\end{exercise}

%Let us solve the previous exercise
%with a quick computer calculation:
%\begin{lstisting}
%\end{lstisting}

%\subsection{Constructions (optional)}

It is now time to prove Galois' great theorem on solvability 
of polynomials. 

\begin{theorem}[Galois]
\index{Galois' theorem}
\label{thm:Galois_great}
    Let $K$ be a field of characteristic zero and $f\in K[X]$. 
    Then $f$ is solvable by radicals if and only if $\Gal(f,K)$ is solvable. 
\end{theorem}

We proved in Theorem~\ref{thm:by_radicals} that solvable 
polynomials have solvable Galois groups. 
For the converse, we need two auxiliary results. 

\begin{lemma}
\label{lem:E=K(beta)}
    Let $E/K$ be a Galois extension of prime degree $p$. Assume that $K$ admits a primitive $p$-root of one. 
    Then $E=K(\beta)$ where $\beta^p\in K$. 
\end{lemma}

\begin{proof}
    Assume that $\Gal(E/K)=\langle\sigma\rangle$. 
    Let $\omega\in K$ be a primitive $p$-root of one. Then $\norm_{E/K}(\omega)=\omega^p=1$. By Hilbert's theorem, 
    $\omega=\beta/\sigma(\beta)$ for some $\beta\in E$. Note that $\beta\not\in K$, as $\omega\ne1$. Moreover, 
    \[
    \sigma(\beta^p)=(\beta\omega^{-1})^p=\beta^p\in\prescript{\Gal(E/K)}{}{E}=K.
    \]
    Since $K\subseteq K(\beta)\subseteq E$ and $[E:K]=p$, we conclude that $E=K(\beta)$ with $\beta^p\in K$. 
\end{proof}

\begin{exercise}
\label{xca:embedding}
    Let $E/K$ be a decomposition field of $f\in K[X]$ and $K^*/K$ be an extension.
    If $E^*/K^*$ is a decomposition field of $f$ containing $E$, then 
    \[
    \Gal(E^*/K^*)\to\Gal(E/K),\quad \sigma\mapsto\sigma|_{E},
    \]
    is an injective group homomorphism. 
\end{exercise}

Now Theorem \ref{thm:Galois_great} will follow from the following theorem. 

\begin{theorem}
    Let $K$ be a field of characteristic zero and $E/K$ be a Galois extension.
    If $\Gal(E/K)$ is solvable, then $E$ can be embedded in a radical extension. 
\end{theorem}

\begin{proof}
    Let $G=\Gal(E/K)$. Since $G$ is solvable, by Proposition \ref{pro:prime_index}, 
    there exists a normal subgroup $H$ of $G$ of prime index $p$. 
    Let $\omega$ be a primitive $p$-th root of one. (It exists because $K$ is a field of characteristic zero.)
    
    We proceed by induction on $[E:K]$. 
    
    If $[E:K]=1$, there is nothing to prove. 
    So assume that $[E:K]>1$. 

    We first assume that $\omega\in K$. 
    The group $\Gal(E/\prescript{H}{}{E})$ is solvable, as it is a subgroup of $G$. Moreover, since 
    \[
    [E:\prescript{H}{}{E}]<[E:K],
    \]
    the inductive hypothesis implies that $\prescript{H}{}{E}/E$ can be embedded in a radical extension, so 
    there exists a radical tower
    is 
    \begin{equation}
    \label{eq:radical_tower1}
    \prescript{H}{}{E}\subseteq R_1\subseteq R_2\subseteq\cdots\subseteq R_m,
    \end{equation}
    where $E\subseteq R_m$. Now $\prescript{H}{}{E}/K$ is a Galois extension, as $H$ is normal in $G$. Moreover, 
    \[ 
    [\prescript{H}{}{E}:K]=(G:H)=p.
    \]
    Since $\omega\in K$, Lemma~\ref{lem:E=K(beta)} implies that 
    $\prescript{H}{}{E}=K(\beta)$ for some $\beta$ such that $\beta^p\in K$. The radical tower \ref{eq:radical_tower1} can be 
    extended by adding $K\subseteq \prescript{H}{}{E}$. 

    For the general case, let $K^*=K(\omega)$ and $E^*=E(\omega)$. Then $E^*/K^*$ is a Galois extension with Galois group 
    $\Gal(E^*/K^*)$. By Exercise~\ref{xca:embedding}, $\Gal(E^*/K^*)$ is solvable. By the previous part, 
    $E^*$ and $E$ can be embedded in a radical extension $R^*/K^*$, so there exists a radical tower
    \begin{equation}
    \label{eq:radical_tower2}
    K^*\subseteq R_1^*\subseteq R_2^*\subseteq\cdots\subseteq R_n^*.
    \end{equation}
    Since $K^*=K(\omega)$ is a pure extension, the radical tower \eqref{eq:radical_tower2} 
    can be extended by adding $K\subseteq K^*$. 
\end{proof}