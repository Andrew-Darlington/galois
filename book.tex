\RequirePackage{amsmath} 

\documentclass[graybox,envcountsect]{svmono}

\usepackage[T1]{fontenc}
\usepackage[utf8]{inputenc}
\usepackage{amsmath}
\usepackage[notref,notcite]{showkeys}
\usepackage{float}
\usepackage{amssymb}
\usepackage{amstext}
\usepackage{mathtools}
\usepackage{xcolor} 
\usepackage{centernot}
\usepackage{listings}
\usepackage{multicol}
\usepackage{mathptmx}
%\let\openbox\relax
\usepackage{newtxtext,newtxmath}
%\usepackage{txfonts}
\usepackage{datetime}
\usepackage{stmaryrd}
\usepackage{tikz-cd}

\usepackage{helvet}
\usepackage{courier}
\usepackage{type1cm}         
\usepackage{makeidx}        
\usepackage{graphicx}        
\usepackage{multicol}        
\usepackage[all]{xy}
\usepackage{hyperref} 
%\usepackage{tikz-cd}
\usepackage{colortbl}
\usepackage{chngcntr}





% Table of contents for lectures and topics
\makeatletter
\newcommand\listtopicsname{List of topics}
\newcommand\listoftopics{
    \chapter*{\listtopicsname}\@starttoc{top}}
\makeatother

\makeatletter
\newcommand\listlecturesname{Contents}
\newcommand\listoflectures{
    \chapter*{\listlecturesname}\@starttoc{lec}}
\makeatother

\newcommand{\lecture}[1]{
    \chapter{#1}
    \addcontentsline{lec}{chapter}{Lecture \thechapter}
}

\newcommand{\topic}[1]{
    \section{#1}
    \addcontentsline{top}{chapter}{\S\thesection\quad #1}
}


\usepackage[small,bf]{caption}

\usepackage{tikz}
\usetikzlibrary{braids}
	
\usepackage[bottom]{footmisc}

% for QED
\let\proof\relax\let\endproof\relax
\let\openbox\relax
\usepackage{amsthm}

\overfullrule=1mm

%%% for Spanish
% \def\abstractname{Resumen}%
% \def\ackname{Agradecimientos}%
% \def\andname{y}%
% \def\bibname{Referencias}%
% \def\lastandname{, y}%
% \def\appendixname{Apéndice}%
\def\chaptername{Lecture}%
% \def\claimname{Afirmación}%
% \def\conjecturename{Conjetura}%
% \def\contentsname{Contenidos}%
% \def\corollaryname{Corolario}%
% \def\definitionname{Definici\'on}%
% \def\emailname{e-mail}%
% \def\examplename{Ejemplo}%
\def\examplesname{Examples}%
% \def\exercisename{Ejercicio}%
\def\figurename{Figure}%
% \def\forewordname{Foreword}%
% \def\keywordname{{\bf Palabras clave:}}%
% \def\indexname{Índice}%
% \def\lemmaname{Lema}%
% \def\listfigurename{Figuras}%
% \def\listtablename{Tablas}%
% \def\notename{Nota}%
% \def\partname{Parte}%
% \def\prefacename{Prefacio}%
\def\problemname{Open problem}%
% \def\proofname{Demostración}%
% \def\propertyname{Propiedad}%
% \def\propositionname{Proposici\'on}%
% \def\questionname{Pregunta}%
% \def\refname{Referencias}%
% \def\remarkname{Observación}%
% \def\seename{see}%
% \def\solutionname{Solución}%
% \def\tablename{Tabla}%
% \def\theoremname{Teorema}
\def\notationname{Notation}
\def\stepsname{Algorithm}
% \def\conventionname{Convención}

% change numbers 
\let\remark\relax
\let\theorem\relax
\let\lemma\relax
\let\definition\relax
\let\proposition\relax
\let\corollary\relax
\let\exercise\relax
\let\example\relax
\let\conjecture\relax

% Numerar con sección y no resetear al cambiar de capítulo
\counterwithout{section}{chapter}
\counterwithout{theorem}{chapter}
\spnewtheorem{theorem}{\theoremname}[section]{\bfseries}{\itshape}

\renewcommand\thetheorem{\thesection.\arabic{theorem}}
\spnewtheorem{lemma}[theorem]{\lemmaname}{\bfseries}{\itshape}
\spnewtheorem{definition}[theorem]{\definitionname}{\bfseries}{\upshape}
\spnewtheorem{proposition}[theorem]{\propositionname}{\bfseries}{\itshape}
\spnewtheorem{corollary}[theorem]{\corollaryname}{\bfseries}{\itshape}
\spnewtheorem{exercise}[theorem]{\exercisename}{\bfseries}{\upshape}
\spnewtheorem{example}[theorem]{\examplename}{\bfseries}{\upshape}
\spnewtheorem{examples}[theorem]{\examplesname}{\bfseries}{\upshape}
\spnewtheorem{remark}[theorem]{\remarkname}{}{\upshape}
\spnewtheorem{conjecture}[theorem]{\conjecturename}{\bfseries}{\upshape}
\spnewtheorem{notation}[theorem]{\notationname}{\bfseries}{\upshape}
\spnewtheorem{steps}[theorem]{\stepsname}{\bfseries}{\upshape}
\spnewtheorem{convention}[theorem]{\conventionname}{\bfseries}{\upshape}

% Numerar con sección y no resetear al cambiar de capítulo
\counterwithout{section}{chapter}

% No sections in TOC
\setcounter{secnumdepth}{1}
\setcounter{tocdepth}{0}

 \usepackage{titlesec}
 \titleformat{\section}
   {\secsize\secstyle}{\S\thesection.}{1em}{}

% para enumerar
\renewcommand{\labelenumi}{\textbf{\arabic{enumi})}}

\makeindex             

\renewcommand{\I}{\operatorname{I}}
\newcommand{\II}{\operatorname{II}}

\newcommand{\GAP}{\textsf{GAP}}
\newcommand{\FK}{\mathcal{E}}
\newcommand{\ad}[1]{\operatorname{ad}\,#1}

%\newcommand{\N}{\mathbb{N}}
\newcommand{\Q}{\mathbb{Q}}
\newcommand{\Z}{\mathbb{Z}}
\newcommand{\F}{\mathbb{F}}
\newcommand{\R}{\mathbb{R}}
\newcommand{\C}{\mathbb{C}}
\renewcommand{\H}{\mathbb{H}}
\newcommand{\A}{\mathbb{A}}
\newcommand{\K}{\mathbb{K}}
\newcommand{\T}{\mathbb{T}}
\renewcommand{\D}{\mathbb{D}}
\newcommand{\B}{\mathbb{B}}
\newcommand{\Fun}{\operatorname{Fun}}
\newcommand{\mpl}{\operatorname{mpl}}
\newcommand{\cL}{\mathcal{L}}
\newcommand{\cE}{\mathcal{E}}
\newcommand{\cH}{\mathcal{H}}

\newcommand{\GF}{\mathsf{GF}}
\newcommand{\MAX}{\operatorname{MAX}}
\newcommand{\MIN}{\operatorname{MIN}}
\newcommand{\cf}{\operatorname{cf}}
\newcommand{\cl}{\operatorname{cl}}
\newcommand{\cd}{\operatorname{cd}}
\newcommand{\bL}{\mathbf{L}}
\newcommand{\bP}{\mathbf{P}}

\newcommand{\Nil}{\operatorname{Nil}}
\newcommand{\rad}{\operatorname{rad}}
\newcommand{\rank}{\operatorname{rank}}

\newcommand{\Aff}{\mathrm{Aff}}
\newcommand{\Ann}{\operatorname{Ann}}
\newcommand{\Der}{\operatorname{Der}}
\newcommand{\Core}{\operatorname{Core}}
\newcommand{\Soc}{\operatorname{Soc}}
\newcommand{\Fix}{\operatorname{Fix}}
\newcommand{\Rad}{\mathrm{rad}}
\newcommand{\Inn}{\mathrm{Inn}}
\newcommand{\dist}{\mathrm{dist}}
\newcommand{\Out}{\mathrm{Out}}
\newcommand{\Ext}{\mathrm{Ext}}
\newcommand{\Img}{\mathrm{im}}
\newcommand{\Hol}{\operatorname{Hol}}
\newcommand{\Hom}{\operatorname{Hom}}
\newcommand{\Alg}{\operatorname{Alg}}
\newcommand{\Bil}{\operatorname{Bil}}
\newcommand{\op}{\operatorname{op}}
\newcommand{\gr}{\operatorname{gr}}
\newcommand{\Syl}{\mathrm{Syl}}
\newcommand{\id}{\operatorname{id}}
\newcommand{\Aut}{\operatorname{Aut}}
\newcommand{\End}{\operatorname{End}}
\newcommand{\Irr}{\operatorname{Irr}}
\newcommand{\Alt}{\mathbb{A}}
\newcommand{\Sym}{\mathbb{S}}
\newcommand{\lcm}{\mathrm{mcm}}
\newcommand{\diag}{\operatorname{diag}}
\newcommand{\spec}{\operatorname{Spec}}
\newcommand{\supp}{\operatorname{supp}}
\newcommand{\trace}{\operatorname{trace}}
\newcommand{\sgn}{\operatorname{sign}}
\newcommand{\ch}{\operatorname{char}}

\newcommand{\inner}{\operatorname{inn}}
\newcommand{\ext}{\operatorname{ext}}
\newcommand{\im}{\operatorname{im}}
\newcommand{\Ret}{\operatorname{Ret}}

\newcommand{\GL}{\mathbf{GL}}
\newcommand{\SL}{\mathbf{SL}}
\newcommand{\PSL}{\mathbf{PSL}}
\newcommand{\PGL}{\mathbf{PGL}}

\newcommand{\legendre}[2]{\left(\frac{#1}{#2}\right)}

%\newcommand{\char}{\operatorname{char}}

% multiset
\def\multiset#1#2{\ensuremath{\left(\kern-.3em\left(\genfrac{}{}{0pt}{}{#1}{#2}\right)\kern-.3em\right)}}

% column vector
\newcount\colveccount
\newcommand*\colvec[1]{
\global\colveccount#1
\begin{pmatrix}
	\colvecnext
	}
	\def\colvecnext#1{
	#1
	\global\advance\colveccount-1
	\ifnum\colveccount>0
	\\
	\expandafter\colvecnext
	\else
\end{pmatrix}
\fi
}


% numero como secciones
\renewcommand{\thesection}{\arabic{section}}
%\renewcommand{\thesubsection}{\Alph{section}}

% To remove Springer from the title page
\usepackage{etoolbox}
\makeatletter
\patchcmd{\@maketitle}{{\Large Springer\par}}{}{}{}
\def\ps@bchap{%
  \let\@oddhead\@empty\let\@evenhead\@empty
  \def\@oddfoot{\reset@font\small\hfil\thepage\hfil}%
  \let\@evenfoot\@oddfoot
}

% Heading 
\def\ps@headings{%
  \let\@mkboth\markboth
  \def\@oddfoot{\reset@font\small\hfil\thepage\hfil}%
  \let\@evenfoot\@oddfoot
  \def\@evenhead{\runheadsize\runheadstyle\hfil\leftmark}%
  \def\@oddhead{\runheadsize\runheadstyle\rightmark\hfil}%
  \def\chaptermark##1{%
    \markboth{%
      {\if@chapnum \thechapter\thechapterend\fi ##1}%
    }{%
      {\if@chapnum \thechapter\thechapterend\fi ##1}}%
    }%
    \def\sectionmark##1{\markright{{\ifnum\c@secnumdepth>\z@
     \S\thesection\seccounterend\hskip\betweenumberspace\fi ##1}}}
}
\makeatother
\pagestyle{headings}

\begin{document}
 
\lstset{language=GAP,
  showstringspaces=false,
  xleftmargin=0.6cm,
  xrightmargin=0.6cm,
  basicstyle=\small\ttfamily,
  frame=single,
  framerule=0pt,
}

\author{Leandro Vendramin}
\title{Galois theory}
\subtitle{Notes}
\maketitle

\frontmatter

%\include{dedic}
%\preface

The notes correspond to the bachelor 
course \emph{Galois theory} of the 
Vrije Universiteit Brussel, 
Faculty of Sciences, 
Department of Mathematics and Data Sciences. The course
is divided into thirteen two-hour lectures. 

The material is somewhat standard. Basic texts on fields and Galois theory 
are for example \cite{MR1645586} and 
\cite{MR3379917}. 

As usual, we also mention a set of great expository papers by 
Keith Conrad available at 
\url{https://kconrad.math.uconn.edu/blurbs/}. 
The notes are extremely well-written and are useful at  
at every stage of a mathematical career. 

Several chapters contain optional paragraphs which give examples of 
how to apply the software package \href{https://oscar.computeralgebra.de/
}{OSCAR Computer Algebra System}
to concrete problems in Galois theory. 

 
Thanks go to Alejandro de la Cueva Merino, Wannes Malfait. 
%Arne van Antwerpen, Luca Descheemaeker, Lucas Simons
%and Geoffrey Jassens. 

This version 
was compiled on \today~at~\currenttime.

\bigskip
\begin{flushright}
Leandro Vendramin\\Brussels, Belgium\par
\end{flushright}


\tableofcontents 
\listoftopics

\mainmatter

%\chapter{}

\topic{Fields}

Recall that a \textbf{field} is a commutative
ring such that $1\ne 0$ and 
that every non-zero element is invertible. Examples
of (infinite) fields are $\Q$, $\R$ and $\C$. If $p$
is a prime number, then $\Z/p$ is a field. 

\begin{example}
	The abelian group $\Z/2\times\Z/2$ is a field
	with multiplication
	\[
		(a,b)(c,d)=(ac+bd,ad+bc+bd).
	\]
\end{example}

\begin{example}
	$\Q(i)=\{a+bi:a,b\in\Q\}$ and 
	$\Q(\sqrt{2})$ are fields.
\end{example}

\[
\begin{tikzcd}
	{\mathbb{C}} \\
	&& {\mathbb{R}} \\
	{\mathbb{Q}(i)} && {\mathbb{Q}(\sqrt{2})} \\
	& {\mathbb{Q}}
	\arrow[no head, from=3-3, to=4-2]
	\arrow[no head, from=2-3, to=3-3]
	\arrow[no head, from=1-1, to=3-1]
	\arrow[no head, from=3-1, to=4-2]
	\arrow[no head, from=1-1, to=2-3]
\end{tikzcd}
\]


\begin{exercise}
	\label{xca:Q(i)}
	Prove that $\Q(i)$ and $\Q(\sqrt{2})$ are not isomorphic as fields.
\end{exercise}

If $R$ is a ring, there exists a unique ring homomorphism
$\Z\to R$, $m\mapsto m1$. The image $\{m1:m\in\Z\}$ 
of this homomorphism is a subring 
of $R$ and it is known as the \textbf{ring of integers} of $R$. The
kernel is a subgroup of $\Z$ and is generated by
some $t\geq0$. The integer $t$ is 
the \textbf{characteristic} of the ring $R$. 

\begin{exercise}
	The characteristic of a field is either zero or
	a prime number. 
\end{exercise}

Recall that a commutative ring $R$ is an \textbf{integral 
domain} if $xy=0\implies x=0$ or $y=0$. Fields
are integral domains. 

\begin{exercise}
	Let $K$ be a field. Prove that
	the following statements are equivalent:
	\begin{enumerate}
		\item $K$ is of characteristic zero.
		\item The additive order of $1$ is infinite. 
		\item The additive order of each $x\ne0$ is infinite.
		\item The ring of integers of $K$ is isomorphic to $\Z$.
	\end{enumerate}
\end{exercise}

\begin{exercise}
	Let $K$ be a field. Prove that
	the following statements are equivalent:
	\begin{enumerate}
		\item $K$ is of characteristic $p$.
		\item The additive order of $1$ is $p$. 
		\item The additive order of each $x\ne0$ is $p$.
		\item The ring of integers of $K$ is isomorphic to $\Z/p$.
	\end{enumerate}
\end{exercise}

% The following exercise is important. 

% \begin{exercise}
% 	Prove that if $K$ is a finite field, then
% 	$|K|=p^m$ for some prime number $p$ and some $m\geq1$. 
% \end{exercise}

\begin{definition}
	\index{Subfield}
	A \textbf{subfield} of a ring $R$ is a subring of $R$ 
	that is also a field.
\end{definition}

Note that if $K$ is a subfield of $E$, then
the characteristic of $K$ coincides
with the characteristic 
of $E$. Moreover, if $K\to L$ is a field homomorphism, then
$K$ and $L$ have the same characteristic. 

\begin{exercise}
	Let $K$ be a field of characteristic $p$. Prove
	that $K\to K$, $x\mapsto x^{p^n}$, is a field homomorphism
	for all $n\in\Z_{\geq 0}$. 
\end{exercise}

Note that finite fields are of characteristic $p$. 

Let $K$ be a subfield of a field $E$. Then $E$ 
is a $K$-vector space with the usual scalar multiplication
$K\times E\to E$, 
$(\lambda, x)\mapsto \lambda x$.

\begin{definition}
	A field $K$ is \textbf{prime} if there are no
	proper subfields of $K$. 
\end{definition}

Examples of prime fields are $\Q$ and $\Z/p$ for a prime number $p$.

\begin{proposition}
	Let $K$ be a field. The following statements hold:
	\begin{enumerate}
		\item $K$ contains a unique prime field, it is known as the 
			\textbf{prime subfield} of $K$.
		\item The prime subfield of $K$ is either isomorphic to $\Q$ if 
			the characteristic of $K$ is zero, or it is isomorphic to $\Z/p$ for
			some prime number $p$ if the characteristic of $K$ is $p$. 
	\end{enumerate}
\end{proposition}

\begin{proof}
	To prove the first claim let $L$ be the intersection
	of all the subfields of $K$. Then $L$ is a subfield of $K$. 
	If $F$ is a subfield of $L$, then $F$ is a subfield
	of $K$. Thus $L\subseteq F$ and hence $F=L$, which proves
	that $L$ is prime. If $L_1$ is a subfield of $K$
	and $L_1$ is prime, then $L\subseteq L_1$ and 
	hence $L=L_1$. 

	Let $K_0$ be the prime field of $K$. Suppose that $K$ is of characteristic
	$p>0$. Then the ring $K_\Z$ of integers of $K$ 
	is a field isomorphic to $\Z/p$ and hence $K_0\simeq
	K_\Z$. Suppose now that the characteristic of $K$ is zero. Let
	$E=\{m1/n1:m,n\in\Z,n\ne 0\}$. We claim that $K_0=E$. Since $K_\Z\subseteq
	K_0$, it follows that $E\subseteq K_0$. Hence $E=K_0$, as $E$ is a subfield
	of $K$.  
\end{proof}

\begin{definition}
	Let $E$ be a field and $K$ be a subfield of $E$. Then 
	$E$ is a \textbf{field extension} of $K$. We will use
	the notation $E/K$. 
\end{definition}

If $E$ is an extension of $K$, then $E$ is a
$K$-vector space. 

\begin{definition}
	The degree of an extension $E$ of $K$ 
	is the integer $\dim_KE$. It will be denoted by $[E:K]$. 
\end{definition}

We say that $E$ is a finite extension of $K$ 
if $[E:K]$ is finite. 

\begin{example}
	Let $K$ be a field. Then $[K:K]=1$. Conversely, 
	if $E$ is an extension of $K$ and $[E:K]=1$, then $K=E$. 
	If not, let $x\in E\setminus K$. We claim that
	$\{1,x\}$ is linearly independent over $K$. Indeed, 
	if $a1+bx=0$ for some $a,b\in K$, then $bx=-a$. If 
	$b\ne 0$, then $x=-a/b\in K$, a contradiction. If $b=0$, then 
	$a=0$. 
\end{example}

We know that $[\C:\R]=2$. 

\begin{example}
	A basis of $\Q(\sqrt{2})$ over $\Q$ 
	is given by $\{1,\sqrt{2}\}$. Then 
	$[\Q(\sqrt{2}):\Q]=2$. The calculations 
	can be easily done by computer: 
\begin{lstlisting}
julia> E, a = quadratic_field(2)
(Real quadratic field defined by x^2 - 2, sqrt(2))
julia> characteristic(E)
0
julia> K = prime_field(E)
Rational Field
julia> degree(E)
2
julia> basis(E)
2-element Vector{nf_elem}:
 1
 sqrt(2)
julia> one(K)==one(E)
true
julia> zero(K)==zero(E)
true
\end{lstlisting}
\end{example}

\begin{example}
	Since $\Q$ is numerable and 
	$\R$ is not, $[\R:\Q]>\aleph_0$. If $\{x_i:i\in\Z_{>0}\}$ 
	is a numerable basis of $\R$ over $\Q$, for each
	$n$ consider the $\Q$-vector space
	$V_n$ generated by $\{x_1,\dots,x_n\}$. Then 
	\[
		\R=\bigcup_{n\geq1}V_n,
	\]
	is numerable, as each $V_n$ is numerable, a contradiction.
\end{example}

If $E$ is an extension of $K$ and $E$ is finite,
then $[E:K]$ is finite. 

\begin{proposition}
	Let $K$ be a finite field. Then $|K|=p^m$ 
	for some prime number $p$ and some $m\geq1$. 
\end{proposition}

\begin{proof}
	We know the prime subfield $K_0$ of $K$ is isomorphic to $\Z/p$. 
	In particular, $|K_0|=p$. Since $K$ is finite, 
	$[K:K_0]=m$ for some $m$. If $\{x_1,\dots,x_m\}$ is a basis
	of $K$ over $K_0$, then each element
	of $K$ can be written uniquely as
	$\sum_{i=1}^ma_ix_i$ for some $a_1,\dots,a_m\in K_0$. Then
	$K\simeq K_0^m$ and hence $|K|=|K_0^m|=p^m$. 
\end{proof}

We now perform some basic calculations 
with a finite field of eight elements: 
\begin{lstlisting}
julia> E, x = FiniteField(2, 3, "x")
(Finite field of degree 3 over F_2, x)
julia> characteristic(E)
2
julia> prime_field(E)
Galois field with characteristic 2
julia> degree(E)
3
julia> size(E)
8
julia> [z for z in E]
8-element Vector{fq_nmod}:
 0
 1
 x
 x + 1
 x^2
 x^2 + 1
 x^2 + x
 x^2 + x + 1
\end{lstlisting}

%julia> F, y = FiniteField(5, 2, "y")
%(Finite field of degree 2 over F_5, y)
%julia> degree(F)
%2
%julia> prime_field(F)
%Galois field with characteristic 5
%julia> size(F)
%25


\begin{definition}
	Let $E$ be an extension of $K$. A \textbf{subextension} $F/K$ 
	of $E/K$ is a subfield $F$ of $E$ that contains $K$, that is
	$K\subseteq F\subseteq E$. 
\end{definition}

\begin{definition}
	Let $E$ and $E_1$ be extensions over $K$. An extension
	\textbf{homomorphism} $E\to E_1$ is a 
	field homomorphism $\sigma\colon E\to E_1$ such that 
	$\sigma(x)=x$ for all $x\in K$. 
\end{definition}

To describe the homomorphism $\sigma\colon E\to E_1$ of the extensions over $K$
one typically writes the commutative diagram 
\[
	\begin{tikzcd}
	K & K \\
	E & {E_1} 
	\arrow["\sigma", from=2-1, to=2-2]
	\arrow[equal, no head, from=1-1, to=1-2]
	\arrow[hook, from=1-1, to=2-1]
	\arrow[hook, from=1-2, to=2-2]
\end{tikzcd}
\]
We write $\Hom(E/K,E_1/K)$ to denote
the set of homomorphism $E\to E_1$ of extensions of $K$. Note
that if $\sigma\in\Hom(E/K,E_1/K)$, then
$\sigma$ is a $K$-linear map, as
\[
	\sigma(\lambda x)=\sigma(\lambda)\sigma(x)=\lambda\sigma(x)
\]
for all $\lambda\in K$ and $x\in E$. 

\begin{example}
	The conjugation map $\C\to\C$, $z\mapsto\overline{z}$, 
	is an endomorphism of $\C$ as an extension over $\R$. Let 
	$\varphi\in\Hom(\C/\R,\C/\R)$. Then 
	\[
	\varphi(x+iy)=\varphi(x)+\varphi(i)\varphi(y)=x+\varphi(i)y
	\]
	for all $x,y\in\R$. Since $\varphi(i)^2=\varphi(i^2)=\varphi(-1)=-1$, 
	it follows that $\varphi(i)\in\{-i,i\}$. Thus either 
	$\varphi(x+iy)=x+iy$ or $\varphi(x+iy)=x-iy$. 
\end{example}

\begin{exercise}
	Prove that if $K$ is a field and $\sigma\colon K\to K$ is a field homomorphism, 
then $\sigma\in\Hom(K/K_0,K/K_0)$. 
\end{exercise}

If $E/K$ is an extension, then
\[
	\Aut(E/K)=\{\sigma:\sigma\colon E\to E\text{ is a bijective extension homomorphism}\}
\]
is a group with composition.

\begin{definition}
	Let $E/K$ be an extension. The \textbf{Galois group}
	of $E/K$ is the group
	$\Aut(E/K)$ and it will be denoted by $\Gal(E/K)$. 
\end{definition}

A typical example: $\Gal(\C/\R)\simeq\Z/2$. 

As an example, we show with the computer that $\Gal(\Q(\sqrt{2})/\Q)\simeq\Z/2$:
\begin{lstlisting}
julia> E, x = quadratic_field(2)
(Real quadratic field defined by x^2 - 2, sqrt(2))
julia> characteristic(E)
0
julia> G, C = galois_group(E);
julia> describe(G)
"C2"
julia> order(G)
2
\end{lstlisting}

\begin{example}
	Let $\theta=\sqrt[3]{2}$ and let $E=\{a+b\theta+c\theta^2:a,b,c\in\Q\}$. Note that 
\[
	a+b\theta+c\theta^2=0 \Longleftrightarrow a=b=c=0. 
\]
% In fact, if $abc\ne 0$, then $aX^2+bX+c\ne 0$ and 
% thus $X^3-2=q(X)(aX^2+bX+c)+r(X)$ for some polynomials
% $q(X)\in\Q[X]$ and $r(X)=eX+f\in\Q[X]$. Evaluate in $\theta$ 
% to obtain that $r(\theta)=0$ and hence $r(X)=0$ in $\Q[X]$. This implies  
% that $aX^2+bX+c$ divides $X^3-2$, a contradiction since
% $X^3-2$ is irreducible in $\Q[X]$. 
Then $E$ is an extension of $\Q$ such that $[E:\Q]=3$. We claim
that $\Gal(E/\Q)$ is trivial. If 
$\sigma\in\Gal(E/\Q)$ and $z=a+b\theta+c\theta^2$, then
$\sigma(z)=a+b\sigma(\theta)+c\sigma^2(\theta)$. Since
$\sigma(\theta)^3=\sigma(\theta^3)=\sigma(2)=2$, it follows
that $\sigma(\theta)=\theta$ and therefore
$\sigma=\id$. 
\end{example}

\begin{exercise}
    Prove that the polynomial $X^3-2$ is irreducible in $\Q[X]$.  
\end{exercise}

The previous exercise can easily be solved using
computers: 
\begin{lstlisting}
julia> R, x = PolynomialRing(QQ, "x");
julia> is_irreducible(x^3-2)
true
\end{lstlisting}

The following exercise is known as the 
\emph{Eisenstein's irreducibility criterion}:

\begin{exercise}
    \index{Eisenstein's criterion}    
    Let $A$ be a unique factorization domain and $K$ be its fraction field. 
    Let $f=\sum_{i=0}^n a_iX^i\in K[X]$ be a polynomial of degree $n>0$. 
    Assume that there exists a prime element $p\in A$ such that
    $p\mid a_i$ for all $i\in\{0,1,\dots,n-1\}$, $p\nmid a_n$ and
    $p^2\nmid a_0$. Then $f$ is irreducible in $K[X]$. 
\end{exercise}

\begin{exercise}
    Prove that
    the polynomials 
    $f=X^{10}+60X^7+82X^6-36X^3+2$ and 
    $g=3X^{10}+15X^2-45$ are irreducible in $\Z[X]$. 
\end{exercise}

\begin{exercise}
    Is the polynomial $f=3(X^{10}+5X^2-15)$ irreducible in $\Z[X]$? 
\end{exercise}

If $E/K$ is an extension and $S$ is a subset of $E$, then
there exists a unique smallest 
subextension $F/K$ of $E/K$ such that
$S\subseteq F$. In fact, 
\[
	F=\bigcap\{T:\text{$T$ is a subfield of $E$ that contains $K\cup S$}\} 
\]
If $L/K$ is a subextension of $E/K$ such that 
$S\subseteq L$, then $F\subseteq L$ by definition. The 
extension $F$ is known as the \textbf{subextension generated by} 
$S$ and
it will be denoted by $K(S)$. 
If $S=\{x_1,\dots,x_n\}$ is finite,
then $K(S)=K(x_1,\dots,x_n)$ is said to be of \textbf{finite type}. 

\begin{example}
	If $\{e_1,\dots,e_n\}$ is a basis of $E$ over $K$, 
	then $E=K(e_1,\dots,e_n)$. 
\end{example}

\begin{example}
	The field $\Q(\sqrt{2})$ is precisely the extension 
	of $\R/\Q$ generated by $\sqrt{2}$. 
\end{example}

Let $E/K$ be an extension and $S$ and $T$ be subsets of $E$.
Then 
\[
	K(S\cup T)=K(S)(T)=K(T)(S).
\]
If, moreover, 
$S\subseteq T$, then $K(S)\subseteq K(T)$. 

\topic{Algebraic extensions}

\begin{definition}
\index{Algebraic!element}
\index{Trascendental!element}
	Let $E/K$ be an extension. An element $x\in E$
	is \textbf{algebraic} over $K$ if there
	exists a non-zero polynomial 
	$f(X)\in K[X]$ such that $f(x)=0$. If $x$ is
	not algebraic over $K$, 
	then it is called \textbf{trascendental} over $K$.
\end{definition}

If $E/K$ is an extension, let 
\[
	\overline{K}_E=\{x\in E:x\text{ is algebraic over }K\}. 
\]
%is the \textbf{algebraic closure} of $K$ in $E$. 

\begin{definition}	
\index{Algebraic!extension}
	An extension $E/K$ is \textbf{algebraic} if 
	every $x\in E$ is algebraic over $K$. 
\end{definition}

If $K$ is a field, every $x\in K$ is algebraic over $K$,
as $x$ is a root of $X-x\in K[X]$. In particular, $K/K$ is
an algebraic extension. 

\begin{example}
	$\C/\R$ is an algebraic extension. If $z\in\C\setminus\R$, then
	$z$ is a root of the polynomial 
	$X^2+(z+\overline{z})X+|z|^2\in\R[X]$. 
\end{example}

If $F/K$ is an algebraic extension $x\in E$ is algebraic
over $K$ for some field $E\supseteq F$, 
then $x$ is algebraic over $F$. 

\begin{example}
	$\Q(\sqrt{2})/\Q$ is algebraic, as the number
	$a+b\sqrt{2}$ is a root of the polynomial
	$X^2-2aX+(a^2-2b^2)\in\Q[X]$. 
\end{example}

\index{Lindemann's theorem}
\index{Hermite's theorem}
The extension $\C/\Q$ is not algebraic. For example, Hermite proved 
that $e$ is transcendental 
over $\Q$; see \cite[Therem 24.4]{MR3379917}. Lindemann's theorem 
states that $\pi$ is 
not algebraic $\Q$; see \cite[Theorem 24.5]{MR3379917}. 

%\section{19/02/2024}

If $E/K$ is an extension and $x\in E$ is algebraic
over $K$, then the evaluation homomorphism 
$K[X]\to E$, $p\mapsto p(x)$, is not injective. In particular,
its kernel is a non-zero ideal. Hence 
it is generated by a monic polynomial $f$. 

\begin{definition}
\index{Minimal polynomial}
	Let $E/K$ be an extension and $x\in E$ be an algebraic element.  The monic
	polynomial that generates the kernel of $K[X]\to E$, $f\mapsto f(x)$, is
	known as the \textbf{minimal polynomial} of $x$ over $K$ and it will be
	denoted by $f(x,K)$. The \textbf{degree} of $x$ over $K$ is then $\deg
	f(x,K)$. 
\end{definition}

Some basic properties of the minimal polynomial of an algebraic element:

\begin{proposition}
	Let $E/K$ be an extension and $x\in E$. Assume that $x$ is algebraic over $K$. 
	\begin{enumerate}
		\item If $g\in K[X]\setminus\{0\}$ is such that $g(x)=0$, then $f(x,K)$ divides $g$ and  
		$\deg f(x,K)\leq\deg g$. 
%		\item If $g(x)=0$ and $g\ne 0$, then $\deg g\geq\gr f(x,K)$.
		\item $f(x,K)$ is irreducible in $K[X]$.
		%\item If $g(x)=0$ and $g(X)$ is monic and irreducible, then
		%	$g=f(x,K)$. 
		\item If $F/K$ is a subextension of $E/K$, then $f(x,F)$ divides
			$f(x,K)$. 
	\end{enumerate}
\end{proposition}

\begin{proof}
	Write $f=f(x,K)$ to denote the minimal polynomial of $x$. 
	To prove 1) note that $g(x)=0$ implies that	$g$ belongs to the kernel of
	the evaluation map, so $g$ is a multiple of $f$. To prove 2) 
	note that if $f=pq$ for some $p,q\in K[X]$ such that
	$0<\deg p,\deg q<\deg f$, then $f(x)=0$ implies that 
	either $p(x)=0$ or $q(x)=0$, a
	contradiction. Finally, we prove 3). Since $f\in K[X]\subseteq F[X]$ 
	and $f(x)=0$, it follows from 1) that $f(x,F)$ divides $f$. 
\end{proof}

Some easy examples: $f(i,\R)=X^2+1$, 
$f(i,\C)=X-i$ and 
$f(\sqrt[3]{2},\Q)=X^3-2$:
\begin{lstlisting}
julia> E, x = radical_extension(3, QQ(2), "x");

julia> minpoly(x)
x^3 - 2

julia> F, y = quadratic_field(-1);

julia> minpoly(y)
x^2 + 1
\end{lstlisting}

\begin{example}
	Let us compute 
	$f(\sqrt{2}+\sqrt{3},\Q)$. Let $\alpha=\sqrt{2}+\sqrt{3}$. 
	Then 
	\begin{align*}
		\alpha-\sqrt{2}=\sqrt{3} & \implies 
		(\alpha-\sqrt{2})^2=3 \implies \alpha^2-2\sqrt{2}\alpha+2=3\\
		&\implies \alpha^2-1=2\sqrt{2}\alpha \implies
		(\alpha^2-1)^2=8\alpha^2\implies
		\alpha^4-10\alpha^2+1=0.
	\end{align*}
	Thus $\alpha$ is a root of $g=X^4-10X^2+1$. To prove that $g=f(\alpha,\Q)$ 
	it is enough to prove that 
	$g$ is irreducible in $\Q[X]$. First note that 
	the roots
	of $g$ are $\sqrt{2}+\sqrt{3}$, $\sqrt{2}-\sqrt{3}$, 
	$-\sqrt{2}+\sqrt{3}$ and $-\sqrt{2}-\sqrt{3}$. This means that
	if $g$ is not irreducible, 
	then $g=hh_1$ for some polynomials $h,h_1\in\Q[X]$ such that
	$\deg h=\deg h_1=2$. This is not possible, as 
	$(\sqrt{2}+\sqrt{3})+(\sqrt{2}-\sqrt{3})=2\sqrt{2}\not\in\Q$, 
	$(\sqrt{2}+\sqrt{3})+(-\sqrt{2}+\sqrt{3})=2\sqrt{3}\not\in\Q$ and 
	$(\sqrt{2}+\sqrt{3})(-\sqrt{2}-\sqrt{3})=-5-2\sqrt{6}\not\in\Q$.
\end{example}

%\begin{lstlisting}
%julia> E, x = quadratic_field(2);
%julia> minpoly(x)
%x^2 - 2
%\end{lstlisting}

\begin{proposition}
\label{pro:multiplicativity of degree}
	Let $F/K$ be a subextension of $E/K$. Then
	\[
	[E:K]=[E:F][F:K].
	\]
\end{proposition}

\begin{proof}
	Let $\{e_i:i\in I\}$ be a basis of $E$ over $F$
	and $\{f_j:j\in J\}$ be a basis of $F$ over $K$. If $x\in E$,
	then $x=\sum_i \lambda_ie_i$ (finite sum) 
	for some $\lambda_i\in F$. For each $i$, 
	$\lambda_i=\sum_j a_{ij}f_j$ (finite sum)
	for some $a_{ij}\in K$. Then 
	$x=\sum_i\sum_j a_{ij}(f_je_i)$. This means
	that $\{f_je_i:i\in I,j\in J\}$ generates
	$E$ as a $K$-vector space. Let us prove that 
	$\{f_je_i:i\in I,j\in J\}$
	is linearly independent. If $\sum_i\sum_j a_{ij}(f_je_i)=0$ (finite sum)
	for some $a_{ij}\in K$, 
	then
	\begin{align*}
		0=\sum_i\left(\sum_j a_{ij}f_j\right)e_i&\implies
		\sum_j a_{ij}f_j=0\text{ for all $i\in I$}\\
		&\implies 
		a_{ij}=0\text{ for all $i\in I$ and $j\in J$}.\qedhere
	\end{align*}
\end{proof}

We state a lemma:

\begin{lemma}
If $A$ is a finite-dimensional commutative algebra over $K$ 
and $A$ is an integral domain, then $A$ is a field. 
\end{lemma}

\begin{proof}
	Let $a\in A\setminus\{0\}$. We need to prove that there exists $b\in A$
	such that $ab=1$. Let $\theta\colon A\to A$, $x\mapsto ax$. Note that 
	$\theta$ is $K$-linear transformation, as 
    \[
    \theta(x+y)=a(x+y)=ax+ay=\theta(x)+\theta(y),\quad
    \theta(\lambda x)=a(\lambda x)=\lambda (ax)=\lambda\theta(x),
    \]
    for all $x,y\in A$ and $\lambda\in K$. 
 It is injective since $A$ is an
	integral domain.  Since $\dim_KA<\infty$, it follows that $\theta$ is an
	isomorphism. In particular, $\theta(A)=A$, which implies that there exists
	$b\in A$ such that $1=ab$. 
\end{proof}

Let $E/K$ be an extension and $x\in E$. 
Then 
\[
K[x]=\{f(x): f\in K[X]\}
\]
is a subring of $E$ that contains $K$. Note that 
$K[x]$ is a $K$-vector space. 

More generally,
if $x_1,\dots,x_n\in E$, then
\[
K[x_1,\dots,x_n]=\{f(x_1,\dots,x_n):f\in K[X_1,\dots,X_n]\}
\]
is a subring of $E$. 
Note that $K[x_1,\dots,x_n]$ is a $K$-vector space. 
Clearly, $K[x_1,\dots,x_n]$ is a domain
and 
\[
K(x_1,\dots,x_n)=\left\{\frac{f(x_1,\dots,x_n)}{g(x_1,\dots,x_n)}:f,g\in K[X_1,\dots,X_n]\text{ with $g(x_1,\dots,x_n)\ne 0$}\right\}
\]
is the extension of $K$ generated by $x_1,\dots,x_n$. 
Note that 
\[
K(x_1,\dots,x_n)=(K(x_1,\dots,x_{n-1}))(x_n).
\]
The previous construction
can be generalized. Let $I$ be a non-empty set. 
For each $i\in I$, let $X_i$ be a variable. Consider
the polynomial ring $K[\{X_i:i\in I\}]$ and let 
$S=\{x_i:i\in I\}$ be a subset of $E$. There exists a unique 
algebra homomorphism 
\[
K[\{X_i:i\in I\}]\to E
\]
such that $X_i\mapsto x_i$ for all $i\in I$. The image 
is denoted by $K[S]$. In particular, an element $z\in K[S]$ 
is of the form 
\[
z=h(x_1,\dots,x_n)
\]
for a polynomial $h\in K[X_1,\dots,X_n]$ 
in finitely many variables $X_1,\dots,X_n$ and 
$x_1,\dots,x_n\in S$. 

\begin{exercise}
\label{xca: Q[sqrt2]=Q(sqrt2)}
    Prove that $\Q[\sqrt{2}]=\Q(\sqrt{2})$. 
\end{exercise}

The exercise is not an accident. 

\begin{theorem}
\label{thm:simple extesnions}
	Let $E/K$ be an extension and $x\in E\setminus K$.
	The following statements are equivalent:
	\begin{enumerate}
		\item $x$ is algebraic over $K$.
		\item $\dim_KK[x]<\infty$.
		\item $K[x]$ is a field.
		\item $K[x]=K(x)$. 
	\end{enumerate}
\end{theorem}

\begin{proof}
	We first prove $1)\implies 2)$. Let $z\in K[x]$, say $z=h(x)$ for some $h\in K[X]$. There exists
	$g\in K[X]$ such that $g\ne 0$ and $g(x)=0$. Divide $h$ by $g$ to obtain 
	polynomials $q,r\in K[X]$ such that $h=gq+r$, where $r=0$ or $\deg r<\deg g$. This implies that
	\[
		z=h(x)=g(x)q(x)+r(x)=r(x).
	\]
	If $\deg g=m$, then $r=\sum_{i=0}^{m-1}a_iX^i$ for some $a_0,\dots,a_{m-1}\in K$. Thus
	\[
 z=\sum_{i=0}^{m-1}a_ix^i 
 \]
 and hence $K[x]\subseteq\langle 1,x,\dots,x^{m-1}\rangle$. 

	The previous lemma proves that $2)\implies 3)$. 

	It is trivial that $3)\implies 4)$. 

	It remains to prove that $4)\implies 1)$. 
	Since $x\ne 0$, $1/x\in K(x)=K[x]$. There exists $a_0,\dots,a_n\in K$ such that
	$1/x=a_0+a_1x+\cdots+a_nx^n$. Thus
	\[
		a_nx^{n+1}+\cdots+a_1x^2+a_0x-1=0, 
	\]
	and hence $x$ is a root of $a_nX^{n+1}+\cdots+a_0X-1\in K[X]\setminus\{0\}$. 
\end{proof}

Note that if $x$ is algebraic over $K$, then
$K[x]\simeq K[X]/(f(x,K))$. 

\begin{exercise}
\label{xca:degree_of_x}
    Let $E/K$ be an extension and $x\in E$ be an algebraic element over $K$.
    Prove that the degree of $x$ over $K$ is equal to $[K(x):K]$. 
\end{exercise}

\begin{corollary}
\label{cor:finite=>algebraic}
	If $E/K$ is finite, then $E/K$ is algebraic. 
\end{corollary}

\begin{proof}
	Let $n=[E:K]$ and $x\in E\setminus K$. The set $\{1,x,\dots,x^n\}$ has $n+1$ elements, so it is linearly dependent. 
	There exist $a_0,\dots,a_n\in K$, not all zero, such that
	\[
        a_0+a_1x+\cdots+a_nx^n=0.
        \]
        Thus $x$ is a root of the non-zero
	polynomial $a_0+a_1X+\cdots+a_nX^n\in K[X]$. 
\end{proof}

In Example~\ref{exa:two_algebraics} we 
proved that $\sqrt{2}+\sqrt[3]{3}$ and $\sqrt{2}\sqrt[3]{3}$ 
are algebraic over $\Q$. This can be easily proved
now with Corollary~\ref{cor:finite=>algebraic}. 

\begin{exercise}
\label{xca:algebraic}
    Let $E/K$ be an extension and 
    $a$ and $b$ be algebraic over $K$. Prove 
    that $a+b$ and $ab$ are algebraic over $K$. 
\end{exercise}

We note that the converse of Corollary~\ref{cor:finite=>algebraic} result does not hold. 

\begin{corollary}
\label{cor:finite type finite}
	If $E/K$ is an extension and $x_1,\dots,x_n\in E$ 
	are algebraic over $K$, then 
	$K(x_1,\dots,x_n)/K$ is finite and
	$K(x_1,\dots,x_n)=K[x_1,\dots,x_n]$. 
\end{corollary}

\begin{proof}
	We proceed by induction on $n$. The case $n=1$ follows immediately from 
	the theorem. So assume the result holds for some $n\geq1$. Since the extensions 
	$K(x_1,\dots,x_n)/K(x_1,\dots,x_{n-1})$ and $K(x_1,\dots,x_{n-1})/K$ are
	both finite, it follows that $K(x_1,\dots,x_n)/K$ is finite. Moreover, 
	\begin{align*}
	K(x_1,\dots,x_n)&=K(x_1,\dots,x_{n-1})(x_n)\\
	&=K(x_1,\dots,x_{n-1})[x_n]=K[x_1,\dots,x_{n-1}][x_n]=K[x_1,\dots,x_n].\qedhere
    \end{align*}
\end{proof}

\begin{corollary}
\label{cor:finite type algebraic}
	Let $E=K(S)$ for some set $S$. Then $E/K$ is algebraic if and only if
	$x$ is algebraic over $K$ for all $x\in S$. 
\end{corollary}

\begin{proof}
	Let us prove the non-trivial implication. Let $z\in K(S)$. In particular, 
	there exists a finite subset $T\subseteq S$ such that 
	$z\in K(T)$. The previous result implies that $K(T)/K$ is algebraic, and
	hence $z$ is algebraic. 
\end{proof}

If $E/K$ is an extension, let 
\[
	\overline{K}_E=\{x\in E:x\text{ is algebraic over }K\}. 
\]
%is the \textbf{algebraic closure} of $K$ in $E$. 

\begin{corollary}
	If $E/K$ is  an extension, then $\overline{K}_E$ 
	is a subfield of $E$ that contains $K$. Moreover, 
	$K(\overline{K}_E)=\overline{K}_E$ and 
        $K(\overline{K}_E)/K$ is algebraic. 
\end{corollary}	

\begin{proof}
    By definition, $K(\overline{K}_E)/K$ is algebraic. 
    Thus $K(\overline{K}_E)\subseteq\overline{K}_E$. From this, it follows that
    $K(\overline{K}_E)=\overline{K}_E$. 
\end{proof}

The following exercise is now almost trivial:

\begin{exercise}
\label{xca:finite type}
\index{Extension!of finite type}
    Let $E/K$ be an extension of finite type; this means 
    that  
    $E=K(S)$ for some finite
    set $S$.  
    Prove that $E/K$ is algebraic if and only if $E/K$ 
    is finite. 
\end{exercise}

Let $\overline{\Q}=\{\alpha\in\C:\alpha\text{ is algebraic over }\Q\}$. 
Then $\overline{\Q}$ is the field of algebraic numbers. 
Can you compute $[\overline{\Q}:\Q]$?

\begin{exercise}
\label{xca:degree of sqrt[3]2}
    Prove that $[\Q[\sqrt[3]{2}]:\Q]=3$. 
\end{exercise}

For the previous exercise, you may use Eisenstein's criterion. 

\begin{exercise}
\label{xca:Q(i,sqrt2)}
    Let $E=\Q[i,\sqrt{2}]=\Q[\sqrt{2}][i]$. Prove that $[E:\Q]=4$.  
\end{exercise}

\begin{exercise}
\label{xca:Q(sqrt2,sqrt[3]5)}
    Let $E=\Q[\sqrt{2},\sqrt[3]{5}]$. 
    \begin{enumerate}
        \item Compute $[E:\Q]$.
        \item Prove that $E=\Q[\sqrt{2}+\sqrt[3]{5}]$.
        \item Find the minimal polynomial of $\sqrt{2}+\sqrt[3]{5}$ over $\Q$. 
    \end{enumerate}
\end{exercise}

\begin{exercise}
\label{xca:isqrt[4]3}
    Find the minimal polynomials of $\sqrt[4]{3}i$ over $\Q[i]$ and over $\Q[\sqrt{3}]$. 
\end{exercise}

\begin{exercise}
\label{xca:sqrt{2}+sqrt[3]{5}i}
    Find the minimal polynomial of $\sqrt{2}+\sqrt[3]{5}i$ over $\Q[i]$.
\end{exercise}

%\begin{theorem}[Galois]
%	\index{Galois' theorem}
%	For every prime number $p$ and every $m\geq1$
%	there exists a field of size $p^m$. 
%\end{theorem}
%
%\begin{proof}
%\end{proof}
%
%
Algebraic field extensions form a nice class of extensions. The same happens
with finite field extensions. 

\begin{proposition}
	Let $F/K$ be a subextension of $E/K$. Then $E/K$ is algebraic 
	if and only if $E/F$ and $F/K$ are algebraic. 
\end{proposition}

\begin{proof}
    If $E/K$ is algebraic, then $E/F$ and $F/K$ are both algebraic, 
    as $K\subseteq F\subseteq E$. 
    Let us assume that $E/F$ and $F/K$ are both algebraic. Let $x\in E$ and 
    let $L$ be the subextension over $K$ generated by the coefficients of $f(x,F)$, 
    the minimal polynomial of $x$ over $F$. Then $L/K$ is finite, since it is generated
    by finitely many algebraic elements. Moreover, $x$ is algebraic over $L$. Since 
    \[
    [L(x):K]=[L(x):L][L:K]<\infty,
    \]
    $L(x)/K$ is algebraic. In particular, $x$ is algebraic over $K$. 
\end{proof}

\begin{exercise}
\label{xca:tower of finite extensions}
	Let $F/K$ be a subextension of $E/K$. Prove that $E/K$ is finite 
	if and only if $E/F$ and $F/K$ are finite. 
\end{exercise}

Let $K$ be a field and $K\subseteq 
F\subseteq L$ and $K\subseteq F\subseteq L$ be fields. The \textbf{composite}  
of $E$ and $F$ is defined~as 
\[
EF=K(E\cup F)=F(E)=E(F)
\]
and it is equal to the 
smallest field that contains $E$ and $F$. Here is the
picture:
\[\begin{tikzcd}
	& L \\
	& EF \\
	E && F \\
	& K
	\arrow[no head, from=1-2, to=2-2]
	\arrow[no head, from=2-2, to=3-1]
	\arrow[no head, from=2-2, to=3-3]
	\arrow[no head, from=3-1, to=4-2]
	\arrow[no head, from=4-2, to=3-3]
\end{tikzcd}\]


\begin{exercise}
\label{xca:composite generated by products}
    Let $E/K$ and $F/K$ be algebraic field extensions. 
    Prove that 
    \[
    EF=\left\{\sum_{i=1}^me_if_i:m\in\Z_{>0},e_i\in E,f_i\in F\text{ for all $i\in\{1,\dots,m\}$}\right\}.
    \]
\end{exercise}

\begin{exercise}
\label{xca:sqrt(2),sqrt(3)}
    If $E=\Q(\sqrt{2})$ and $F=\Q(\sqrt{3})$, then $EF=\Q(\sqrt{2},\sqrt{3})$. 
    Compute $[\Q(\sqrt{2},\sqrt{3}):\Q]$ and 
    $\Q(\sqrt{2})\cap\Q(\sqrt{3})$. 
\end{exercise}

\begin{exercise}
\label{xca:sqrt[3]2,3rd root of 1}
    Let $\xi\in\C$ be a primitive cubic root of one. 
    If $E=\Q(\sqrt[3]{2})$ and $F=\Q(\xi)$, then $EF=\Q(\sqrt[3]{2},\xi)$. 
    Compute $[\Q(\sqrt[3]{2},\xi):\Q]$ and 
    $\Q(\sqrt[3]{2})\cap\Q(\xi)$. 
\end{exercise}

\begin{exercise}
\label{xca: shift for algebraic}
	Let $E/K$ and $F/K$ be extensions, where both $E$ and $F$ are subfields of 
	a field $L$. If $F/K$ is algebraic, then $EF/E$ is algebraic.
\end{exercise}

% \begin{proof}
%     If $F/K$ is algebraic, then $EF/E=E(F)/E$ is algebraic, as it is generated by 
%     algebraic elements over $E$.  
% \end{proof}

\begin{exercise}
\label{xca: shift for finite}
	Let $E/K$ and $F/K$ be extensions, where both $E$ and $F$ are subfields of 
	a field $L$. If $F/K$ is finite, then $EF/E$ is finite.
\end{exercise}

The solution to the previous exercise shows, in particular, that $[EF:E]\leq [F:K]$. 

%\chapter{}

%\begin{theorem}[Galois]
%	\index{Galois' theorem}
%	For every prime number $p$ and every $m\geq1$
%	there exists a field of size $p^m$. 
%\end{theorem}
%
%\begin{proof}
%\end{proof}
%
%
Algebraic field extensions form a nice class of extensions. The same happens
with finite field extensions. 

\begin{proposition}
	Let $F/K$ be a subextension of $E/K$. Then $E/K$ is algebraic 
	if and only if $E/F$ and $F/K$ are algebraic. 
\end{proposition}

\begin{proof}
    If $E/K$ is algebraic, then $E/F$ and $F/K$ are both algebraic, 
    as $K\subseteq F\subseteq E$. 
    Let us assume that $E/F$ and $F/K$ are both algebraic. Let $x\in E$ and 
    let $L$ be the subextension over $K$ generated by the coefficients of $f(x,F)$, 
    the minimal polynomial of $x$ over $F$. Then $L/K$ is finite, since it is generated
    by finitely many algebraic elements. Moreover, $x$ is algebraic over $L$. Since 
    \[
    [L(x):K]=[L(x):L][L:K]<\infty,
    \]
    $L(x)/K$ is algebraic. In particular, $x$ is algebraic over $K$. 
\end{proof}

\begin{exercise}
	Let $F/K$ be a subextension of $E/K$. Prove that $E/K$ is finite 
	if and only if $E/F$ and $F/K$ are finite. 
\end{exercise}

Let $K$ be a field and $K\subseteq 
F\subseteq L$ and $K\subseteq F\subseteq L$ be fields. The \textbf{composite}  
of $E$ and $F$ is defined as 
\[
EF=K(E\cup F)=F(E)=E(F)
\]
and it is equal to the 
smallest field that contains $E$ and $F$. Here is the
picture:
\[\begin{tikzcd}
	& E \\
	& EF \\
	E && F \\
	& K
	\arrow[no head, from=1-2, to=2-2]
	\arrow[no head, from=2-2, to=3-1]
	\arrow[no head, from=2-2, to=3-3]
	\arrow[no head, from=3-1, to=4-2]
	\arrow[no head, from=4-2, to=3-3]
\end{tikzcd}\]


\begin{exercise}
    Let $E$ and $F$ be fields. 
    Prove that 
    \[
    EF=\left\{\sum_{i=1}^me_if_i:m\in\Z_{>0},e_i\in E,f_i\in F\text{ for all $i\in\{1,\dots,m\}$}\right\}.
    \]
\end{exercise}

\begin{exercise}
    If $F=\Q(\sqrt{2})$ and $L=\Q(\sqrt{3})$, then $FL=\Q(\sqrt{2},\sqrt{3})$. 
    Compute $[\Q(\sqrt{2},\sqrt{3}):\Q]$ and 
    $\Q(\sqrt{2})\cap\Q(\sqrt{3})$. 
\end{exercise}

\begin{exercise}
    Let $\xi\in\C$ be a primitive cubic root of one. 
    If $F=\Q(\sqrt[3]{2})$ and $L=\Q(\xi)$, then $FL=\Q(\sqrt[3]{2},\xi)$. 
    Compute $[\Q(\sqrt[3]{2},\xi):\Q]$ and 
    $\Q(\sqrt[3]{2})\cap\Q(\xi)$. 
\end{exercise}

\begin{exercise}
	Let $E/K$ and $F/K$ be extensions, where both $E$ and $F$ are subfields of 
	a field $L$. If $F/K$ is algebraic, then $EF/E$ is algebraic.
\end{exercise}

% \begin{proof}
%     If $F/K$ is algebraic, then $EF/E=E(F)/E$ is algebraic, as it is generated by 
%     algebraic elements over $E$.  
% \end{proof}

\begin{exercise}
	Let $E/K$ and $F/K$ be extensions, where both $E$ and $F$ are subfields of 
	a field $L$. If $F/K$ is finite, then $EF/E$ is finite.
\end{exercise}

The solution to the previous exercise shows, in particular, that $[EF:E]\leq [F:K]$. 



%\begin{theorem}[Galois]
%	\index{Galois' theorem}
%	For every prime number $p$ and every $m\geq1$
%	there exists a field of size $p^m$. 
%\end{theorem}
%
%\begin{proof}
%\end{proof}
%
%
% Algebraic field extensions form a nice class of extensions. The same happens
% with finite field extensions. 

% \begin{proposition}
% 	Let $F/K$ is a subextension of $E/K$. Then $E/K$ is algebraic (resp. finite)
% 	if and only if $E/F$ and $F/K$ are algebraic (resp. finite). 
% \end{proposition}

% \begin{proof}
% 	From the formula 
% 	$[E:K]=[E:F][F:K]$ it follows that 
% 	$E/K$ is finite if and only if $E/F$ and $F/K$ are
% 	both finite. 

% 	If $E/K$ is algebraic, then $E/F$ and $F/K$ are both algebraic. Conversely,
% 	suppose now that both $E/F$ and $F/K$ are algebraic. For $x\in E$ let $L$
% 	be the extension of $K$ generated by the coefficients of $f(x,F)$, the
% 	minimal polynomial of $x$ over $F$. Then $L$ is finite, as it is generated
% 	by finitely many algebraic elements. Moreover, $x$ is algebraic over $L$.
% 	Since $[L(x):K]=[L(x):L][L:K]<\infty$, $L(x)/K$ is algebraic. In
% 	particular, $x$ is algebraic over $K$. 
% \end{proof}

% \begin{proposition}
% 	Let $E/K$ and $F/K$ be extensions, where both $E$ and $F$ are subfields of
% 	a field $L$. If $F/K$ is algebraic (resp. finite), then $EF/E$ is algebraic
% 	(resp. finite).
% \end{proposition}

% \begin{proof}
% 	Now we prove that if $F/K$ is finite, then $EF/E$ is finite. For that purpose,
% 	we show that $[EF:E]<[F:K]$. Recall that $EF=E(F)$. The elements of $F$ are
% 	algebraic over $K$, so they are algebraic over $E$. In particular, $E(F)/E$ is algebraic
% 	and $E(F)=E[F]$. Let $z\in EF$, say $z=\sum_i x_it_i$ for some $x_i\in E$ and $t_i\in F$. 
% 	The extension $F/K$ is finite, so let $\{f_1,\dots,f_m\}$ be a basis of $F$ over $K$. Then
% 	each $t_i$ can be written as $t_i=\sum_ja_{ij}f_j$ for some $a_{ij}\in K$. Then
% 	\[
% 		z=\sum_j\left(\sum_i a_{ij}x_i\right)f_j
% 	\]
% 	and thus $\{f_1,\dots,f_m\}$ generates $EF$ as a vector space over $E$. 
% \end{proof}

% \[\begin{tikzcd}
% 	& EF \\
% 	E && F \\
% 	& K
% 	\arrow[no head, from=1-2, to=2-1]
% 	\arrow[no head, from=2-1, to=3-2]
% 	\arrow[no head, from=3-2, to=2-3]
% 	\arrow[no head, from=1-2, to=2-3]
% \end{tikzcd}\]

\begin{lemma}
\label{lem:exists_bijective}
	Let $\sigma\colon K\to L$ be a field homomorphism. Then there exists an extension
	$E/K$ and a field isomorphism $\varphi\colon E\to L$
	such that $\varphi|_K=\sigma$. 
\end{lemma}

\begin{proof}
    Note that $\sigma\colon K\to\sigma(K)$ is bijective. 
    Let $A$ be a set in bijection with $L\setminus\sigma(K)$ and disjoint with $K$. 
	Let $E=K\cup A$. If $\theta\colon A\to L\setminus\sigma(K)$ is bijective, then 
	let 
	\[
		\varphi\colon E\to L,
		\quad
		\varphi(x)=\begin{cases}
			\sigma(x) & \text{if $x\in K$},\\
			\theta(x) & \text{if $x\in A$}.
		\end{cases}
	\]
	Then $\varphi$ is a bijective map such that $\varphi|_K=\sigma$. 
	Transport the operations of $L$ onto $E$, that is 
	to define binary operations on $E$ as follows: 
	\begin{align*}
		&(x,y)\mapsto x\oplus y=\varphi^{-1}(\varphi(x)+\varphi(y)), && 
		(x,y)\mapsto x\odot y=\varphi^{-1}(\varphi(x)\varphi(y)).
	\end{align*}
	Then, for example, 
	\[
		x\oplus y=\varphi^{-1}(\varphi(x)+\varphi(y))=\varphi^{-1}(\sigma(x)+\sigma(y))
		=\varphi^{-1}(\sigma(x+y))=\varphi^{-1}(\varphi(x+y))=x+y
	\]
	for all $x,y\in K$. 
\end{proof}

If $\sigma\colon A\to B$ is a ring homomorphism, then $\sigma$ induces a ring
homomorphism $\overline{\sigma}\colon A[X]\to B[X]$,
$\sum_ia_iX^i\mapsto\sum_i\sigma(a_i)X^i$. 

\begin{theorem}
	Let $K$ be a field and $f\in K[X]$ be such that $\deg f>0$. Then 
	there exists an extension $E/K$ such that $f$ admits a root in $E$. 
\end{theorem}

\begin{proof}
	We may assume that $f$ is irreducible over $K$. Let $L=K[X]/(f)$ and 
	$\pi\colon K[X]\to L$ be the canonical map. Then $L$ 
	is a field (the reader should explain why). 
	Let $\sigma\colon K\to L$, $a\mapsto \pi(aX^0)$, and 
	$g=\overline{\sigma}(f)\in L[X]$. 

	We claim that $\pi(X)$ is a root of $g$ in $L$. Suppose that $f=\sum_i a_iX^i$. 
	Then 
	\begin{align*}
		g(\pi(X))&=\overline{\sigma}(f)(\pi(X))\\
		&=\sum_i \sigma(a_i)\pi(X)^i
		=\sum_i\pi(a_iX^0)\pi(X^i)=\pi(\sum a_iX^i)=\pi(f)=0.
	\end{align*}
	The previous lemma states that 
	there exists an extension $E/K$ and an isomorphism $\varphi\colon E\to L$
	such that $\varphi|_K=\sigma$. Note that
	$\varphi(x)=0$ if and only if $x=0$. If $u=\pi(X)$, then $\varphi^{-1}(u)$ is a root of $f$ in $E$, 
	as 
	\begin{align*}
		\varphi(f(\varphi^{-1}(u)))&=\varphi\left(\sum_ia_i\varphi^{-1}(u)^i\right)
		=\varphi\left(\sum_ia_i\varphi^{-1}(u^i)\right)\\
		&=\sum_i\varphi(a_i)u^i=\sum_i\sigma(a_i)u^i=g(u)=0.\qedhere
	\end{align*}
\end{proof}

As a corollary, if $K$ is a field and $f_1,\dots,f_n\in K[X]$ are polynomials 
of positive degree, then there exists an extension $E/K$  such that 
each $f_i$ admits a root in $E$. This is proved by induction on $n$.  

\begin{definition}
	A field $K$ is \textbf{algebraically closed} if each $f\in K[X]$ 
	of positive degree admits a root in $K$. 
\end{definition}

The \emph{fundamental theorem of algebra} states that $\C$ is algebraically closed. A
typical proof uses complex analysis.  Later we will give a proof of this result
using Galois theory. 

\begin{proposition}
	The following statements are equivalent:
	\begin{enumerate}
		\item $K$ is algebraically closed.
		\item If $f\in K[X]$ is irreducible, then $\deg f=1$.
		\item If $f\in K[X]$ is non-zero, then $f$ decomposes linearly in $K[X]$, that is
			\[
				f=a\prod_{i=1}^n(X-\alpha_i)^{m_i}
			\]
			for some $a\in K$ and $\alpha_1,\dots,\alpha_n\in K$. 
		\item If $E/K$ is algebraic, then $E=K$. 
	\end{enumerate}
\end{proposition}

\begin{proof}
	$1)\implies 2\implies 3)$ are exercises.  
	
	Let us prove that $3)\implies
	4)$. Let $x\in E$. Decompose $f(x,K)$ linearly in $K[X]$ as
	$f(x,K)=a\prod_{i=1}^n(X-\alpha_i)^{m_i}$ and evaluate on $x$ to obtain that
	$x=\alpha_j$ for some $j$. 
	
	To prove that $4)\implies 1)$ let $f\in K[X]$ be
	such that $\deg f>0$. There exists an extension $E/K$ such that $f$ has a
	root $x$ in $E$. The extension $K(x)/K$ is algebraic and hence $K(x)=K$, so
	$x\in K$. 
\end{proof}



\topic{Artin's theorem}

\begin{definition}
	The \textbf{algebraic closure} of a field $K$ is an algebraic extension $C/K$ 
	such that $C$ is algebraically closed. 
\end{definition}

For example, $\C/\R$ is an algebraic closure but $\C/\Q$ is not. 

\begin{proposition}
\label{pro:Artin}
	Let $C$ be algebraically closed and $\sigma\colon K\to C$ be a field homomorphism. If $E/K$ 
	is algebraic, then there exists a field homomorphism 
	$\varphi\colon E\to C$ such that 
	$\varphi|_K=\sigma$. 
\end{proposition}

\begin{proof}
	Suppose first that $E=K(x)$ and let $f=f(x,K)$. Let $\overline{\sigma}(f)\in C[X]$ 
	and let $y\in C$ be a root of $\overline{\sigma}(f)$. If $z\in E$, then $z=g(x)$ for
	some $g\in K[X]$. Let $\varphi\colon E\to C$, $z\mapsto \overline{\sigma}(g)(y)$. 

	The map $\varphi$ is well-defined. If $z=h(x)$ for some $h\in K[X]$, then
	\[
	0=g(x)-h(x)=(g-h)(x)
	\]
	and thus $f$ divides $g-h$. In particular, $\overline{\sigma}(f)$ divides
    $\overline{\sigma}(g-h)=\overline{\sigma}(g)-\overline{\sigma}(h)$ and hence
    $(\overline{\sigma}(g)-\overline{\sigma}(h))(y)=0$. 

	It is an exercise to show that the map $\varphi$ is a ring homomorphism.
	
	Let $a\in K$. It follows that $\varphi|_K=\sigma$, as 
	\[
	\varphi(a)=\overline{\sigma}(aX^0)(y)=\sigma(a)
	\]
	%and 
	%$\varphi(x)=\overline{\sigma}(X)(y)=y$. 
	
	Let us now prove the proposition in full generality. Let 
	$X$ be the set of pairs $(F,\tau)$, where $F$ is a subfield of $E$ that contains $K$ and
	$\tau\colon F\to C$ is a field homomorphism such that $\tau|_K=\sigma$. Note that
	$(K,\sigma)\in X$, so $X$ is non-empty. Moreover, $X$ is partially ordered by
	\[
	(F,\tau)\leq (F_1,\tau_1)\Longleftrightarrow F\subseteq F_1\text{ and }\tau_1|_F=\tau.
	\]
	If $\{(F_i,\tau_i):i\in I\}$ is a chain in $X$, then $F=\cup_{i\in I}F_i$ is a subfield of $E$
	that contains $K$. Moreover, if $z\in F$, then $z\in F_i$ for some $i\in I$ and 
	then one defines $\tau(z)=\tau_i(z)$. It is an exercise to prove that $\tau$ is well-defined.
	Since $(F,\tau)\in X$ is an upper bound, Zorn's lemma implies that there exists
	a maximal element 
	$(E_1,\theta)\in X$. We claim that $E=E_1$. If not, let $z\in E\setminus E_1$. 
	Since we know the proposition is true for the extension $E_1(z)/E_1$, 
	let  
	$\rho\colon E_1(z)\to C$ be a field homomorphism such that $\rho|_{E_1}=\theta$. Then, in particular, 
	$\rho|_K=\sigma$. This implies that $(E_1(z),\rho)\in X$ and hence
	$(E_1,\theta)<(E_1(z),\rho)$, a contradiction to the maximality of $(E_1,\theta)$. 
\end{proof}


%\chapter{}

The previous proposition will be used to prove 
that the algebraic closure always exists. 

\begin{theorem}[Artin]
	\index{Artin's theorem}
	Let $K$ be a field. Then $K$ admits an algebraic closure $C/K$. If $C_1/K$
	is an algebraic closure, then the extensions $C/K$ and $C_1/K$ are
	isomorphic. 
\end{theorem}

\begin{proof}
    Let us first prove the uniqueness. The previous proposition implies the existence of 
    an extensions homomorphism $\varphi\colon C\to C_1$. Let $y\in C_1$ and $f=f(y,K)$ be 
    the minimal polynomial of $y$ in $K$. Since $f$ admits a factorization
    \[
        f=\lambda\prod (X-\alpha_i)^{m_i}
    \]
    in $C[X]$, it follows that
    \[
    f=\overline{\varphi}(f)=\prod (X-\varphi(\alpha_i))^{m_i}
    \]
    Since $0=f(y)$, we conclude that $y=\varphi(\alpha_j)$ for some $j$. In particular, $\varphi$ is
    surjective and hence $\varphi$ is bijective. 
    
    We now prove the existence. Let us assume that $K$ admits an extension $E/K$ 
    with $E$ algebraically closed. Let 
    \[
    	F=\overline{K}_E=\{x\in E:x\text{ is algebraic over }K\}. 
    \]
    Then $F/K$ is algebraic. Let $g\in F[X]$ be such that
    $\deg g>0$. Since $E$ is algebraically closed, $g$ admits a root $\alpha$ in $E$. In particular, $\alpha$
    is algebraic over $F$ and hence $\alpha$ is algebraic over $K$. This implies that $\alpha\in F$, thus
    $F$ is algebraically closed. This proves that $F/K$ is an algebraic closure. 
    
    Let us prove that there exists an extension $E_1/K$ such that
    every polynomial $f\in K[X]$ with $\deg f>0$ has a root in $E_1$. Let 
    $\{f_i:i\in I\}$ be the family of monic irreducible polynomials with coefficients in $K$. 
    We may think that $f_i=f_i(X_i)$. 
    Let $R=K[\{X_i:i\in I\}]$ and let $J$ be the ideal of $R$ 
    generated by the $f_i(X_i)$. We claim that $J\ne R$. If not, $1\in J$, so
    \[
    1=\sum_{j=1}^m g_jf_{i_j}(X_j)
    \]
    for some $g_1,\dots,g_m\in R$. There exists an extension $F/K$ such that
    $f_{i_j}$ has a root $\alpha_j$ in $F$ for all $j$. Let 
    \[
    \sigma\colon R\to F,\quad
    \sigma(X_k)=\begin{cases}
        \alpha_j & \text{if $k=i_j$},\\
        0 & \text{if $k\not\in\{i_1,\dots,i_m\}$}.
        \end{cases}
    \]
    Then $1=\sigma(1)=\sum_{j=1}^m\sigma(g_j)f_{i_j}(\alpha_j)$, a contradiction. 
    
    Since $J$ is a proper ideal, it is contained in a maximal ideal $M$. Let $L=R/M$ 
    and let $\sigma\colon K\to L$ be given by...
    Then $\pi(X_i)$ is a root of $\overline{\sigma}(f_i)$ for all $i$ 
    and there exists an extension $E_1/K$ such that
    every $f_i$ has a root in $E_1$. Proceeding in this way, we construct
    a sequence
    \[
    E_1\subseteq E_2\subseteq\cdots
    \]
    of fields such that every polynomial of positive degree and coefficients in $E_k$ 
    admits a root in $E_{k+1}$. Let $E=\cup E_k$. We claim that $E$ is algebraically closed. In fact, 
    let $g\in E[X]$ be such that $\deg g>0$. Then, since $g\in E_r[X]$ for some $r$, it follows
    that $g$ has a root in $E_{r+1}\subseteq E$. 
\end{proof}

\topic{Decomposition fields}

\begin{definition}
\index{Decomposition field}
	Let $K$ be a field and $f\in K[X]$ be such that $\deg f>0$. A \textbf{decomposition field}
	of $f$ over $K$ is field $E$ that contains $K$ and that satisfies the following properties:
	\begin{enumerate}
		\item $f$ factorizes linearly in $E[X]$. 
		\item if $F$ is a field such that $K\subseteq F\subseteq E$ and 
			$f$ factorizes
			linearly in $F[X]$, then $F=E$. 
	\end{enumerate}
\end{definition}

Easy examples: 

\begin{example}
	$\C$ is a decomposition field of $X^2+1\in\R[X]$. 
\end{example}

\begin{example}
	$\Q[\sqrt{2}]$ is a decomposition field of $X^2-2\in\Q[X]$. 
\end{example}

\begin{example}
	$\Q(\sqrt[3]{2})$ is not a decomposition field of $X^3-2\in\Q[X]$. However, if
	$\omega$ ia a primitive cubic root of one, then 
	$\Q(\sqrt[3]{2},\omega)$ is is a decomposition field of $X^3-2\in\Q[X]$. 
\end{example}

\begin{proposition}
	$E$ is a decomposition field of $f\in K[X]$ if and only if
	$f$ factorizes linearly in $E[X]$ and $E=K(x_1,\dots,x_n)$ where 
	$x_1,\dots,x_n$ are roots of $f$. 
\end{proposition}

\begin{proof}
    Let $f=a\prod_{i=1}^r(X-x_i)^{n_i}$ and $F=K(x_1,\dots,x_r)$ with $x_1,\dots,x_r\in E$. Since $f$
    factorizes linearly in $F[X]$, it follows that $F=E$. 
    Conversely, let $E=K(x_1,\dots,x_r)$ and assume that $f$ factorizes linearly
    in $F[X]$. Then, in particular, $x_1,\dots,x_r\in F$. Hence $E\subseteq F$ and
    $F=E$. 
\end{proof}

One immediately obtains the following consequence:
If $E$ is a decomposition field of $f\in K[X]$, then $E/K$ is finite. 

\begin{theorem}
    Let $f\in K[X]$. There exists a (unique up to extension isomorphism) 
    decomposition field of $f$ over $K$. 
\end{theorem}

\begin{proof}
    Let $C/K$ be an algebraic closure. Write $f=a\prod_{i=1}^r(X-x_i)^{n_i}$ in $C[X]$. 
    Then $E=K(x_1,\dots,x_r)$ is a decomposition field of $f$ over $K$. Let us prove
    uniqueness: if $E_1/K$ is a decomposition field of $f$ over $K$, 
    then $E_1/K$ is algebraic and thus Proposition
    \ref{pro:Artin} implies that 
    there exists $\varphi\in\Hom(E_1/K,C/K)$ such that $\varphi|_K$ is the identity.
    Factorize $f$ linearly in $E_1[X]$ and apply $\overline{\varphi}$:
    \[
    f=a\prod_{j=1}^s(X-y_j)^{m_j}
    \implies
    \overline{\varphi}(f)=\varphi(a)\prod_{j=1}^s(X-\varphi(y_j))^{m_j}
    \]
    so $f$ factorizes linearly in $\varphi(E_1)$. Moreover, 
    $E_1=K(y_1,\dots,y_s)$ and it follows that 
    $\varphi(E_1)=K(\varphi(y_1),\dots,\varphi(y_s))$. Thus
    $\varphi(E_1)$ is a decomposition field of $f$. Since  
    $\varphi(E_1)\subseteq C$, it follows that $\varphi(E_1)=E$. 
\end{proof}

\begin{exercise}
If $E/K$ is finite and $\varphi\in\Hom(E/K,E/K)$, 
then $\varphi$ is an isomorphism. 
\end{exercise}

Let $C$ be an algebraic closure of $K$ and 
$G=\Gal(C/K)$. The group $G$ acts on $C$
\[
\sigma\cdot x=\sigma(x),\quad
\sigma\in G,\,x\in C.
\]
The orbits 
are of the form 
\[
O_G(x)=\{\sigma(x):\sigma\in G\}
=\{y\in C:y=\sigma(x)\text{ for some $\sigma\in G$}\}
\]
The elements $x,y\in C$ are \textbf{conjugate} 
if $y=\sigma(x)$ for some $\sigma\in G$. 

\begin{proposition}
    Let $C$ be an algebraic closure of $K$ and $x,y\in C$. Then 
    $x$ and $y$ are conjugate if and only if $f(x,K)=f(y,K)$. In particular, 
    $O_G(x)$ is finite. 
\end{proposition}

\begin{proof}
    Let $G=\Gal(C/K)$. 
    If $x$ and $y$ are conjugate, say $y=\sigma(x)$ for some $\sigma\in G$, 
    let us write $g=f(x,K)$ as 
    \[
    g=X^n+\sum_{i=0}^{n-1} a_iX^i. 
    \]
    Then $0=g(x)=x^n+\sum_{i=0}^{n-1}a_ix^i$ and hence $y$ is
    a root of $g$, as 
    \begin{align*}
    0=\sigma\left(x^n+\sum_{i=0}^{n-1}a_ix^i\right)
    &=\sigma(x)^n+\sum_{i=0}^{n-1}\sigma(a_i)\sigma(x)^i\\
    &=\sigma(x)^n+\sum_{i=0}^{n-1}a_i\sigma(x)^i
    =y^n+\sum_{i=0}^{n-1}a_iy^i.
    \end{align*}
    Thus $f(y,K)=g$. 
    
    Conversely, assume that $f(x,K)=f(y,K)$. Let
    $g=f(x,K)=f(y,K)$ and let 
    \[
    \varphi\colon K[x]\to K[y],
    \quad
    h(x)\mapsto h(y).
    \]
    Let us show that the map $\varphi$ is well-defined: we need to show 
    that if 
    $h_1(x)=h_2(x)$, then $h_1(y)=\varphi(h_1(x))=\varphi(h_2(x))=h_2(y)$. 
    If $h_1(x)=h_2(x)$, then 
    \[
    (h_1-h_2)(x)=h_1(x)-h_2(x)=0.
    \]
    Thus implies
    that $g$ divides $h_1-h_2$. In particular, $h_1(y)=h_2(y)$.
    
    A straightforward calculation shows that $\varphi$ is a field 
    homomorphism such that $\varphi|_K=\id$, so $\varphi$ is 
    an extension homomorphism such that $\varphi(x)=y$. There exists
    $\sigma\in\Hom(C/K,C/K)$ such that 
    $\sigma|_{K[x]}=\varphi$. Since $\sigma$ is a bijective, 
    $\sigma(x)=\varphi(x)=y$ and hence 
    $O_G(x)=O_G(y)$. 
\end{proof}

\begin{proposition}
    Let $C$ be an algebraic closure of $K$ and $x$. Then 
    \[
    f(x,K)=\prod_{y\in O_G(x)}(X-y)^m
    \]
    for some $m$. 
\end{proposition}

\begin{proof}
    For each $y\in O_G(x)$ let $m_y$ be the multiplicity
    of $y$ in $f(x,K)$. 
    Then, for example,
    $f(x,K)=(X-x)^{m_x}g$ for some $g$. If $y\in O_G(x)$, 
    then $y=\sigma(x)$ for some $\sigma\in\Gal(C/K)$. Since
    \[
    \overline{\sigma}(f(x,K))=f(x,K)=(X-y)^{m_x}\overline{\sigma}(g), 
    \]
    it follows that $m_y\geq m_x$. By symmetry, 
    we conclude that $m_x=m_y$. 
\end{proof}

The previous proposition shows, in particular, 
that all the roots of 
an irreducible polynomial $f\in K[X]$ 
in an algebraic closure $C$ of $K$
have the same multiplicity. This is clearly 
not true if $f$ is not irreducible. Find an example.

%Take 
%for example $f=(X-1)^2(X^2+1)\in\Q[X]$. 

% pag 20

\begin{definition}
\index{Decomposition field}
    Let $K$ be a field and $\{f_i:i\in I\}$ be a non-empty 
    family of polynomials of positive degree
    with coefficients in $K$. A \textbf{decomposition field} 
    of the $\{f_i:i\in I\}$ is an extension $E/K$
    such that every $f_i$ factorizes linearly in $E[X]$ and 
    if $F/K$ is a subextension of $E/K$ such that every $f_i$ 
    factorizes linearly in $F[X]$, then $F=E$. 
\end{definition}

\begin{exercise}
    Prove that $E/K$ is a decomposition field of
    $\{f_i:i\in I\}$ if and only if every $f_i$ factorizes linearly 
    in $E[X]$ and $E=K(S)$ where $S=\{\text{roots of $f_i$ for all $i$}\}$. 
\end{exercise}

\begin{exercise}
    Prove that if $E/K$ is a decomposition field
    of $\{f_i:i\in I\}$, then $E/K$ is algebraic. If, moreover, 
    $I$ is finite, then $E/K$ is a decomposition field
    of $\prod_{i\in I}f_i$. 
\end{exercise}

\begin{exercise}
    Prove that 
    there exists a decomposition field of $\{f_i:i\in I\}$ 
    and it is unique up to extension isomorphism. 
\end{exercise}

\topic{Normal extensions}

\begin{proposition}
    Let $E/K$ be an algebraic extension and $\sigma\in\Hom(E/K,E/K)$. 
    Then $\sigma$ is bijective. 
\end{proposition}

\begin{proof}
    Let $x\in E$ and 
    $C$ be an algebraic closure of $K$ that contains $E$. There
    exists $\varphi\colon C\to C$ such that $\varphi|_E=\sigma$. 
\end{proof}
%\chapter{}


\topic{Normal extensions}

% \begin{proposition}
%     Let $E/K$ be an algebraic extension and $\sigma\in\Hom(E/K,E/K)$. 
%     Then $\sigma$ is bijective. 
% \end{proposition}

% \begin{proof}
%     Let $x\in E$ and 
%     $C$ be an algebraic closure of $K$ that contains $E$. There
%     exists a field homomorphism $\varphi\colon C\to C$ such that $\varphi|_E=\sigma$. 
%     Thus $\varphi|_K=\sigma|_K=\id_K$. Let $G=\Gal(C/K)$. Then  
%     $\varphi\in G$. If $z\in O_G(x)$, 
%     then $z=\tau(x)$ for some $\tau\in G$ and hence
%     \[
%     \varphi(z)=\varphi(\tau(x))=(\varphi\tau)(x). 
%     \]
%     This implies that $\varphi(z)\in O_G(x)$ 
%     and $\varphi(O_G(x))=O_G(x)$. The restriction
%     $\sigma|_{E\cap O_G(x)}$ is injective. Then 
%     \begin{align*}
%         \sigma(E\cap O_G(x))&=\varphi(E\cap O_G(x))\\
%         &=\varphi(E)\cap\varphi(O_G(x))
%         =\sigma(E)\cap O_G(x)\subseteq E\cap O_G(x).
%         \end{align*}
%     Since $|E\cap O_G(x)|<\infty$, it follows that $E\cap O_G(x)=\sigma(E\cap O_G(x))$ and
%     hence $x$ belongs to the image of $\sigma$. 
% \end{proof}


% pag 21 example

\begin{definition}
    Let $E/K$ be an algebraic extension and $C$ be an algebraic closure of $K$ containing $E$. Then $E/K$ is \textbf{normal} if 
    $\sigma(E)\subseteq E$ for all $\sigma\in\Hom(E/K,C/K)$. 
\end{definition}

Note that $\sigma(E)\subseteq E$ in the previous definition
is equivalent to $\sigma(E)=E$. 

\begin{example}
    The extension $\Q(\sqrt[3]{2})/\Q$ is not normal. Why?
\end{example}

Some trivial examples of normal extensions: $K/K$ is normal
and if $C$ is an algebraic closure of $K$, then $C/K$ is normal. 

\begin{example}
    The extension $\Q(\sqrt{2})/\Q$ is normal. 
    Every extension generated by algebraic elements of degree two is normal. 
\end{example}

\begin{exercise}
    Let $\xi$ be a primitive cubic root of one. Then 
    $\Q(\sqrt[3]{2},\xi)/\Q$ is normal. 
\end{exercise}

The following result is practical but technical. That is why we leave the proof
as an exercise. 

\begin{exercise}
    Prove that the previous definition depends only on $E$ (and not on the
    algebraic closure $C$). 
\end{exercise}

Some properties:

\begin{proposition}
    Let $E/K$ be a normal extension and $f\in K[X]$ be an irreducible polynomial
    that admits a root $x$ in $E$. Then $f$ factorizes
    linearly in $E$.
\end{proposition}

\begin{proof}
    We may assume that $f$ is monic. Let $C/K$ be an algebraic closure of $K$ containing $E$. 
    Let $y$ be a root of $f$ in $C$. Since $f=f(x,K)=f(y,K)$, 
    it follows that $y=\sigma(x)$ for some $\sigma\in\Gal(C/K)$. Since 
    $E/K$ is normal, $\sigma|_E\colon E\to C$ is an automorphism of $E/K$, that is
    $\sigma(E)\subseteq E$. In particular, $y\in E$. 
\end{proof}

Let $K\subseteq F\subseteq E$ be a tower of fields. 
If $E/K$ is normal, then $E/F$ is normal. However, 
Note that $E/K$ normal does not imply $F/K$ normal, as this would imply 
that every extension is normal. Moreover, 
$E/F$ normal and $F/K$ normal do not imply $E/K$ normal.
    
\begin{example}
The extensions $\Q(\sqrt[4]{2})/\Q(\sqrt{2})$ and $\Q(\sqrt{2})/\Q$ are both
normal, but $\Q(\sqrt[4]{2})/\Q$ is not normal, 
as the roots of $X^4-2$ are
$\sqrt[4]{2}$, $-\sqrt[4]{2}$, $\sqrt[4]{2}i$ and $-\sqrt[4]{2}i$.
\end{example}


Recall that if $C$ is an algebraic closure of $K$ and $x\in C$,
then 
\[
f(x,K)=\prod(X-y)^m,
\]
where the product is taken over all 
$y\in O_{\Gal(C/K)}(x)$. 
If $E/K$ is normal and $x\in E$, then there exists $m$ such that 
\[
f(x,K)=\prod(X-y)^m,
\]
where the product is taken over all 
$y\in O_{\Gal(E/K)}(x)$. 

\begin{proposition}
    Let $E/K$ and $F/K$ be extensions. If $F/K$ 
    is normal, then $EF/E$ is normal.
\end{proposition}

\begin{proof}
    Let $C$ be an algebraic closure of $E$ containing $EF$ (this exists because $EF/E$ is algebraic). 
    Let $\sigma\in\Hom(EF/E,C/E)$. We claim that $\sigma(EF)=EF$. Let 
    \[
    \overline{K}=\{x\in C:x\text{ is algebraic over $K$}\}.
    \]
    Then $\overline{K}$ is an algebraic closure over $K$ and $F\subseteq\overline{K}$. 
    Since $F/K$ is normal and $\sigma|_F\in\Hom(F/K,\overline{K}/K)$, 
    it follows that $\sigma(F)=F$. If $z\in EF$, then
    $z=\sum_{i=1}^m e_if_i$ for some $e_1,\dots,e_m\in E$ and 
    $f_1,\dots,f_m\in F$. Since $\sigma(e_i)=e_i$ for all $i$,  
    \[
    \sigma(z)=\sum_{i=1}^m\sigma(e_i)\sigma(f_i)=\sum_{i=1}^m e_i\sigma(f_i)\in EF.\qedhere 
    \]
\end{proof}

What is the relation between 
normal extensions and decomposition fields? The notions look
deeply related. The following proposition serves as an explanation: 

\begin{proposition}
    Let $E/K$ be an algebraic extension. Then 
    $E/K$ is normal if and only if $E/K$ is the decomposition field
    of a family of polynomials of $K[X]$ of positive degree.
\end{proposition}

\begin{proof}
    Assume first that $E/K$ is a normal extension. 
    Let $G=\Gal(E/K)$.  If $x\in E$ and $f(x,K)=\prod_{y\in O_G(x)}(X-y)^m$, 
    then $f(x,K)$ factorizes linearly in $E[X]$. Thus 
    $E/K$ is a decomposition field of the family 
    $\{f(x,K):x\in E\}$. 
    
    Conversely, assume that $E/K$ is a decomposition field of the family 
    $\{f_i:i\in I\}$. Then $E=K(S)$ where $S$ is the set of roots
    of the polynomials $f_i$. Let $C/K$ be an algebraic closure
    of $K$ that contains $E$ and let $\sigma\in\Hom(E/K,C/K)$. Let $x\in S$. 
    Then $x$ is a root of some $f_j=\sum a_kX^k$. Since $f_j(x)=0$, 
    it follows that $\sigma(x)$ is a root of $f_j$, as 
    \[
    f_j(\sigma(x))=\sum a_k\sigma(x)^k
    =\sum\sigma(a_k)\sigma(x^k)
    =\sigma\left(\sum a_kx^k\right)=\sigma(0)=0.
    \]
    Hence $\sigma(E)\subseteq E$. 
\end{proof}

\begin{exercise}
    Let $E=\Q[\sqrt[4]{7}+\sqrt{2}]$. 
    \begin{enumerate}
        \item Prove that $E/\Q$ is not normal. 
        \item Compute $[E:\Q]$.
        \item Compute $\Gal(E/\Q)$. 
    \end{enumerate}
\end{exercise}


\topic{Dedekind's theorem}

Note that every extension homomorphism $E/K\to F/K$ is, in particular, 
a $K$-linear map $E\to F$, that is
\[
\Hom(E/K,F/K)\subseteq\Hom_K(E,F).
\]
If $F/K$ is an extension and $V$ 
is a $K$-vector space, the set
$\Hom_K(V,F)$ of $K$-linear maps
is a vector space over $F$ with
$(a\cdot f)(v)=af(v)$ for $a\in F$, 
$f\in\Hom_K(V,F)$ and $v\in V$. 

\begin{exercise}
\label{xca:dim}
    Let $V$ be a $K$-vector space. 
    Prove that $\dim_F\Hom_K(V,F)\geq\dim_KV$. Moreover, if 
     $\dim_KV<\infty$, then $\dim_F\Hom_K(V,F)=\dim_KV$.
\end{exercise}

If $V$ is a vector space and $S$ is a (possibly infinite) subset of $V$, 
then $S$ is linearly independent if every finite subset of $S$ is linearly independent. 

\begin{theorem}[Dedekind]
\index{Dedekind's theorem}
Let $E/K$ and $F/K$ be extensions and let 
$\{\varphi_i:i\in I\}$ 
be a subset of
$\Hom(E/K,F/K)$, i.e. a 
family of extension homomorphisms. Assume that 
$\varphi_i\ne \varphi_j$ if $i\ne j$. Then 
the subset $\{\varphi_i:i\in I\}\subseteq\Hom_K(E,F)$ 
is linearly independent over $F$. 
\end{theorem}

\begin{proof}
    Assume it is not. Let $\{\varphi_1,\dots,\varphi_n\}$ 
    be linearly dependent over $F$ with $n$ minimal. Clearly, $n>1$. 
    We may assume that 
    \begin{equation}
        \label{eq:Dedekind1}
        \sum_{i=1}^n a_i\varphi_i=0
    \end{equation}
    for some $a_1,\dots,a_n\in F$ all different from zero. 
    Let
    $z\in E\setminus\{0\}$ be such that $\varphi_1(z)\ne\varphi_2(z)$. If $x\in E$, then
    \[
    0=\left(\sum_{i=1}^na_i\varphi_i\right)(xz)=\sum_{i=1}^na_i\varphi_i(xz)
    =\sum_{i=1}^na_i\varphi_i(x)\varphi_i(z)
    =\left(\sum_{i=1}^n (a_i\varphi_i(z))\varphi_i\right)(x).
    \]
    Thus 
    \begin{equation}
        \label{eq:Dedekind2}
        \sum_{i=1}^n (a_i\varphi_i(z))\varphi_i=0.
    \end{equation}
    Since $\sum_{i=1}^na_i\varphi_i=0$ and $\varphi_1(z)\ne0$, 
    \[
    a_1\varphi_1+a_2\frac{\varphi_2(z)}{\varphi_1(z)}\varphi_2+\cdots+a_n\frac{\varphi_n(z)}{\varphi_1(z)}\varphi_n=0.
    \]
    Thus, 
    subtracting \eqref{eq:Dedekind1} and \eqref{eq:Dedekind2}, 
    \[
    \left(a_2-a_2\frac{\varphi_2(z)}{\varphi_1(z)}\right)\varphi_2
    +\cdots+\left(a_n-a_n\frac{\varphi_n(z)}{\varphi_1(z)}\right)\varphi_n=0.
    \]
    Since $a_n\ne 0$ and $\varphi_2(z)\ne\varphi_1(z)$, 
    the scalar $a_2-a_2\frac{\varphi_2(z)}{\varphi_1(z)}\ne 0$ and hence 
    $\{\varphi_2,\dots,\varphi_n\}$ is linearly dependent, a contradiction. 
\end{proof}

If $E/K$ and $F/K$ are extensions, 
let $\gamma(E/K,F/K)=|\Hom(E/K,F/K)|$. 

\begin{exercise}
Prove the following statements:
\begin{enumerate}
    \item $\gamma(E/K,F/K)\leq\dim_F\Hom_K(E,F)$.
    \item If $[E:K]<\infty$, then $\gamma(E/K,F/K)\leq[E:K]$. 
    \item If $x$ is algebraic over $K$, then $\gamma(K(x)/K,F/K)\leq\deg f(x,K)$.
\end{enumerate}
\end{exercise}

If $C$ is an algebraic closure of $K$,
then we define $\gamma(E/K)=\gamma(E/K,C/K)$. This definition does
not depend on the algebraic closure. 

\begin{exercise}
\label{xca:gamma_C}
    If $C$ and $C_1$ are algebraic closures of $K$, then
    \[
    |\Hom(E/K,C/K)|=|\Hom(E/K,C_1/K)|.
    \]
\end{exercise}

\begin{proposition}
\label{pro:gamma_orbit}
    Let $C$ be an algebraic closure of $K$ and $G=\Gal(C/K)$. 
    If $x\in C$, then $\gamma(K(x)/K)=|O_G(x)|$. 
\end{proposition}

\begin{proof}
    If $\sigma\in\Hom(K(x)/K,C/K)$, then there exists $\phi\in G$ such that
    $\phi|_{K(x)}=\sigma$. Thus $\sigma(x)=\phi(x)\in O_G(x)$. Conversely,
    if $y\in O_G(x)$, then there exists $\tau\in G$ such that
    $y=\tau(x)$. Hence $\tau|_{K(x)}\in\Hom(K(x)/K,C/K)$ and 
    $\tau|_{K(x)}(x)=y$. Since our sets are then in bijective correspondence, 
    the claim follows. 
\end{proof}


\begin{exercise}
If $E/K$ is finite, then $|\Gal(E/K)|\leq [E:K]$. Moreover, 
$E/K$ is normal if and only if $|\Gal(E/K)|=\gamma(E/K)$. 
\end{exercise}


%\chapter{}

If $t\colon A\to B$ is a surjective map, then 
$a\sim a_1\Longleftrightarrow t(a)=t(a_1)$ 
defines an equivalence relation on $A$. The set $\overline{A}$ 
of equivalence classes is in bijective correspondence with $B$,
$\overline{A}\to B$, $\overline{a}\mapsto t(a)$. 
Moreover, if $|t^{-1}(\{b\})|=m$ for all $b\in B$, then 
$|A|=m|\overline{A}|=m|B|$. 

\begin{proposition}
    Let $E/K$ be algebraic and $F/K$ be a subextension such that 
    $E/F$ is finite. Then $\gamma(E/K)=\gamma(E/F)\gamma(F/K)$. 
\end{proposition}

 \begin{proof}
    Assume that $E=F(x)$. Let $f=f(x,F)=\sum b_iX^i$ and
    let $G=\Gal(E/F)$. Let $C$ be an algebraic closure of $K$ containing $E$. 
    The map
    \[
    \lambda\colon \Hom(E/K,C/K)\to\Hom(F/K,C/K),\quad
    \sigma\mapsto\sigma|_F,
    \]
    is well-defined. It is surjective: if $\varphi\in\Hom(F/K,C/K)$, then $\varphi\colon F\to C$ is, 
    in particular, a field homomorphism. Since $E/F$ is algebraic, by Proposition \ref{pro:Artin} 
    there exists a field homomorphism 
    $\sigma\colon E\to C$ such that $\sigma|_F=\varphi$. Since $\sigma|_K=\varphi|_K=\id$, in particular 
    $\sigma\in\Hom(E/K,C/K)$. 
    
    For $\varphi\in\Hom(F/K,C/K)$,  
    \[
    \lambda^{-1}(\{\varphi\})=\{\sigma\in\Hom(E/K,C/K):\sigma|_F=\varphi\}
    \]
    and let $R_\varphi$ be the set of roots (in $C$) of the polynomial $\overline{\varphi}(f)=\sum\varphi(b_i)X^i$. 
    
    \begin{claim}
        The map $\alpha\colon \lambda^{-1}(\{\varphi\})\to R_{\varphi}$, $\sigma\mapsto\sigma(x)$, is well-defined. 
    \end{claim}
    
    We need to show that $\sigma(x)$ is a root of $\overline{\varphi}(f)$:
    \begin{align*}
    \overline{\varphi}(f)(\sigma(x))&=\sum \varphi(b_i)\sigma(x)^i
    =\sum\sigma(b_i)\sigma(x^i)\\
    &=\sum\sigma(b_ix^i)=\sigma\left(\sum b_ix^i\right)=\sigma(f(x))=\sigma(0)=0.
    \end{align*}
    
    \begin{claim}
    The map $\beta\colon R_{\varphi}\to \lambda^{-1}(\{\varphi\})$, $y\mapsto\sigma_y$, 
    where $\sigma_y(z)=\overline{\varphi}(h)(y)$
    if $z=h(x)$, is well-defined. 
    \end{claim}
    
    We need to show that if $z=h(x)$ and 
    $z=h_1(x)$ for some $h,h_1\in F[X]$, then 
    $\overline{\varphi}(h)(y)=\overline{\varphi}(h_1)(y)$. 
    The assumptions imply that 
    $(h-h_1)(x)=0$ and hence $f$ divides $h-h_1$. Since
    $\overline{\varphi}$ is a ring homomorphism, 
    $\overline{\varphi}(f)$ divides $\overline{\varphi}(h)-\overline{\varphi}(h_1)$. 
    This implies $(\overline{\varphi}(h)-\overline{\varphi}(h_1))(y)=0$. We also need to show that 
    $\sigma_y|_F=\varphi$: if $a\in F$, then 
    write $a=aX^0\in F[X]$. Thus 
    $\sigma_y(a)=\overline{\varphi}(aX^0)(y)=\varphi(a)\in C$. 
    It is now an exercise to prove that $\sigma_y\in\Hom(E/K,C/K)$. 
    
    \begin{claim}
        $|\lambda^{-1}(\{\varphi\})|=|R_\varphi|$. 
    \end{claim}
    
    For this we need to show that $\beta$ is
    the inverse of $\alpha$, that is 
    $\alpha\circ\beta=\id$ and $\beta\circ\alpha=\id$. 
    To prove that $\beta\circ\alpha=\id$ 
    let $\sigma$ be such that $\sigma|_F=\varphi$. 
    Then $y=\sigma(x)\in R_\varphi$. Let
    $z=h(x)=\sum a_ix^i\in F[x]=E$. Then  
    \[
    \overline{\varphi}(h)(y)=\sum\varphi(a_i)y^i=\sum\sigma(a_i)y^i
    =\sigma\left(\sum a_ix^i\right)=\sigma(y).
    \]
    Conversely, if $y\in R_\varphi$, then
    \[
    \alpha(\sigma_y)=\sigma_y(x)=y,
    \]
    as $\sigma_y(x)=\overline{\varphi}(X)(y)=y$.
    
    \begin{claim}
        If $\phi\in G$ is such that $\phi|_F=\varphi$, then 
        $O_{G}(x)=\phi^{-1}(R_\varphi)$.
    \end{claim}

    Let us first prove $O_{G}(x)\supseteq \phi^{-1}(R_\varphi)$.
    If $y\in R_{\varphi}$, 
    then 
    \begin{align*}
    f(\phi^{-1}(y))&=\sum b_i\phi^{-1}(y^i)=\phi^{-1}\left(\sum\phi(b_i)y^i\right)\\
    &=\phi^{-1}\left(\sum\varphi(b_i)y^i\right)=\phi^{-1}\overline{\varphi}(f)(y))=\phi^{-1}(0)=0.
    \end{align*}
    Now we prove $O_{G}(x)\subseteq\phi^{-1}(R_\varphi)$.
    Let $z\in O_{G}(x)$ and $y\in C$ be such that 
    $\phi^{-1}(y)=z$. Then $\overline{\varphi}(f)(y)=0$, as
    \begin{align*}
    \overline{\varphi}(f)(y)&=\sum\varphi(b_i)y^i\\
    &=\sum\varphi(b_i)\phi(z^i)
    =\sum\phi(b_i)\phi(z^i)=\phi\left(\sum b_iz^i\right)=\phi(f(z))=\phi(0)=0.
    \end{align*}
    
    \medskip
    It follows that $|\lambda^{-1}(\varphi)|=|O_{G}(x)|$ for
    all $\varphi$. By using the argument
    before the proposition, 
    \begin{align*}
    \gamma(E/K)&=|\Hom(E/K,C/K)|\\
        &=|O_{G}(x)||\Hom(F/K,C/K)|\\
        &=|O_{G}(x)|\gamma(F/K).
    \end{align*}
    Since $\gamma(K(x)/K)=|O_{G}(x)|$ by Proposition \ref{pro:gamma_orbit}, the claim follows. 
    
    \medskip 
    For the general case we assume that $E=F(x_1,\dots,x_n)$. We proceed
    by induction on $n$. If $n=0$, then $E=F$ and the result is trivial. 
    If $n>0$, let $L=F[x_1,\dots,x_{n-1}]$ and $E=L(x_n)$. The 
    case proved 
    implies that $\gamma(E/F)=\gamma(E/L)\gamma(L/F)$. By the inductive 
    hypothesis, $\gamma(L/K)=\gamma(L/F)\gamma(F/K)$. Thus 
    \[
    \gamma(E/F)\gamma(F/K)=\gamma(E/L)\gamma(L/F)\gamma(F/K)
    =\gamma(E/L)\gamma(L/K)=\gamma(E/K),
    \]
    again using the previous case. 
\end{proof}

\topic{Separable extensions}

\begin{definition}
    Let $E/K$ be an algebraic extension and $x\in E$. Then
    $x$ is \textbf{separable} over $K$ if $x$ is a simple root
    of $f(x,K)$. 
\end{definition}

An algebraic extension $E/K$ is \textbf{separable} 
if every $x\in E$ is separable over $K$. Clearly, $K/K$ is separable. 

\begin{exercise}
    Prove that 
    an element $x$ is separable over $K$ if and only if $x$ is a simple root
    of a polynomial with coefficients in $K$. 
\end{exercise}

If $F/K$ is a subextension of $E/K$ and $x\in E$ is separable over $K$, then
$x$ is separable over $F$. 

\begin{exercise}
    If $C$ is an algebraic closure of $K$, $x\in C$ and $G=\Gal(C/K)$. 
    Prove that the following statements are equivalent:
    \begin{enumerate}
        \item $x$ is separable over $K$.
        \item Every $y\in O_G(x)$ is separable over $K$.
        \item $\gamma(K(x)/K)=[K(x):K]=\deg f(x,K)$. 
    \end{enumerate}
\end{exercise}

Let $K$ be any field and $g\in K[X]$. Let $z$ be a root of $g$. 
Then $z$ is a multiple root of $g$ if and only if $z$ is a root of $g'$. 

\begin{exercise}
Prove that if $K$ has characteristic zero or $K$ is finite, then 
every algebraic extension of $K$ is separable. 
\end{exercise}

A consequence: 
Let $E/K$ be a finite extension. Then $E/K$ is separable
if and only if $\gamma(E/K)=[E:K]$. 

\begin{example}
    Let $E=\Q(\sqrt{2},\sqrt{3})$. Then 
    $[E:\Q]=4$ and 
    $\Gal(E/Q)\simeq C_2\times C_2$. The extension $E/Q$ is normal, 
    as it is the decomposition field of $(X^2-2)(X^2-3)$ and 
    it is separable as $\Q$ has characteristic zero. 
    % If $\sigma\in\Gal(E/\Q)$, then
    % $\sigma(\sqrt{2})\in\{-\sqrt{2},\sqrt{2}\}$ and 
    % $\sigma(\sqrt{3})\in\{-\sqrt{3},\sqrt{3}\}$. 
\end{example}

\begin{example}
    Let $E$ be a decomposition field of $X^4-2$ over $\Q$. 
    Then $E/\Q$ is normal and separable. Note that
    $E=\Q(\sqrt[4]{2},i)$, so $[E:\Q]=8=|\Gal(E/\Q)|$. 
    
    Let us compute
    $\Gal(E/\Q)$. If $\sigma\in\Gal(E/\Q)$, then 
    $\sigma(\sqrt[4]{2})\in\{\sqrt[4]{2},-\sqrt[4]{2},\sqrt[4]{2}i,-\sqrt[4]{2}i\}$ and 
    $\sigma(i)\in\{-i,i\}$. Two examples are 
    \[
    \alpha\colon\begin{cases}
    \sqrt[4]{2}\mapsto\sqrt[4]{2}i,\\
    i\mapsto i,
    \end{cases}
    \quad
    \beta\colon\begin{cases}
    \sqrt[4]{2}\mapsto\sqrt[4]{2},\\
    i\mapsto -i.
    \end{cases}
    \]
    It follows that 
    $\Gal(E/\Q)$ is isomorphic to the group $\langle\alpha,\beta\rangle$, which turns out to be
    isomorphic to the dihedral group
    of eight elements. 
\end{example}

Another consequence: If $E=K(S)$, then $E/K$ is separable if and only if
every $x\in S$ is separable over $K$. One first does the case $E=K(x)$ 
and then proceeds by induction. 

\begin{exercise}
\label{xca:separable1}
    Let $K\subseteq F\subseteq E$ be a tower of fields. Prove that 
    if $E/K$ is separable, then $F/K$ and $E/F$ are separable. 
\end{exercise}

\begin{exercise}
\label{xca:separable2}
    Let $E/K$ and $F/K$ be extensions. Prove that if $E/K$ is separable, 
    then $EF/E$ is separable. 
\end{exercise}

% This follows because $EF/E$ is algebraic and 
% $EF/E=E(F)/E$ is generated by separable elements.


%\section{24/03/2024}


\begin{proposition}
    Let $E/K$ be a finite extension. Then $E=K(x)$ for some $x\in E$ 
    if and only if $E/K$ admits finitely many subextensions. 
\end{proposition}

\begin{proof}
     We may assume that $K$ is infinite; otherwise, the result is trivial. 
    We first prove $\implies$. 
    Let us 
    assume that $E=K(x)$ for some $x$. We claim that the map
    \begin{align*}
    \Psi\colon \{F:K\subseteq F\subseteq E\}&\to\{g\in K[X]:g\text{ is a monic divisor of $f(x,K)$}\},\\
    F&\mapsto f(x,F),
    \end{align*}
    is injective. 
    Take $F_0$ such that $K\subseteq F_0\subseteq F\subseteq E$ and  
    $f(x,F)=f(x,F_0)$. Then  
%    Let $\Psi(F)=g\in F[X]$ and write $g=\sum_{i=0}^ma_iX^i$, where $m=\deg g$. 
%    Thus $a_m=1$. Let $F_0=K(a_0,\dots,a_m)$. Then $F_0\subseteq F$. Since $g=f(x,F)$, the polynomial $g$ is irreducible 
%    in $F[X]$ and hence it is irreducible in $F_0[X]$. Now
    \[
    [E:F_0]=[F_0(x):F_0]=\deg f(x,F_0)=m=[F(x):F]=[E:F]
    \]
    and hence $F=F_0$.
    
    In general, let  $F_1$ and $F_2$ be such that 
    $K\subseteq F_1,F_2\subseteq E$ and $f(x,F_1)=f(x,F_2)$.
    Let $F_0=F_1\cap F_2$. 
    Then $f=f(x,F_1)=f(x,F_2)\in F_0[X]$ and hence $f(x,F_0)=f$.
    Hence we can apply what we proved before to $F_0\subseteq F_1$
    and $F_0\subseteq F_2$, to obtain that $F_1=F_0=F_2$.
    It follows that $\Psi$ is injective 
    and hence there are finitely many fields between $K$ and $E$. 
    
    Let us prove $\impliedby$.  
    Let us assume that $E=K(x,y)$. For each $a\in K$, we consider
    the extension $K(ay+x)/K$. By assumption, there exist $a,b\in K$ such that
    $a\ne b$ and 
    \[
    K(x+ay)=K(x+by)=L.
    \]
    We claim that $L=E$. Note that 
    $x+ay\in L$ and $x+by\in L$, so $(a-b)y\in L$ and hence, since $K\subseteq L$, it follows that
    $y\in L$. Thus $x\in L$ and therefore $L=E$. 
\end{proof}

As a consequence, if $E/K$ is finite and separable, then $E/K$ admits
finitely many subextensions. 

\subsection{Galois extensions}

Let $E/K$ be an algebraic extension. Assume that $E=K(S)$ and
let $C$ be an algebraic closure of $K$ containing $E$. Let 
\[
T=\{y\in C:y\text{ is a root of $f(x,K)$ for $x\in S$}\}
\]
and let $L=K(T)$. Then $E\subseteq L$, as $S\subseteq T$. The extension
$L/K$ is normal, as $L/K$ is a decomposition field of the family $\{f(x,K):x\in S\}$. 
Moreover, $L$ is the smallest normal extension of $K$ containing $E$. The field
$L$ is the \textbf{normal closure} of $E$ (with respect to $C$). 

\begin{exercise}
If $E/K$ is finite, then $L/K$ is finite
\end{exercise}

\begin{exercise}
If $E/K$ is separable, then $L/K$ is separable.
\end{exercise}

Let $E/K$ be an extension and $S\subseteq\Gal(E/K)$ be a subset. 
the set 
\[
    \prescript{S}{}{E}=\{x\in E:\sigma(x)=x\text{ for all $\sigma\in S$}\}
\]
is a subfield of $E$ that contains $K$. The subfield $\prescript{S}{}{E}$
is known as the \textbf{fixed field} of $S$. 

\begin{definition}
    \index{Extension!Galois}
    Let $E/K$ be an algebraic extension and $G=\Gal(E/K$). 
    Then $E/K$ is a \textbf{Galois extension} if $\prescript{G}{}{E}=K$. 
\end{definition}

Clearly, $K/K$ is a Galois extension. 
Note that $\Q(\sqrt[3]{2})/\Q$ is not a Galois extension. Why?

\begin{exercise}
    Prove that $\Q(\sqrt{2},\sqrt{3})/\Q$ is a Galois extension. 
\end{exercise}

\begin{exercise}
If the characteristic of $K$ is different from two, 
then every quadratic extension of $K$ is a Galois extension. 
\end{exercise}

\begin{exercise}
    Let $E/K$ be an algebraic extension and $G=\Gal(E/K)$. Let
    $F=\prescript{G}{}{E}$. Prove that $\Gal(E/F)=G$ and hence $E/F$ is a Galois extension. 
\end{exercise}

\begin{proposition}
\label{pro:normal+separable}
    Let $E/K$ be an algebraic extension. Then $E/K$ is a Galois extension
    if and only if $E/K$ is normal and separable. 
\end{proposition}

\begin{proof}
    Let $G=\Gal(E/K)$. Let us first assume that $E/K$ is Galois. For $x\in E$ let 
    \[
    f_x=\prod_{y\in O_G(x)}(X-y)=\sum a_iX^i\in E[X].
    \]
    If $\varphi\in G$, then 
    \[
    \overline{\varphi}(f_x)=\prod_{y\in O_G(x)}(X-\varphi(y))=f_x,
    \]
    as if $O_G(x)=\{\sigma_1(x),\dots,\sigma_r(x)\}$, then 
    $\varphi(\sigma_i(x))=(\varphi\sigma_i)(x)=\sigma_j(x)$ for some $j$. 
    Since 
    \[
    \sum a_iX^i=f_x=\overline{\varphi}(f_x)=\sum\varphi(a_i)X^i,
    \]
    it follows that $a_i\in\prescript{G}{}{E}=K$ for all $i$. 
    Thus $f_x\in K[X]$
    and $E/K$ is a decomposition field of the family $\{f_x:x\in E\}$. In particular, 
    $E/K$ is normal. Moreover, $x$ is a simple root of $f_x\in K[X]$ and hence
    $x$ is separable over $K$. 

    Conversely, let $x\in \prescript{G}{}{E}$. Since $E/K$ is normal, then 
    $f(x,K)=\prod_{y\in O_G(x)}(X-y)^m$ for some $m$.
    Since $E/K$ is separable, 
    $m=1$.
    Moreover $x\in \prescript{G}{}{E}$, so $O_G(x)=\{x\}$.
    Thus $f(x,K)=\prod_{y\in O_G(x)}(X-y)=X-x$ and $x\in K$. 
\end{proof}

\begin{definition}
Let $K$ be a field and $f\in K[X]$. Then $f$ is \textbf{separable}
if all roots of $f$ are simple (in some algebraic closure of $K$). 
\end{definition}

\begin{proposition}
    Let $E/K$ be a finite extension. Then $E/K$ is a Galois extension 
    if and only if $E$ is a decomposition field over $K$ 
    of a separable polynomial $f\in K[X]$. 
\end{proposition}

\begin{proof}
    Let us assume first that $E/K$ is a Galois extension. Since
    $E/K$ is finite and separable, $E=K(x)$ by Proposition \ref{pro:monogenic}. 
    Then $E/K$ is a decomposition field of $f(x,K)$ since
    $E/K$ is normal. Since $E/K$ is separable, $x$ is separable over $K$. Thus $x$ is 
    a simple root of $f(x,K)$ and hence $f(x,K)$ is separable. 
    Conversely, let $x_1,\dots,x_r$ be the roots of a separable polynomial $f\in K[X]$.
    Then $E=K(x_1,\dots,x_r)$ is separable and normal.  
\end{proof}

In the previous case, $\Gal(E/K)$ is known as the \textbf{Galois group}
of the polynomial $f$. The notation is $\Gal(f,K)$. If $n=\deg f$ and
$x_1,\dots,x_n$ are the roots of $f$, then any 
$\varphi\in\Gal(f,K)$ permutes the roots of $f$, that is
$\varphi$ permutes the 
set $\{x_1,\dots,x_n\}$. In particular, $\Gal(f,K)$ is isomorphic to a subgroup of
$\Sym_n$ and hence $|\Gal(f,K)|$ divides $n!$. 

\begin{proposition}
    Let $E/K$ be a normal extension and $F$ be the separable
    closure of $K$ with respect to $E$. 
    Then $F/K$ is a Galois extension.
\end{proposition}

\begin{proof} 
    Let $C/K$ be an algebraic closure such that $E\subseteq C$. Let $\sigma\in\Hom(F/K,C/K)$. 
    and let $\varphi\in\Hom(E/K,C/K)$ be such that $\varphi|_F=\sigma$. Since $E/K$ is normal, 
    $\varphi(E)=E$. Let $x\in F$. Then $\sigma(x)=\varphi(x)\in E$. Thus
    $f(\sigma(x),K)=f(x,K)$ and $\sigma(x)$ is separable over $K$, which 
    implies that $\sigma(x)\in F$. Thus $F/K$ is normal. Since $F/K$ is separable, it follows
    that $F/K$ is a Galois extension by Proposition \ref{pro:normal+separable}.
\end{proof}

Some easy facts.

\begin{exercise}
    Let $E/K$ be a separable extension and $L/K$ be the normal 
    closure of $E$ in some algebraic closure $C$
    that contains $E$. Prove that $L/K$ is a Galois extension.
\end{exercise}

\begin{exercise}
    Let $E/K$ be a finite extension. Prove that $E/K$ is Galois
    if and only if $[E:K]=|\Gal(E/K)|$.
\end{exercise}

For the previous exercise, 
note that if $E/K$ is a finite extension, then  
\[
|\Gal(E/K)|\leq\gamma(E/K)\leq [E:K].
\]
The first inequality
is equality if and only if $E/K$ is normal. The second
inequality is equality if and only if $E/K$ is separable.

\begin{exercise}
    Let $E/K$ be a Galois extension and $F/K$ be a subextension of $E/K$. 
    Prove that $E/F$ is a Galois extension. 
\end{exercise}


\begin{theorem}[Artin]
\index{Artin's theorem}
\label{thm:ArtinGalois}
    Let $E$ be a field and $G$ be a finite group of automorphisms of $E$. 
    If $K=\prescript{G}{}{E}$, then $E/K$ is a Galois extension,
    $[E:K]=|G|$ and $\Gal(E/K)=G$. 
\end{theorem}

Before proving the theorem, we need a lemma.

\begin{lemma}
    Let $E/K$ be a separable extension such that $\deg f(x,K)\leq m$
    for all $x\in E$. Then $E/K$ is finite and $[E:K]\leq m$. 
\end{lemma}

\begin{proof}
   Let $z\in E$ be of maximal degree. If $x\in E$, 
   then $K(x,z)/K$ is separable. Then $K(x,z)=K(y)$ for some $y$. 
   It follows that 
   \[
   K(z)\subseteq K(x,z)=K(y).
   \]
   Since 
   $\deg f(z,K)\leq\deg f(y,K)$, 
   $\deg f(z,K)=\deg f(y,K)$. Hence 
   $K(y)=K(z)$. In particular, $x\in K(z)$ and
   therefore $E=K(z)$. 
\end{proof}

Now we are ready to prove Artin's theorem: 

\begin{proof}[Proof of Theorem \ref{thm:ArtinGalois}]
    Note that $G\subseteq\Gal(E/K)$. Let $x\in E$ and 
    \[
    f_x=\prod_{y\in O_G(x)}(X-y).
    \]
    Since $f_x\in K[X]$, the extension $E/K$ is normal and separable (as it is a decomposition
    field of a family of separable polynomials), so $E/K$ is a Galois extension. Moreover, 
    \[
    \deg f(x,K)\leq \deg f_x=|O_G(x)|\leq |G|.
    \]
    By the previous lemma, $E/K$ is finite and $[E:K]\leq |G|$. This
    implies that
    $|\Gal(E/K)|=[E:K]\leq |G|$ and hence $|\Gal(E/K)|=|G|$. 
\end{proof}

\begin{example}
    Let $E=K(X,Y)$ and $\sigma\colon K[X,Y]\to E$ be the ring homomorphism given by $\sigma(X)=Y$ and $\sigma(Y)=X$. Note that $\sigma$ is bijective, as $\sigma^2=\id$. The map $\sigma$ induces
    a field homomorphism $\overline{\sigma}\colon E\to E$ such that 
    $\overline{\sigma}^2=\id$. Recall that such a homomorphism is given by 
    $f/g\mapsto \sigma(f)/\sigma(g)$. Let $G=\langle\overline{\sigma}\rangle$. Then $|G|=2$. 
    We claim that $\prescript{G}{}{E}=K(X+Y,XY)$. Let $F=K(X+Y,XY)$. We only prove
    that $\prescript{G}{}{E}\subseteq F$, as the other inclusion is trivial. Artin's theorem
    implies that $[E:\prescript{G}{}{E}]=2$ and $E=F(X)$, as $X$ is a root
    of the polynomial $Z^2-(X+Y)Z+XY$. Then $[E:F]\leq 2$ and $[\prescript{G}{}{E}:F]=1$.
\end{example}

%\chapter{}



\begin{theorem}[Artin]
\index{Artin's theorem}
\label{thm:ArtinGalois}
    Let $E$ be a field and $G$ be a finite group of automorphisms of $E$. 
    If $K=\prescript{G}{}{E}$, then $E/K$ is a Galois extension,
    $[E:K]=|G|$ and $\Gal(E/K)=G$. 
\end{theorem}

Before proving the theorem, we need a lemma.

\begin{lemma}
    Let $E/K$ be a separable extension such that $\deg(x,K)\leq m$
    for all $x\in E$. Then $E/K$ is finite and $[E:K]\leq m$. 
\end{lemma}

\begin{proof}
   Let $z\in E$ be of maximal degree. If $x\in E$, 
   then $K(x,z)/K$ is separable. Then $K(x,z)=K(y)$ for some $y$. 
   It follows that 
   \[
   K(z)\subseteq K(x,z)=K(y).
   \]
   Since 
   $\deg(z,K)\leq\deg(y,K)$, 
   $\deg(z,K)=\deg(y,K)$. Hence 
   $K(y)=K(z)$. In particular, $x\in K(z)$ and
   therefore $E=K(z)$. 
\end{proof}

Now we are ready to prove Artin's theorem: 

\begin{proof}[Proof of Theorem \ref{thm:ArtinGalois}]
    Note that $G\subseteq\Gal(E/K)$. Let $x\in E$ and 
    \[
    f_x=\prod_{y\in O_G(x)}(X-y).
    \]
    Since $f_x\in K[X]$, it follows
    that the extension $E/K$ is normal and separable (as it is a decomposition
    field of a family of separable polynomials), so $E/K$ is a Galois extension. Moreover, 
    \[
    \deg(x,K)\leq \deg f_x=|O_G(x)|\leq |G|.
    \]
    By the previous lemma, $E/K$ is finite and $[E:K]\leq |G|$. This
    implies that
    $|G(E/K)|=[E:K]\leq |G|$ and hence $|G(E/K)|=|G|$. 
\end{proof}

\begin{example}
    Let $E=K(X,Y)$ and $\sigma\colon K[X,Y]\to E$ be the ring homomorphism given by $\sigma(X)=Y$ and $\sigma(Y)=X$. Note that $\sigma$ is bijective, as $\sigma^2=\id$. The map $\sigma$ induces
    a field homomorphism $\overline{\sigma}\colon E\to E$ such that 
    $\overline{\sigma}^2=\id$. Recall that such a homomorphism is given by 
    $f/g\mapsto \sigma(f)/\sigma(g)$. Let $G=\langle\overline{\sigma}\rangle$. Then $|G|=2$. 
    We claim that $\prescript{G}{}{E}=K(X+Y,XY)$. Let $F=K(X+Y,XY)$. We only prove
    that $\prescript{G}{}{E}\subseteq F$, as the other inclusion is trivial. Artin's theorem
    implies that $[E:\prescript{G}{}{E}]=2$ and $E=F(X)$, as $X$ is a root
    of the polynomial $Z^2-(X+Y)Z+XY$. Then $[E:F]\leq 2$ and $[\prescript{G}{}{E}:F]=1$.
\end{example}

\topic{Galois' correspondence}

\begin{theorem}[Galois]
\index{Galois' theorem}
    Let $E/K$ be a finite Galois extension and $G=\Gal(E/K)$. 
    There exists a bijective correspondence
    \[
    \{F:K\subseteq F\subseteq E\text{ subfields}\}\leftrightarrow
    \{\text{subgroups of $G$}\}
    \]
    The correspondence is given by $F\mapsto G(E/F)$ and 
    $\prescript{S}{}{E}\mapsfrom S$. Moreover, 
    normal subextensions of $E/K$ correspond 
    to normal subgroups of $G$. 
\end{theorem}

\[
\begin{tikzcd}
	&& E \\
	& F && {\{1\}} \\
	K && S \\
	& G
	\arrow[no head, from=1-3, to=2-2]
	\arrow[no head, from=2-2, to=3-1]
	\arrow[no head, from=3-1, to=4-2]
	\arrow[no head, from=4-2, to=3-3]
	\arrow[no head, from=2-2, to=3-3]
	\arrow[no head, from=1-3, to=2-4]
	\arrow[no head, from=2-4, to=3-3]
\end{tikzcd}
\]

\begin{proof}
Let $\alpha$ and $\beta$ be the maps $\alpha(F)=\Gal(E/F)$ and $\beta(S)=\prescript{S}{}{E}$. A routine
exercise shows that $\alpha$ and $\beta$ are well-defined. 
We first note that
\begin{align*}
   &\beta(\alpha(F))=\beta(\Gal(E/F))=\prescript{\Gal(E/F)}{}{E}=F
\end{align*}
since $E/F$ is a Galois extension. Moreover,
\begin{align*}
   &\alpha(\beta(S))=\alpha(\prescript{S}{}{E})=\Gal(E/\prescript{S}{}{E})=S
\end{align*}
by Artin's theorem, as $S$ is finite. 

Let $F$ be a subfield of $E$ containing $K$ and 
$S=\alpha(F)$. Then
\[
[F:K]=\frac{[E:K]}{[E:F]}=\frac{|G|}{|S|}=(G:S).
\]

Let $C$ be an algebraic closure of $K$ that contains $E$. 
If $S=\Gal(E/F)$, then $F=\prescript{S}{}{E}$. 

We need to prove that $F/K$ is normal if and only if $S$ is normal in $G$. 
Let us first prove $\implies$. Let $\tau\in S$ and $\sigma\in G$. Since
$F/K$ is normal, $\sigma|_F\in\Aut(F)$. Thus $\sigma^{-1}(F)=F$. In particular, 
if $x\in F$, then $\sigma^{-1}(x)\in F$ and 
\[
\sigma\tau\sigma^{-1}(x)=\sigma\sigma^{-1}(x)=x.
\]
Conversely, let $\varphi\in\Hom(F/K,C/K)$. There exists 
$\Phi\colon E\to C$ such that $\Phi|_F=\varphi$. Since $E/K$ is normal, 
$\Phi(E)=E$ and hence $\Phi\in G$. We claim that $\varphi(x)\in F$ for all $x\in F$. 
Note that $F=\prescript{S}{}{E}$, so
 \[
 \tau\varphi(x)=\tau\Phi(x)=\Phi\Phi^{-1}\tau\Phi(x)=\Phi(x)=\varphi(x)
 \]
 for all $\tau\in S$, as $\Phi^{-1}\tau\Phi\in S$. This means that $\varphi(x)\in\prescript{S}{}{E}=F$. 
 
 Let us compute $\Gal(F/K)$. Since $F/K$ is normal, 
 the map 
 $\lambda\colon G\to\Gal(F/K)$, $\sigma\mapsto\sigma|_F$, 
 is a surjective group homomorphism such that $\ker\lambda=S$. The first isomorphism 
 theorem implies that $\Gal(F/K)\simeq G/S$. 
\end{proof}

Some easy consequences.

\begin{exercise}
    If $E/K$ is a Galois extension of degree $n$ and
    $p$ is a prime number dividing $n$, then $E/K$ admits
    a subextension of degree $n/p$. 
\end{exercise}
    
\begin{exercise}
    If $E/K$ is a Galois extension of degree $p^\alpha m$ with
    $p$ a prime number coprime with $m$, then $E/K$ admits 
    a subextension of degree $m$. 
    %This follows from Sylow's theorem
    %and Galois's theorem.
\end{exercise}

\begin{definition}
\index{Extension!abelian}
    An extension $E/K$ is \textbf{abelian} if $E/K$ is a Galois extension
    with $\Gal(E/K)$ abelian.
\end{definition}

\begin{exercise}
    If $E/K$ is an abelian extension of degree $n$ and $d$ divides
    $n$, then $E/K$ admits a subextension of degree $d$. 
\end{exercise}

\begin{definition}
    \index{Extension!cyclic}
    An extension $E/K$ is \textbf{cyclic} if $E/K$ is 
    a Galois extension with $\Gal(E/K)$ cyclic. 
\end{definition}

\begin{example}
    The extension $\Q(\sqrt{2},\sqrt{3})/\Q$ admits
    exactly three non-trivial subextensions: 
    \[
    \Q(\sqrt{2})/\Q,
    \quad
    \Q(\sqrt{3})/\Q,
    \quad 
    \Q(\sqrt{6})/\Q,
    \]
    as $\Gal(\Q(\sqrt{2},\sqrt{3})/Q)\simeq C_2\times C_2$. 
    % Note that if $\sigma\in\Gal(\Q(\sqrt{2},\sqrt{3})/Q)$, then
    % $\sigma(\sqrt{2})\in\{\sqrt{2},-\sqrt{2}\}$ and
    % $\sigma(\sqrt{3})\in\{\sqrt{3},-\sqrt{3}\}$.
\end{example}

\begin{example}
    Let $\omega\in\C\setminus\{1\}$ be such that $\omega^5=1$.
    Then 
    \[
    f(\omega,\Q)=1+X+X^2+X^3+X^4
    \]
    and $\Q(\omega)/\Q$ has
    degree four. 
    Moreover, $\Q(\omega)/\Q$ is a Galois extension
    and 
    $\Gal(\Q(\omega)/\Q)\simeq C_4$. If $\sigma\in \Gal(\Q(\omega)/\Q)$,
    then $\sigma(\omega)=\omega^i$ for some $i\in\{1,\dots,4\}$. 
    Moreover, for every $i\in\{1,\dots,4\}$ 
    the map $\omega\mapsto\omega^i$ induces an automorphism
    of $\Q(\omega)/\Q$. Thus $|\Gal(\Q(\omega)/\Q)|=4$. Now 
    \[
    \sigma_i^k=\id\Longleftrightarrow
    \omega^{i^k}=\sigma_i^k(\omega)=\omega\Longleftrightarrow
    i^k\equiv1\bmod 5.
    \]
    Thus the map $\sigma_2$ given 
    by $\omega\mapsto\omega^2$ has order four. 
    
    Since $\Gal(\Q(\omega)/\Q)=\langle\sigma\rangle$,
    where $\sigma(\omega)=\omega^2$, 
    is cyclic of order four, 
    the extension $\Q(\omega)/\Q$ has a unique degree-two 
    subtextension $F/\Q$. Note that $|\langle\sigma^2\rangle|=2$ 
    and $\sigma^2(\omega)=\omega^4=\omega^{-1}$. Thus 
    $F=\prescript{\langle\sigma^2\rangle}{}{\Q(\omega)}$. Let 
    $\theta=\omega+\omega^{-1}$. Then 
    \[
    \theta^2=\omega^2+\omega^3+2=-(1+\omega+\omega^{-1})+2=1-\theta
    \]
    and hence $\theta$ is a root of $X^2+X-1$. Since $\theta\not\in\Q$, 
    it follows that 
    \[
    \theta\in\{(-1+\sqrt{5})/2,(-1-\sqrt{5})/2\}.
    \]
    Therefore
    $F=\Q(\sqrt{5})$. 
\end{example}

Let us mention some other consequences.

\begin{exercise}
    Let $E/K$ be a finite Galois extension 
    and $F_1,\dots,F_n$ fields 
    such that $K\subseteq F_i\subseteq E$ for 
    all $i\in\{1,\dots,n\}$. For every 
    $i$ let $S_i=\Gal(E/F_i)$. Then
    \[
    \Gal\left(E/\bigcap_{i=1}^nF_i\right)=\left\langle\bigcup_{i=1}^nS_i\right\rangle,
    \quad
    \Gal\left(E/\prod_{i=1}^nF_i\right)=\bigcap_{i=1}^nS_i.
    \]
\end{exercise}

The following statement is a concrete application of the 
previous exercise.

\begin{exercise}
    Let $E/K$ be a finite Galois extension and $G=\Gal(E/K)$.
    Assume that $G$ is the direct product
    $G=S\times T$
    of the groups $S$ and $T$. Let 
    $F=\prescript{S}{}{E}$ and
    $L=\prescript{T}{}{E}$. Then $F\cap L=K$ and $FL=E$.
\end{exercise}

\begin{proposition}
Let $E_1/K,\dots,E_r/K$ be Galois extensions. 
If $E=\prod_{i=1}^rE_i$, then $E/K$ is a Galois extension. If, moreover, each $E_i/K$ is finite,
then 
\[
\theta\colon \Gal(E/K)\to \Gal(E_1/K)\times\cdots\times\Gal(E_r/K),
\quad
\sigma\mapsto(\sigma|_{E_1},\dots,\sigma|_{E_r}),
\]
is an injective group homomorphism.
\end{proposition}

\begin{proof}
    We only do the first part in the case $r=2$, the general case is left as an exercise. Since $E_1/K$ is algebraic, 
    then $E_1E_2/E_2$ is algebraic. Since $E_2/K$ is algebraic, $E_1E_2/K$ is algebraic. Similarly, 
    $E_1E_2/K$ is separable. 
    
    Let $C/K$ be an algebraic closure such that $E_1E_2\subseteq C$. If $\sigma\in\Hom(E_1E_2/K,C/K)$, then 
    $\sigma(E_1E_2)\subseteq\sigma(E_1)\sigma(E_2)=E_1E_2$ (do this calculation as an exercise). 
    Thus $E_1E_2/K$ is normal. 
    
    If both $E_1/K$ and $E_2/K$ are finite, then $E_1E_2/K$ is finite. 
    
    Clearly, $\theta$ is a group homomorphism. We claim that the map $\theta$ is injective. Let $\sigma\in\ker\theta$. Then
    $\sigma|_{E_i}=\id_{E_i}$ for all $i\in\{1,\dots,r\}$. Let $S=\langle\sigma\rangle\subseteq\Gal(E/K)$ and
    $F=\prescript{S}{}{E}$. Then $E_i\subseteq F$ for all $i\in\{1,\dots,r\}$ and
    hence $E\subseteq F$. It follows that $F=E=\prescript{\{\id\}}{}{E}$ and therefore $S=\{\id\}$, so 
    $\sigma=\id$. 
\end{proof}

\begin{exercise}
    Let $E_1/K,\dots,E_r/K$ be finite Galois extensions such that for each $j$ 
    one has $E_j\cap (E_1\cdots E_{j-1}E_{j+1}\cdots E_r)=K$. Then 
    \[
    \Gal(E/K)\simeq\Gal(E_1/K)\times\cdots\times\Gal(E_r/K).
    \]
    In this case, $[E:K]=\prod_{i=1}^r[E_i:K]$. 
\end{exercise}


%\section{Lecture -- Week 9}

\subsection{Norm and trace}

\begin{definition}
\index{Trace}
\index{Norm}
    Let $E/K$ be a finite extension and $C/K$ be an algebraic closure 
    that contains $E$. Let $A=\Hom(E/K,C/K)$. For $x\in E$
    we define the \emph{trace} of $x$ in $E/K$ 
    as 
    \[
    \trace_{E/K}(x)=[E:K]_{\operatorname{ins}}\sum_{\varphi\in A}\varphi(x)
    \]
    and the \emph{norm} of $x$ in $E/K$ as
    \[
    \norm_{E/K}(x)=\left(\prod_{\varphi\in A}\varphi(x)\right)^{[E:K]_{\operatorname{ins}}}.
    \]
\end{definition}

As an optional exercise, one can show that these definitions do not depend on the algebraic closure. 

We collect some basic properties as an exercise:

\begin{exercise}
\label{xca:norm_and_trace}
    Let $E/K$ be a finite extension. The following statements hold:
    \begin{enumerate}
        \item If $E/K$ is not separable, then $\trace_{E/K}(x)=0$ for all $x\in E$.
        \item If $x\in K$, then $\trace_{E/K}(x)=[E:K]x$.
        \item $\trace_{E/K}(x)\in K$ for all $x\in E$.
        \item $\norm_{E/K}(x)=0$ if and only if $x=0$. 
        \item If $x\in K$, then $\norm_{E/K}(x)=x^{[E:K]}$. 
        \item $\norm_{E/K}(x)\in K$ for all $x\in E$. 
    \end{enumerate}
\end{exercise}

One proves, moreover, that  
$\trace_{E/K}\colon E\to K$ 
satisfies
\[
\trace_{E/K}(x+\lambda y)=
\trace_{E/K}(x)+\lambda\trace_{E/K}(y)
\]
for all $x,y\in E$ and $\lambda\in K$, that is to say that 
$\trace_{E/K}\colon E\to K$ 
is a 
linear form in $E$ The norm  
$\norm_{E/K}\colon E^{\times}\to K^{\times}$ 
is a group homomorphism. 

\begin{exercise}
        Let $E/K$ be a finite extension and
        $x\in E$ be algebraic. If
        \[
        f(x,K)=X^n+a_{n-1}X^{n-1}+\cdots+a_1X+a_0,
        \]
        then 
        $\norm_{E/K}(x)=\left((-1)^na_0\right)^{[E:K(x)]}$ and 
        $\trace_{E/K}(x)=-[E:K(x)]a_{n-1}$. 
\end{exercise}

\begin{example}
    Let $E=\Q(\sqrt{2},\sqrt{3})$. Then 
    \begin{align*}
    &\trace_{E/\Q}(\sqrt{2})=0,
    &&
    \norm_{E/\Q}(\sqrt{2})=4,\\
    &\trace_{E/\Q(\sqrt{2})}(\sqrt{2})=2\sqrt{2},
    &&\norm_{E/\Q(\sqrt{2})}(\sqrt{2})=2.    
    \end{align*}
\end{example}

\begin{example}
    If $E/K$ is a finite Galois extension, then 
    \[
    \trace_{E/K}(x)=\sum_{\sigma\in\Gal(E/K)}\sigma(x)
    \quad\text{and}\quad
    \norm_{E/K}(x)=\prod_{\sigma\in\Gal(E/K)}\sigma(x)
    \]
    for all $x\in E$. In particular, since $E=K(y)$ for some
    $y$ by Proposition \ref{pro:monogenic}, 
    \[
    \trace_{E/K}(y)=-a_{n-1}
    \quad\text{and}\quad
    \norm_{E/K}(y)=(-1)^na_0,
    \]
    where
    $f(y,K)=X^n+a_{n-1}X^{n-1}+\cdots+a_1X+a_0$.
\end{example}        

\subsection{Finite fields}

In this section, $p$ will be a prime number. 

\begin{proposition}
    Let $m$ be a positive integer. 
    Up to isomorphism, there exists a unique 
    field $F_{p^m}$ of size $p^m$. 
\end{proposition}

\begin{proof}
    Let $C$ be an algebraic closure of the field $\Z/p$ and 
    let $F_{p^m}=\{x\in C:x^{p^m}=x\}$ be the set of roots of $X^{p^m}-X$. Since 
    the polynomial $X^{p^m}-X$ has no multiple roots, $|F_{p^m}|=p^m$. Moreover, 
    $F_{p^m}$ is the unique subfield of $C$ of size $p^m$. 
    
    To prove the uniqueness, it is enough to note that 
    if $K$ is a field of $p^m$ elements, then
    $K$ is the splitting field of $X^{p^m}-X$ over $\Z/p$.  
\end{proof}

Let $K=\Z/p$ and $C$ be an algebraic closure of $K$. 
We claim that $C=\bigcup_k F_{p^k}$. If $x\in C$, then $x$ is algebraic over $K$. 
Since $K(x)/K$ is finite, $K(x)$ is a finite field, say 
$|K(x)|=p^r$ for some $r$. Then $x^{p^r}=x$ and hence $x\in F_{p^r}$. 

\begin{exercise}
    Prove the following statements:
    \begin{enumerate}
        \item If $x\in F_{p^r}$, then $x^{p^{rk}}=x$ for all $k\geq0$.
        %\item If $m\mid n$, then $F_{p^m}\subseteq F_{p^n}$. 
        \item $F_{p^m}\subseteq F_{p^n}$ if and only if $m\mid n$. 
        \item $F_{p^m}\cap F_{p^n}=F_{p^{\gcd(m,n)}}$.
    \end{enumerate}
\end{exercise}

\begin{proposition}
    Every finite extension of a finite field is cyclic. 
\end{proposition}

\begin{proof}
    Let $K=\Z/p$. It is enough to show that $F_{p^n}/F_{p^m}$ is cyclic if $m$ divides $n$. 
    
    We first prove that $F_{p^n}/K$ is cyclic. 
    Let 
    \[
    \sigma\colon F_{p^n}\to F_{p^n},\quad 
    x\mapsto x^p.
    \]
    Then 
    $\sigma\in\Gal(F_{p^n}/K)$ (it is bijective because all field homomorphisms 
    are injective and $F_{p^n}$ is finite). 

    Note that 
    $F_{p^n}/K$ is a Galois extension, as $F_{p^n}$ is the splitting
    field over $K$ 
    of the separable polynomial $X^{p^n}-X\in K[X]$. 
    Thus $|\Gal(F_{p^n}/K)|=[F_{p^n}:K]=n$. 
    
    We claim that $\sigma$ generated $\Gal(F_{p^n}/K)$. Since 
    $\sigma^i(x)=x^{p^i}$ for all $i\geq 0$, in particular, 
    \[
    \sigma^n(x)=x^{p^n}=x.
    \]
    Thus $\sigma^n=\id$ and hence $|\sigma|$ divides $n$. Let 
    $s=|\sigma|$. We know that $F_{p^n}^{\times}=F_{p^n}\setminus\{0\}$ is
    cyclic, say $F_{p^n}^{\times}=\langle g\rangle$. Since $|g|=p^n-1$, 
    \[
    g=\sigma^s(g)=g^{p^s}
    \]
    and hence $p^s\equiv 1\bmod (p^n-1)$. Thus $p^n-1$ divides $p^s-1$ and
    hence $n$ divides $s$. Therefore $n=s$ and $\Gal(F_{p^n}/K)=\langle\sigma\rangle$. 
    
    For the general case, note that if $m$ divides $n$, 
    then the Galois group 
    $\Gal(F_{p^n}/F_{p^m})$ is a subgroup of $\Gal(F_{p^n}/K)$. Since  $\Gal(F_{p^n}/K)$ is cyclic, 
    the claim follows.
\end{proof}

\index{Frobenius automorphism}
If $K=\Z/p$ and 
$m$ divides $n$, the subextension $F_{p^m}$ corresponds 
to the unique
subgroup of index $m$ of $\Gal(F_{p^n}/K)=\langle\sigma\rangle$. This subgroup
is $\langle\sigma^m\rangle$, where
\[
\sigma^m(x)=x^{p^m}=x^{|F_{p^m}|}.
\]
Note that $\Gal(F_{p^n}/F_{p^m})=\langle\sigma^m\rangle$. 
The map $\sigma^m$ is known as 
the \emph{Frobenius automorphism}. 

\begin{exercise}
    Let $E/K$ be an extension of finite fields. Then $E/K$ 
    is cyclic. Moreover, $\Gal(E/K)=\langle\tau\rangle$, where $\tau(x)=x^{|K|}$. 
\end{exercise}

% page 96
% number of irreducible polynomials
% Moebius inversion formula in commutative rings



\subsection{Cyclotomic extensions}

For $n\geq1$ let $G_n(K)=\{x\in K:x^n=1\}$ be the 
set of $n$-roots of one in $K$. Note that
$G_n(K)$ is a cyclic subgroup of $K^{\times}$ and that 
$|G_n(K)|$ divides $n$. 

\begin{example}
    $G_n(\R)=\{-1,1\}$ if $n$ is odd and $G_{n}(\R)=\{1\}$ if $n$ is even.
\end{example}

\begin{exercise}
    Let $K$ be a field of characteristic $p>0$. Let $n=p^sm$ for some $m$ not divisible by $p$. 
    Then $G_n(K)=G_m(K)$. 
\end{exercise}

\begin{exercise}
    Let $q$ be a prime number. Then $G_n(\Z/q)\simeq\Z/\gcd(n,q-1)$. 
\end{exercise}

Similarly, one can prove that if $K$ is a finite field, then $G_n(K)$ is a cyclic group
of order $\gcd(n,|K^{\times}|)$. 

\begin{example}
    If $C$ is algebraically closed of characteristic coprime with $n$, 
    then $G_n(C)$ is cyclic of order $n$, as $X^n-1$ 
    has all its roots in $C$ and does not contain multiple roots. 
\end{example}

Let $K$ be an algebraically closed field and $n$ be
such that $n$ is coprime with the characteristic of $K$. The set of 
\emph{primitive $n$-roots} is defined as 
\[
H_n(K)=\{x\in G_n(K):|x|=n\}.
\]

\begin{definition}
\index{Cyclotomic polynomial}
    Let $K$ be an algebraically closed field and $n$ be
    such that $n$ is coprime with the characteristic of $K$. The \emph{$n$-th cyclotomic
    polynomial} is defined as 
    \[
    \Phi_n=\prod_{x\in H_n(K)}(X-x)\in K[X].
    \]
\end{definition}

\index{Euler's $\phi$ function}
For $n\geq1$ the Euler's function is defined as 
\[
\varphi(n)=|\{k:1\leq k\leq n,\;\gcd(k,n)=1\}|.
\]
For example, $\varphi(4)=2$, $\varphi(8)=\varphi(10)=4$ and $\varphi(p)=p-1$ for every prime $p$. 

\begin{proposition}
    Let $K$ be an algebraically closed field and $n$ be
    such that $n$ is coprime with the characteristic of $K$. Let $A$ be
    the prime subring of $K$. 
    \begin{enumerate}
        \item $\deg\Phi_n=\varphi(n)$.
        \item $\Phi_n\in A[X]$.
    \end{enumerate}
\end{proposition}

\begin{proof}
    The first statement is clear. Let us prove 2) by induction on $n$. The case $n=1$ is
    trivial, as $\Phi_1=X-1$. Assume that $\Phi_d\in A[X]$ for all $d$ such that $d<n$. 
    In particular,
    \[
    \gamma=\prod_{\substack{d\mid n\\d\ne n}}\Phi_d\in A[X].
    \]
    Since $\gamma$ is monic, it follows that 
    $\frac{X^n-1}{\gamma}\in A[X]$. Now the claim follows from 
    \[
    X^n-1=\prod_{d\mid n}\Phi_d=\Phi_n\prod_{\substack{d\mid n\\d\ne n}}\Phi_d=\Phi_n\gamma.\qedhere
    \]
\end{proof}

By taking degree in the equality 
$X^n-1=\prod_{d\mid n}\Phi_d$ 
one gets 
\[
n=\sum_{d\mid n}\varphi(d).
\]

\begin{definition}
\label{defn:cyclotomic}
\index{Extension!cyclotomic}
    Let $n\geq2$ and $K$ be a field of characteristic coprime with $n$. A 
    \emph{cyclotomic extension} of $K$ of index $n$ is a 
    splitting field of $X^n-1$ over $K$. 
\end{definition}

Let $C$ be an algebraic closure of $K$ and $n\geq2$ be coprime with the characteristic of $K$. 
It follows from Definition \ref{defn:cyclotomic} 
that a cyclotomic extension of $K$ of index $n$ is of the form 
$K(\omega)/K$ for some $\omega\in H_n(C)$. 

\begin{proposition}
    A cyclotomic extension of index $n$ is abelian and of degree a divisor of $\varphi(n)$. 
\end{proposition}

\begin{proof}
    Let $C$ be an algebraic closure of $K$ and $n\geq2$ be coprime with the characteristic of $K$. 
    Let $\omega\in H_n(C)$ and $K(\omega)/K$ be a cyclotomic extension. Then $K(\omega)/K$
    is a Galois extension, as it is a splitting field of a separable polynomial. 
    Let $U=\mathcal{U}(\Z/n)$ be the group of units of $\Z/n$ and 
    \[
    \lambda\colon \Gal(K(\omega)/K)\to U,
    \quad
    \sigma\mapsto m_{\sigma},
    \]
    where $m_{\sigma}$ is such that $\sigma(\omega)=\omega^{m_{\sigma}}$. The map $\lambda$ is well-defined and
    it is a group homomorphism, as if $\sigma,\tau\in\Gal(K(\omega)/K)$, then, since 
    \[
        (\tau\sigma)(\omega)=\tau(\sigma(\omega))=\tau(\omega^{m_\sigma})=\left(\omega^{m_\sigma}\right)^{m_\tau}=\omega^{m_\sigma m_\tau},
    \]
    it follows that $\lambda(\sigma)\lambda(\tau)=\lambda(\sigma\tau)$. Since 
    $\lambda$ is injective, $\Gal(K(\omega)/K)$ is isomorphic to a subgroup 
    of the abelian group $U$. Hence $\Gal(K(\omega)/K)$ is abelian. Moreover, 
    \[
    [K(\omega):K]=|\Gal(K(\omega)/K)|
    \]
    is a divisor of $|U|=\varphi(n)$. 
\end{proof}

\begin{exercise}
    Prove that a cyclotomic extension $K(\omega)/K$ has degree $\varphi(n)$ if and only if 
    $\Phi_n$ is irreducible over $K$. 
\end{exercise}

Note that $\Phi_n$ is irreducible over $\Q$. Some concrete examples:
\[
\Phi_1=X-1,
\quad
\Phi_2=X+1,
\quad
\Phi_3=X^2+X+1,
\quad
\Phi_6=X^2-X+1.
\]
If $p$ is a prime number, then $\Phi_p=X^{p-1}+\cdots+X+1$. 

\begin{example}
    $\Phi_5$ is irreducible over $\Z/2$. First note that
    $\Phi_5=X^{4}+\cdots+X+1$ does not have roots in $\Z/2$. If 
    $\Phi_5$ is reducible, then, since
    $X^2+X+1$ is the unique degree-two 
    monic irreducible polynomial 
    over $\Z/2$, it follows that
    \[
    \Phi_5=(X^2+X+1)(X^2+X+1)=(X^2+X+1)^2=X^4+X^2+1,
    \]
    a contradiction.
\end{example}

\begin{exercise}
Prove that
$\Phi_{12}=X^4-X^2+1$ is not irreducible over $\Z/5$. 
\end{exercise}

\subsection{Hilbert's theorem 90}

\begin{theorem}[Hilbert]
    Let $E/K$ be a cyclic extension. Assume that 
    $\Gal(E/K)$ is generated by $\tau$. For 
    $a\in E$, $\norm_{E/K}(a)=1$ if and only 
    if $a=b/\tau(b)$ for some $b\in E\setminus\{0\}$. 
\end{theorem}

\begin{proof}
    Let $n=|\Gal(E/K)|$. We first prove $\impliedby$. If $a=b/\tau(b)$ and $b\ne 0$, then 
    \[
    \norm_{E/K}(a)=a\tau(a)\tau^2(a)\cdots\tau^{n-1}(a)
    =\frac{b}{\tau(b)}\frac{\tau(b)}{\tau^2(b)}\cdots\frac{\tau^{n-1}(b)}{\tau^n(b)}=1.
    \]

    Now we prove $\implies$. Let $a\in E$ be such that $\norm_{E/K}(a)=1$. For 
    $c\in E$ let 
    \begin{align*}
        d_0 &= ac,\\
        d_1 &= a\tau(a)\tau(c),\\
        d_2 &= a\tau(a)\tau^2(a)\tau^2(c),\\
        &\vdots\\
        d_{n-1} &= \underbrace{a\tau(a)\cdots\tau^{n-1}(a)}_{=\norm_{E/K}(a)}\tau^{n-1}(c)=\tau^{n-1}(c).
    \end{align*}
    Then 
    \[
    a\tau(d_j)=a\tau(a)\cdots\tau^{j+1}(a)\tau^{j+1}(c)=d_{j+1}
    \]
    for all $j\in\{0,\dots,n-2\}$. Let $b=d_0+\cdots+d_{n-1}$. We claim that 
    $b\ne 0$ for some $c$. Suppose this is not true, say $b=0$ for all $c$. Then 
    \begin{align*}
    0&=ac+(a\tau(a))\tau(c)+\cdots+(a\tau(a)\cdots\tau^{n-1}(a))\tau^{n-1}(c)
    \end{align*}
    for every $c\in E$. This 
    implies that $a=0$ by Dedekind's theorem, a contradiction. 
    
    So let $c\in E$ be
    such that $b\ne 0$. Then 
    \begin{align*}
    \tau(b)&=\tau(d_0)+\cdots+\tau(d_{n-1})\\
    &=\tau(ac)+\tau(a\tau(a)\tau(c))+\cdots+\tau(\tau^{n-1}(c))\\
    &=\frac{1}{a}(d_1+\cdots+d_{n-1})+\tau^n(c)\\
    &=\frac{1}{a}(d_0+\cdots+d_{n-1})\\
    &=b/a.\qedhere
    \end{align*}
\end{proof}

\begin{exercise}
    Let $E/K$ be a cyclic extension. Assume that 
    $\Gal(E/K)$ is generated by $\tau$. Prove that for 
    $a\in E$, $\trace_{E/K}(a)=0$ if an only 
    if $a=b-\tau(b)$ for some $b\in E\setminus\{0\}$.  
\end{exercise}

%\include{13}

\backmatter

%\addcontentsline{lec}{chapter}{Some hints}
%\include{exercises}
\include{hints}

%\addcontentsline{lec}{chapter}{Some solutions}
\section*{Some solutions}

% \pagestyle{plain}
% \fancyhf{}
% \fancyhead[LE,RO]{Rings and modules}
% \fancyhead[RE,LO]{Some solutions}
% \fancyfoot[CE,CO]{\leftmark}
% \fancyfoot[LE,RO]{\thepage}

%\addcontentsline{toc}{chapter}{Some solutions}

% \begin{sol}{xca:algebraic_bijective}
%     Let $\sigma\colon C\to C$ be a homomorphism. As $\sigma$ is
%     injective, we need to prove that $\sigma$ is surjective. Let $y\in C$. Note that $y$ is algebraic over $K$. Let 
%     $R$ be the set of roots of the minimal polynomial 
%     $f(y,K)$ of $y$ over $K$. 
%     The map 
%     $\sigma|_R\colon R\to R$ is injective. Since 
%     $R$ is finite, $\sigma|_R$ is then bijective. In particular, 
%     there exists $x\in R$ such that $y=\sigma(x)$.
% \end{sol}


\begin{sol}{xca:Q(i)}
Assume that $\Q[i]$ and $\Q[\sqrt{2}]$ were isomorphic 
and let $\varphi:\Q[i]\to \Q[\sqrt{2}]$
be a field isomorphism.
Then 
    \[
    \varphi(i)^2=
    \varphi(i^2)=
    \varphi(-1)=
    -\varphi(1)=-1.
    \]
But $\varphi(i)\in\Q[\sqrt{2}]$ and
$\Q[\sqrt{2}]\subseteq \R$ where every square is 
positive, a contradiction.
\end{sol}

\begin{sol}{xca: field characteristic}
    Let $t>0$ be the characteristic of a field $K$ and 
    let $\varphi:\Z\to K,$ $x\mapsto x1$.
    Then, by definition, $\ker\varphi$ is 
    the ideal generated by $t$.
    On the other hand, the image $\varphi(\Z)$ is 
    a domain, being a subring of a field
    and is isomorphic to $\Z/\ker\varphi$.
    Therefore, $\ker\varphi$ is a prime ideal of $\Z$,
    i.e. $t$ is a prime number.
\end{sol}

\begin{sol}{xca: char 0}
Let $\varphi:\Z\to K,$ $x\mapsto x1$.

We first prove that \textbf{1)} implies all the other 
properties. So suppose that the characteristic of $K$ is zero, i.e. $\ker\varphi=\{0\}$. 

Then $m1=0$ if and only if $m=0$, i.e.
the order of $1$ is infinite.

Let $0\neq x\in K$. If $mx=0$, then $0=mx=(m1)x$.
But $K$ is a field and $x\neq 0$, hence $m1=0$,
so $m\in\ker\varphi=\{0\}$.
Hence the order of $x$ is infinite.

By definition, the ring of integers of $K$ is
the image of $\varphi$, which 
so it is isomorphic to $\Z/\ker\varphi=\Z$.

Finally, we prove that \textbf{4)} implies \textbf{1)}.
Take $m\in\ker\varphi$. Then $m1=0$,
but 1 has infinite order, hence $m=0$.
Therefore $\ker\varphi=\{0\}$, 
i.e. $K$ has characteristic 0.
\end{sol}

\begin{sol}{xca: char p}
Let $\varphi:\Z\to K,$ $x\mapsto x1$.

We first prove that \textbf{1)} implies all the other 
properties. So suppose that the characteristic of $K$ is $p>0$ i.e. $\ker\varphi$ is the ideal generated by $p$. 

Then $m1=0$ if and only if $p$ divides $m$, i.e.
the order of $1$ is $p$.

Let $0\neq x\in K$. If $mx=0$, then $0=mx=(m1)x$.
But $K$ is a field and $x\neq 0$, hence $m1=0$,
so $p$ divides $m$.
Hence $x$ has order $p$.

By definition, the ring of integers of $K$ is
the image of $\varphi$, which 
so it is isomorphic to $\Z/\ker\varphi\cong \Z/p$.

Finally, we prove that \textbf{4)} implies \textbf{1)}.
Take $m\in\ker\varphi$. Then $m1=0$,
but 1 has order $p$, hence $p$ divides $m$.
Therefore $\ker\varphi$ is generated by $p$, 
i.e. $K$ has characteristic 0.
\end{sol}

\begin{sol}{xca: Frobenius hom}
Let $\Phi: K\to K$ be the map 
$x\mapsto x^{p}$.
Since the map $x\mapsto x^{p^n}$
is exactly $\Phi^n$,
it is enough to prove that $\Phi$ is a field homomorphism.

As $K$ is commutative under multiplication, 
for all $x,y \in K$ 
        \[
        \Phi(xy) = (xy)^p = x^py^p = \Phi(x) \Phi(y).
        \]
Moreover, for all $x,y \in K$ 
\[
\Phi(x+y) =
(x+y)^p \sum_{k=0}^p \binom{p}{k}x^py^{p-k}=
x^p + y^p + \sum_{k=1}^{p-1} \binom{p}{k}x^ky^{p-k} ,
\]
where $\binom{p}{k} = \frac{p!}{k!(p-k)!}$,
which can also be written as
        \[
        p! = \binom{p}{k} \cdot k! \cdot (p-k)!
        \]
But $p$ divides $p!$, so $p$
has to divide at least one 
factor on the right side. 
But $p$ doesn't divide $i$
 for $1 \leq i \leq p-1$, therefore if $k \leq p-1$, $p$ doesn't divide
$k!$ and if $1 \leq  k$, $p$ doesn't divide $(p-k)!$. 
Hence, if $1 \leq k \leq p-1$, $p$ has to divide $\binom{p}{k}$ and 
\[
\sum_{k=1}^{p-1} \binom{p}{k}x^ky^{p-k}=0.
\]
Therefore, $\Phi$ is a field homomorphism.
\end{sol}

\begin{sol}{xca: prime field is fixed}
    By definition $K_0=\{m1\colon m\in \Z\}$ and 
    $\sigma:K\to K$ is a field homomorphism,
    so $\sigma(1)=1$.
    Hence, for every $m\in\Z$
    \[
    \sigma(m1)=m\sigma(1)=m1,
    \]
    i.e. $\sigma|_{K_0}$ is the identity.
\end{sol}

\begin{sol}{xca: X^3-2}
    If $X^3-2$ were reducible, since it has degree 3,
    it would have a linear
    factor in the decomposition in irreducibles.
    Therefore it would have a rational root.
    But the roots of $X^3-2$ are $\sqrt[3]{2},\sqrt[3]{2}\xi,\sqrt[3]{2}\xi^2$,
    where $\xi$ is a primitive third root of unity,
    So all the roots are not in $\Q$, a contradiction.
\end{sol}

\begin{sol}{xca:Eisenstein's criterion}
Recall first the following:
\begin{lemma}[Gauss' Lemma]
    Let $A$ be a unique factorization domain and $K$ be its fraction field.
    A non-constant polynomial $f\in A[X]$ is irreducible if and only if it is primitive and irreducible in $K[X]$.
\end{lemma}

    Suppose that $f$ is reducible in $K[X]$.
    Then $g=c^{-1}f$, where $c$ is the content of $f$,
    would be reducible and primitive.
    Hence, by Gauss' Lemma, $g$ is also 
    reducible in $A[X]$.
    So $c^{-1}f=g=hl$, for some non-constant polynomials $h,l\in A[X]$.
    Now consider $\pi:A\to A/(p)$, $a\mapsto \overline{a}$ the natural surjection.
    We know that $\overline{a_i}=0$ for all 
    $i\in\{0,1,\dots, n-1\}$ and $\overline{a_n}\neq 0$.
    Therefore 
    \[
    \overline{\pi}(ch)\overline{\pi}(l)=\overline{c}\:\overline{\pi}(h)\overline{\pi}(l)=\overline{\pi}(f)=\overline{a_n}X^n\in  A/(p)[X].
    \]
    But $A/(p)[X]$ is a UFD so the only possibility
    is that $\overline{\pi}(ch)=\overline{d}X^t$ and $\overline{\pi}(l)=\overline{f}X^s$, for
    some $f,d\in A/(p)\setminus\{\overline{0}\}$
    and $t,s\in\{1,\dots, n-1\}$.
    In particular, $\overline{\pi}(ch)$ and $\overline{\pi}(l)$ have both constant term 
    equal to 0.
    Hence $p$ divides $ch(0)$ and $l(0)$ in $A$.
    Therefore $p^2$ divides $ch(0)l(0)=f(0)$,
    a contradiction.
\end{sol}

\begin{sol}{xca: using Eisenstein}
It is easy to see that $f$ satisfies the Eisenstein criterion for $p=2$
and $g$ satisfies it for $p=5$. 
\end{sol}

\begin{sol}{xca: counterexample os Eisenstein in Z}
    $f=3(X^10+5X^2-15)$ is a product of $3$ and $(X^10+5X^2-15)$,
    which are both non-invertible elements of $\Z[X]$.
    Hence $f$ is reducible.
\end{sol}

\begin{sol}{xca: Q[sqrt2]=Q(sqrt2)}
Clearly for every field extension $L/K$ and 
    every $\alpha\in L$ we have that
    $K[\alpha]\subseteq K(\alpha)$.

    Vice versa take $\frac{a+\sqrt{2}b}{c+\sqrt{2}d}\in \Q(\sqrt{2})$,
    then we can write:
    \[
    \frac{a+\sqrt{2}b}{c+\sqrt{2}d}=\frac{(a+\sqrt{2}b)(c-\sqrt{2}d)}{(c+\sqrt{2}d)(c-\sqrt{2}d)}=
    \frac{ac-2bd+(bc-ad)\sqrt{2}}{c^2-2d^2}.
    \]
    Hence
    \[
    \frac{a+\sqrt{2}b}{c+\sqrt{2}d}=\frac{ac-2bd}{c^2-2d^2}+\frac{bc-ad}{c^2-2d^2}\sqrt{2}\in \Q[\sqrt{2}].
    \]
\end{sol}


\begin{sol}{xca:degree_of_x}
    Let $f=f(x,K)$ be the minimal polynomial of $x$ over $K$ of degree $\deg(f)=n$.
    We claim that $\{1,x,\dots, x^{n-1}\}$ is a basis of $K(x)$ as a $K$-vector space. 

    To prove that $\{1,x,\dots, x^{n-1}\}$ is a generating set, recall that $K(x)=K[x]$, since $x$ is algebraic over $K$. 
    Let $z\in K(x)=K[x]$, say $z=h(x)$ for some $h\in K[X]$. 
    Divide $h$ by $f$ to obtain polynomials $q,r\in K[X]$ 
    such that $h=fq+r$, where either $r=0$ or $\deg r<\deg f=n$. Then 
    \[
		z=h(x)=f(x)q(x)+r(x)=r(x).
	\]
	Write $r=\sum_{i=0}^{n-1}c_iX^i$ for some $c_0,\dots,c_{n-1}\in K$. 
    Thus $z=\sum_{i=0}^{n-1}a_ix^i\in \langle 1,x,\dots,x^{n-1}\rangle$.
        
    We now prove that $\{1,x,\dots, x^{n-1}\}$ is linearly independent. If not, 
    there exists a linear combination
    $0=\sum_{i=0}^{n-1}a_ix^i$ with $a_0,\dots,a_{n-1}\in K$ not all zero. 
    Then $h(X)=\sum_{i=0}^{n-1}a_iX^i\in K[X]\setminus\{0\}$
    has $x$ as a root and 
        \[
        n=\deg(f)\leq \deg(h)\leq n-1,
        \]
       a contradiction. 
\end{sol}

\begin{sol}{xca:algebraic}
$a$ is algebraic over $K$, 
so, by Theorem \ref{thm:simple extesnions},
it has finite degree over $K$
and $K[a]=K(a)$.
$b$ is algebraic over $K$,
so it is also algebraic over $K(a)$, hence it has finite degree over $K(a)$ and 
$K(a)[b]=K(a,b)$.
This implies that the extension
$K(a,b)/K$ is a finite extension
since it is a tower of finite extensions.
Hence, by Corollary \ref{cor:finite=>algebraic}, $K(a,b)/K$ is an algebraic extension.
Therefore, since $a+b,ab\in K(a,b)$,
this implies that $a+b$ and $ab$
are algebraic over $K$.
\end{sol}

\begin{sol}{xca:finite type}
    Assume that $K(S)/K$ is algebraic, then, by Corollary \ref{cor:finite type algebraic}, 
    $x$ is algebraic over $K$ for all $x\in S$.
    By Corollary \ref{cor:finite type finite}, we conclude that $K(S)/K$ is finite.

    On the other hand, if $K(S)/K$ is finite, then
    $K\subseteq K(x)\subseteq K(S)$ for all $x\in S$, so
    $K(x)/K$ is finite for all $x\in S$.
    Then, by Theorem \ref{thm:simple extesnions},
    $x$ is algebraic over $K$ for all $x\in S$.
    Hence, by Corollary \ref{cor:finite type algebraic}, $K(S)/K$ is algebraic.
 \end{sol}

\begin{sol}{xca:degree of sqrt[3]2}
    $\sqrt[3]{2}$ is a root of the monic polynomial $f=X^3-2\in \Q[X]$.
    Therefore $\sqrt[3]{2}$ is algebraic over $\Q$ 
    and $\Q[\sqrt[3]{2}]=\Q(\sqrt[3]{2})$.
    In Exercise \ref{xca: X^3-2}, we proved that
    $f$ is irreducible in $\Q[X]$.
    Hence $f$ is the minimal polynomial of $\sqrt[3]{2}$
    over $\Q$ and, by Theorem \ref{thm:simple extesnions},
    $[\Q(\sqrt[3]{2}):\Q]=\deg f=3$.
\end{sol}

\begin{sol}{xca:Q(i,sqrt2)}
    $i$ is a root of the monic polynomial $X^2+1\in\Q[X]$
    and $\sqrt{2}$ is a root of the monic polynomial of 
    $X^2-2\in\Q[X]$.
    So, by Corollary \ref{cor:finite type finite},
     $\Q[i,\sqrt{2}]=\Q(i,\sqrt{2})$ and it is algebraic over $\Q$.
    By Eisenstein's criterion with $p=2$,
    $X^2-2$ is irreducible in $\Q[X]$, so 
    $[\Q(\sqrt{2}):\Q]=2$.
    Since $i$ is a root of $X^2+1\in\Q[X]$,
    then $[E:\Q(\sqrt{2})]\leq 2$.
    Moreover, $i\notin \R\supseteq \Q(\sqrt{2})$.
    Therefore $[E:\Q(\sqrt{2})]=2$ and,
    by Proposition \ref{pro:multiplicativity of degree},
    $$[E:\Q]=[E:\Q(\sqrt{2})][\Q(\sqrt{2}):\Q]=4$$.
\end{sol}

\begin{sol}{xca:Q(sqrt2,sqrt[3]5)}
    
\end{sol}

\begin{sol}{xca:dec field X^4-5X^2+5}
Note that, since $f=X^4-5X^2+5$ is an even polynomial
if $\alpha\in \C$ is a root of $f$,
then also $-\alpha$ is a root of $f$.
 Hence, given two roots $\alpha,\beta\in \C$
 such that $\beta\neq-\alpha$,
we have that the decomposition field of $f$ over $\Q$ is
$E=\Q(\alpha,-\alpha,\beta,-\beta)$.
 But $-\alpha,-\beta\in \Q(\alpha,\beta)\subseteq E$
 and so 
\[
 E=\Q(\alpha,-\alpha,\beta,-\beta)\subseteq \Q(\alpha,\beta)\subseteq\Q(\alpha,-\alpha,\beta,-\beta)=E,
 \]
 which means that $E=\Q(\alpha,\beta)$.
 Moreover we can decompose $f$ in $\C[X]$ as 
 \[
 (X-\alpha)(X+\alpha)(X-\beta)(X+\beta)=(X^2-\alpha^2)(X^2-\beta^2)=X^4-(\alpha^2+\beta^2)X^2+\alpha^2\beta^2.
 \]
This implies in particular that $\alpha^2\beta^2=5$, hence $\beta=\pm \frac{\sqrt{5}}{\alpha}\in \Q(\alpha,\sqrt{5})$.

Therefore $E=\Q(\alpha,\beta)\subseteq \Q(\alpha,\sqrt{5})$.
On the other hand $\sqrt{5}=\pm\alpha\beta\in\Q(\alpha,\beta)$,
 hence $\Q(\alpha,\sqrt{5})\subseteq\Q(\alpha,\beta)=E$.
 So we can conclude that $E=\Q(\alpha,\sqrt{5})$.
Using the multiplicative of the degree of finite extension we get that
 \[
 [E:\Q]=[E:\Q(\alpha)][\Q(\alpha):\Q].
\]
But $[\Q(\alpha):\Q]=\deg(f(\alpha,\Q))$.
 Using Eisenstein criterion (Exercise \ref{xca:Eisenstein's criterion}) with $p=5$,
 we have that $f$ is irreducible (and monic), so $f=f(\alpha,\Q)$.
Thus $[\Q(\alpha):\Q]=\deg f=4$.
It remains to compute $[E:\Q(\alpha)]=[\Q(\alpha,\sqrt{5}):\Q(\alpha)]$.
We have the following situation:
\[
\begin{tikzcd}
	& {E=\Q(\alpha,\sqrt{5})}\arrow[rd,no head]\\
	{\Q(\alpha)}\arrow[ru,no head,"\leq 2"]&& {\Q(\sqrt{5})}\arrow[ld,no head,"2"]  \\
	& {\Q}\arrow[lu,no head,"4"] 
 \end{tikzcd}
\]
Observe that $\Q(\alpha,\sqrt{5})$ is equal to the composite of $\Q(\alpha)$ and $\Q(\sqrt{5})$.
 We can use the property of composite extension,
$[LF:L]\leq[F:K]$, to deduce that 
 \[
 [\Q(\alpha,\sqrt{5}):\Q(\alpha)]\leq [\Q(\sqrt{5}):\Q]=2.
 \]
 The last equality is because $f(\sqrt{5},\Q)=X^2-5$,
 as it is monic has $\sqrt{5}$ as a root 
 and it's irreducible 
 (due to Eisenstein's criterion or
 because it is of degree 2 with 2 non-rational roots).
 Finally, we want to understand whether $[\Q(\alpha,\sqrt{5}):\Q(\alpha)]$ is 1 or 2.
 Note that $\alpha^4-5\alpha^2+5=0$, so we can solve the equation for
$\alpha^2$ as it is a root of $X^2-5X+5$, i.e.
 \[
\alpha^2=\frac{5\pm \sqrt{25-20}}{2}=\frac{5\pm \sqrt{5}}{2},
 \]
 hence $\sqrt{5}=\pm (2\alpha^2-5)\in\Q(\alpha)$.
 So $\Q(\alpha,\sqrt{5})\subseteq\Q(\alpha)\subseteq \Q(\alpha,\sqrt{5})$, 
which means that $E=\Q(\alpha)$ and
$[E:\Q]=[\Q(\alpha):\Q]=4$.
\end{sol}


\begin{sol}{xca:Q(sqrt[3]{2},xi) normal}
First of all, note that $\sqrt[3]{2}$ is a root of
the polynomial $f(X)=X^3-2$.
To prove that $\Q(\sqrt[3]{2},\xi)$ is a normal extension 
we use Proposition \ref{pro:normal<=>dec},
so it is enough to prove
that $\Q(\sqrt[3]{2},\xi)$ is the decomposition field of $f$.
We know that the decomposition field $E$ of $f$ over $\Q$ is
$\Q$ extended with the roots of $f$, i.e.
$E=\Q(\sqrt[3]{2},\sqrt[3]{2}\xi,\sqrt[3]{2}\xi^2)$.
But it's easy to see that actually 
\[
\Q(\sqrt[3]{2},\xi)=\Q(\sqrt[3]{2},\sqrt[3]{2}\xi,\sqrt[3]{2}\xi^2)=E.
\]
The inclusion $\subseteq$ is because $\sqrt[3]{2},  \xi=\frac{\sqrt[3]{2}\xi}{\sqrt[3]{2}}\in E$. 
Vice versa $\supseteq$ is due to the fact that 
the roots of $f$ are products of $\sqrt[3]{2}$ and $\xi$, elements in $\Q(\sqrt[3]{2},\xi)$.
\end{sol}


\begin{sol}{xca:Q[sqrt[4]{7}+sqrt{2}]}
Let $\alpha=\sqrt[4]{7}+\sqrt{2}$. Then $(\alpha - \sqrt{2})^4 -7 = 0$.
By expanding the left side, we get
\[
0  =\alpha^4 - 4\sqrt{2}\alpha^3 + 12 \alpha^2 - 8\sqrt{2}\alpha - 3 
 = (\alpha^4 + 12\alpha^2 - 3) -  (4 \alpha^3 + 8 \alpha )\sqrt{2}.
\]
But $4 \alpha^3 + 8 \alpha = 4\alpha (\alpha^2+2) \neq 0$, otherwise $\alpha\in\{0,\pm i\sqrt{2})$.
Therefore $\sqrt{2}=\frac{\alpha^4 + 12\alpha^2 - 3}{4 \alpha^3 + 8 \alpha}\in \Q(\alpha) $.

This allows us to prove that $\Q(\sqrt{2} , \sqrt[4]{7}) = \Q(\alpha)$.
From the definition of $\alpha$ it's clear that 
$\Q(\alpha) \subseteq \Q(\sqrt{2} , \sqrt[4]{7})$.
On the other hand, we just proved that $\sqrt{2} \in \Q(\alpha)$.
As $\sqrt[4]{7} = \alpha - \sqrt{2} \in \Q(\alpha)$,
we also see that $\sqrt[4]{7} \in \Q(\alpha)$.
It follows that $\Q(\sqrt{2} , \sqrt[4]{7}) \subseteq \Q(\alpha)$.

Moreover, $\sqrt{2}\not\in \Q(\sqrt[4]{7})$. 
Otherwise, as $[\Q(\sqrt[4]{7}):\Q] = 4$ and $[\Q(\sqrt{2}):\Q] = 2$
we would get that $[\Q(\sqrt[4]{7}):\Q(\sqrt{2})] = 2$.
Let $f(\sqrt[4]{7},\Q(\sqrt{2})) = X^2 + \beta X + \gamma$,
with $\beta, \gamma \in \Q(\sqrt{2})$. 
So 
$$0=f(\sqrt[4]{7})=\sqrt{7} + \beta \sqrt[4]{7} + \gamma.$$
Therefore 
$$\beta^2 \sqrt{7}  = (- \sqrt{7} - \gamma)^2 =7 + 2\gamma \sqrt{7}  + \gamma^2.$$
So $(\beta^2 - 2 \gamma)\sqrt{7}= \gamma^2 + 7$.
But $\beta^2 - 2 \gamma\neq 0$ because 
$\gamma^2 + \beta \gamma + \frac{\beta^2}{2} = 0$ 
holds only for $\gamma =\frac{\beta}{2} (-1 \pm i)\in \C\setminus\R$,
which is clearly not in $\Q(\sqrt{2})$.
Thus we would have 
$\sqrt{7} = \frac{\gamma^2 + 7}{\beta^2 - 2 \gamma} \in \Q(\sqrt{2})$, which 
is a contradiction.

To sum up we have that $\sqrt{2}\not\in \Q(\sqrt[4]{7})$ and
\[
\begin{tikzcd}
	& {E=\Q(\alpha)=\Q(\sqrt{2},\sqrt[4]{7})}\arrow[rd,no head]\\
	{\Q(\sqrt{2})}\arrow[ru,no head,]&& {\Q(\sqrt[4]{7})}\arrow[ld,no head,"4"]  \\
	& {\Q}\arrow[lu,no head,"2"] 
 \end{tikzcd}
\]
\begin{enumerate}
    \item We know that $\sqrt[4]{7} \in \Q(\alpha)$ 
    which has minimal polynomial $f(\sqrt[4]{7},\Q) = x^4-7$.
    One root of this polynomial is $i\sqrt[4]{7}$.
    This root is not in $\Q(\alpha)\subseteq \R$ as it is in $\C\setminus\R$. 
    Therefore $\Q(\alpha)/\Q$ is not normal by Proposition \ref{pro:linear_factorization}.
    \item  As $\sqrt{2} \not\in \Q(\sqrt[4]{7})$,
    we see that $[\Q(\alpha):\Q(\sqrt[4]{7})] > 1$.
    On the other hand, 
    $$[\Q(\alpha):\Q(\sqrt[4]{7})] \leq [\Q(\sqrt{2}:\Q]=2,$$
    which proves that $[\Q(\alpha):\Q(\sqrt[4]{7})] = 2$.
    Therefore,
    \[
     [E:\Q]=[\Q(\alpha):\Q] = [\Q(\alpha):\Q(\sqrt[4]{7})] \cdot [\Q(\sqrt[4]{7}): \Q] = 2 \cdot 4 = 8.
    \]
    \item Let $\sigma\in G=\Gal(E/\Q)$.
    Since $E=\Q(\sqrt{2},\sqrt[4]{7})$ and 
    $\sqrt{2}$ and $\sqrt[4]{7}$ are independent 
    because $\sqrt{2}\not\in \Q(\sqrt[4]{7})$,
    we know that $\sigma$ is completely determined
    by $\sigma(\sqrt{2})$ and $\sigma(\sqrt[4]{7})$.
    By Proposition \ref{pro:conjugate},
    $\sigma(\sqrt{2})\in E$ has to be a root of $f(\sqrt{2},\Q)=X^2-2$ 
    and $\sigma(\sqrt[4]{7})\in E$ has to be a root of $f(\sqrt[4]{7},\Q)=X^4-7$.
    So $\sigma(\sqrt{2})=\pm\sqrt{2}$ and, since $E\subseteq \R$,
    $$\sigma(\sqrt[4]{7})\in E\cap\{\sqrt[4]{7}i^j\mid j\in\{0,1,2,3\}\}=\{\pm\sqrt[4]{7}\}.$$
    Therefore $G$ contains 4 elements $\sigma_{k,l}$ for $k,l\in \Z/2$
    such that $\sigma_{k,l}(\sqrt{2})=(-1)^k\sqrt{2}$ 
    and $\sigma_{k,l}(\sqrt[4]{7})=(-1)^l\sqrt[4]{7}$.
    This gives directly the isomorphism between $G$ and $\Z/2\times\Z/2$. 
\end{enumerate}
\end{sol}



\begin{sol}{xca:dim}
 Let $\{v_i:i\in I\}$ be a basis of $V$ over $K$. For each $i\in I$
 let $f_i\colon V\to F$, $f_i(v_j)=\delta_{ij}$. Then $\{f_i:i\in I\}$ is linearly 
 independent over $F$. In fact, let 
 $\sum a_if_i=0$, where each $a_i\in F$. Then 
 $a_i=0$ for almost all $i$. If $j\in I$, then 
 \[
 0=\left(\sum a_if_i\right)(v_j)=\sum a_if_i(v_j)=a_j.
 \]
 Now assume that $\dim_KV=n$. Let $\{v_1,\dots,v_n\}$ be a basis of $V$ over $K$.
 We claim that $\{f_1,\dots,f_n\}$ is a basis of $\Hom_K(V,F)$ over $F$. If 
 $g\in\Hom_K(V,F)$, then $g=\sum g(v_i)f_i$. If $1\leq k\leq n$, then
 \[
 \left(\sum g(v_i)f_i\right)(v_k)=\sum g(v_i)f_i(v_k)=g(v_k).
 \]
\end{sol}



\begin{sol}{xca:gamma_C}
We need to find a bijective map 
\[
\Hom(E/K,C/K)\to\Hom(E/K,C_1/K).
\]
If $\sigma\in\Hom(E/K,C/K)$, then $\theta^{-1}\sigma\in\Hom(E/K,C_1/K)$. 
If $\varphi\in\Hom(E/K,C_1/K)$, then $\theta\varphi\in\Hom(E/K,C/K)$. The 
maps $\sigma\mapsto\theta^{-1}\sigma$ and 
$\varphi\mapsto\theta\varphi$ are inverse to each other. 
\end{sol}

\begin{sol}{xca:order_reversing}
    We first prove that for every order
    reversing function $\varphi$
    and every element $s,t$ in its domain,
    \[
    \varphi(s\vee t)\leq \varphi(s)\wedge \varphi(t) \text{ and }
    \varphi(s)\vee \varphi(t)\leq \varphi(s\wedge t).
    \]
    Since that $s\leq s\vee t$
    and $t\leq s\vee t$ and $\varphi$
    is order reversing,
    we have that $\varphi(s\vee t)\leq \varphi(s)$ 
    and $\varphi(s\vee t)\leq \varphi(t)$.
    Hence $\varphi(s\vee t)\leq \varphi(s)\wedge \varphi(t).$
    Moreover $s\wedge t\leq s$
    and $s\wedge t\leq t$.
    So $\varphi(s)\leq \varphi(s\wedge t)$ and $\varphi(t)\leq \varphi(s\wedge t)$.
    Hence 
    $\varphi(s)\vee \varphi(t)\leq \varphi(s\wedge t).$

    We can now apply this result
    for $\varphi=f$, $s=x$ and $t=y$
    obtaining that
    \[
    f(x\vee y)\leq f(x)\wedge f(y), \quad
    f(x)\vee f(y)\leq f(x\wedge y).
    \]
    On the other hand, for $\varphi=f^{-1}$, 
    $s=f(x)$ and $t=f(y)$,
    we obtain that 
    \[
    f^{-1}(f(x)\vee f(y))\leq
    f^{-1}(f(x))\wedge f^{-1}(f(y))=
    x\wedge y,
    \]
    and
    \[
    x\vee y=f^{-1}(f(x))\vee f^{-1}(f(y))\leq f^{-1}\big(f(x)\wedge f(y)\big).
    \]
    Thus, applying $f$, which is order reversing,
    it implies that 
    \[
    f(x\wedge y)\leq f\big(f^{-1}(f(x)\vee f(y))\big)=
    f(x)\vee f(y)
    \]
    and
    \[
    f(x)\wedge f(y)=f\big(f^{-1}\big(f(x)\wedge f(y)\big)\big)\leq f(x\vee y).
    \]
 \end{sol}

 \begin{sol}{xca:Cauchy+Galois}
     Since $E/K$ is a Galois extension,
     the order of $\Gal(E/K)$ is precisely $[E:K]=n$.
     So, by Cauchy's Theorem, there exists a subgroup $S$ 
     of $\Gal(E/K)$
     of order $p$.
     Then, by Galois' Theorem, the subextension
     ${}^SE/K$
     has degree equal to the index of $S$,
     which is $n/p$.
 \end{sol}

 \begin{sol}{xca:Sylow+Galois}
     Since $E/K$ is a Galois extension,
     $|\Gal(E/K)|=[E:K]=p^\alpha m$.
     So, by Sylow's Theorem, there exists a subgroup $P$ 
     of $\Gal(E/K)$
     of order $p^{\alpha}$.
     Then, by Galois' Theorem, the subextension
     ${}^PE/K$
     has degree $(\Gal(E/K):P)=m$.
 \end{sol}

\begin{sol}{xca:separable_charp}
    Write $f=X^n+a_{n-1}X^{n-1}+\cdots+a_1X+a_0$. Then 
    \[ 
    f'=nX^{n-1}+(n-1)a_{n-1}X^{n-2}+\cdots+2a_2X+a_1.
    \]
    Since $f$ is not separable, $f'=0$. Thus $n=ka_k=0$ in $K$ for all $k\in\{0,\dots,n-1\}$. This implies
    that $p$ divides $k$ whenever $a_k\ne 0$. This means that the only terms in $f$ occur in degree 
    that are multiples of $p$. In particular, $n=pm$ for some $m$. Hence 
    \[
    f=X^{pm}+a_{p(n-1)}X^{p(m-1)}+\cdots+a_pX^p+a_0=g(X^p)
    \]
    for some $g\in K[X]$.  
\end{sol}

% \begin{sol}\
%     \begin{enumerate}
%         \item If $E/K$ is not separable, 
%         then $[E:K]_{\operatorname{ins}}=p^s$ for some $s$. 
%         Then the trace is zero because $K$ is of characteristic $p$. 
%         \item 
%     \end{enumerate}
% \end{sol}

\begin{sol}{xca:solvable+simple}
 If $G$ is solvable, then $[G,G]$ is a proper normal subgroup of $G$. 
 Since $G$ is simple, $[G,G]=\{1\}$ and $G$ is abelian. Thus $G$ is cyclic of prime order.
\end{sol}

\begin{sol}{xca:diagonal}
 Assume that $G$ is simple. Let $A=G\times\{1\}$, $B=\{1\}\times G$ and
 $D=\{(x,x):x\in G\}$ the diagonal subgroup of $G\times G$. 
 Since 
 \[
 (g,h)=(g,1)(1,h)=(gh^{-1},1)(h,h)
 \]
 it follows that $G=AB=AD$. Let $M$ be a subgroup of $G\times G$ that contains $D$. 
 Note that
 \[
 M=M\cap (G\times G)=M\cap AD=(M\cap A)D. 
 \]
 Similarly, $M=(M\cap B)D$. Since $A$ is normal in $G\times G$, $M\cap A$ is normal in $G\times G$ 
 and $(M\cap A)B$ is normal in $MB=G\times G$. Using the second isomorphism theorem, we see that
 \[
 M\cap A\simeq \frac{(M\cap A)B}{B}
 \]
 is a normal subgroup of $(G\times G)/B\simeq A$. Since $A\simeq G$ is simple, either 
 $M\cap A=\{1\}$ or $M\cap A=A$. Thus either $M=D$ or $BD=G\times G$. Therefore $D$ is maximal.
 \end{sol}


%\addcontentsline{lec}{chapter}{References}
\bibliographystyle{abbrv}
\bibliography{refs}


\printindex     
%\phantom{Trick}
%\addcontentsline{lec}{chapter}{\indexname}

\end{document}





