\section{Lecture -- Week 5}


\subsection{Normal extensions}

\begin{proposition}
    Let $E/K$ be an algebraic extension and $\sigma\in\Hom(E/K,E/K)$. 
    Then $\sigma$ is bijective. 
\end{proposition}

\begin{proof}
    It is enough to prove that $\sigma$ is surjective. Why? 
    
    Let $x\in E$ and 
    $C$ be an algebraic closure of $K$ that contains $E$. By Proposition~\ref{pro:Artin}, there 
    exists a field homomorphism $\varphi\colon C\to C$ such that $\varphi|_E=\sigma$. 
    Thus $\varphi|_K=\sigma|_K=\id_K$. Let $G=\Gal(C/K)$. Then  
    $\varphi\in G$. If $z\in O_G(x)$, 
    then $z=\tau(x)$ for some $\tau\in G$ and hence
    \[
    \varphi(z)=\varphi(\tau(x))=(\varphi\tau)(x). 
    \]
    This implies that $\varphi(z)\in O_G(x)$ 
    and $\varphi(O_G(x))=O_G(x)$. The restriction
    $\sigma|_{E\cap O_G(x)}$ is injective. Then 
    \begin{align*}
        \sigma(E\cap O_G(x))&=\varphi(E\cap O_G(x))\\
        &=\varphi(E)\cap\varphi(O_G(x))
        =\sigma(E)\cap O_G(x)\subseteq E\cap O_G(x).
        \end{align*}
    Since $|E\cap O_G(x)|<\infty$, it follows that $E\cap O_G(x)=\sigma(E\cap O_G(x))$ and
    hence $x$ belongs to the image of $\sigma$. 
\end{proof}


% pag 21 example

\begin{definition}
    Let $E/K$ be an algebraic extension and $C$ be an algebraic closure of $K$ containing $E$. Then $E/K$ is \emph{normal} if 
    $\sigma(E)\subseteq E$ for all $\sigma\in\Hom(E/K,C/K)$. 
\end{definition}

Note that $\sigma(E)\subseteq E$ in the previous definition
is equivalent to $\sigma(E)=E$. 

\begin{example}
    The extension $\Q(\sqrt[3]{2})/\Q$ is not normal. Why?
\end{example}

Some trivial examples of normal extensions: $K/K$ is normal
and if $C$ is an algebraic closure of $K$, then $C/K$ is normal. 

\begin{example}
    The extension $\Q(\sqrt{2})/\Q$ is normal. 
    Every extension generated by algebraic elements of degree two is normal. 
\end{example}

\begin{exercise}
\label{xca:Q(sqrt[3]{2},xi) normal}
    Let $\xi$ be a primitive cubic root of one. Then 
    $\Q(\sqrt[3]{2},\xi)/\Q$ is normal. 
\end{exercise}

The following result is practical but technical. That is why we leave the proof
as an exercise. 

\begin{exercise}
    Prove that the previous definition depends only on $E$ (and not on the
    algebraic closure $C$). 
\end{exercise}

Some properties:

\begin{proposition}
\label{pro:linear_factorization}
    Let $E/K$ be a normal extension and $f\in K[X]$ be an irreducible polynomial
    that admits a root $x$ in $E$. Then $f$ factorizes
    linearly in $E$.
\end{proposition}

\begin{proof}
    We may assume that $f$ is monic. Let $C/K$ be an algebraic closure of $K$ containing $E$. 
    Let $y$ be a root of $f$ in $C$. Since $f=f(x,K)=f(y,K)$, 
    it follows that $y=\sigma(x)$ for some $\sigma\in\Gal(C/K)$. Since 
    $E/K$ is normal, $\sigma|_E\colon E\to C$ is an automorphism of $E/K$, that is
    $\sigma(E)\subseteq E$. In particular, $y\in E$. 
\end{proof}

Let $K\subseteq F\subseteq E$ be a tower of fields. 
If $E/K$ is normal, then $E/F$ is normal. However, 
Note that $E/K$ normal does not imply $F/K$ normal, as this would imply 
that every extension is normal. Moreover, 
$E/F$ normal and $F/K$ normal do not imply $E/K$ normal.
    
\begin{example}
The extensions $\Q(\sqrt[4]{2})/\Q(\sqrt{2})$ and $\Q(\sqrt{2})/\Q$ are both
normal, but the extension $\Q(\sqrt[4]{2})/\Q$ is not, 
as the roots of $X^4-2$ are
$\sqrt[4]{2}$, $-\sqrt[4]{2}$, $\sqrt[4]{2}i$ and $-\sqrt[4]{2}i$.
\end{example}


Recall that if $C$ is an algebraic closure of $K$ and $x\in C$,
then 
\[
f(x,K)=\prod(X-y)^m,
\]
where the product is taken over all 
$y\in O_{\Gal(C/K)}(x)$. 
If $E/K$ is normal and $x\in E$, then there exists $m$ such that 
\[
f(x,K)=\prod(X-y)^m,
\]
where the product is taken over all 
$y\in O_{\Gal(E/K)}(x)$. 

\begin{proposition}
    Let $E/K$ and $F/K$ be extensions. If $F/K$ 
    is normal, then $EF/E$ is normal.
\end{proposition}

\begin{proof}
    Let $C$ be an algebraic closure of $E$ containing $EF$ (this exists because $EF/E$ is algebraic). 
    Let $\sigma\in\Hom(EF/E,C/E)$. We claim that $\sigma(EF)=EF$. Let 
    \[
    \overline{K}=\{x\in C:x\text{ is algebraic over $K$}\}.
    \]
    Then $\overline{K}$ is an algebraic closure over $K$ and $F\subseteq\overline{K}$. 
    Since $F/K$ is normal and $\sigma|_F\in\Hom(F/K,\overline{K}/K)$, 
    it follows that $\sigma(F)=F$. If $z\in EF$, then
    $z=\sum_{i=1}^m e_if_i$ for some $e_1,\dots,e_m\in E$ and 
    $f_1,\dots,f_m\in F$. Since $\sigma(e_i)=e_i$ for all $i$,  
    \[
    \sigma(z)=\sum_{i=1}^m\sigma(e_i)\sigma(f_i)=\sum_{i=1}^m e_i\sigma(f_i)\in EF.\qedhere 
    \]
\end{proof}

What is the relation between 
normal extensions and splitting fields? The notions look
deeply related. The following proposition serves as an explanation: 

\begin{proposition}
\label{pro:normal<=>dec}
    Let $E/K$ be an algebraic extension. Then 
    $E/K$ is normal if and only if $E/K$ is the splitting field
    of a family of polynomials of $K[X]$ of positive degree.
\end{proposition}

\begin{proof}
    Assume first that $E/K$ is a normal extension. 
    Let $G=\Gal(E/K)$.  If $x\in E$ and $f(x,K)=\prod_{y\in O_G(x)}(X-y)^m$, 
    then $f(x,K)$ factorizes linearly in $E[X]$. Thus 
    $E/K$ is a splitting field of the family 
    $\{f(x,K):x\in E\}$. 
    
    Conversely, assume that $E/K$ is a splitting field of the family 
    $\{f_i:i\in I\}$. Then $E=K(S)$ where $S$ is the set of roots
    of the polynomials $f_i$. Let $C/K$ be an algebraic closure
    of $K$ that contains $E$ and let $\sigma\in\Hom(E/K,C/K)$. Let $x\in S$. 
    Then $x$ is a root of some $f_j=\sum a_kX^k$. Since $f_j(x)=0$, 
    it follows that $\sigma(x)$ is a root of $f_j$, as 
    \[
    f_j(\sigma(x))=\sum a_k\sigma(x)^k
    =\sum\sigma(a_k)\sigma(x^k)
    =\sigma\left(\sum a_kx^k\right)=\sigma(0)=0.
    \]
    Hence $\sigma(E)\subseteq E$. 
\end{proof}

\begin{exercise}
\label{xca:Q[sqrt[4]{7}+sqrt{2}]}
    Let $E=\Q[\sqrt[4]{7}+\sqrt{2}]$. 
    \begin{enumerate}
        \item Prove that $E/\Q$ is not normal. 
        \item Compute $[E:\Q]$.
        \item Compute $\Gal(E/\Q)$. 
    \end{enumerate}
\end{exercise}


\subsection{Dedekind's theorem}

Note that every extension homomorphism $E/K\to F/K$ is, in particular, 
a $K$-linear map $E\to F$, that is
\[
\Hom(E/K,F/K)\subseteq\Hom_K(E,F).
\]
If $F/K$ is an extension and $V$ 
is a $K$-vector space, the set
$\Hom_K(V,F)$ of $K$-linear maps
is a vector space over $F$ with
$(a\cdot f)(v)=af(v)$ for $a\in F$, 
$f\in\Hom_K(V,F)$ and $v\in V$. 

\begin{exercise}
\label{xca:dim}
    Let $V$ be a $K$-vector space. 
    Prove that $\dim_F\Hom_K(V,F)\geq\dim_KV$. Moreover, if 
     $\dim_KV<\infty$, then $\dim_F\Hom_K(V,F)=\dim_KV$.
\end{exercise}

If $V$ is a vector space and $S$ is a (possibly infinite) subset of $V$, 
then $S$ is linearly independent if every finite subset of $S$ is linearly independent. 

\begin{theorem}[Dedekind]
\index{Dedekind's theorem}
Let $E/K$ and $F/K$ be extensions and let 
$\{\varphi_i:i\in I\}$ 
be a subset of
$\Hom(E/K,F/K)$, i.e. a 
family of extension homomorphisms. Assume that 
$\varphi_i\ne \varphi_j$ if $i\ne j$. Then 
the subset $\{\varphi_i:i\in I\}\subseteq\Hom_K(E,F)$ 
is linearly independent over $F$. 
\end{theorem}

\begin{proof}
    Assume it is not. Let $\{\varphi_1,\dots,\varphi_n\}$ 
    be linearly dependent over $F$ with $n$ minimal. Clearly, $n>1$. 
    Without loss of generality, we may assume that 
    \begin{equation}
        \label{eq:Dedekind1}
        \sum_{i=1}^n a_i\varphi_i=0
    \end{equation}
    for some $a_1,\dots,a_n\in F$ all different from zero. 
    Let
    $z\in E\setminus\{0\}$ be such that $\varphi_1(z)\ne\varphi_2(z)$. If $x\in E$, then
    \[
    0=\left(\sum_{i=1}^na_i\varphi_i\right)(xz)=\sum_{i=1}^na_i\varphi_i(xz)
    =\sum_{i=1}^na_i\varphi_i(x)\varphi_i(z)
    =\left(\sum_{i=1}^n (a_i\varphi_i(z))\varphi_i\right)(x).
    \]
    Thus 
    \[
        \sum_{i=1}^n (a_i\varphi_i(z))\varphi_i=0.
    \]
    Since $\varphi_1(z)\ne0$, 
    \begin{equation}
    \label{eq:Dedekind2}
        a_1\varphi_1+a_2\frac{\varphi_2(z)}{\varphi_1(z)}\varphi_2+\cdots+a_n\frac{\varphi_n(z)}{\varphi_1(z)}\varphi_n=0.
    \end{equation}

    Thus, 
    subtracting \eqref{eq:Dedekind1} and \eqref{eq:Dedekind2}, 
    \[
    \left(a_2-a_2\frac{\varphi_2(z)}{\varphi_1(z)}\right)\varphi_2
    +\cdots+\left(a_n-a_n\frac{\varphi_n(z)}{\varphi_1(z)}\right)\varphi_n=0.
    \]
    Since $a_n\ne 0$ and $\varphi_2(z)\ne\varphi_1(z)$, 
    the scalar $a_2-a_2\frac{\varphi_2(z)}{\varphi_1(z)}\ne 0$ and hence 
    $\{\varphi_2,\dots,\varphi_n\}$ is linearly dependent, a contradiction. 
\end{proof}

If $E/K$ and $F/K$ are extensions, 
let $\gamma(E/K,F/K)=|\Hom(E/K,F/K)|$. 

\begin{exercise}
Prove the following statements:
\begin{enumerate}
    \item $\gamma(E/K,F/K)\leq\dim_F\Hom_K(E,F)$.
    \item If $[E:K]<\infty$, then $\gamma(E/K,F/K)\leq[E:K]$. 
    \item If $x$ is algebraic over $K$, then $\gamma(K(x)/K,F/K)\leq\deg f(x,K)$.
\end{enumerate}
\end{exercise}

If $C$ is an algebraic closure of $K$,
then we define $\gamma(E/K)=\gamma(E/K,C/K)$. This definition does
not depend on the algebraic closure. 

\begin{exercise}
\label{xca:gamma_C}
    If $C$ and $C_1$ are algebraic closures of $K$, then
    \[
    |\Hom(E/K,C/K)|=|\Hom(E/K,C_1/K)|.
    \]
\end{exercise}

\begin{proposition}
\label{pro:gamma_orbit}
    Let $C$ be an algebraic closure of $K$ and $G=\Gal(C/K)$. 
    If $x\in C$, then $\gamma(K(x)/K)=|O_G(x)|$. 
\end{proposition}

\begin{proof}
    If $\sigma\in\Hom(K(x)/K,C/K)$, then there exists $\phi\in G$ such that
    $\phi|_{K(x)}=\sigma$. Thus 
    \[
    \sigma(x)=\phi(x)\in O_G(x).
    \]
    Conversely,
    if $y\in O_G(x)$, then there exists $\tau\in G$ such that
    $y=\tau(x)$. Hence 
    \[
    \tau|_{K(x)}\in\Hom(K(x)/K,C/K)
    \]
    and 
    $\tau|_{K(x)}(x)=y$. Since our sets are then in bijective correspondence, 
    the claim follows. 
\end{proof}


\begin{exercise}
If $E/K$ is finite, then $|\Gal(E/K)|\leq [E:K]$. Moreover, 
$E/K$ is normal if and only if $|\Gal(E/K)|=\gamma(E/K)$. 
\end{exercise}

